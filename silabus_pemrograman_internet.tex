\documentclass[12pt,a4paper]{article}
\usepackage[utf8]{inputenc}
\usepackage[T1]{fontenc}
\usepackage[indonesian]{babel}
\usepackage{xurl}
\usepackage{geometry}
\usepackage{longtable}
\usepackage{array}
\usepackage{booktabs}
\usepackage{enumitem}
\usepackage[hidelinks,breaklinks]{hyperref}
\geometry{margin=2.5cm}

\begin{document}

\begin{center}
\textbf{\Large RENCANA PEMBELAJARAN SEMESTER (RPS)}\\
\textbf{Berbasis Outcome-Based Education (OBE)}\\[0.5cm]
\textbf{PROGRAM STUDI TEKNIK INFORMATIKA}\\
\textbf{FAKULTAS TEKNOLOGI INFORMASI}\\
\textbf{UNIVERSITAS BALE BANDUNG}\\
\end{center}

\vspace{1cm}

% ============================================================
% 1. IDENTITAS MATA KULIAH
% ============================================================
\section*{1. Identitas Mata Kuliah}
\begin{tabular}{ll}
Nama Program Studi & : Teknik Informatika \\
Nama Mata Kuliah & : Pemrograman Internet \\
Kode Mata Kuliah & : \lbrack Diisi oleh Program Studi\rbrack \\
Semester & : \lbrack Diisi sesuai kurikulum\rbrack \\
SKS / Bobot Kredit & : 2 SKS \\
Jumlah Pertemuan & : 16 pertemuan \\
Komposisi & : 70\% Praktek, 30\% Teori \\
Dosen Pengampu & : \lbrack Nama Dosen, Gelar\rbrack \\
Tanggal Penyusunan & : \lbrack Tanggal\rbrack \\
\end{tabular}

\vspace{0.5cm}

% ============================================================
% 2. CAPAIAN PEMBELAJARAN LULUSAN (CPL)
% ============================================================
\section*{2. Capaian Pembelajaran Lulusan (CPL)}

CPL yang dibebankan pada mata kuliah ini mencakup kompetensi lulusan dalam aspek pengetahuan, keterampilan, dan sikap:

\begin{itemize}[leftmargin=*]
  \item \textbf{CPL-1 (Pengetahuan):} Menguasai konsep teoretis arsitektur web, protokol HTTP, pemrograman client-side (HTML, CSS, JavaScript) dan server-side (PHP, MySQL) secara mendalam, serta mampu memformulasikan penyelesaian masalah dalam pengembangan aplikasi berbasis web.
  
  \item \textbf{CPL-2 (Keterampilan Umum):} Mampu menerapkan pemikiran logis, kritis, sistematis, dan inovatif dalam konteks pengembangan aplikasi internet yang memperhatikan dan menerapkan nilai humaniora.
  
  \item \textbf{CPL-3 (Keterampilan Khusus):} Mampu merancang, mengimplementasikan, dan mengevaluasi aplikasi web dengan memanfaatkan teknologi HTML, CSS, JavaScript, PHP, MySQL serta prinsip keamanan dasar.
  
  \item \textbf{CPL-4 (Sikap):} Menunjukkan sikap bertanggung jawab atas pekerjaan di bidang keahliannya secara mandiri dan mampu bekerja sama dalam tim.
\end{itemize}

\vspace{0.5cm}

% ============================================================
% 3. CAPAIAN PEMBELAJARAN MATA KULIAH (CPMK)
% ============================================================
\section*{3. Capaian Pembelajaran Mata Kuliah (CPMK)}

Kemampuan atau kompetensi spesifik yang diharapkan mahasiswa kuasai setelah menyelesaikan mata kuliah:

\begin{itemize}[leftmargin=*]
  \item \textbf{CPMK-1:} Mahasiswa mampu memahami dan menjelaskan arsitektur web, protokol HTTP, serta peran pemrograman client-side dan server-side dalam aplikasi internet.
  
  \item \textbf{CPMK-2:} Mahasiswa mampu merancang dan membangun antarmuka web dengan HTML, CSS, dan JavaScript yang responsif dan memenuhi prinsip aksesibilitas dasar.
  
  \item \textbf{CPMK-3:} Mahasiswa mampu mengimplementasikan logika server-side (PHP) dan integrasi basis data (MySQL) serta menghubungkannya dengan front-end.
  
  \item \textbf{CPMK-4:} Mahasiswa mampu menerapkan prinsip keamanan web (validasi input, prepared statement, session) dalam aplikasi berbasis web.
\end{itemize}

\vspace{0.5cm}

% ============================================================
% 4. SUB-CPMK / INDIKATOR PENCAPAIAN
% ============================================================
\section*{4. Sub-CPMK / Indikator Pencapaian}

Penjabaran CPMK menjadi indikator yang lebih terukur dan dapat diuji:

\begin{itemize}[leftmargin=*]
  \item \textbf{Sub-CPMK 1.1:} Menjelaskan arsitektur client-server dan alur komunikasi HTTP request-response
  \item \textbf{Sub-CPMK 1.2:} Membedakan peran dan teknologi client-side (browser) dan server-side
  \item \textbf{Sub-CPMK 2.1:} Membuat halaman web dengan HTML5 (struktur, form, semantic elements)
  \item \textbf{Sub-CPMK 2.2:} Menerapkan CSS3 untuk layout (Flexbox/Grid) dan desain responsif
  \item \textbf{Sub-CPMK 2.3:} Menggunakan JavaScript untuk manipulasi DOM, event handling, dan validasi form
  \item \textbf{Sub-CPMK 3.1:} Mengimplementasikan script PHP untuk pemrosesan form dan koneksi basis data MySQL
  \item \textbf{Sub-CPMK 3.2:} Membuat operasi CRUD berbasis web dengan integrasi PHP dan MySQL
  \item \textbf{Sub-CPMK 4.1:} Menerapkan validasi input dan session dalam aplikasi web
  \item \textbf{Sub-CPMK 4.2:} Menerapkan prepared statement, sanitasi, dan prinsip keamanan web dasar (XSS, CSRF)
\end{itemize}

\vspace{0.5cm}

% ============================================================
% 5. MATERI PEMBELAJARAN (BAHAN KAJIAN)
% ============================================================
\section*{5. Materi Pembelajaran (Bahan Kajian)}

Daftar topik materi yang relevan dengan Sub-CPMK dan CPMK. \textbf{Fokus teknologi:} HTML, CSS, JavaScript, MySQL, PHP. \textbf{Alat yang digunakan:} XAMPP, VS Code (dengan addon untuk HTML/CSS/JS/PHP), GitHub Copilot.

\begin{enumerate}[leftmargin=*]
  \item Pengenalan pemrograman internet, arsitektur client-server, alat (XAMPP, VS Code)
  \item Protokol HTTP dan interaksi browser--server
  \item HTML5: struktur dokumen, elemen semantik, link, gambar, tabel
  \item HTML5: form dan elemen input
  \item CSS3: selector, box model, warna, typography
  \item CSS3: Flexbox, Grid, desain responsif
  \item JavaScript: sintaks, tipe data, kontrol alur, fungsi
  \item JavaScript: DOM, event handling, validasi form
  \item UTS / Review materi pertemuan 1--8
  \item PHP: pengenalan, environment XAMPP, request-response
  \item PHP + MySQL: koneksi, query, CRUD
  \item Session, cookie, autentikasi dasar
  \item Pembuatan aplikasi web integratif (HTML/CSS/JS/PHP/MySQL)
  \item Keamanan web: validasi input, prepared statement, sanitasi
  \item Proyek kelompok: aplikasi web CRUD
  \item Presentasi proyek, review semester, UAS
\end{enumerate}

\vspace{0.5cm}

\textbf{Rencana Perkuliahan (16 Pertemuan):}

\small
\begin{longtable}{|c|>{\raggedright\arraybackslash}p{4.5cm}|>{\raggedright\arraybackslash}p{3cm}|>{\raggedright\arraybackslash}p{2.5cm}|}
\hline
\textbf{Pert.} & \textbf{Materi} & \textbf{Sub-CPMK} & \textbf{Metode} \\
\hline
\endfirsthead
\hline
\textbf{Pert.} & \textbf{Materi} & \textbf{Sub-CPMK} & \textbf{Metode} \\
\hline
\endhead
\hline
\endfoot
1 & Pengenalan pemrograman internet, arsitektur web, alat (XAMPP, VS Code) & 1.1, 1.2 & Ceramah, praktikum \\
\hline
2 & Protokol HTTP, interaksi browser--server & 1.1, 1.2 & Ceramah, praktikum \\
\hline
3 & HTML5: struktur, elemen semantik, link, gambar, tabel & 2.1 & Praktikum, live coding \\
\hline
4 & HTML5: form dan elemen input & 2.1 & Praktikum, live coding \\
\hline
5 & CSS3: selector, box model, warna, typography & 2.2 & Praktikum \\
\hline
6 & CSS3: Flexbox, Grid, desain responsif & 2.2 & Praktikum \\
\hline
7 & JavaScript: sintaks, DOM, event handling & 2.3 & Praktikum \\
\hline
8 & JavaScript: validasi form, integrasi HTML/CSS & 2.3 & Praktikum, kuis \\
\hline
9 & UTS / Review materi pertemuan 1--8 & CPMK-1, CPMK-2 & Ujian, diskusi \\
\hline
10 & PHP: pengenalan, environment XAMPP, request-response & 3.1 & Ceramah, praktikum \\
\hline
11 & PHP + MySQL: koneksi, query, CRUD & 3.1, 3.2 & Praktikum \\
\hline
12 & Session, cookie, autentikasi dasar & 4.1 & Praktikum \\
\hline
13 & Integrasi web (HTML/CSS/JS/PHP/MySQL) & 3.2, 4.1 & Praktikum \\
\hline
14 & Keamanan web: validasi, prepared statement, sanitasi & 4.2 & Ceramah, praktikum \\
\hline
15 & Proyek kelompok: aplikasi web CRUD & 3.2, 4.1, 4.2 & Project-based \\
\hline
16 & Presentasi proyek, review semester, UAS & Semua CPMK & Presentasi, ujian \\
\hline
\end{longtable}
\normalsize

\vspace{0.5cm}

% ============================================================
% 6. METODE PEMBELAJARAN
% ============================================================
\section*{6. Metode Pembelajaran}

Strategi atau pendekatan pembelajaran yang dipilih sesuai OBE dengan komposisi 70\% praktek dan 30\% teori:

\begin{itemize}[leftmargin=*]
  \item \textbf{Ceramah Interaktif (30\% teori):} Penjelasan konsep arsitektur web, HTTP, dan keamanan dengan diskusi tanya jawab
  \item \textbf{Praktikum Terbimbing (70\% praktek):} Latihan HTML/CSS/JavaScript, PHP, MySQL dengan XAMPP dan VS Code
  \item \textbf{Live Coding dan Demonstrasi:} Memperagakan pembuatan halaman web, form, koneksi database dengan alat XAMPP dan VS Code
  \item \textbf{Problem-Based Learning (PBL):} Mahasiswa menyelesaikan permasalahan nyata (membuat situs, CRUD) menggunakan teknologi HTML, CSS, JS, PHP, MySQL
  \item \textbf{Project-Based Learning:} Pengembangan aplikasi web mini sebagai proyek kelompok
  \item \textbf{Peer Review:} Mahasiswa melakukan review kode atau tampilan web terhadap pekerjaan rekan
  \item \textbf{Penggunaan GitHub Copilot:} Alat bantu coding (dengan etika akademik dan panduan kejujuran)
\end{itemize}

\vspace{0.5cm}

% ============================================================
% 7. PENGALAMAN BELAJAR MAHASISWA
% ============================================================
\section*{7. Pengalaman Belajar Mahasiswa}

Deskripsi tugas, aktivitas, atau pengalaman belajar yang mendukung pencapaian Sub-CPMK:

\begin{itemize}[leftmargin=*]
  \item Membuat halaman web statis (HTML, CSS, JavaScript) dengan VS Code
  \item Merancang dan mengimplementasikan antarmuka web (landing page, form registrasi/login) dengan VS Code
  \item Praktik koneksi PHP--MySQL dengan XAMPP
  \item Membangun modul CRUD berbasis web dengan integrasi PHP dan MySQL
  \item Menerapkan validasi input, session, dan prinsip keamanan (prepared statement, sanitasi) dalam tugas atau proyek
  \item Berkolaborasi dalam tim untuk mengembangkan proyek aplikasi web
  \item Mempresentasikan hasil proyek atau melakukan peer review terhadap kode/tampilan web rekan
  \item Menggunakan GitHub Copilot untuk asistensi coding (sesuai panduan kejujuran akademik)
\end{itemize}

\vspace{0.5cm}

% ============================================================
% 8. KRITERIA, INDIKATOR, DAN BOBOT PENILAIAN
% ============================================================
\section*{8. Kriteria, Indikator, dan Bobot Penilaian}

Teknik/alat asesmen dipetakan ke Sub-CPMK/CPMK dengan bobot yang jelas (sesuai komposisi 70\% praktek):

\begin{longtable}{|>{\raggedright\arraybackslash}p{2.4cm}|>{\raggedright\arraybackslash}p{3.4cm}|>{\raggedright\arraybackslash}p{4.9cm}|>{\raggedright\arraybackslash}p{1.4cm}|}
\hline
\textbf{Komponen} & \textbf{Teknik Asesmen} & \textbf{Indikator/CPMK} & \textbf{Bobot (\%)} \\
\hline
\endfirsthead
\hline
\textbf{Komponen} & \textbf{Teknik Asesmen} & \textbf{Indikator/CPMK} & \textbf{Bobot (\%)} \\
\hline
\endhead
\hline
\endfoot

Tugas Individu & Tugas coding web (HTML/CSS/JS, form, layout) & Sub-CPMK 2.1, 2.2, 2.3 & 15 \\
\hline
Kuis & Multiple Choice \& Essay & Sub-CPMK 1.1, 1.2, 2.1, 2.2 & 10 \\
\hline
Praktikum & Lab exercise (PHP, MySQL, CRUD) & Sub-CPMK 3.1, 3.2, 4.1, 4.2 & 20 \\
\hline
UTS & Written exam \& praktik coding & CPMK-1, CPMK-2 & 20 \\
\hline
Proyek Kelompok & Aplikasi web (front-end + back-end) & CPMK-2, CPMK-3, CPMK-4 & 20 \\
\hline
UAS & Ujian komprehensif & CPMK-1, CPMK-2, CPMK-3, CPMK-4 & 15 \\
\hline
\textbf{Total} & & & \textbf{100} \\
\hline
\end{longtable}

\textbf{Kriteria Penilaian:}
\begin{itemize}[leftmargin=*]
  \item A (85--100): Menguasai semua CPMK dengan sangat baik, mampu menerapkan dalam kasus kompleks
  \item B (70--84): Menguasai sebagian besar CPMK dengan baik
  \item C (60--69): Menguasai CPMK dasar dengan cukup
  \item D (50--59): Menguasai sebagian kecil CPMK
  \item E (<50): Belum menguasai CPMK yang ditetapkan
\end{itemize}

\vspace{0.5cm}

% ============================================================
% 9. EVALUASI DAN REFLEKSI PEMBELAJARAN (OPSIONAL)
% ============================================================
\section*{9. Evaluasi dan Refleksi Pembelajaran}

Penilaian sumatif/formatif untuk memantau ketercapaian outcome secara menyeluruh:

\begin{itemize}[leftmargin=*]
  \item \textbf{Evaluasi Formatif:} Kuis, latihan coding web (HTML/CSS/JS/PHP/MySQL), dan peer review untuk memberikan feedback berkelanjutan
  \item \textbf{Evaluasi Sumatif:} UTS dan UAS untuk mengukur pencapaian CPMK Pemrograman Internet secara komprehensif
  \item \textbf{Refleksi Mahasiswa:} Jurnal belajar atau refleksi singkat terhadap pemahaman materi (arsitektur web, HTML, CSS, JavaScript, PHP, MySQL, keamanan)
  \item \textbf{Evaluasi Dosen:} Survey kepuasan mahasiswa di tengah dan akhir semester
  \item \textbf{Continuous Improvement:} Analisis hasil penilaian untuk perbaikan RPS di semester berikutnya
\end{itemize}

\vspace{0.5cm}

% ============================================================
% 10. DAFTAR REFERENSI
% ============================================================
\section*{10. Daftar Referensi}

Sumber belajar utama (sumber terbuka dari internet) yang digunakan dalam penyusunan materi dan asesmen:

\sloppy
\begin{enumerate}[leftmargin=*]
  \item MDN Web Docs. \textit{Learn Web Development}. \url{https://developer.mozilla.org/en-US/docs/Learn}
  \item MDN Web Docs. \textit{HTML: HyperText Markup Language}. \url{https://developer.mozilla.org/en-US/docs/Web/HTML}
  \item MDN Web Docs. \textit{CSS: Cascading Style Sheets}. \url{https://developer.mozilla.org/en-US/docs/Web/CSS}
  \item MDN Web Docs. \textit{JavaScript}. \url{https://developer.mozilla.org/en-US/docs/Web/JavaScript}
  \item MDN Web Docs. \textit{Using the Fetch API}. \url{https://developer.mozilla.org/en-US/docs/Web/API/Fetch\_API/Using\_Fetch}
  \item MDN Web Docs. \textit{Web APIs}. \url{https://developer.mozilla.org/en-US/docs/Web/API}
  \item MDN Web Docs. \textit{HTTP}. \url{https://developer.mozilla.org/en-US/docs/Web/HTTP}
  \item MDN Web Docs. \textit{Structuring the Web with HTML}. \url{https://developer.mozilla.org/en-US/docs/Learn/HTML}
  \item MDN Web Docs. \textit{Styling the Web with CSS}. \url{https://developer.mozilla.org/en-US/docs/Learn/CSS}
  \item MDN Web Docs. \textit{JavaScript---Dynamic Client-Side Scripting}. \url{https://developer.mozilla.org/en-US/docs/Learn/JavaScript}
  \item Haverbeke, M. \textit{Eloquent JavaScript} (edisi daring gratis). \url{https://eloquentjavascript.net}
  \item JavaScript.info. \textit{The Modern JavaScript Tutorial}. \url{https://javascript.info}
  \item W3Schools. \textit{Web Development Tutorials}. \url{https://www.w3schools.com}
  \item W3Schools. \textit{PHP Tutorial}. \url{https://www.w3schools.com/php}
  \item W3Schools. \textit{SQL Tutorial}. \url{https://www.w3schools.com/sql}
  \item web.dev. \textit{Learn Web Development}. \url{https://web.dev/learn}
  \item PHP. \textit{PHP Manual}. \url{https://www.php.net/manual/en}
  \item PHP The Right Way. \textit{Reference for PHP Best Practices}. \url{https://phptherightway.com}
  \item MySQL. \textit{MySQL Documentation}. \url{https://dev.mysql.com/doc}
  \item WHATWG. \textit{HTML Living Standard}. \url{https://html.spec.whatwg.org}
  \item W3C. \textit{HTML Standards}. \url{https://www.w3.org/html}
  \item W3C. \textit{CSS Specifications}. \url{https://www.w3.org/Style/CSS}
  \item OWASP. \textit{Cheat Sheet Series}. \url{https://cheatsheetseries.owasp.org}
  \item OWASP. \textit{Cross-Site Scripting (XSS) Prevention Cheat Sheet}. \url{https://cheatsheetseries.owasp.org/cheatsheets/Cross\_Site\_Scripting\_Prevention\_Cheat\_Sheet.html}
  \item OWASP. \textit{SQL Injection Prevention Cheat Sheet}. \url{https://cheatsheetseries.owasp.org/cheatsheets/SQL\_Injection\_Prevention\_Cheat\_Sheet.html}
  \item OWASP. \textit{OWASP Top Ten}. \url{https://owasp.org/www-project-top-ten}
  \item OWASP. \textit{Authentication Cheat Sheet}. \url{https://cheatsheetseries.owasp.org/cheatsheets/Authentication\_Cheat\_Sheet.html}
  \item OWASP. \textit{Session Management Cheat Sheet}. \url{https://cheatsheetseries.owasp.org/cheatsheets/Session\_Management\_Cheat\_Sheet.html}
  \item OWASP. \textit{Input Validation Cheat Sheet}. \url{https://cheatsheetseries.owasp.org/cheatsheets/Input\_Validation\_Cheat\_Sheet.html}
  \item OWASP. \textit{Cross-Site Request Forgery Prevention Cheat Sheet}. \url{https://cheatsheetseries.owasp.org/cheatsheets/Cross-Site\_Request\_Forgery\_Prevention\_Cheat\_Sheet.html}
  \item OWASP. \textit{REST Security Cheat Sheet}. \url{https://cheatsheetseries.owasp.org/cheatsheets/REST\_Security\_Cheat\_Sheet.html}
  \item FreeCodeCamp. \textit{Learn to Code}. \url{https://www.freecodecamp.org/learn}
  \item CSS-Tricks. \textit{CSS-Tricks Guides}. \url{https://css-tricks.com/guides}
  \item JSON. \textit{Introducing JSON}. \url{https://www.json.org}
  \item REST API Tutorial. \textit{REST API Tutorial}. \url{https://www.restapitutorial.com}
  \item W3C WAI. \textit{Web Accessibility Initiative}. \url{https://www.w3.org/WAI}
  \item W3C. \textit{Web Content Accessibility Guidelines (WCAG) 2.1}. \url{https://www.w3.org/TR/WCAG21}
  \item W3C WAI. \textit{Accessibility Tutorials}. \url{https://www.w3.org/WAI/tutorials}
  \item Can I Use. \textit{Can I use\ldots}. \url{https://caniuse.com}
  \item HTTP Status Codes. \textit{HTTP Status Codes}. \url{https://httpstatuses.com}
  \item DevDocs. \textit{API Documentation Browser}. \url{https://devdocs.io}
  \item GitHub. \textit{GitHub Docs}. \url{https://docs.github.com}
  \item Git. \textit{Git Documentation}. \url{https://git-scm.com/doc}
  \item Postman. \textit{Postman Learning Center}. \url{https://learning.postman.com}
  \item Bootstrap. \textit{Bootstrap Documentation}. \url{https://getbootstrap.com/docs}
  \item npm. \textit{npm Documentation}. \url{https://docs.npmjs.com}
  \item Visual Studio Code. \textit{VS Code Documentation}. \url{https://code.visualstudio.com/docs}
  \item Apache Friends. \textit{XAMPP}. \url{https://www.apachefriends.org}
  \item GitHub. \textit{GitHub Copilot Documentation}. \url{https://docs.github.com/copilot}
  \item PHP. \textit{PHP: Getting Started}. \url{https://www.php.net/manual/en/getting-started.php}
  \item MySQL. \textit{MySQL Getting Started}. \url{https://dev.mysql.com/doc/mysql-getting-started/en}
  \item MDN Web Docs. \textit{Server-side website programming}. \url{https://developer.mozilla.org/en-US/docs/Learn/Server-side}
\end{enumerate}

\vspace{1cm}

\begin{flushright}
\begin{tabular}{c}
Disusun oleh,\\[2cm]
\textbf{\lbrack Nama Dosen, Gelar\rbrack}\\
NIP. \lbrack Nomor Induk Pegawai\rbrack
\end{tabular}
\end{flushright}

\end{document}
