\section{Asesmen Akhir Komprehensif}

Asesmen ini dirancang untuk mengukur pencapaian CPMK-1 sampai CPMK-4 Pemrograman Internet secara menyeluruh \cite{mdn-learn}. Kerjakan dengan jujur dan mandiri.

\textbf{Alokasi Waktu}:
\begin{itemize}
  \item Bagian A (Pilihan Ganda): 25 menit
  \item Bagian B (Essay): 40 menit
  \item Bagian C (Analisis Kode): 35 menit
  \item Bagian D (Proyek Aplikasi Web): sesuai jadwal
\end{itemize}

\subsection{Bagian A: Pilihan Ganda (CPMK-1, CPMK-2)}

\begin{enumerate}
  \item Dalam arsitektur client-server, browser bertindak sebagai:
  \begin{enumerate}
    \item Server \item Client \item Gateway \item Proxy
  \end{enumerate}
  \item Metode HTTP untuk mengirim data form ke server adalah:
  \begin{enumerate}
    \item GET \item POST \item FETCH \item SEND
  \end{enumerate}
  \item Elemen HTML5 untuk navigasi utama adalah:
  \begin{enumerate}
    \item \texttt{<header>} \item \texttt{<nav>} \item \texttt{<menu>} \item \texttt{<link>}
  \end{enumerate}
  \item Untuk layout responsif, properti CSS yang umum digunakan adalah:
  \begin{enumerate}
    \item \texttt{@media} \item \texttt{@import} \item \texttt{@keyframes} \item \texttt{@font}
  \end{enumerate}
  \item API DOM untuk memilih elemen berdasarkan ID adalah:
  \begin{enumerate}
    \item \texttt{querySelector()} \item \texttt{getElementById()} \item \texttt{selectById()} \item \texttt{find()}
  \end{enumerate}
  \item Serangan yang menyuntikkan script berbahaya ke halaman disebut:
  \begin{enumerate}
    \item SQL Injection \item XSS \item CSRF \item DDoS
  \end{enumerate}
  \item Format pertukaran data yang ringan dan umum untuk aplikasi web adalah:
  \begin{enumerate}
    \item XML \item JSON \item CSV \item HTML
  \end{enumerate}
  \item Untuk mencegah SQL injection, yang harus digunakan adalah:
  \begin{enumerate}
    \item String concatenation \item Prepared statement \item Escape manual \item Validasi client saja
  \end{enumerate}
\end{enumerate}

\subsection{Bagian B: Essay (CPMK-1, CPMK-2, CPMK-3, CPMK-4)}

\begin{enumerate}
  \item Jelaskan arsitektur client-server dan alur HTTP request-response!
  \item Jelaskan perbedaan client-side dan server-side beserta contoh teknologinya!
  \item Mengapa validasi di server wajib meskipun sudah ada validasi di client?
  \item Jelaskan pencegahan XSS dan CSRF!
  \item Jelaskan cara kerja session dan cookie dalam autentikasi web!
\end{enumerate}

\subsection{Bagian C: Analisis Kode}

Analisis kode PHP/HTML berikut terkait keamanan dan best practices:
\begin{lstlisting}[basicstyle=\ttfamily\footnotesize, frame=single]
<?php
$id = $_GET['id'];
$sql = "SELECT * FROM users WHERE id = " . $id;
$result = mysqli_query($conn, $sql);
echo "<div>" . $_POST['nama'] . "</div>";
?>
\end{lstlisting}

Pertanyaan: Identifikasi minimal 3 masalah keamanan. Perbaiki kode tersebut!

\subsection{Bagian D: Proyek Aplikasi Web (CPMK-2, CPMK-3, CPMK-4)}

Implementasikan aplikasi web sederhana (mis. daftar tugas, buku tamu) dengan:
\begin{itemize}
  \item Front-end: HTML5, CSS3, JavaScript (layout responsif)
  \item Back-end: PHP
  \item Basis data: MySQL
  \item CRUD lengkap, form validasi, pencegahan XSS/SQL injection
\end{itemize}
