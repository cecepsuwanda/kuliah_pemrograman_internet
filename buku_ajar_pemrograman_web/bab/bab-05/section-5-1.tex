\section{CSS: Selector, Box Model, Warna, dan Typography}

CSS (Cascading Style Sheets) mengontrol tampilan dan layout halaman web \cite{mdn-css}. CSS dipisahkan dari HTML untuk memisahkan struktur konten dan presentasi, memudahkan pemeliharaan dan penggunaan ulang gaya \cite{mdn-learn-css}. Selector menentukan elemen mana yang diberi gaya: selector elemen (\texttt{p}, \texttt{h1}), class (\texttt{.nama-class}), ID (\texttt{\#nama-id}), dan selector gabungan \cite{w3c-css}.

Box model adalah fondasi layout CSS: setiap elemen direpresentasikan sebagai kotak dengan \texttt{content}, \texttt{padding}, \texttt{border}, dan \texttt{margin} \cite{csstricks}. Warna dapat dinyatakan dengan nama, hex (\texttt{\#ff0000}), RGB, atau HSL. Typography diatur dengan \texttt{font-family}, \texttt{font-size}, \texttt{line-height}, dan \texttt{text-align} \cite{mdn-css}. Memahami box model dan tipografi penting untuk desain yang konsisten \cite{webdev}.
