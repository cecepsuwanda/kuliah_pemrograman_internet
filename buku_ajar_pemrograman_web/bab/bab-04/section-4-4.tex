\section{Validasi Form HTML5 Lanjutan dan Keamanan}

Validasi form merupakan aspek krusial dalam pengembangan aplikasi web yang aman dan user-friendly. HTML5 menyediakan berbagai mekanisme validasi bawaan, namun perlu dikombinasikan dengan validasi server-side untuk keamanan maksimal \cite{mdn-form-validation}.

\subsection{Atribut Validasi HTML5}

HTML5 memperkenalkan atribut-atribut validasi yang bekerja di browser sebelum form dikirim:

\begin{itemize}
  \item \texttt{required}: Field wajib diisi
  \item \texttt{pattern}: Regex pattern untuk validasi format
  \item \texttt{minlength} dan \texttt{maxlength}: Panjang minimum dan maksimum teks
  \item \texttt{min} dan \texttt{max}: Rentang nilai untuk tipe numerik
  \item \texttt{step}: Interval nilai yang diperbolehkan
\end{itemize}

\begin{lstlisting}[caption={Contoh Validasi dengan Pattern}, basicstyle=\ttfamily\small, frame=single]
<form>
  <!-- Validasi username: alphanumeric, 3-20 karakter -->
  <label for="username">Username:</label>
  <input type="text" id="username" name="username" 
         pattern="[a-zA-Z0-9]{3,20}" 
         title="Username harus 3-20 karakter alphanumeric"
         required>
  
  <!-- Validasi password: minimal 8 karakter, ada huruf dan angka -->
  <label for="password">Password:</label>
  <input type="password" id="password" name="password" 
         pattern="(?=.*[A-Za-z])(?=.*\d)[A-Za-z\d]{8,}"
         title="Password minimal 8 karakter, mengandung huruf dan angka"
         required>
  
  <!-- Validasi kode pos Indonesia -->
  <label for="kodepos">Kode Pos:</label>
  <input type="text" id="kodepos" name="kodepos" 
         pattern="[0-9]{5}" 
         title="Kode pos harus 5 digit angka"
         required>
</form>
\end{lstlisting}

\subsection{Pseudo-classes CSS untuk Validasi}

CSS menyediakan pseudo-classes untuk styling form berdasarkan status validasi:

\begin{lstlisting}[caption={Styling Form dengan CSS Validasi}, basicstyle=\ttfamily\small, frame=single]
/* Style untuk input valid */
input:valid {
  border-color: green;
  background-color: #f0fff0;
}

/* Style untuk input invalid */
input:invalid {
  border-color: red;
  background-color: #fff0f0;
}

/* Style untuk input yang sedang difokuskan */
input:focus {
  outline: none;
  box-shadow: 0 0 5px rgba(0, 123, 255, 0.5);
}

/* Style untuk input wajib */
input:required {
  border-left: 3px solid orange;
}
\end{lstlisting}

\subsection{JavaScript API untuk Validasi}

HTML5 menyediakan Constraint Validation API yang memungkinkan validasi kustom menggunakan JavaScript:

\begin{lstlisting}[caption={Validasi dengan JavaScript}, basicstyle=\ttfamily\small, frame=single]
<script>
const form = document.querySelector('form');
const email = document.getElementById('email');

// Custom validation message
email.addEventListener('input', function() {
  if (email.validity.typeMismatch) {
    email.setCustomValidity('Masukkan alamat email yang valid!');
  } else {
    email.setCustomValidity('');
  }
});

// Validasi saat submit
form.addEventListener('submit', function(event) {
  if (!form.checkValidity()) {
    event.preventDefault();
    // Tampilkan error message
    alert('Mohon perbaiki input yang salah!');
  }
});
</script>
\end{lstlisting}

\subsection{Praktik Keamanan Validasi}

Validasi client-side hanya untuk UX yang lebih baik. Validasi server-side WAJIB untuk keamanan:

\begin{enumerate}
  \item \textbf{Jangan percaya input client}: Data dapat dimanipulasi menggunakan browser developer tools
  
  \item \textbf{Sanitasi semua input}: Bersihkan input dari karakter berbahaya sebelum diproses
  
  \item \textbf{Gunakan parameterized queries}: Hindari SQL injection dengan prepared statements
  
  \item \textbf{Validasi tipe data}: Pastikan data sesuai tipe yang diharapkan (integer, string, date)
  
  \item \textbf{Batasi ukuran file}: Untuk upload, batasi ukuran dan tipe file yang diperbolehkan
\end{enumerate}

\begin{lstlisting}[caption={Contoh Validasi Server-Side (PHP)}, basicstyle=\ttfamily\small, frame=single]
<?php
// Validasi dan sanitasi email
$email = filter_input(INPUT_POST, 'email', FILTER_VALIDATE_EMAIL);
if (!$email) {
    die("Email tidak valid!");
}

// Validasi panjang password
$password = $_POST['password'];
if (strlen($password) < 8) {
    die("Password minimal 8 karakter!");
}

// Sanitasi input string
$nama = htmlspecialchars(trim($_POST['nama']), ENT_QUOTES, 'UTF-8');

// Prepared statement untuk database
$stmt = $pdo->prepare("INSERT INTO users (email, nama) VALUES (?, ?)");
$stmt->execute([$email, $nama]);
?>
\end{lstlisting}

\subsection{Error Handling dan User Feedback}

Berikan feedback yang jelas kepada pengguna tentang error validasi:

\begin{itemize}
  \item Gunakan warna yang kontras untuk menandai field error
  \item Tampilkan pesan error yang spesifik dan actionable
  \item Hindari pesan error teknis yang membingungkan
  \item Berikan contoh format yang benar (contoh: "Format: 081234567890")
  \item Validasi real-time untuk UX yang lebih baik
\end{itemize}

Implementasi validasi yang komprehensif di sisi client dan server adalah kunci aplikasi web yang aman dan user-friendly \cite{owasp-validation}.

