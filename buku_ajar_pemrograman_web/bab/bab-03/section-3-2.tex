\section{Tag-Tag Fundamental HTML}

HTML (HyperText Markup Language) menggunakan tag-tag untuk mendefinisikan struktur dan format konten web. Tag HTML ditulis dalam kurung sudut `<tag>` dan biasanya berpasangan dengan tag penutup `</tag>` \cite{w3schools-html}. Pemahaman tag-tag fundamental ini adalah dasar untuk semua pengembangan web \cite{mdn-html}.

\subsection{Heading Tags}

Heading tags digunakan untuk membuat hierarki konten dari yang paling penting hingga yang kurang penting:

\begin{itemize}
  \item \texttt{<h1>} - Heading paling penting (biasanya judul utama)
  \item \texttt{<h2>} - Subheading penting
  \item \texttt{<h3>} - Subheading level 3
  \item \texttt{<h4>} - Subheading level 4
  \item \texttt{<h5>} - Subheading level 5
  \item \texttt{<h6>} - Subheading paling kecil
\end{itemize}

\begin{lstlisting}[caption={Contoh Heading Tags}, basicstyle=\ttfamily\small, frame=single]
<!DOCTYPE html>
<html>
<head>
  <title>Contoh Heading</title>
</head>
<body>
  <h1>Judul Utama Halaman</h1>
  <h2>Bab 1: Pengenalan HTML</h2>
  <h3>1.1 Sejarah HTML</h3>
  <h4>1.1.1 HTML 1.0</h4>
  <h5>1.1.1.1 Fitur Awal</h5>
  <h6>1.1.1.1.1 Detail Teknis</h6>
</body>
</html>
\end{lstlisting}

\textbf{Hasil di Browser:}
Browser akan menampilkan teks dengan ukuran font yang berbeda-beda:
- `<h1>`: Font terbesar dan tebal
- `<h6>`: Font terkecil dan kurang tebal
- Setiap heading otomatis memiliki line break sebelum dan sesudahnya

\subsection{Paragraph dan Line Break}

Tag paragraph dan line break mengatur alur teks dalam dokumen:

\begin{itemize}
  \item \texttt{<p>} - Membuat paragraf baru (dengan jarak spasi)
  \item \texttt{<br>} - Line break (pindah baris tanpa spasi)
  \item \texttt{<hr>} - Horizontal rule (garis horizontal pembatas)
\end{itemize}

\begin{lstlisting}[caption={Contoh Paragraph dan Line Break}, basicstyle=\ttfamily\small, frame=single]
<!DOCTYPE html>
<html>
<head>
  <title>Contoh Paragraph</title>
</head>
<body>
  <p>Ini adalah paragraf pertama. 
  Paragraf akan otomatis wrap text dan memiliki jarak dengan paragraf lain.</p>
  
  <p>Ini paragraf kedua.<br>
  Teks ini pindah ke baris baru dengan tag br.<br>
  Dan ini baris ketiga.</p>
  
  <hr>
  
  <p>Garis horizontal di atas memisahkan konten.</p>
</body>
</html>
\end{lstlisting}

\textbf{Hasil di Browser:}
- Setiap `<p>` memiliki jarak vertikal antar paragraf
- `<br>` membuat teks pindah baris tanpa jarak tambahan
- `<hr>` menampilkan garis horizontal pembatas konten

\subsection{Divisions dan Spans}

Tag `<div>` dan `<span>` adalah container untuk mengelompokkan elemen:

\begin{itemize}
  \item \texttt{<div>} - Block-level container (mengambil full width)
  \item \texttt{<span>} - Inline container (mengambil width sesuai konten)
\end{itemize}

\begin{lstlisting}[caption={Contoh Div dan Span}, basicstyle=\ttfamily\small, frame=single]
<!DOCTYPE html>
<html>
<head>
  <title>Contoh Div dan Span</title>
</head>
<body>
  <div style="background-color: lightblue; padding: 10px;">
    <h3>Container Div</h3>
    <p>Div adalah block-level element yang mengambil full width.</p>
  </div>
  
  <p>Teks normal dengan <span style="color: red; font-weight: bold;">span merah tebal</span> di dalamnya.</p>
  
  <div style="border: 1px solid black; margin: 10px;">
    <span style="background-color: yellow;">Span 1</span>
    <span style="background-color: lightgreen;">Span 2</span>
    <span style="background-color: orange;">Span 3</span>
  </div>
</body>
</html>
\end{lstlisting}

\textbf{Hasil di Browser:}
- `<div>` menampilkan kotak dengan background biru muda
- Teks dalam `<span>` memiliki warna dan style berbeda
- `<div>` kedua menampilkan 3 span dengan background berbeda dalam satu baris

\subsection{HTML Comments}

Comments digunakan untuk dokumentasi dan tidak akan ditampilkan di browser:

\begin{lstlisting}[caption={Contoh HTML Comments}, basicstyle=\ttfamily\small, frame=single]
<!DOCTYPE html>
<html>
<head>
  <title>Contoh Comments</title>
</head>
<body>
  <!-- Ini adalah comment yang tidak akan ditampilkan -->
  <h1>Judul Halaman</h1>
  
  <!-- 
    Comment bisa 
    multiple lines
    untuk dokumentasi
  -->
  
  <p>Ini paragraf yang akan ditampilkan.</p>
  <!-- <p>Paragraf ini di-comment, tidak akan muncul</p> -->
</body>
</html>
\end{lstlisting}

\textbf{Hasil di Browser:}
- Hanya `<h1>` dan `<p>` pertama yang ditampilkan
- Semua comments tidak terlihat di browser
- Berguna untuk debugging dan dokumentasi kode

\subsection{Contoh Lengkap: Halaman Web Sederhana}

Berikut contoh lengkap halaman web menggunakan tag-tag fundamental HTML:

\begin{lstlisting}[caption={Halaman Web Lengkap dengan Tag Fundamental}, basicstyle=\ttfamily\small, frame=single]
<!DOCTYPE html>
<html>
<head>
  <title>Toko Buku Online</title>
</head>
<body>
  <!-- Header Section -->
  <div style="text-align: center; background-color: #f0f0f0; padding: 20px;">
    <h1>Toko Buku "ilmu"</h1>
    <hr>
    <p>Koleksi buku berkualitas untuk semua usia</p>
  </div>
  
  <!-- Main Content -->
  <div style="padding: 20px;">
    <h2>Buku Terlaris Minggu Ini</h2>
    
    <div style="border: 1px solid #ccc; margin: 10px; padding: 10px;">
      <h3>JavaScript untuk Pemula</h3>
      <p>Penulis: <span style="font-style: italic;">John Doe</span></p>
      <p>Harga: <span style="color: red; font-weight: bold;">Rp 150.000</span></p>
      <p>Deskripsi: Buku ini cocok untuk pemula yang ingin belajar JavaScript dari dasar.</p>
    </div>
    
    <div style="border: 1px solid #ccc; margin: 10px; padding: 10px;">
      <h3>HTML5 dan CSS3 Modern</h3>
      <p>Penulis: <span style="font-style: italic;">Jane Smith</span></p>
      <p>Harga: <span style="color: red; font-weight: bold;">Rp 200.000</span></p>
      <p>Deskripsi: Panduan lengkap membuat website modern dengan HTML5 dan CSS3.</p>
    </div>
  </div>
  
  <!-- Footer -->
  <div style="background-color: #333; color: white; text-align: center; padding: 10px;">
    <p>&copy; 2024 Toko Buku "ilmu". All rights reserved.</p>
    <br>
    <p>Contact: info@tokoilmu.com | Phone: 021-12345678</p>
  </div>
</body>
</html>
\end{lstlisting}

\textbf{Hasil di Browser:}
- Header dengan background abu-abu dan judul toko
- Konten utama dengan 2 card buku terlaris
- Footer dengan background gelap dan informasi kontak
- Layout terstruktur dengan penggunaan `<div>` dan styling inline

Pemahaman tag-tag fundamental ini adalah fondasi untuk membangun halaman web yang kompleks dan terstruktur \cite{w3schools-html}.
