\section{HTML5 Semantic Elements Lengkap}

HTML5 memperkenalkan elemen-elemen semantik yang memberikan makna struktur pada dokumen web. Elemen semantik tidak hanya mengatur tampilan, tetapi juga memberikan konteks pada konten untuk browser, search engine, dan teknologi assistif \cite{mdn-html}. Penggunaan elemen semantik yang tepat adalah fondasi web modern yang accessible dan SEO-friendly \cite{w3schools-html}.

\subsection{Semantic Structural Elements}

Elemen struktural semantik mendefinisikan bagian-bagian utama halaman:

\begin{itemize}
  \item \texttt{<header>} - Kepala halaman atau section
  \item \texttt{<nav>} - Navigasi utama website
  \item \texttt{<main>} - Konten utama dan unik halaman
  \item \texttt{<article>} - Konten mandiri yang berdiri sendiri
  \item \texttt{<section>} - Pengelompokan konten tematik
  \item \texttt{<aside>} - Konten samping/tambahan
  \item \texttt{<footer>} - Kaki halaman atau section
\end{itemize}

\begin{lstlisting}[caption={Contoh Semantic Structure}, basicstyle=\ttfamily\small, frame=single]
<!DOCTYPE html>
<html lang="id">
<head>
  <meta charset="UTF-8">
  <meta name="viewport" content="width=device-width, initial-scale=1.0">
  <title>Website Modern dengan Semantic HTML5</title>
  <style>
    body { font-family: Arial, sans-serif; line-height: 1.6; margin: 0; padding: 0; }
    header { background-color: #2c3e50; color: white; padding: 1rem; text-align: center; }
    nav { background-color: #333; padding: 1rem; }
    nav ul { list-style-type: none; margin: 0; padding: 0; display: flex; justify-content: center; }
    nav li { margin: 0 1rem; }
    nav a { color: white; text-decoration: none; padding: 0.5rem 1rem; border-radius: 4px; }
    main { padding: 2rem; max-width: 1200px; margin: 0 auto; }
    article { background-color: white; margin-bottom: 2rem; padding: 2rem; border-radius: 8px; }
    section { margin-bottom: 2rem; }
    aside { background-color: #e1f5fe; padding: 2rem; border-radius: 8px; margin-bottom: 2rem; }
    footer { background-color: #333; color: white; text-align: center; padding: 2rem; margin-top: 2rem; }
  </style>
</head>
<body>
  <header>
    <h1>TechBlog Indonesia</h1>
  </header>
  <nav>
    <ul>
      <li><a href="#beranda">Beranda</a></li>
      <li><a href="#tutorial">Tutorial</a></li>
      <li><a href="#review">Review</a></li>
      <li><a href="#tentang">Tentang</a></li>
    </ul>
  </nav>
  <main>
    <article id="beranda">
      <header><h2>Artikel Terbaru: HTML5 Semantic Elements</h2></header>
      <section>
        <h3>Pengenalan Semantic HTML5</h3>
        <p>HTML5 membawa revolusi dalam cara kita menulis dan memahami struktur dokumen web.</p>
      </section>
    </article>
    <aside>
      <h3>Artikel Terkait</h3>
      <ul><li><a href="#article2">CSS Grid vs Flexbox</a></li></ul>
    </aside>
  </main>
  <footer>
    <p>&copy; 2024 TechBlog Indonesia</p>
  </footer>
</body>
</html>
\end{lstlisting}

\textbf{Hasil di Browser:}
- Header dengan branding dan navigasi
- Main content dengan featured article
- Aside dengan related articles
- Footer dengan copyright
- Semantic structure yang jelas dan accessible
