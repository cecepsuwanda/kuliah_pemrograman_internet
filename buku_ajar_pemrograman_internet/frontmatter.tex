\frontmatter

% ============================================================
% Halaman Sampul
% ============================================================
\begin{titlepage}
  \centering
  \vspace*{1cm}
  {\LARGE\bfseries Pemrograman Internet\par}
  \vspace{0.5cm}
  {\large Buku Ajar Berbasis Outcome-Based Education (OBE)\par}
  \vspace{1cm}
  {\large Mata Kuliah: Pemrograman Internet\par}
  {\large 2 SKS \textbar{} 16 Pertemuan\par}
  \vspace{2cm}
  {\large Program Studi Teknik Informatika\par}
  {\large Fakultas Teknologi Informasi\par}
  {\large Universitas Bale Bandung\par}
  \vfill
  {\large 2026\par}
\end{titlepage}

\cleardoublepage

% ============================================================
% Kata Pengantar
% ============================================================
\chapter*{Kata Pengantar}
\addcontentsline{toc}{chapter}{Kata Pengantar}

Puji syukur kehadirat Tuhan Yang Maha Esa atas terselesaikannya buku ajar \textit{Pemrograman Internet} ini. Buku ini disusun dengan pendekatan \textbf{Outcome-Based Education (OBE)}, yang berfokus pada pencapaian kompetensi terukur sesuai dengan Capaian Pembelajaran Lulusan (CPL) dan Capaian Pembelajaran Mata Kuliah (CPMK).

Pemrograman Internet merupakan keterampilan fundamental yang harus dikuasai oleh setiap mahasiswa Teknik Informatika. Mata kuliah ini tidak hanya mengajarkan cara membuat halaman web, tetapi juga cara berpikir dalam merancang dan mengimplementasikan aplikasi web yang meliputi antarmuka (HTML, CSS, JavaScript), logika server-side, basis data, dan keamanan dasar.

Buku ini dirancang untuk mendukung pembelajaran mahasiswa dengan pendekatan student\-centered learning. Setiap bab dilengkapi dengan:
\begin{itemize}
  \item Sub-CPMK yang jelas dan terukur
  \item Materi pokok dengan contoh kode HTML, CSS, JavaScript, dan PHP
  \item Aktivitas pembelajaran yang mendorong eksplorasi mandiri
  \item Latihan dan refleksi untuk penguatan pemahaman
  \item Asesmen untuk mengukur pencapaian kompetensi
  \item Checklist kompetensi untuk self-assessment
\end{itemize}

Kami berharap buku ini dapat menjadi panduan yang efektif dalam perjalanan pembelajaran Anda menguasai Pemrograman Internet.

\vspace{1cm}
\begin{flushright}
Penyusun\\
\lbrack Nama Dosen, Gelar\rbrack
\end{flushright}

\cleardoublepage

% ============================================================
% Cara Menggunakan Buku Ini
% ============================================================
\chapter*{Cara Menggunakan Buku Ini}
\addcontentsline{toc}{chapter}{Cara Menggunakan Buku Ini}

Buku ajar ini dirancang dengan pendekatan OBE untuk memaksimalkan pencapaian pembelajaran Anda. Berikut panduan penggunaan buku ini:

\section*{Struktur Buku}

\textbf{Bab I: Pendahuluan dan Orientasi}\\
Memperkenalkan tujuan buku, keterkaitan dengan RPS, dan konteks kurikulum OBE.

\textbf{Bab II: Landasan Teori}\\
Menyajikan fondasi arsitektur web dan protokol HTTP yang menjadi basis pembelajaran.

\textbf{Bab III--VIII: Unit Materi Inti (Front-End)}\\
HTML5, CSS3, dan JavaScript untuk membangun antarmuka web.

\textbf{Bab IX--XIV: Unit Materi Inti (Back-End dan Integrasi)}\\
Server-side, basis data, REST API, keamanan web, dan deployment.

\textbf{Bab XIV: Evaluasi dan Integrasi}\\
Berisi asesmen komprehensif dan panduan refleksi untuk mengukur pencapaian kompetensi.

\textbf{Lampiran}\\
Menyediakan rubrik penilaian, glosarium istilah pemrograman web, dan referensi tambahan.

\section*{Komponen dalam Setiap Bab}

\begin{enumerate}
  \item \textbf{Sub-CPMK}: Baca dengan seksama untuk memahami kompetensi yang harus dicapai
  \item \textbf{Materi Pokok}: Pelajari dengan cermat, jalankan semua contoh kode
  \item \textbf{Aktivitas Pembelajaran}: Lakukan secara mandiri atau berkelompok
  \item \textbf{Latihan}: Kerjakan untuk menguji pemahaman Anda
  \item \textbf{Asesmen}: Gunakan untuk mengukur pencapaian Sub-CPMK
  \item \textbf{Checklist}: Centang setelah yakin menguasai setiap indikator
\end{enumerate}

\section*{Tips Belajar Efektif}

\begin{itemize}
  \item Gunakan browser dan editor teks (VS Code, Sublime) untuk praktik
  \item Jalankan contoh kode di setiap bab dan modifikasi untuk eksperimen
  \item Manfaatkan dokumentasi MDN, W3Schools, dan sumber daring di daftar referensi
  \item Kerjakan proyek mini untuk mengintegrasikan HTML, CSS, JavaScript, dan server-side
  \item Diskusikan konsep yang sulit dengan teman atau dosen
  \item Manfaatkan checklist untuk self-assessment berkala
\end{itemize}

\cleardoublepage

% ============================================================
% Identitas Mata Kuliah
% ============================================================
\chapter*{Identitas Mata Kuliah}
\addcontentsline{toc}{chapter}{Identitas Mata Kuliah}

\begin{tabular}{ll}
  Nama Program Studi & : Teknik Informatika \\
  Nama Mata Kuliah & : Pemrograman Internet \\
  Kode Mata Kuliah & : \lbrack Diisi oleh Program Studi\rbrack \\
  Semester & : \lbrack Diisi sesuai kurikulum\rbrack \\
  SKS / Bobot Kredit & : 2 SKS \\
  Jumlah Pertemuan & : 16 pertemuan \\
  Dosen Pengampu & : \lbrack Nama Dosen, Gelar\rbrack \\
  Tanggal Penyusunan & : \lbrack Tanggal\rbrack \\
\end{tabular}

\vspace{1cm}

\section*{Capaian Pembelajaran Lulusan (CPL)}

CPL yang dibebankan pada mata kuliah ini mencakup kompetensi lulusan dalam aspek pengetahuan, keterampilan, dan sikap:

\begin{enumerate}
  \item \textbf{CPL-1 (Pengetahuan):} Menguasai konsep teoretis arsitektur web, protokol HTTP, serta pemrograman client-side dan server-side secara mendalam.
  
  \item \textbf{CPL-2 (Keterampilan Umum):} Mampu menerapkan pemikiran logis, kritis, sistematis, dan inovatif dalam konteks pengembangan aplikasi internet.
  
  \item \textbf{CPL-3 (Keterampilan Khusus):} Mampu merancang, mengimplementasikan, dan mengevaluasi aplikasi web dengan teknologi front-end dan back-end serta prinsip keamanan dasar.
  
  \item \textbf{CPL-4 (Sikap):} Menunjukkan sikap bertanggung jawab atas pekerjaan di bidang keahliannya secara mandiri dan mampu bekerja sama dalam tim.
\end{enumerate}

\section*{Capaian Pembelajaran Mata Kuliah (CPMK)}

Kemampuan atau kompetensi spesifik yang diharapkan mahasiswa kuasai setelah menyelesaikan mata kuliah:

\begin{enumerate}
  \item \textbf{CPMK-1:} Mahasiswa mampu memahami dan menjelaskan arsitektur web, protokol HTTP, serta peran pemrograman client-side dan server-side.
  
  \item \textbf{CPMK-2:} Mahasiswa mampu merancang dan membangun antarmuka web dengan HTML, CSS, dan JavaScript yang responsif dan aksesibel.
  
  \item \textbf{CPMK-3:} Mahasiswa mampu mengimplementasikan logika server-side dan integrasi basis data serta menghubungkannya dengan front-end.
  
  \item \textbf{CPMK-4:} Mahasiswa mampu membangun dan mengonsumsi API (REST), serta menerapkan prinsip keamanan dasar (autentikasi, validasi input, HTTPS).
\end{enumerate}

\section*{Matriks Keterkaitan CPL dan CPMK}

\begin{table}[h]
\centering
\small
\begin{tabular}{|c|c|c|p{4.5cm}|}
  \hline
  \textbf{CPL} & \textbf{CPMK} & \textbf{Kontribusi} & \textbf{Keterangan} \\
  \hline
  CPL-1 & CPMK-1 & Tinggi & Penguasaan konsep arsitektur web dan HTTP \\
  \hline
  CPL-1 & CPMK-2 & Tinggi & Pemahaman client-side dan desain antarmuka \\
  \hline
  CPL-2 & CPMK-2, CPMK-3, CPMK-4 & Tinggi & Penerapan pemikiran sistematis dalam implementasi web \\
  \hline
  CPL-3 & CPMK-2 & Tinggi & Perancangan dan implementasi antarmuka web \\
  \hline
  CPL-3 & CPMK-3 & Tinggi & Implementasi server-side dan basis data \\
  \hline
  CPL-3 & CPMK-4 & Tinggi & Implementasi API dan keamanan web \\
  \hline
  CPL-4 & CPMK-2, CPMK-3, CPMK-4 & Sedang & Tanggung jawab dan kerja sama dalam proyek web \\
  \hline
\end{tabular}
\caption{Matriks Keterkaitan CPL dan CPMK}
\end{table}

\cleardoublepage

% ============================================================
% Daftar Isi
% ============================================================
\phantomsection
\addcontentsline{toc}{chapter}{Daftar Isi}
\tableofcontents
\cleardoublepage

\clearpage
\phantomsection
\addcontentsline{toc}{chapter}{Daftar Gambar}
\listoffigures
\cleardoublepage

\phantomsection
\addcontentsline{toc}{chapter}{Daftar Tabel}
\listoftables
\cleardoublepage

\phantomsection
\addcontentsline{toc}{chapter}{Daftar Kode Program}
\lstlistoflistings
\cleardoublepage
