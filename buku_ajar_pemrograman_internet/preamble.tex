% ============================================================
% Preamble untuk Buku Ajar OBE - Pemrograman Internet
% ============================================================

% --- Encoding dan Bahasa ---
\usepackage[utf8]{inputenc}
\usepackage[T1]{fontenc}
\usepackage[indonesian]{babel}
\usepackage{lmodern}

% --- Layout Halaman ---
\usepackage{geometry}
\geometry{a4paper, margin=2.5cm, headheight=15pt}

% --- Grafik dan Tabel ---
\usepackage{graphicx}
\usepackage{booktabs}
\usepackage{longtable}
\usepackage{array}
\usepackage{multirow}
\usepackage{xcolor}
\usepackage{amssymb}

% --- TikZ untuk Diagram ---
\usepackage{tikz}
\usetikzlibrary{shapes,arrows,positioning,calc}
\tikzset{
  client/.style={rectangle, draw=black, fill=blue!10, minimum width=2.5cm, minimum height=1cm, text centered},
  server/.style={rectangle, draw=black, fill=green!10, minimum width=2.5cm, minimum height=1cm, text centered},
  arrow/.style={thick, ->, >=stealth}
}

% --- List dan Enumerasi ---
\usepackage{enumitem}

% --- Hyperlinks ---
\usepackage{xurl}
\usepackage{hyperref}
\hypersetup{
  colorlinks=true,
  linkcolor=blue,
  urlcolor=blue,
  citecolor=blue,
  bookmarksnumbered=true,
  pdfstartview=FitH
}

% --- Subfiles untuk Modular Structure ---
\usepackage{subfiles}

% --- Header dan Footer ---
\usepackage{fancyhdr}
\pagestyle{fancy}
\fancyhf{}
\fancyhead[LE,RO]{\thepage}
\fancyhead[RE]{\leftmark}
\fancyhead[LO]{\rightmark}
\renewcommand{\headrulewidth}{0.4pt}

% --- Code Listings untuk HTML, CSS, JavaScript, PHP ---
\usepackage{listings}
\usepackage{xcolor}
\definecolor{codegray}{rgb}{0.5,0.5,0.5}
\definecolor{codebg}{rgb}{0.95,0.95,0.95}
\definecolor{htmlblue}{rgb}{0,0.4,0.6}
\definecolor{htmlgreen}{rgb}{0.25,0.5,0.35}
\definecolor{htmlred}{rgb}{0.6,0,0}
\definecolor{htmlpurple}{rgb}{0.5,0,0.35}

% Style untuk HTML (bila language=HTML tidak dikenali, gunakan empty)
\lstdefinestyle{htmlstyle}{
  language=,
  basicstyle=\ttfamily\small,
  keywordstyle=\color{htmlblue}\bfseries,
  stringstyle=\color{htmlred},
  commentstyle=\color{htmlgreen},
  numbers=left,
  numberstyle=\tiny\color{codegray},
  backgroundcolor=\color{codebg},
  frame=single,
  tabsize=2,
  breaklines=true
}

% Style untuk CSS
\lstdefinestyle{cssstyle}{
  language=,
  basicstyle=\ttfamily\small,
  keywordstyle=\color{htmlblue}\bfseries,
  stringstyle=\color{htmlred},
  commentstyle=\color{htmlgreen},
  numbers=left,
  numberstyle=\tiny\color{codegray},
  backgroundcolor=\color{codebg},
  frame=single,
  tabsize=2,
  breaklines=true
}

% Style untuk JavaScript
\lstdefinestyle{jsstyle}{
  language=,
  basicstyle=\ttfamily\small,
  keywordstyle=\color{htmlpurple}\bfseries,
  stringstyle=\color{htmlred},
  commentstyle=\color{htmlgreen},
  numbers=left,
  numberstyle=\tiny\color{codegray},
  backgroundcolor=\color{codebg},
  frame=single,
  tabsize=2,
  breaklines=true
}

% Style untuk PHP
\lstdefinestyle{phpstyle}{
  language=PHP,
  basicstyle=\ttfamily\small,
  keywordstyle=\color{htmlpurple}\bfseries,
  stringstyle=\color{htmlred},
  commentstyle=\color{htmlgreen},
  numbers=left,
  numberstyle=\tiny\color{codegray},
  backgroundcolor=\color{codebg},
  frame=single,
  tabsize=2,
  breaklines=true
}

\renewcommand{\lstlistlistingname}{Daftar Kode Program}
\renewcommand{\lstlistingname}{Kode Program}

% Environment untuk kode web
\lstnewenvironment{htmlcode}[1][]{\lstset{style=htmlstyle, #1}}{}
\lstnewenvironment{csscode}[1][]{\lstset{style=cssstyle, #1}}{}
\lstnewenvironment{jscode}[1][]{\lstset{style=jsstyle, #1}}{}
\lstnewenvironment{phpcode}[1][]{\lstset{style=phpstyle, #1}}{}

% --- Boxes untuk Highlight ---
\usepackage{tcolorbox}
\tcbuselibrary{skins,breakable}

\newtcolorbox{konsep}{
  colback=blue!5!white,
  colframe=blue!75!black,
  fonttitle=\bfseries,
  title=Konsep Penting,
  breakable
}

\newtcolorbox{catatan}{
  colback=yellow!5!white,
  colframe=yellow!75!black,
  fonttitle=\bfseries,
  title=Catatan,
  breakable
}

\newtcolorbox{contoh}{
  colback=green!5!white,
  colframe=green!75!black,
  fonttitle=\bfseries,
  title=Contoh,
  breakable
}

% ============================================================
% Custom Environments untuk OBE
% ============================================================
\newenvironment{subcpmk}{
  \vspace{0.5cm}
  \noindent\colorbox{blue!10}{\parbox{\dimexpr\textwidth-2\fboxsep}{
    \textbf{\large Sub-CPMK yang Dicakup dalam Bab Ini:}
  }}
  \vspace{0.3cm}
  \begin{itemize}[leftmargin=*, itemsep=5pt]
}{
  \end{itemize}
  \vspace{0.5cm}
}

\newenvironment{aktivitas}{
  \vspace{0.5cm}
  \noindent\colorbox{green!10}{\parbox{\dimexpr\textwidth-2\fboxsep}{
    \textbf{\large Aktivitas Pembelajaran}
  }}
  \vspace{0.3cm}
  \begin{enumerate}[leftmargin=*, itemsep=8pt]
}{
  \end{enumerate}
  \vspace{0.5cm}
}

\newenvironment{latihan}{
  \vspace{0.5cm}
  \noindent\colorbox{orange!10}{\parbox{\dimexpr\textwidth-2\fboxsep}{
    \textbf{\large Latihan dan Refleksi}
  }}
  \vspace{0.3cm}
  \begin{enumerate}[leftmargin=*, itemsep=8pt]
}{
  \end{enumerate}
  \vspace{0.5cm}
}

\newenvironment{asesmen}{
  \vspace{0.5cm}
  \noindent\colorbox{red!10}{\parbox{\dimexpr\textwidth-2\fboxsep}{
    \textbf{\large Asesmen (Evaluasi Kinerja)}
  }}
  \vspace{0.3cm}
}{
  \vspace{0.5cm}
}

\newenvironment{checklist}{
  \vspace{0.5cm}
  \noindent\colorbox{purple!10}{\parbox{\dimexpr\textwidth-2\fboxsep}{
    \textbf{\large Checklist Pencapaian Kompetensi}
  }}
  \vspace{0.3cm}
  \noindent\textit{Centang item berikut setelah Anda yakin telah menguasainya:}
  \vspace{0.2cm}
  \begin{itemize}[leftmargin=*, itemsep=5pt, label=$\square$]
}{
  \end{itemize}
  \vspace{0.5cm}
}

\newenvironment{rangkuman}{
  \vspace{0.5cm}
  \noindent\colorbox{gray!10}{\parbox{\dimexpr\textwidth-2\fboxsep}{
    \textbf{\large Rangkuman}
  }}
  \vspace{0.3cm}
}{
  \vspace{0.5cm}
}

% ============================================================
% Custom Commands
% ============================================================
\newcommand{\code}[1]{\texttt{#1}}
\newcommand{\keyword}[1]{\texttt{\textbf{#1}}}

% ============================================================
% Pengaturan Tambahan
% ============================================================
\setlength{\parskip}{0.5em}
\setlength{\parindent}{0pt}
\setcounter{tocdepth}{2}
\setcounter{secnumdepth}{3}
