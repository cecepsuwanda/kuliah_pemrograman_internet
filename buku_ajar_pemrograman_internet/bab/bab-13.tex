\documentclass[../main]{subfiles}
\ifSubfilesClassLoaded{\setcounter{chapter}{12}}{}
\begin{document}

\chapter{Keamanan Web}

\begin{subcpmk}
  \item Sub-CPMK 4.2: Validasi input, XSS, CSRF, sanitasi, HTTPS
\end{subcpmk}

\section{Validasi Input dan Sanitasi}

Validasi input memastikan data memenuhi format dan aturan yang diharapkan \cite{owasp-input}. Validasi di server wajib karena client dapat di-bypass; validasi di client untuk UX \cite{owasp-series}. Sanitasi membersihkan input dari karakter berbahaya (mis. escape HTML untuk mencegah XSS) \cite{owasp-xss}. Whitelist (hanya menerima nilai yang diizinkan) lebih aman daripada blacklist \cite{owasp-input}. Untuk SQL, selalu gunakan prepared statement; jangan menyusun query dengan string concatenation \cite{owasp-sql}.

\section{XSS, CSRF, dan HTTPS}

XSS (Cross-Site Scripting) terjadi ketika input pengguna dieksekusi sebagai script di browser korban \cite{owasp-xss}. Pencegahan: escape output (encode HTML entities), gunakan \texttt{Content-Security-Policy}, validasi input \cite{owasp-xss}.

CSRF (Cross-Site Request Forgery) memanfaatkan session pengguna untuk memicu request tidak sah dari situs lain \cite{owasp-csrf}. Pencegahan: token CSRF di form, atribut \texttt{SameSite} pada cookie, verifikasi Origin/Referer \cite{owasp-csrf}. HTTPS mengenkripsi komunikasi client-server; wajib untuk data sensitif dan login \cite{owasp-rest}. OWASP Top Ten merangkum risiko keamanan web paling kritis \cite{owasp-top10}.


\begin{aktivitas}
  \item Terapkan escape output pada aplikasi Anda.
  \item Implementasikan token CSRF pada form.
  \item Pelajari OWASP Top Ten.
\end{aktivitas}

\begin{checklist}
  \item Memahami validasi dan sanitasi
  \item Memahami XSS dan pencegahannya
  \item Memahami CSRF dan pencegahannya
  \item Mengetahui pentingnya HTTPS
\end{checklist}

\begin{rangkuman}
Keamanan web meliputi validasi input, sanitasi, pencegahan XSS dan CSRF, serta penggunaan HTTPS. Referensi OWASP memberikan panduan praktis \cite{owasp-series}, \cite{owasp-top10}.
\end{rangkuman}

\ifSubfilesClassLoaded{
  \renewcommand{\bibname}{Daftar Pustaka}
  \bibliographystyle{plain}
  \bibliography{../references}
}{}
\end{document}
