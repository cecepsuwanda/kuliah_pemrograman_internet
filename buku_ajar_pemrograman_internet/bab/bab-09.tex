\documentclass[../main]{subfiles}
\ifSubfilesClassLoaded{\setcounter{chapter}{8}}{}
\begin{document}

\chapter{Pemrograman Server-Side (PHP/Node.js)}

\begin{subcpmk}
  \item Sub-CPMK 3.1: Mengimplementasikan script server-side untuk pemrosesan form
\end{subcpmk}

\section{Pengenalan Pemrograman Server-Side}

Pemrograman server-side menangani logika yang dijalankan di server, seperti pemrosesan form, akses basis data, dan pembuatan respons dinamis \cite{mdn-learn}. Bahasa server-side populer meliputi PHP, Node.js (JavaScript), Python (Django, Flask), dan Java (Spring) \cite{php-manual}, \cite{nodejs}. Server menerima HTTP request, memproses data, berinteraksi dengan basis data jika perlu, dan mengembalikan HTML, JSON, atau format lain sebagai respons \cite{mdn-express}. Environment pengembangan biasanya memerlukan web server (Apache, Nginx) dan interpreter/runtime untuk bahasa yang dipilih \cite{php-right-way}.

\section{PHP dan Node.js: Request-Response}

PHP adalah bahasa server-side yang tertanam dalam HTML dan dijalankan oleh web server \cite{php-manual}. Variabel \texttt{\$\_GET} dan \texttt{\$\_POST} menyimpan data dari request; \texttt{\$\_SERVER} berisi informasi server dan request \cite{w3schools-php}. Node.js menjalankan JavaScript di server; framework Express menyederhanakan penanganan route dan middleware \cite{express}, \cite{mdn-express}. Baik PHP maupun Node.js dapat membaca body request, mengembalikan HTML atau JSON, dan mengatur header respons \cite{nodejs}. Pemilihan teknologi bergantung pada kebutuhan proyek, tim, dan ekosistem yang tersedia \cite{php-right-way}.

\begin{lstlisting}[caption={Contoh PHP Memproses Form}, basicstyle=\ttfamily\small, frame=single]
<?php
if ($_SERVER['REQUEST_METHOD'] === 'POST') {
  $nama = htmlspecialchars($_POST['nama']);
  echo "Halo, " . $nama;
}
?>
\end{lstlisting}


\begin{aktivitas}
  \item Setup environment PHP atau Node.js lokal.
  \item Buat endpoint yang menerima form POST dan mengembalikan respons.
\end{aktivitas}

\begin{checklist}
  \item Memahami peran server-side
  \item Dapat memproses request (GET/POST)
  \item Dapat mengembalikan respons
\end{checklist}

\begin{rangkuman}
Pemrograman server-side menangani logika di server. PHP dan Node.js adalah pilihan populer; keduanya memproses request dan mengembalikan respons \cite{php-manual}, \cite{nodejs}.
\end{rangkuman}

\ifSubfilesClassLoaded{
  \renewcommand{\bibname}{Daftar Pustaka}
  \bibliographystyle{plain}
  \bibliography{../references}
}{}
\end{document}
