\documentclass[../main]{subfiles}
\ifSubfilesClassLoaded{\setcounter{chapter}{1}}{}
\begin{document}

\chapter{Landasan Teori: Arsitektur Web dan Protokol HTTP}

\begin{subcpmk}
  \item Sub-CPMK 1.1: Menjelaskan arsitektur client-server dan alur komunikasi HTTP request-response
  \item Sub-CPMK 1.2: Membedakan peran dan teknologi client-side (browser) dan server-side
\end{subcpmk}

\section{Arsitektur Web dan Model Client-Server}

Aplikasi web modern didasarkan pada arsitektur client-server, di mana peran dibagi antara klien (browser) yang meminta sumber daya dan server yang menyediakan serta memproses permintaan tersebut \cite{mdn-learn}. Pemahaman mendalam tentang arsitektur ini sangat penting karena seluruh komunikasi di web dibangun di atas model ini. Klien mengirim permintaan HTTP ke server, server memproses permintaan dan mengembalikan respons, lalu klien menampilkan hasil kepada pengguna.

Model client-server memungkinkan pemisahan tanggung jawab yang jelas: klien menangani tampilan dan interaksi pengguna, sementara server menangani logika bisnis, akses data, dan keamanan \cite{mdn-http}. Pemisahan ini memudahkan pengembangan, pemeliharaan, dan skalabilitas aplikasi. Selain itu, banyak klien dapat mengakses server yang sama secara bersamaan, yang merupakan dasar dari web yang kita kenal saat ini.

\begin{figure}[h]
\centering
\begin{tikzpicture}[node distance=3cm, >=stealth]
  \node[client] (client) {Browser\\\small(Client)};
  \node[server, right=of client] (server) {Web Server\\\small(Server)};
  \draw[arrow] (client) -- node[above, align=center] {1. HTTP Request\\(GET, POST, dll.)} (server);
  \draw[arrow] (server) -- node[below, align=center] {2. HTTP Response\\(HTML, JSON, dll.)} (client);
\end{tikzpicture}
\caption{Arsitektur Client-Server dalam Aplikasi Web}
\end{figure}

\begin{table}[h]
\centering
\small
\begin{tabular}{|l|p{5cm}|p{5cm}|}
\hline
\textbf{Aspek} & \textbf{Client (Browser)} & \textbf{Server} \\
\hline
Lokasi eksekusi & Komputer pengguna & Komputer hosting \\
\hline
Teknologi & HTML, CSS, JavaScript & PHP, Node.js, Python, Java \\
\hline
Peran utama & Menampilkan, berinteraksi & Memproses, menyimpan data \\
\hline
Akses data & Melalui API/request & Langsung ke basis data \\
\hline
\end{tabular}
\caption{Perbandingan Peran Client dan Server}
\end{table}

\section{Protokol HTTP dan Alur Request-Response}

HTTP (HyperText Transfer Protocol) adalah protokol lapisan aplikasi yang menjadi fondasi komunikasi data di World Wide Web \cite{mdn-http}. HTTP mendefinisikan bagaimana pesan diformat dan ditransmisikan, serta bagaimana server dan browser merespons berbagai perintah. Setiap kali pengguna membuka halaman web, mengirim form, atau memuat sumber daya, serangkaian pertukaran HTTP terjadi di balik layar.

Siklus HTTP bersifat request-response: klien mengirim permintaan, server mengembalikan respons. Permintaan HTTP mengandung metode (GET, POST, PUT, DELETE, dll.), URL tujuan, header (metadata), dan kadang body data \cite{http-status}. Respons HTTP mengandung status code (200 OK, 404 Not Found, 500 Server Error), header, dan body yang berisi konten yang diminta.

\begin{table}[htbp]
\centering
\small
\begin{tabular}{|l|>{\raggedright\arraybackslash}p{7.2cm}|}
\hline
\textbf{Metode HTTP} & \textbf{Keterangan} \\
\hline
GET & Mengambil sumber daya; tidak mengubah data di server \\
\hline
POST & Mengirim data untuk diproses; sering untuk form, membuat sumber baru \\
\hline
PUT & Mengganti sumber daya yang ada \\
\hline
DELETE & Menghapus sumber daya \\
\hline
\end{tabular}
\caption{Metode HTTP Utama}
\end{table}

Status code HTTP mengindikasikan hasil permintaan \cite{http-status}. Kode 2xx menandakan sukses; 3xx untuk pengalihan; 4xx untuk kesalahan klien (misalnya 404 Not Found); 5xx untuk kesalahan server. Memahami status code penting untuk debugging dan menangani error dengan benar dalam aplikasi web.

\section{Client-Side vs Server-Side}

Pemrograman web dibagi menjadi dua ranah utama: client-side dan server-side \cite{mdn-learn}. Client-side merujuk pada kode yang dijalankan di browser pengguna, sedangkan server-side merujuk pada kode yang dijalankan di server. Keduanya memiliki peran yang saling melengkapi dan sering bekerja bersama dalam satu aplikasi web yang lengkap.

Teknologi client-side meliputi HTML (struktur konten), CSS (tampilan dan layout), dan JavaScript (perilaku dan interaktivitas) \cite{mdn-html}, \cite{mdn-css}, \cite{mdn-js}. Kode client-side bersifat terbuka—pengguna dapat melihat dan memeriksa kode melalui alat pengembang browser. Oleh karena itu, logika sensitif dan data rahasia tidak boleh disimpan atau diproses di client-side.

Teknologi server-side mencakup PHP, Node.js, Python (Django, Flask), Java (Spring), dan bahasa lain yang berjalan di server \cite{php-manual}, \cite{nodejs}. Kode server-side bersifat tertutup dari pengguna; server memproses permintaan, mengakses basis data, menerapkan logika bisnis, dan mengembalikan hasil. Data sensitif dan validasi penting harus dilakukan di server untuk keamanan \cite{owasp-input}.

\section{Peta Konsep Pemrograman Internet}

Gambar \ref{fig:peta-konsep} memperlihatkan peta konsep utama dalam pemrograman internet. Alur pembelajaran dimulai dari fondasi (arsitektur, HTTP), kemudian membangun antarmuka (HTML, CSS, JavaScript), lalu logika server dan basis data, serta diakhiri dengan API dan keamanan.

\begin{figure}[h]
\centering
\begin{tikzpicture}[
  node distance=0.8cm,
  box/.style={rectangle, draw, fill=blue!10, minimum width=2.2cm, minimum height=0.6cm, font=\small, align=center}
]
  \node[box] (arch) {Arsitektur\\Web \& HTTP};
  \node[box, below=of arch] (html) {HTML};
  \node[box, left=1.5cm of html] (css) {CSS};
  \node[box, right=1.5cm of html] (js) {JavaScript};
  \node[box, below=of html] (server) {Server-Side};
  \node[box, below=of server] (db) {Basis Data};
  \node[box, below=of db] (api) {API \& Keamanan};
  \draw[->] (arch) -- (html);
  \draw[->] (arch) -- (css);
  \draw[->] (arch) -- (js);
  \draw[->] (html) -- (server);
  \draw[->] (css) -- (server);
  \draw[->] (js) -- (server);
  \draw[->] (server) -- (db);
  \draw[->] (db) -- (api);
\end{tikzpicture}
\caption{Peta Konsep Materi Pemrograman Internet}
\label{fig:peta-konsep}
\end{figure}

Memahami hubungan antar komponen ini membantu mahasiswa melihat gambaran besar sebelum masuk ke detail teknis setiap topik \cite{webdev}. Setiap lapisan dibangun di atas lapisan sebelumnya, sehingga penguasaan bertahap dari fondasi ke tingkat lanjut akan memudahkan pemahaman keseluruhan.

\section{Ringkasan Landasan Teori}

Landasan teori pemrograman internet mencakup arsitektur client-server, protokol HTTP, dan pemisahan peran client-side serta server-side. Arsitektur client-server adalah fondasi di mana browser (client) berkomunikasi dengan web server melalui pertukaran permintaan dan respons HTTP \cite{mdn-http}.

Protokol HTTP mendefinisikan metode (GET, POST, PUT, DELETE), status code, dan format pesan yang memungkinkan interoperabilitas antara berbagai klien dan server. Di sisi lain, pemahaman yang jelas tentang client-side versus server-side membantu pengembang memilih teknologi dan lapisan yang tepat untuk setiap kebutuhan \cite{devdocs}. Referensi daring seperti MDN, W3C, dan WHATWG menyediakan dokumentasi yang komprehensif untuk mempelajari standar web secara mendalam \cite{whatwg}.


\begin{aktivitas}
  \item Buat diagram sederhana (gambar atau TikZ) yang menggambarkan alur HTTP request dari browser ke server dan respons kembali.
  \item Jelaskan dengan kata-kata Anda sendiri perbedaan client-side dan server-side. Berikan 2 contoh fitur web yang dijalankan di client dan 2 contoh yang dijalankan di server.
  \item Gunakan alat pengembang browser (F12) untuk memeriksa tab Network. Buka sebuah situs web dan identifikasi request HTTP (metode, status code, URL).
\end{aktivitas}

\begin{latihan}
  \item Jelaskan arsitektur client-server dan mengapa model ini digunakan untuk aplikasi web!
  \item Sebutkan minimal 4 metode HTTP dan kegunaannya!
  \item Apa perbedaan antara kode yang dijalankan di client dan di server? Mengapa data sensitif harus diproses di server?
  \item \textbf{Refleksi}: Bagaimana pemahaman Anda tentang arsitektur web sebelum dan sesudah mempelajari bab ini?
\end{latihan}

\begin{asesmen}
\textbf{Instrumen untuk Sub-CPMK 1.1 dan 1.2}

\textbf{A. Pilihan Ganda}
\begin{enumerate}
  \item Dalam model client-server, browser bertindak sebagai:
  \begin{enumerate}
    \item Server
    \item Client
    \item Gateway
    \item Proxy
  \end{enumerate}
  \item Metode HTTP yang digunakan untuk mengirim data form ke server adalah:
  \begin{enumerate}
    \item GET
    \item POST
    \item FETCH
    \item SEND
  \end{enumerate}
\end{enumerate}

\textbf{B. Essay:} Jelaskan alur lengkap ketika pengguna mengakses halaman web dari browser (dari ketik URL hingga halaman tampil). Sebutkan peran client dan server dalam proses tersebut.

\textbf{Rubrik}: Lihat Lampiran.
\end{asesmen}

\begin{checklist}
  \item Saya dapat menjelaskan arsitektur client-server
  \item Saya dapat menjelaskan alur HTTP request-response
  \item Saya dapat membedakan client-side dan server-side
  \item Saya dapat menyebutkan teknologi untuk client dan server
\end{checklist}

\begin{rangkuman}
Bab ini membahas landasan teori pemrograman internet: arsitektur client-server, protokol HTTP, serta peran dan teknologi client-side serta server-side. Pemahaman ini menjadi fondasi untuk bab-bab berikutnya.

\textbf{Poin Kunci:} Arsitektur client-server; HTTP request-response; metode GET, POST, PUT, DELETE; status code; client-side (HTML, CSS, JS); server-side (PHP, Node.js, dll.).
\end{rangkuman}

\ifSubfilesClassLoaded{
  \renewcommand{\bibname}{Daftar Pustaka}
  \bibliographystyle{plain}
  \bibliography{references}
}{}
\end{document}
