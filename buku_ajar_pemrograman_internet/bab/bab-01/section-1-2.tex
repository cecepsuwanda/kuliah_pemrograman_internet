\section{Keterkaitan Buku Ajar dengan RPS Berbasis OBE}

Buku ajar ini dirancang selaras dengan Rencana Pembelajaran Semester (RPS) mata kuliah Pemrograman Internet yang berbasis Outcome-Based Education (OBE). Keterkaitan ini diwujudkan melalui pemetaan eksplisit setiap bab ke Sub-CPMK dan CPMK yang telah ditetapkan dalam silabus. Pendekatan ini memastikan bahwa materi pembelajaran, aktivitas, dan asesmen sejalan dengan capaian yang diharapkan oleh program studi.

\textbf{Alignment dengan CPL dan CPMK:} Setiap bab dalam buku ini dipetakan ke Sub-CPMK yang berkontribusi pada pencapaian CPMK dan CPL. Bab II--IV mencakup Sub-CPMK 1.1, 1.2 (arsitektur web) dan Sub-CPMK 2.1 (HTML). Bab V--VIII mencakup Sub-CPMK 2.2, 2.3 (CSS dan JavaScript). Bab IX--XIV mencakup Sub-CPMK 3.1, 3.2 (server-side, database) dan Sub-CPMK 4.1, 4.2 (API, keamanan).

\textbf{Integrasi Metode Pembelajaran:} Buku ini mengintegrasikan metode pembelajaran yang tercantum dalam RPS, meliputi ceramah interaktif, demonstrasi dan live coding, praktikum terbimbing, Problem-Based Learning (PBL), Project-Based Learning, peer review, dan studi kasus. Setiap bab menyediakan aktivitas yang mendukung metode tersebut sesuai konteks materi \cite{freecodecamp}.

\textbf{Sistem Penilaian:} Komponen asesmen dalam buku ini sejalan dengan bobot penilaian RPS: Tugas Individu (15\%), Kuis (10\%), Praktikum (15\%), UTS (20\%), Proyek Kelompok (20\%), dan UAS (20\%).
