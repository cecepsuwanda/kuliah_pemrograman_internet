\section{Tujuan Buku Ajar}

Buku ajar ini dirancang sebagai panduan komprehensif untuk menguasai \textit{Pemrograman Internet} dengan memanfaatkan teknologi web modern (\cite{mdn-learn}). Fokus utama buku ini adalah pada penguasaan arsitektur web, pemrograman client-side (HTML, CSS, JavaScript), pemrograman server-side, integrasi basis data, dan prinsip keamanan dasar. Tujuan spesifik buku ini adalah memberikan landasan yang kuat bagi mahasiswa untuk mengembangkan aplikasi web yang fungsional dan aman.

Pembelajaran pemrograman internet mencakup berbagai lapisan teknologi yang saling melengkapi. Di sisi client, mahasiswa akan mempelajari struktur konten (HTML), presentasi visual (CSS), dan interaktivitas (JavaScript) sesuai standar web yang ditetapkan oleh W3C dan WHATWG \cite{whatwg}. Di sisi server, mahasiswa akan diperkenalkan pada pemrosesan request, koneksi basis data, dan integrasi aplikasi web dinamis.

Setelah mempelajari buku ini secara menyeluruh, mahasiswa diharapkan mampu:
\begin{enumerate}
  \item Memahami arsitektur client-server dan protokol HTTP \cite{mdn-http}
  \item Membangun antarmuka web responsif dengan HTML5, CSS3, dan JavaScript
  \item Mengimplementasikan logika server-side dan integrasi basis data
  \item Mengintegrasikan aplikasi web dengan PHP dan MySQL
  \item Menerapkan prinsip keamanan web dasar
\end{enumerate}
