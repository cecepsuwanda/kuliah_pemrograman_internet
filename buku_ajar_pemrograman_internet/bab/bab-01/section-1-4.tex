\section{Konteks Kurikulum OBE}

\textbf{Outcome-Based Education (OBE)} adalah pendekatan pembelajaran yang berfokus pada pencapaian hasil (\textit{outcomes}) yang terukur. Dalam OBE, proses pembelajaran dirancang secara sistematis untuk memastikan mahasiswa mencapai kompetensi yang telah ditetapkan. Prinsip utama OBE meliputi: kejelasan fokus pada outcomes, perancangan kurikulum secara top-down dari outcomes, ekspektasi tinggi, dan kesempatan beragam untuk mencapai outcomes.

Buku ini mengimplementasikan OBE melalui komponen-komponen yang terstruktur. Setiap bab diawali dengan Sub-CPMK yang eksplisit; materi disusun dari dasar ke lanjut secara sistematis; aktivitas beragam (latihan, studi kasus, proyek, peer review) mendukung pencapaian kompetensi; asesmen terukur dengan rubrik yang jelas; dan checklist memberikan umpan balik berkelanjutan bagi mahasiswa.

\begin{table}[h]
\centering
\small
\begin{tabular}{|l|p{8cm}|}
\hline
\textbf{Komponen OBE} & \textbf{Implementasi dalam Buku} \\
\hline
Outcomes yang Jelas & Sub-CPMK eksplisit di setiap bab \\
\hline
Pembelajaran Terstruktur & Materi dari arsitektur web $\rightarrow$ HTML $\rightarrow$ CSS $\rightarrow$ JS $\rightarrow$ server $\rightarrow$ basis data $\rightarrow$ keamanan \\
\hline
Aktivitas Beragam & Latihan coding, studi kasus web, proyek aplikasi, peer review \\
\hline
Asesmen Terukur & Rubrik penilaian untuk tugas, praktikum, proyek \\
\hline
Feedback Berkelanjutan & Checklist untuk self-assessment per bab \\
\hline
\end{tabular}
\caption{Implementasi OBE dalam Buku Ajar Pemrograman Internet}
\end{table}
