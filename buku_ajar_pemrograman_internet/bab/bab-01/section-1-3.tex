\section{Petunjuk Penggunaan Buku Ajar}

\subsection{Untuk Mahasiswa}

Buku ini dirancang untuk dipelajari secara bertahap selama 16 pertemuan (2 SKS). Sebelum perkuliahan, bacalah Sub-CPMK di awal setiap bab untuk memahami target pembelajaran, pelajari materi pokok dengan seksama, dan jalankan semua contoh kode menggunakan browser dan editor teks. Mencatat pertanyaan atau konsep yang belum dipahami akan memperkaya diskusi di kelas.

Selama perkuliahan, manfaatkan waktu untuk diskusi, praktikum, dan live coding. Kerjakan aktivitas pembelajaran secara aktif, tanyakan hal-hal yang belum jelas, dan berpartisipasi dalam peer review terhadap tampilan atau kode web yang dibuat rekan. Setelah perkuliahan, kerjakan latihan dan refleksi, lakukan asesmen mandiri, centang checklist kompetensi yang telah dikuasai, dan kerjakan proyek mini untuk memperdalam pemahaman \cite{webdev}.

\subsection{Untuk Dosen}

Buku ini dapat digunakan sebagai bahan ajar utama, sumber latihan dan tugas, referensi untuk menyusun soal ujian, dan panduan untuk merancang aktivitas pembelajaran. Dokumentasi daring seperti MDN, W3Schools, dan OWASP dapat dilengkapi sebagai referensi tambahan sesuai kebutuhan \cite{mdn-learn}.
