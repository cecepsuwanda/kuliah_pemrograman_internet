\section{Peta Konsep Pemrograman Internet}

Mata kuliah Pemrograman Internet mencakup topik-topik utama yang saling terkait dan disusun mengikuti alur pembelajaran dari dasar ke lanjut \cite{mdn-learn}. Peta konsep ini membantu mahasiswa memvisualisasikan struktur materi secara keseluruhan.

\begin{enumerate}
  \item \textbf{Bab II}: Landasan Teori --- Arsitektur web, HTTP, client-server
  \item \textbf{Bab III}: HTML5 --- Struktur dokumen, elemen semantik, link, gambar, tabel
  \item \textbf{Bab IV}: HTML5 Form --- Form dan elemen input
  \item \textbf{Bab V}: CSS3 --- Selector, box model, warna, typography
  \item \textbf{Bab VI}: CSS3 Layout --- Flexbox, Grid, desain responsif
  \item \textbf{Bab VII}: JavaScript --- Sintaks, DOM, event handling
  \item \textbf{Bab VIII}: JavaScript --- Validasi form, integrasi dengan HTML/CSS
  \item \textbf{Bab IX}: Server-Side --- Pengenalan PHP/Node.js, request-response
  \item \textbf{Bab X}: Basis Data --- Integrasi, CRUD
  \item \textbf{Bab XI}: Session, Cookie, Autentikasi
  \item \textbf{Bab XII}: REST API --- Endpoint, Fetch/AJAX
  \item \textbf{Bab XIII}: Keamanan Web --- XSS, CSRF, validasi, sanitasi
  \item \textbf{Bab XIV}: Evaluasi dan Integrasi Kompetensi
\end{enumerate}

\textbf{Alur Pembelajaran:} Bab II--IV membangun fondasi arsitektur dan struktur konten; Bab V--VIII membangun antarmuka dan interaktivitas; Bab IX--XII membangun logika server, basis data, dan API; Bab XIII menambahkan lapisan keamanan; Bab XIV mengintegrasikan seluruh kompetensi melalui evaluasi komprehensif.
