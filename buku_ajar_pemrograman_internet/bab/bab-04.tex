\documentclass[../main]{subfiles}
\ifSubfilesClassLoaded{\setcounter{chapter}{3}}{}
\begin{document}

\chapter{HTML5 Lanjutan: Form, API, dan Aksesibilitas}

\begin{subcpmk}
  \item Sub-CPMK 2.1, 2.3: Membuat form HTML5, menggunakan API browser, dan menerapkan aksesibilitas.
\end{subcpmk}

\section{Forms Dasar HTML}

Forms adalah elemen fundamental HTML untuk mengumpulkan data dari pengguna. Forms memungkinkan interaksi antara user dan website, mulai dari login form hingga survey kompleks \cite{w3schools-html}. Pemahaman forms adalah kunci untuk web development yang interaktif \cite{mdn-html}.

\subsection{Basic Form Structure}

Form HTML dibangun menggunakan tag-tag berikut:

\begin{itemize}
  \item \texttt{<form>} - Container utama untuk form
  \item \texttt{<input>} - Elemen input untuk berbagai tipe data
  \item \texttt{<label>} - Label untuk accessibility dan user experience
  \item \texttt{<button>} - Tombol untuk submit atau aksi lain
  \item \texttt{<textarea>} - Input teks multi-line
  \item \texttt{<select>} dan \texttt{<option>} - Dropdown/select box
  \item \texttt{<fieldset>} dan \texttt{<legend>} - Grouping related fields
\end{itemize}

\begin{lstlisting}[caption={Contoh Form Dasar}, basicstyle=\ttfamily\small, frame=single]
<!DOCTYPE html>
<html>
<head>
  <title>Contoh Form Dasar</title>
</head>
<body>
  <h1>Demonstrasi HTML Forms</h1>
  
  <!-- Form sederhana -->
  <h3>Form Kontak Sederhana</h3>
  <form action="/submit-contact" method="post">
    <!-- Text Input dengan Label -->
    <p>
      <label for="nama">Nama Lengkap:</label><br>
      <input type="text" id="nama" name="nama" size="30">
    </p>
    
    <p>
      <label for="email">Email:</label><br>
      <input type="text" id="email" name="email" size="30">
    </p>
    
    <!-- Password Input -->
    <p>
      <label for="password">Password:</label><br>
      <input type="password" id="password" name="password" size="30">
    </p>
    
    <!-- Submit Button -->
    <p>
      <input type="submit" value="Kirim">
      <input type="reset" value="Reset">
    </p>
  </form>
  
  <!-- Form dengan berbagai input types -->
  <h3>Form dengan Berbagai Input Types</h3>
  <form action="/register" method="post">
    <fieldset>
      <legend>Informasi Pribadi</legend>
      
      <p>
        <label for="fullname">Nama Lengkap:</label><br>
        <input type="text" id="fullname" name="fullname" maxlength="50">
      </p>
      
      <p>
        <label for="age">Umur:</label><br>
        <input type="text" id="age" name="age" size="5" maxlength="3">
      </p>
      
      <p>
        <label for="gender">Jenis Kelamin:</label><br>
        <input type="radio" id="male" name="gender" value="L">
        <label for="male">Laki-laki</label>
        <input type="radio" id="female" name="gender" value="P">
        <label for="female">Perempuan</label>
      </p>
    </fieldset>
    
    <fieldset>
      <legend>Preferensi</legend>
      
      <p>
        <label for="newsletter">Subscribe Newsletter:</label><br>
        <input type="checkbox" id="newsletter" name="newsletter" value="yes">
        <label for="newsletter">Ya, saya ingin menerima newsletter</label>
      </p>
      
      <p>
        <label for="country">Negara:</label><br>
        <select id="country" name="country">
          <option value="">-- Pilih Negara --</option>
          <option value="ID">Indonesia</option>
          <option value="MY">Malaysia</option>
          <option value="SG">Singapore</option>
          <option value="US">United States</option>
        </select>
      </p>
    </fieldset>
    
    <p>
      <label for="comments">Komentar:</label><br>
      <textarea id="comments" name="comments" rows="4" cols="40">
        Tulis komentar Anda di sini...
      </textarea>
    </p>
    
    <p>
      <input type="submit" value="Daftar">
      <input type="button" value="Batal" onclick="alert('Form dibatalkan!')">
    </p>
  </form>
</body>
</html>
\end{lstlisting}

\textbf{Hasil di Browser:}
- Form dengan fieldset untuk grouping related fields
- Label yang terhubung dengan input untuk accessibility
- Berbagai input types: text, password, radio, checkbox, select, textarea
- Submit dan reset button dengan fungsi berbeda
- Legend untuk fieldset description

\subsection{Input Types Detail}

HTML menyediakan berbagai tipe input untuk keperluan spesifik:

\begin{itemize}
  \item \textbf{Text Inputs}: text, password, hidden
  \item \textbf{Choice Inputs}: radio, checkbox
  \item \textbf{Date/Time}: date, time, datetime-local
  \item \textbf{Numeric}: number, range, tel
  \item \textbf{File Inputs}: file, image
  \item \textbf{Action Inputs}: submit, reset, button
\end{itemize}

\begin{lstlisting}[caption={Berbagai Input Types}, basicstyle=\ttfamily\small, frame=single]
<!DOCTYPE html>
<html>
<head>
  <title>Berbagai Input Types</title>
</head>
<body>
  <h1>Demonstrasi Input Types</h1>
  
  <form action="/process" method="post">
    <!-- Text Inputs -->
    <fieldset>
      <legend>Text Inputs</legend>
      
      <p><label for="username">Username:</label><br>
         <input type="text" id="username" name="username" 
                placeholder="Masukkan username" maxlength="20" required>
      </p>
      
      <p><label for="search">Pencarian:</label><br>
         <input type="search" id="search" name="search" 
                placeholder="Cari..." results="5">
      </p>
      
      <p><label for="url">Website:</label><br>
         <input type="url" id="url" name="url" 
                placeholder="https://www.example.com">
      </p>
      
      <p><label for="email">Email:</label><br>
         <input type="email" id="email" name="email" 
                placeholder="nama@email.com" required>
      </p>
      
      <p><label for="password">Password:</label><br>
         <input type="password" id="password" name="password" 
                placeholder="Minimal 8 karakter" minlength="8" required>
      </p>
      
      <p><label for="hidden">Hidden Field:</label><br>
         <input type="hidden" id="hidden" name="csrf_token" value="abc123">
      </p>
    </fieldset>
    
    <!-- Numeric Inputs -->
    <fieldset>
      <legend>Numeric Inputs</legend>
      
      <p><label for="age">Umur:</label><br>
         <input type="number" id="age" name="age" 
                min="1" max="120" step="1" value="25">
      </p>
      
      <p><label for="range">Rating:</label><br>
         <input type="range" id="range" name="rating" 
                min="1" max="10" value="5" oninput="showValue(this.value)">
         <span id="rangeValue">5</span>
      </p>
      
      <p><label for="phone">Telepon:</label><br>
         <input type="tel" id="phone" name="phone" 
                placeholder="0812-3456-789" pattern="[0-9]{3}-[0-9]{4}-[0-9]{4}">
      </p>
    </fieldset>
    
    <!-- Date/Time Inputs -->
    <fieldset>
      <legend>Date and Time Inputs</legend>
      
      <p><label for="birthdate">Tanggal Lahir:</label><br>
         <input type="date" id="birthdate" name="birthdate" 
                min="1900-01-01" max="2024-12-31">
      </p>
      
      <p><label for="meeting-time">Waktu Meeting:</label><br>
         <input type="time" id="meeting-time" name="meeting_time" 
                min="09:00" max="17:00">
      </p>
      
      <p><label for="event-datetime">Acara:</label><br>
         <input type="datetime-local" id="event-datetime" name="event_datetime">
      </p>
      
      <p><label for="month">Bulan:</label><br>
         <input type="month" id="month" name="month" value="2024-12">
      </p>
      
      <p><label for="week">Minggu:</label><br>
         <input type="week" id="week" name="week" value="2024-W52">
      </p>
    </fieldset>
    
    <!-- Choice Inputs -->
    <fieldset>
      <legend>Choice Inputs</legend>
      
      <p><label for="hobbies">Hobi:</label><br>
         <input type="checkbox" id="reading" name="hobbies[]" value="membaca">
         <label for="reading">Membaca</label><br>
         
         <input type="checkbox" id="sports" name="hobbies[]" value="olahraga">
         <label for="sports">Olahraga</label><br>
         
         <input type="checkbox" id="music" name="hobbies[]" value="musik">
         <label for="music">Musik</label>
      </p>
      
      <p><label for="education">Pendidikan Terakhir:</label><br>
         <input type="radio" id="sma" name="education" value="SMA">
         <label for="sma">SMA</label><br>
         
         <input type="radio" id="s1" name="education" value="S1">
         <label for="s1">S1/Diploma</label><br>
         
         <input type="radio" id="s2" name="education" value="S2">
         <label for="s2">S2/Sarjana</label>
      </p>
    </fieldset>
    
    <!-- File Inputs -->
    <fieldset>
      <legend>File Inputs</legend>
      
      <p><label for="avatar">Avatar:</label><br>
         <input type="file" id="avatar" name="avatar" 
                accept="image/*" capture="user">
      </p>
      
      <p><label for="documents">Upload Dokumen:</label><br>
         <input type="file" id="documents" name="documents[]" 
                accept=".pdf,.doc,.docx" multiple>
      </p>
      
      <p><label for="photo">Ambil Foto:</label><br>
         <input type="image" id="photo" name="photo" 
                accept="image/*" alt="Camera input">
      </p>
    </fieldset>
    
    <!-- Action Buttons -->
    <fieldset>
      <legend>Form Actions</legend>
      
      <p>
         <input type="submit" value="Submit Form" style="background-color: #4CAF50; color: white;">
         <input type="reset" value="Reset Form" style="background-color: #f44336; color: white;">
         <input type="button" value="Batal" onclick="alert('Dibatalkan!')" 
                style="background-color: #ff9800; color: white;">
      </p>
    </fieldset>
  </form>
  
  <script>
    function showValue(value) {
      document.getElementById('rangeValue').textContent = value;
    }
  </script>
</body>
</html>
\end{lstlisting}

\textbf{Hasil di Browser:}
- Text inputs dengan placeholder dan validation
- Numeric inputs dengan min/max/step controls
- Date/time inputs dengan native calendar picker
- Radio buttons untuk single choice
- Checkboxes untuk multiple choices
- File inputs dengan drag-and-drop support
- Styled action buttons dengan different behaviors

\subsection{Form Attributes dan Validation}

HTML5 menyediakan atribut untuk validasi dan user experience:

\begin{itemize}
  \item \textbf{Validation}: required, pattern, min, max, minlength, maxlength
  \item \textbf{User Experience}: placeholder, autofocus, autocomplete, readonly, disabled
  \item \textbf{Form Attributes}: action, method, enctype, target, novalidate
\end{itemize}

\begin{lstlisting}[caption={Form Attributes dan Validation}, basicstyle=\ttfamily\small, frame=single]
<!DOCTYPE html>
<html>
<head>
  <title>Form Validation dan Attributes</title>
  <style>
    .form-group {
      margin-bottom: 15px;
    }
    label {
      display: block;
      margin-bottom: 5px;
      font-weight: bold;
    }
    input, select, textarea {
      width: 100%;
      padding: 8px;
      border: 1px solid #ddd;
      border-radius: 4px;
      box-sizing: border-box;
    }
    input:focus, select:focus, textarea:focus {
      border-color: #4CAF50;
      outline: none;
    }
    .error {
      border-color: #f44336;
    }
    .success {
      border-color: #4CAF50;
    }
  </style>
</head>
<body>
  <h1>Form dengan Validation dan UX Attributes</h1>
  
  <form action="/register" method="post" novalidate>
    <!-- Required Fields -->
    <fieldset>
      <legend>Informasi Wajib</legend>
      
      <div class="form-group">
        <label for="fullname">Nama Lengkap:</label>
        <input type="text" id="fullname" name="fullname" 
               required minlength="3" maxlength="50"
               placeholder="Wajib diisi (3-50 karakter)">
      </div>
      
      <div class="form-group">
        <label for="email">Email:</label>
        <input type="email" id="email" name="email" 
               required placeholder="email@domain.com">
      </div>
      
      <div class="form-group">
        <label for="password">Password:</label>
        <input type="password" id="password" name="password" 
               required minlength="8" maxlength="20"
               placeholder="Minimal 8 karakter">
      </div>
      
      <div class="form-group">
        <label for="confirm-password">Konfirmasi Password:</label>
        <input type="password" id="confirm-password" name="confirm_password" 
               required minlength="8" maxlength="20"
               placeholder="Ketik ulang password">
      </div>
    </fieldset>
    
    <!-- Optional Fields -->
    <fieldset>
      <legend>Informasi Opsional</legend>
      
      <div class="form-group">
        <label for="phone">Telepon:</label>
        <input type="tel" id="phone" name="phone" 
               pattern="[0-9]{3}-[0-9]{4}-[0-9]{4}"
               placeholder="0812-3456-789">
      </div>
      
      <div class="form-group">
        <label for="website">Website:</label>
        <input type="url" id="website" name="website" 
               placeholder="https://www.example.com">
      </div>
      
      <div class="form-group">
        <label for="age">Umur:</label>
        <input type="number" id="age" name="age" 
               min="17" max="100" step="1" value="25">
      </div>
    </fieldset>
    
    <!-- Special Attributes -->
    <fieldset>
      <legend>Form Attributes</legend>
      
      <div class="form-group">
        <label for="search">Pencarian:</label>
        <input type="search" id="search" name="search" 
               autofocus autocomplete="on"
               placeholder="Cari produk..." results="10">
      </div>
      
      <div class="form-group">
        <label for="comments">Komentar Tambahan:</label>
        <textarea id="comments" name="comments" rows="4" 
                  placeholder="Tulis komentar opsional di sini..."></textarea>
      </div>
      
      <div class="form-group">
        <label for="newsletter">
          <input type="checkbox" id="newsletter" name="newsletter" value="yes" checked>
          Saya ingin menerima newsletter dan promo
        </label>
      </div>
    </fieldset>
    
    <!-- Form Actions -->
    <div class="form-group">
      <input type="submit" value="Daftar Sekarang" 
             style="background-color: #4CAF50; color: white; padding: 10px 20px; border: none; border-radius: 4px; cursor: pointer;">
      
      <input type="reset" value="Reset Form" 
             style="background-color: #f44336; color: white; padding: 10px 20px; border: none; border-radius: 4px; cursor: pointer; margin-left: 10px;">
    </div>
  </form>
  
  <script>
    // Client-side validation example
    document.querySelector('form').addEventListener('submit', function(e) {
      const password = document.getElementById('password').value;
      const confirmPassword = document.getElementById('confirm-password').value;
      
      if (password !== confirmPassword) {
        e.preventDefault();
        alert('Password tidak cocok!');
        document.getElementById('confirm-password').classList.add('error');
      } else {
        document.getElementById('confirm-password').classList.add('success');
      }
    });
  </script>
</body>
</html>
\end{lstlisting}

\textbf{Hasil di Browser:}
- Required fields dengan browser validation
- Pattern validation untuk phone format
- Min/max validation untuk numeric input
- Autofocus pada search field
- Checked state untuk newsletter checkbox
- Client-side validation dengan JavaScript
- Styled submit dan reset buttons

\subsection{Contoh Lengkap: Multi-Step Registration Form}

Berikut contoh registration form yang kompleks dengan multiple sections:

\begin{lstlisting}[caption={Multi-Step Registration Form}, basicstyle=\ttfamily\small, frame=single]
<!DOCTYPE html>
<html>
<head>
  <title>Multi-Step Registration</title>
  <style>
    .form-container {
      max-width: 600px;
      margin: 0 auto;
      padding: 20px;
      border: 1px solid #ddd;
      border-radius: 8px;
    }
    .step {
      display: none;
    }
    .step.active {
      display: block;
    }
    .step-indicator {
      text-align: center;
      margin-bottom: 20px;
    }
    .step-dot {
      display: inline-block;
      width: 12px;
      height: 12px;
      border-radius: 50%;
      background-color: #ccc;
      margin: 0 5px;
    }
    .step-dot.active {
      background-color: #4CAF50;
    }
    .form-group {
      margin-bottom: 15px;
    }
    label {
      display: block;
      margin-bottom: 5px;
      font-weight: bold;
    }
    input, select {
      width: 100%;
      padding: 8px;
      border: 1px solid #ddd;
      border-radius: 4px;
    }
    .btn {
      background-color: #4CAF50;
      color: white;
      padding: 10px 20px;
      border: none;
      border-radius: 4px;
      cursor: pointer;
    }
    .btn-secondary {
      background-color: #6c757d;
    }
  </style>
</head>
<body>
  <h1>Form Registrasi Multi-Step</h1>
  
  <div class="form-container">
    <!-- Step Indicators -->
    <div class="step-indicator">
      <span class="step-dot active" id="step1-dot"></span>
      <span class="step-dot" id="step2-dot"></span>
      <span class="step-dot" id="step3-dot"></span>
    </div>
    
    <form action="/register" method="post" id="registrationForm">
      <!-- Step 1: Personal Information -->
      <div class="step active" id="step1">
        <h2>Langkah 1: Informasi Pribadi</h2>
        
        <div class="form-group">
          <label for="firstName">Nama Depan:</label>
          <input type="text" id="firstName" name="firstName" required>
        </div>
        
        <div class="form-group">
          <label for="lastName">Nama Belakang:</label>
          <input type="text" id="lastName" name="lastName" required>
        </div>
        
        <div class="form-group">
          <label for="email">Email:</label>
          <input type="email" id="email" name="email" required>
        </div>
        
        <div class="form-group">
          <label for="phone">Telepon:</label>
          <input type="tel" id="phone" name="phone">
        </div>
        
        <button type="button" class="btn" onclick="nextStep(2)">Selanjutnya</button>
      </div>
      
      <!-- Step 2: Account Information -->
      <div class="step" id="step2">
        <h2>Langkah 2: Informasi Akun</h2>
        
        <div class="form-group">
          <label for="username">Username:</label>
          <input type="text" id="username" name="username" required minlength="4">
        </div>
        
        <div class="form-group">
          <label for="password">Password:</label>
          <input type="password" id="password" name="password" required minlength="8">
        </div>
        
        <div class="form-group">
          <label for="confirmPassword">Konfirmasi Password:</label>
          <input type="password" id="confirmPassword" name="confirmPassword" required>
        </div>
        
        <div class="form-group">
          <label for="securityQuestion">Pertanyaan Keamanan:</label>
          <select id="securityQuestion" name="securityQuestion" required>
            <option value="">-- Pilih Pertanyaan --</option>
            <option value="pet">Nama hewan peliharaan pertama?</option>
            <option value="school">Nama sekolah dasar?</option>
            <option value="city">Kota kelahiran?</option>
          </select>
        </div>
        
        <div class="form-group">
          <label for="securityAnswer">Jawaban:</label>
          <input type="text" id="securityAnswer" name="securityAnswer" required>
        </div>
        
        <button type="button" class="btn btn-secondary" onclick="previousStep(1)">Sebelumnya</button>
        <button type="button" class="btn" onclick="nextStep(3)">Selanjutnya</button>
      </div>
      
      <!-- Step 3: Preferences -->
      <div class="step" id="step3">
        <h2>Langkah 3: Preferensi</h2>
        
        <div class="form-group">
          <label for="newsletter">
            <input type="checkbox" id="newsletter" name="newsletter" value="yes" checked>
            Saya ingin menerima newsletter
          </label>
        </div>
        
        <div class="form-group">
          <label for="notifications">
            <input type="checkbox" id="notifications" name="notifications" value="yes">
            Aktifkan notifikasi email
          </label>
        </div>
        
        <div class="form-group">
          <label for="language">Bahasa Preferensi:</label>
          <select id="language" name="language">
            <option value="id">Bahasa Indonesia</option>
            <option value="en">English</option>
            <option value="zh">中文</option>
          </select>
        </div>
        
        <div class="form-group">
          <label for="timezone">Timezone:</label>
          <select id="timezone" name="timezone">
            <option value="WIB">WIB (GMT+7)</option>
            <option value="WITA">WITA (GMT+8)</option>
            <option value="WIT">WIT (GMT+9)</option>
          </select>
        </div>
        
        <button type="button" class="btn btn-secondary" onclick="previousStep(2)">Sebelumnya</button>
        <button type="submit" class="btn">Daftar Sekarang</button>
      </div>
    </form>
  </div>
  
  <script>
    let currentStep = 1;
    
    function showStep(stepNumber) {
      // Hide all steps
      document.querySelectorAll('.step').forEach(step => {
        step.classList.remove('active');
      });
      
      // Show current step
      document.getElementById('step' + stepNumber).classList.add('active');
      
      // Update indicators
      document.querySelectorAll('.step-dot').forEach(dot => {
        dot.classList.remove('active');
      });
      document.getElementById('step' + stepNumber + '-dot').classList.add('active');
      
      currentStep = stepNumber;
    }
    
    function nextStep(step) {
      if (validateStep(currentStep)) {
        showStep(step);
      }
    }
    
    function previousStep(step) {
      showStep(step);
    }
    
    function validateStep(step) {
      let isValid = true;
      
      if (step === 1) {
        const firstName = document.getElementById('firstName').value;
        const email = document.getElementById('email').value;
        
        if (!firstName || !email) {
          alert('Nama dan email wajib diisi!');
          isValid = false;
        }
      } else if (step === 2) {
        const username = document.getElementById('username').value;
        const password = document.getElementById('password').value;
        const confirmPassword = document.getElementById('confirmPassword').value;
        
        if (!username || !password || password !== confirmPassword) {
          alert('Username dan password harus diisi dan cocok!');
          isValid = false;
        }
      }
      
      return isValid;
    }
  </script>
</body>
</html>
\end{lstlisting}

\textbf{Hasil di Browser:}
- Multi-step form dengan step indicators
- Validasi antar step sebelum lanjut
- Smooth transitions antar sections
- Progress indicators dengan dots
- Form validation yang komprehensif
- User-friendly navigation dengan previous/next buttons

Forms yang baik memungkinkan pengumpulan data yang efektif dan user-friendly \cite{w3schools-html}. Kombinasi berbagai input types, validation attributes, dan user experience features akan menciptakan forms yang optimal untuk berbagai keperluan \cite{mdn-html}.

\section{Validasi Dasar Form di HTML5}

HTML5 menyediakan validasi bawaan melalui atribut seperti \texttt{required}, \texttt{pattern}, \texttt{min}, \texttt{max}, dan \texttt{minlength} \cite{mdn-html}. Atribut \texttt{required} mencegah pengiriman form jika field kosong. Atribut \texttt{pattern} menerima ekspresi reguler untuk memvalidasi format input. Validasi ini dijalankan di browser sebelum form dikirim, memberikan umpan balik cepat kepada pengguna \cite{webdev}.

Namun, validasi di client saja tidak cukup untuk keamanan \cite{owasp-input}. Data harus selalu divalidasi kembali di server karena pengguna dapat menonaktifkan validasi JavaScript/HTML5 atau mengirim request langsung. Kombinasi validasi client (untuk UX) dan server (untuk keamanan) merupakan praktik yang disarankan \cite{owasp-series}. Bab VIII akan membahas validasi form dengan JavaScript untuk kontrol yang lebih halus.

\section{Form HTML5 Lanjutan}

HTML5 membawa evolusi signifikan dalam form development dengan menambahkan tipe input baru, atribut validasi yang lebih kuat, dan API untuk kontrol form yang lebih baik \cite{mdn-html}. Form HTML5 modern tidak hanya mengumpulkan data tetapi juga memberikan user experience yang lebih baik dengan validasi real-time dan feedback yang intuitif \cite{w3schools-html}.

\subsection{Advanced Input Types HTML5}

HTML5 memperkenalkan tipe input baru yang memberikan validasi dan user experience yang lebih baik:

\begin{itemize}
  \item \textbf{Email dan URL}: \texttt{type="email"} dan \texttt{type="url"} dengan validasi format otomatis
  \item \textbf{Numeric Controls}: \texttt{type="number"}, \texttt{type="range"}, \texttt{type="tel"} dengan kontrol khusus
  \item \textbf{Date/Time Inputs}: \texttt{type="date"}, \texttt{type="time"}, \texttt{type="datetime-local"}
  \item \textbf{Color Picker}: \texttt{type="color"} untuk pemilihan warna
  \item \textbf{Search}: \texttt{type="search"} dengan optimasi untuk pencarian
  \item \textbf{File Upload}: \texttt{type="file"} dengan atribut \texttt{accept} dan \texttt{multiple}
\end{itemize}

\begin{lstlisting}[caption={Advanced HTML5 Input Types}, basicstyle=\ttfamily\small, frame=single]
<!DOCTYPE html>
<html lang="id">
<head>
  <meta charset="UTF-8">
  <title>Advanced HTML5 Form</title>
  <style>
    body { font-family: Arial, sans-serif; max-width: 800px; margin: 0 auto; padding: 20px; }
    .form-group { margin-bottom: 20px; }
    label { display: block; margin-bottom: 8px; font-weight: bold; }
    input { width: 100%; padding: 12px; border: 2px solid #ddd; border-radius: 6px; box-sizing: border-box; }
    input:focus { border-color: #4CAF50; outline: none; }
    .input-group { display: grid; grid-template-columns: 1fr 1fr; gap: 15px; }
    .btn { background-color: #007bff; color: white; padding: 12px 24px; border: none; border-radius: 6px; cursor: pointer; }
  </style>
</head>
<body>
  <h1>Form HTML5 Lanjutan</h1>
  
  <form action="/process-advanced-form" method="post">
    <!-- Personal Information dengan Advanced Inputs -->
    <fieldset>
      <legend>Informasi Personal</legend>
      
      <div class="form-group">
        <label for="email">Email:</label>
        <input type="email" id="email" name="email" required placeholder="nama@email.com">
      </div>
      
      <div class="form-group">
        <label for="website">Website:</label>
        <input type="url" id="website" name="website" placeholder="https://www.example.com">
      </div>
      
      <div class="form-group">
        <label for="phone">Telepon:</label>
        <input type="tel" id="phone" name="phone" placeholder="0812-3456-789">
      </div>
    </fieldset>
    
    <!-- Numeric Controls -->
    <fieldset>
      <legend>Kontrol Numerik</legend>
      
      <div class="input-group">
        <div>
          <label for="age">Umur:</label>
          <input type="number" id="age" name="age" min="17" max="100" value="25">
        </div>
        
        <div>
          <label for="rating">Rating (1-10):</label>
          <input type="range" id="rating" name="rating" min="1" max="10" value="7">
        </div>
      </div>
    </fieldset>
    
    <!-- Date/Time Inputs -->
    <fieldset>
      <legend>Informasi Waktu</legend>
      
      <div class="form-group">
        <label for="birthdate">Tanggal Lahir:</label>
        <input type="date" id="birthdate" name="birthdate" min="1900-01-01" max="2024-12-31">
      </div>
      
      <div class="form-group">
        <label for="appointment">Waktu Janji Temu:</label>
        <input type="time" id="appointment" name="appointment" min="09:00" max="17:00">
      </div>
    </fieldset>
    
    <!-- Color Picker -->
    <fieldset>
      <legend>Preferensi Warna</legend>
      
      <div class="form-group">
        <label for="themeColor">Warna Tema:</label>
        <input type="color" id="themeColor" name="themeColor" value="#4CAF50">
      </div>
    </fieldset>
    
    <!-- File Upload -->
    <fieldset>
      <legend>Upload Dokumen</legend>
      
      <div class="form-group">
        <label for="avatar">Avatar:</label>
        <input type="file" id="avatar" name="avatar" accept="image/*">
      </div>
      
      <div class="form-group">
        <label for="documents">Dokumen:</label>
        <input type="file" id="documents" name="documents[]" accept=".pdf,.doc,.docx" multiple>
      </div>
    </fieldset>
    
    <div class="form-group">
      <button type="submit" class="btn">Simpan Data</button>
    </div>
  </form>
</body>
</html>
\end{lstlisting}

\textbf{Hasil di Browser:}
- Email dan URL input dengan validasi format otomatis
- Numeric controls dengan range slider dan number spinner
- Date/time inputs dengan native calendar picker
- Color picker untuk pemilihan warna
- File upload dengan multiple selection

\section{Text Formatting dan Typography}

HTML menyediakan berbagai tag untuk memformat teks dan mengatur tampilan typography. Tag-tag ini memungkinkan penekanan, gaya, dan struktur teks yang lebih terorganisir \cite{w3schools-html}. Meskipun sebagian besar formatting sekarang menggunakan CSS, tag HTML formatting tetap penting untuk dipahami sebagai fondasi web development \cite{mdn-html}.

\subsection{Text Formatting Tags}

Tag formatting digunakan untuk memberikan penekanan atau gaya tertentu pada teks:

\begin{itemize}
  \item \texttt{<b>} - Bold text (teks tebal)
  \item \texttt{<strong>} - Strong importance (semantik, teks tebal)
  \item \texttt{<i>} - Italic text (teks miring)
  \item \texttt{<em>} - Emphasis (semantik, teks miring)
  \item \texttt{<u>} - Underlined text (teks bergaris bawah)
  \item \texttt{<s>} - Strikethrough text (teks dicoret)
  \item \texttt{<mark>} - Marked/highlighted text (teks disorot)
\end{itemize}

\begin{lstlisting}[caption={Contoh Text Formatting}, basicstyle=\ttfamily\small, frame=single]
<!DOCTYPE html>
<html>
<head>
  <title>Contoh Text Formatting</title>
</head>
<body>
  <h1>Demonstrasi Text Formatting</h1>
  
  <p>Ini adalah teks <b>bold</b> menggunakan tag b.</p>
  <p>Ini adalah teks <strong>strong</strong> menggunakan tag strong.</p>
  
  <p>Ini adalah teks <i>italic</i> menggunakan tag i.</p>
  <p>Ini adalah teks <em>emphasized</em> menggunakan tag em.</p>
  
  <p>Ini adalah teks <u>underlined</u> menggunakan tag u.</p>
  <p>Ini adalah teks <s>strikethrough</s> menggunakan tag s.</p>
  
  <p>Ini adalah teks <mark>highlighted</mark> menggunakan tag mark.</p>
  
  <p><b>Kombinasi:</b> <i><u>Teks miring dan bergaris bawah</u></i></p>
  <p><strong><mark>Teks penting dan disorot</mark></strong></p>
</body>
</html>
\end{lstlisting}

\textbf{Hasil di Browser:}
- `<b>` dan `<strong>` menampilkan teks tebal (visually sama)
- `<i>` dan `<em>` menampilkan teks miring (visually sama)
- `<u>` menambahkan garis bawah pada teks
- `<s>` menambahkan garis coret pada teks
- `<mark>` menyorot teks dengan background kuning
- **Best Practice**: Gunakan `<strong>` dan `<em>` untuk semantic meaning, bukan `<b>` dan `<i>`

\subsection{Font Tags dan Typography}

HTML menyediakan tag untuk mengatur font dan ukuran teks:

\begin{itemize}
  \item \texttt{<font>} - Mengatur font family, size, dan color
  \item \texttt{<basefont>} - Font default untuk seluruh dokumen
  \item \texttt{<big>} - Membesarkan teks
  \item \texttt{<small>} - Mengecilkan teks
  \item \texttt{<sup>} - Superscript (teks di atas garis)
  \item \texttt{<sub>} - Subscript (teks di bawah garis)
\end{itemize}

\begin{lstlisting}[caption={Contoh Font dan Typography}, basicstyle=\ttfamily\small, frame=single]
<!DOCTYPE html>
<html>
<head>
  <title>Contoh Font Tags</title>
</head>
<body>
  <h1>Demonstrasi Font dan Typography</h1>
  
  <p><font face="Arial" size="5" color="blue">Teks dengan Arial, size 5, warna biru</font></p>
  <p><font face="Times New Roman" size="4" color="red">Teks dengan Times New Roman, size 4, warna merah</font></p>
  <p><font face="Courier New" size="3" color="green">Teks dengan Courier New, size 3, warna hijau</font></p>
  
  <p>Teks normal <big>lebih besar</big> dan <small>lebih kecil</small>.</p>
  
  <p>Rumus kimia: H<sub>2</sub>O (air)</p>
  <p>Pangkat: 10<sup>2</sup> = 100</p>
  <p>Tanggal: 25<sup>th</sup> Desember 2024</p>
  
  <p><font face="Verdana" size="6" color="purple">
    <b><u>Teks kombinasi:</u></b><br>
    Font Verdana, size 6, warna ungu,<br>
    <i>miring</i> dan <s>dicoret</s>
  </font></p>
</body>
</html>
\end{lstlisting}

\textbf{Hasil di Browser:}
- Tag `<font>` dengan atribut `face`, `size`, `color` mengubah tampilan teks
- `<big>` menampilkan teks lebih besar dari normal
- `<small>` menampilkan teks lebih kecil dari normal
- `<sub>` menampilkan teks subscript (bawah)
- `<sup>` menampilkan teks superscript (atas)
- **Catatan**: Tag `<font>` deprecated di HTML5, gunakan CSS sebagai gantinya

\subsection{Preformatted Text dan Code}

Tag untuk menampilkan teks dengan format asli (tanpa formatting otomatis):

\begin{itemize}
  \item \texttt{<pre>} - Preformatted text (mempertahankan spasi dan line break)
  \item \texttt{<code>} - Code snippet (monospace font)
  \item \texttt{<kbd>} - Keyboard input (teks yang diketik user)
  \item \texttt{<samp>} - Sample output (output program)
  \item \texttt{<var>} - Variable (nama variabel)
\end{itemize}

\begin{lstlisting}[caption={Contoh Preformatted Text dan Code}, basicstyle=\ttfamily\small, frame=single]
<!DOCTYPE html>
<html>
<head>
  <title>Contoh Preformatted Text</title>
</head>
<body>
  <h1>Preformatted Text dan Code</h1>
  
  <h3>Normal Paragraph:</h3>
  <p>    Spasi di awal
    dan indentasi
    akan diabaikan
    oleh browser</p>
  
  <h3>Preformatted Text:</h3>
  <pre>    Spasi di awal
    dan indentasi
    akan dipertahankan
    sesuai aslinya</pre>
  
  <h3>Code Examples:</h3>
  <p>Gunakan fungsi <code>console.log()</code> untuk debugging JavaScript.</p>
  
  <p>Tekan tombol <kbd>Ctrl</kbd> + <kbd>C</kbd> untuk copy.</p>
  
  <p>Output program: <samp>Hello, World!</samp></p>
  
  <p>Nilai variabel <var>x</var> adalah 10.</p>
  
  <pre><code>
function calculateArea(radius) {
    const pi = 3.14159;
    return pi * radius * radius;
}
  </code></pre>
</body>
</html>
\end{lstlisting}

\textbf{Hasil di Browser:}
- `<pre>` menampilkan teks dengan spasi dan indentasi asli
- `<code>` menampilkan teks dengan font monospace
- `<kbd>` menampilkan teks dengan border (seperti tombol keyboard)
- `<samp>` menampilkan output program dengan font monospace
- `<var>` menampilkan nama variabel dengan italic
- Kombinasi `<pre><code>` ideal untuk menampilkan source code

\subsection{Quotations dan Citations}

Tag untuk menampilkan kutipan dan referensi:

\begin{itemize}
  \item \texttt{<blockquote>} - Block quotation (kutipan panjang)
  \item \texttt{<q>} - Inline quotation (kutipan pendek)
  \item \texttt{<cite>} - Citation (sumber kutipan)
  \item \texttt{<abbr>} - Abbreviation (singkatan)
  \item \texttt{<dfn>} - Definition (definisi istilah)
\end{itemize}

\begin{lstlisting}[caption={Contoh Quotations dan Citations}, basicstyle=\ttfamily\small, frame=single]
<!DOCTYPE html>
<html>
<head>
  <title>Contoh Quotations</title>
</head>
<body>
  <h1>Quotations dan Citations</h1>
  
  <h3>Block Quotation:</h3>
  <blockquote>
    "The best way to predict the future is to invent it."
    <br>
    <cite>- Alan Kay</cite>
  </blockquote>
  
  <h3>Inline Quotation:</h3>
  <p>Steve Jobs pernah berkata, <q>Stay hungry, stay foolish.</q> dalam pidato terkenalnya.</p>
  
  <h3>Abbreviations:</h3>
  <p><abbr title="World Wide Web">WWW</abbr> adalah singkatan dari World Wide Web.</p>
  <p><abbr title="HyperText Markup Language">HTML</abbr> adalah bahasa markup untuk web.</p>
  
  <h3>Definitions:</h3>
  <p><dfn>HTML</dfn> adalah singkatan dari HyperText Markup Language, 
  yaitu bahasa standar untuk membuat halaman web.</p>
  
  <h3>Kombinasi:</h3>
  <blockquote>
    <q>Learning HTML is the first step to become a web developer.</q>
    <br>
    <cite>- Web Development Expert</cite>
    <br>
    <small><dfn>HTML</dfn> adalah fondasi dari semua website.</small>
  </blockquote>
</body>
</html>
\end{lstlisting}

\textbf{Hasil di Browser:}
- `<blockquote>` menampilkan kutipan dengan indentasi (biasanya italic)
- `<q>` menampilkan kutipan dengan tanda kutip otomatis
- `<abbr>` menampilkan singkatan dengan tooltip saat hover
- `<dfn>` menampilkan definisi dengan italic
- `<cite>` menampilkan sumber dengan italic
- Browser otomatis menambahkan tanda kutip pada `<q>` dan indentasi pada `<blockquote>`

\subsection{Contoh Lengkap: Artikel dengan Typography}

Berikut contoh artikel blog yang menggunakan berbagai text formatting tags:

\begin{lstlisting}[caption={Artikel Blog Lengkap dengan Text Formatting}, basicstyle=\ttfamily\small, frame=single]
<!DOCTYPE html>
<html>
<head>
  <title>Tutorial HTML untuk Pemula</title>
</head>
<body>
  <!-- Header -->
  <h1><u>Tutorial HTML Lengkap</u></h1>
  <p><strong>Published:</strong> <em>25 Desember 2024</em> | <strong>Author:</strong> <cite>Web Developer Team</cite></p>
  <hr>
  
  <!-- Introduction -->
  <h2>Pengenalan HTML</h2>
  <p>HTML atau <abbr title="HyperText Markup Language">HTML</abbr> adalah <mark>bahasa fundamental</mark> untuk membuat halaman web. 
  Setiap <dfn>developer web</dfn> harus menguasai HTML sebelum belajar teknologi lain.</p>
  
  <!-- Why Learn HTML -->
  <h2>Mengapa Harus Belajar HTML?</h2>
  <blockquote>
    "HTML is the foundation of web development. Without HTML, there would be no web pages as we know them today."
    <br>
    <cite>- MDN Web Docs</cite>
  </blockquote>
  
  <h3>Alasan Utama:</h3>
  <ol>
    <li><b>Universality:</b> HTML bekerja di semua browser</li>
    <li><b>Simplicity:</b> Mudah dipelajari pemula</li>
    <li><b>Flexibility:</b> Dapat dikombinasi dengan CSS dan JavaScript</li>
  </ol>
  
  <!-- Code Example -->
  <h2>Contoh Kode Dasar</h2>
  <p>Berikut struktur <big>HTML paling sederhana</big>:</p>
  
  <pre><code>
&lt;!DOCTYPE html&gt;
&lt;html&gt;
&lt;head&gt;
  &lt;title&gt;Halaman Pertama&lt;/title&gt;
&lt;/head&gt;
&lt;body&gt;
  &lt;h1&gt;Hello, World!&lt;/h1&gt;
&lt;/body&gt;
&lt;/html&gt;
  </code></pre>
  
  <p><small><em>Catatan:</em> Gunakan <kbd>Save</kbd> untuk menyimpan file dengan ekstensi <samp>.html</samp></small></p>
  
  <!-- Mathematical Example -->
  <h2>Contoh Matematika</h2>
  <p>Rumus luas lingkaran: L = &pi; &times; r<sup>2</sup></p>
  <p>Dimana &pi; &asymp; 3.14159 dan r adalah jari-jari.</p>
  <p>Contoh: Jika r = 5cm, maka L = 3.14159 &times; 5<sup>2</sup> = 78.54 cm<sup>2</sup></p>
  
  <!-- Conclusion -->
  <h2>Kesimpulan</h2>
  <p>HTML adalah <mark>skill esensial</mark> yang harus dikuasai. 
  Dengan <strong>pemahaman yang baik</strong> tentang text formatting, 
  Anda dapat membuat <i>konten yang menarik dan terstruktur</i>.</p>
  
  <p><s>Jangan takut untuk bereksperimen!</s> <u>Terus belajar dan praktik!</u></p>
</body>
</html>
\end{lstlisting}

\textbf{Hasil di Browser:}
- Header dengan underline dan informasi publikasi
- Kutipan dengan indentasi dan sumber
- Singkatan dengan tooltip saat hover
- Code dalam preformatted block dengan monospace font
- Rumus matematika dengan superscript
- Kombinasi formatting yang beragam untuk artikel yang menarik

Pemahaman text formatting tags ini memungkinkan penciptaan konten yang kaya dan terstruktur tanpa bergantung pada CSS \cite{w3schools-html}. Namun, untuk proyek modern, disarankan menggunakan CSS untuk styling dan mempertahankan HTML untuk struktur semantik \cite{mdn-html}.

\section{Storage dan APIs HTML5}

HTML5 memperkenalkan berbagai APIs untuk menyimpan data di client-side dan mengakses fitur device. Web Storage API menyediakan cara yang lebih efisien dan aman untuk menyimpan data dibandingkan cookies \cite{mdn-web-storage}. IndexedDB dan File API memungkinkan penyimpanan data yang lebih kompleks dan struktur \cite{w3schools-html}.

\subsection{LocalStorage dan SessionStorage}

Web Storage API menyediakan dua mekanisme penyimpanan key-value di browser:

\begin{itemize}
  \item \textbf{localStorage}: Data persisten, tidak hilang saat browser ditutup
  \item \textbf{sessionStorage}: Data hanya untuk satu session, hilang saat tab ditutup
  \item \textbf{Storage Limit}: Biasanya 5-10MB per domain
  \item \textbf{Synchronous API}: Operasi storage bersifat synchronous
\end{itemize}

\begin{lstlisting}[caption={Web Storage API}, basicstyle=\ttfamily\small, frame=single]
<!DOCTYPE html>
<html lang="id">
<head>
  <meta charset="UTF-8">
  <meta name="viewport" content="width=device-width, initial-scale=1.0">
  <title>Web Storage API</title>
  <style>
    body { font-family: Arial, sans-serif; line-height: 1.6; max-width: 800px; margin: 0 auto; padding: 20px; background-color: #f8f9fa; }
    .storage-demo { background-color: white; border-radius: 8px; padding: 20px; margin-bottom: 20px; box-shadow: 0 2px 8px rgba(0,0,0,0.1); }
    .form-group { margin-bottom: 15px; }
    label { display: block; margin-bottom: 5px; font-weight: bold; }
    input { width: 100%; padding: 8px; border: 1px solid #ddd; border-radius: 4px; box-sizing: border-box; }
    .btn { background-color: #007bff; color: white; border: none; padding: 10px 20px; border-radius: 4px; cursor: pointer; margin-right: 10px; }
    .btn:hover { background-color: #0056b3; }
    .storage-info { background-color: #e7f3ff; border: 1px solid #b3d9ff; border-radius: 4px; padding: 15px; margin-top: 20px; }
    .data-display { background-color: #f8f9fa; border: 1px solid #e9ecef; border-radius: 4px; padding: 15px; margin-top: 15px; max-height: 200px; overflow-y: auto; }
  </style>
</head>
<body>
  <h1>Web Storage API Demo</h1>
  
  <!-- LocalStorage Demo -->
  <div class="storage-demo">
    <h2>💾 LocalStorage (Persistent)</h2>
    
    <div class="form-group">
      <label for="localKey">Key:</label>
      <input type="text" id="localKey" placeholder="Masukkan key (contoh: username)">
    </div>
    
    <div class="form-group">
      <label for="localValue">Value:</label>
      <input type="text" id="localValue" placeholder="Masukkan value (contoh: JohnDoe)">
    </div>
    
    <div>
      <button class="btn" onclick="saveToLocal()">Simpan ke LocalStorage</button>
      <button class="btn" onclick="loadFromLocal()">Load dari LocalStorage</button>
      <button class="btn" onclick="clearLocal()">Clear LocalStorage</button>
    </div>
    
    <div class="storage-info">
      <h4>Data LocalStorage:</h4>
      <div id="localData" class="data-display">Belum ada data</div>
    </div>
  </div>
  
  <!-- SessionStorage Demo -->
  <div class="storage-demo">
    <h2>⏱️ SessionStorage (Session Only)</h2>
    
    <div class="form-group">
      <label for="sessionKey">Key:</label>
      <input type="text" id="sessionKey" placeholder="Masukkan key">
    </div>
    
    <div class="form-group">
      <label for="sessionValue">Value:</label>
      <input type="text" id="sessionValue" placeholder="Masukkan value">
    </div>
    
    <div>
      <button class="btn" onclick="saveToSession()">Simpan ke SessionStorage</button>
      <button class="btn" onclick="loadFromSession()">Load dari SessionStorage</button>
      <button class="btn" onclick="clearSession()">Clear SessionStorage</button>
    </div>
    
    <div class="storage-info">
      <h4>Data SessionStorage:</h4>
      <div id="sessionData" class="data-display">Belum ada data</div>
    </div>
  </div>
  
  <!-- Storage Info -->
  <div class="storage-demo">
    <h2>📊 Storage Information</h2>
    <div class="storage-info">
      <p><strong>LocalStorage Size:</strong> <span id="localSize">0</span> items</p>
      <p><strong>SessionStorage Size:</strong> <span id="sessionSize">0</span> items</p>
      <p><strong>Remaining Space:</strong> ~5-10MB (browser dependent)</p>
    </div>
    <button class="btn" onclick="updateStorageInfo()">Update Info</button>
  </div>
  
  <script>
    // LocalStorage Functions
    function saveToLocal() {
      const key = document.getElementById('localKey').value;
      const value = document.getElementById('localValue').value;
      
      if (key && value) {
        localStorage.setItem(key, value);
        displayLocalData();
        updateStorageInfo();
        alert('Data disimpan ke LocalStorage!');
      } else {
        alert('Key dan value harus diisi!');
      }
    }
    
    function loadFromLocal() {
      const key = document.getElementById('localKey').value;
      const value = localStorage.getItem(key);
      
      if (value) {
        document.getElementById('localValue').value = value;
        alert('Data ditemukan: ' + value);
      } else {
        alert('Data tidak ditemukan!');
      }
    }
    
    function clearLocal() {
      localStorage.clear();
      displayLocalData();
      updateStorageInfo();
      alert('LocalStorage dibersihkan!');
    }
    
    function displayLocalData() {
      const dataDiv = document.getElementById('localData');
      let html = '';
      
      for (let i = 0; i < localStorage.length; i++) {
        const key = localStorage.key(i);
        const value = localStorage.getItem(key);
        html += `<p><strong>${key}:</strong> ${value}</p>`;
      }
      
      dataDiv.innerHTML = html || 'Belum ada data';
    }
    
    // SessionStorage Functions
    function saveToSession() {
      const key = document.getElementById('sessionKey').value;
      const value = document.getElementById('sessionValue').value;
      
      if (key && value) {
        sessionStorage.setItem(key, value);
        displaySessionData();
        updateStorageInfo();
        alert('Data disimpan ke SessionStorage!');
      } else {
        alert('Key dan value harus diisi!');
      }
    }
    
    function loadFromSession() {
      const key = document.getElementById('sessionKey').value;
      const value = sessionStorage.getItem(key);
      
      if (value) {
        document.getElementById('sessionValue').value = value;
        alert('Data ditemukan: ' + value);
      } else {
        alert('Data tidak ditemukan!');
      }
    }
    
    function clearSession() {
      sessionStorage.clear();
      displaySessionData();
      updateStorageInfo();
      alert('SessionStorage dibersihkan!');
    }
    
    function displaySessionData() {
      const dataDiv = document.getElementById('sessionData');
      let html = '';
      
      for (let i = 0; i < sessionStorage.length; i++) {
        const key = sessionStorage.key(i);
        const value = sessionStorage.getItem(key);
        html += `<p><strong>${key}:</strong> ${value}</p>`;
      }
      
      dataDiv.innerHTML = html || 'Belum ada data';
    }
    
    function updateStorageInfo() {
      document.getElementById('localSize').textContent = localStorage.length;
      document.getElementById('sessionSize').textContent = sessionStorage.length;
    }
    
    // Initialize
    displayLocalData();
    displaySessionData();
    updateStorageInfo();
  </script>
</body>
</html>
\end{lstlisting}

\textbf{Hasil di Browser:}
- Form untuk menyimpan key-value pairs ke LocalStorage dan SessionStorage
- Display data yang tersimpan secara real-time
- Tombol untuk save, load, dan clear storage
- Storage information showing jumlah items
- Demo interaktif dengan JavaScript API

\subsection{IndexedDB}

IndexedDB adalah database NoSQL di browser untuk penyimpanan data struktur yang kompleks:

\begin{itemize}
  \item \textbf{Object Store}: Penyimpanan data seperti tabel dalam database
  \item \textbf{Index}: Untuk pencarian data yang efisien
  \item \textbf{Transactions}: Operasi database dalam transaksi
  \item \textbf{Async API}: Operasi bersifat asynchronous dengan promises
  \item \textbf{Large Storage}: Dapat menyimpan data dalam jumlah besar (hundreds of MB)
\end{itemize}

\begin{lstlisting}[caption={IndexedDB Implementation}, basicstyle=\ttfamily\small, frame=single]
<!DOCTYPE html>
<html lang="id">
<head>
  <meta charset="UTF-8">
  <meta name="viewport" content="width=device-width, initial-scale=1.0">
  <title>IndexedDB Demo</title>
  <style>
    body { font-family: Arial, sans-serif; line-height: 1.6; max-width: 900px; margin: 0 auto; padding: 20px; background-color: #f8f9fa; }
    .db-demo { background-color: white; border-radius: 8px; padding: 20px; margin-bottom: 20px; box-shadow: 0 2px 8px rgba(0,0,0,0.1); }
    .form-group { margin-bottom: 15px; }
    label { display: block; margin-bottom: 5px; font-weight: bold; }
    input { width: 100%; padding: 8px; border: 1px solid #ddd; border-radius: 4px; box-sizing: border-box; }
    .btn { background-color: #007bff; color: white; border: none; padding: 10px 20px; border-radius: 4px; cursor: pointer; margin-right: 10px; }
    .btn:hover { background-color: #0056b3; }
    .data-table { width: 100%; border-collapse: collapse; margin-top: 15px; }
    .data-table th, .data-table td { border: 1px solid #ddd; padding: 10px; text-align: left; }
    .data-table th { background-color: #f2f2f2; }
    .status { padding: 10px; border-radius: 4px; margin-top: 10px; }
    .status.success { background-color: #d4edda; color: #155724; }
    .status.error { background-color: #f8d7da; color: #721c24; }
  </style>
</head>
<body>
  <h1>🗄️ IndexedDB Demo</h1>
  
  <div class="db-demo">
    <h2>Tambah Data Mahasiswa</h2>
    
    <div class="form-group">
      <label for="studentId">ID Mahasiswa:</label>
      <input type="number" id="studentId" placeholder="Contoh: 12345">
    </div>
    
    <div class="form-group">
      <label for="studentName">Nama:</label>
      <input type="text" id="studentName" placeholder="Contoh: John Doe">
    </div>
    
    <div class="form-group">
      <label for="studentEmail">Email:</label>
      <input type="email" id="studentEmail" placeholder="Contoh: john@email.com">
    </div>
    
    <div class="form-group">
      <label for="studentMajor">Jurusan:</label>
      <input type="text" id="studentMajor" placeholder="Contoh: Teknik Informatika">
    </div>
    
    <div>
      <button class="btn" onclick="addStudent()">Tambah Mahasiswa</button>
      <button class="btn" onclick="loadAllStudents()">Load Semua Data</button>
      <button class="btn" onclick="clearAllData()">Hapus Semua Data</button>
    </div>
    
    <div id="status" class="status" style="display: none;"></div>
  </div>
  
  <div class="db-demo">
    <h2>Data Mahasiswa</h2>
    <table class="data-table">
      <thead>
        <tr>
          <th>ID</th>
          <th>Nama</th>
          <th>Email</th>
          <th>Jurusan</th>
          <th>Aksi</th>
        </tr>
      </thead>
      <tbody id="studentsTable">
        <tr>
          <td colspan="5" style="text-align: center; color: #666;">Belum ada data</td>
        </tr>
      </tbody>
    </table>
  </div>
  
  <script>
    let db;
    const DB_NAME = 'StudentDB';
    const STORE_NAME = 'students';
    const DB_VERSION = 1;
    
    // Initialize IndexedDB
    function initDB() {
      return new Promise((resolve, reject) => {
        const request = indexedDB.open(DB_NAME, DB_VERSION);
        
        request.onerror = () => reject(request.error);
        request.onsuccess = () => {
          db = request.result;
          resolve(db);
        };
        
        request.onupgradeneeded = (event) => {
          const database = event.target.result;
          
          // Create object store dengan keyPath 'id'
          if (!database.objectStoreNames.contains(STORE_NAME)) {
            const store = database.createObjectStore(STORE_NAME, { keyPath: 'id' });
            
            // Create indexes untuk pencarian
            store.createIndex('name', 'name', { unique: false });
            store.createIndex('email', 'email', { unique: true });
            store.createIndex('major', 'major', { unique: false });
          }
        };
      });
    }
    
    // Add student to IndexedDB
    async function addStudent() {
      try {
        const id = parseInt(document.getElementById('studentId').value);
        const name = document.getElementById('studentName').value;
        const email = document.getElementById('studentEmail').value;
        const major = document.getElementById('studentMajor').value;
        
        if (!id || !name || !email || !major) {
          showStatus('Semua field harus diisi!', 'error');
          return;
        }
        
        const student = { id, name, email, major, createdAt: new Date().toISOString() };
        
        const transaction = db.transaction([STORE_NAME], 'readwrite');
        const store = transaction.objectStore(STORE_NAME);
        
        await new Promise((resolve, reject) => {
          const request = store.add(student);
          request.onsuccess = () => resolve();
          request.onerror = () => reject(request.error);
        });
        
        showStatus('Mahasiswa berhasil ditambahkan!', 'success');
        clearForm();
        loadAllStudents();
        
      } catch (error) {
        showStatus('Error: ' + error.message, 'error');
      }
    }
    
    // Load all students from IndexedDB
    async function loadAllStudents() {
      try {
        const transaction = db.transaction([STORE_NAME], 'readonly');
        const store = transaction.objectStore(STORE_NAME);
        
        const students = await new Promise((resolve, reject) => {
          const request = store.getAll();
          request.onsuccess = () => resolve(request.result);
          request.onerror = () => reject(request.error);
        });
        
        displayStudents(students);
        
      } catch (error) {
        showStatus('Error loading data: ' + error.message, 'error');
      }
    }
    
    // Display students in table
    function displayStudents(students) {
      const tbody = document.getElementById('studentsTable');
      
      if (students.length === 0) {
        tbody.innerHTML = '<tr><td colspan="5" style="text-align: center; color: #666;">Belum ada data</td></tr>';
        return;
      }
      
      tbody.innerHTML = students.map(student => `
        <tr>
          <td>${student.id}</td>
          <td>${student.name}</td>
          <td>${student.email}</td>
          <td>${student.major}</td>
          <td>
            <button class="btn" style="padding: 5px 10px; font-size: 12px;" 
                    onclick="deleteStudent(${student.id})">Hapus</button>
          </td>
        </tr>
      `).join('');
    }
    
    // Delete student from IndexedDB
    async function deleteStudent(id) {
      try {
        const transaction = db.transaction([STORE_NAME], 'readwrite');
        const store = transaction.objectStore(STORE_NAME);
        
        await new Promise((resolve, reject) => {
          const request = store.delete(id);
          request.onsuccess = () => resolve();
          request.onerror = () => reject(request.error);
        });
        
        showStatus('Mahasiswa berhasil dihapus!', 'success');
        loadAllStudents();
        
      } catch (error) {
        showStatus('Error deleting student: ' + error.message, 'error');
      }
    }
    
    // Clear all data from IndexedDB
    async function clearAllData() {
      try {
        const transaction = db.transaction([STORE_NAME], 'readwrite');
        const store = transaction.objectStore(STORE_NAME);
        
        await new Promise((resolve, reject) => {
          const request = store.clear();
          request.onsuccess = () => resolve();
          request.onerror = () => reject(request.error);
        });
        
        showStatus('Semua data berhasil dihapus!', 'success');
        loadAllStudents();
        
      } catch (error) {
        showStatus('Error clearing data: ' + error.message, 'error');
      }
    }
    
    // Clear form
    function clearForm() {
      document.getElementById('studentId').value = '';
      document.getElementById('studentName').value = '';
      document.getElementById('studentEmail').value = '';
      document.getElementById('studentMajor').value = '';
    }
    
    // Show status message
    function showStatus(message, type) {
      const statusDiv = document.getElementById('status');
      statusDiv.textContent = message;
      statusDiv.className = 'status ' + type;
      statusDiv.style.display = 'block';
      
      setTimeout(() => {
        statusDiv.style.display = 'none';
      }, 3000);
    }
    
    // Initialize database when page loads
    initDB().then(() => {
      console.log('IndexedDB initialized successfully');
      loadAllStudents();
    }).catch(error => {
      console.error('Error initializing IndexedDB:', error);
      showStatus('Error initializing database: ' + error.message, 'error');
    });
  </script>
</body>
</html>
\end{lstlisting}

\textbf{Hasil di Browser:}
- Form untuk menambah data mahasiswa dengan ID, nama, email, dan jurusan
- Data disimpan dalam IndexedDB dengan object store dan indexes
- Tabel menampilkan semua data mahasiswa yang tersimpan
- Fitur CRUD (Create, Read, Update, Delete) operations
- Status notifications untuk feedback user

\subsection{File API}

File API memungkinkan web applications untuk berinteraksi dengan file di device user:

\begin{itemize}
  \item \textbf{File Input}: Mengakses file melalui file input atau drag-and-drop
  \item \textbf{FileReader}: Membaca konten file (text, data URL, binary)
  \item \textbf{FileWriter}: Menulis file (dengan File System Access API)
  \item \textbf{Blob}: Mengolah binary large objects
  \item \textbf{FileList}: Koleksi file dari input atau drag-and-drop
\end{itemize}

\begin{lstlisting}[caption={File API Implementation}, basicstyle=\ttfamily\small, frame=single]
<!DOCTYPE html>
<html lang="id">
<head>
  <meta charset="UTF-8">
  <meta name="viewport" content="width=device-width, initial-scale=1.0">
  <title>File API Demo</title>
  <style>
    body { font-family: Arial, sans-serif; line-height: 1.6; max-width: 900px; margin: 0 auto; padding: 20px; background-color: #f8f9fa; }
    .file-demo { background-color: white; border-radius: 8px; padding: 20px; margin-bottom: 20px; box-shadow: 0 2px 8px rgba(0,0,0,0.1); }
    .drop-zone { border: 2px dashed #ccc; border-radius: 8px; padding: 40px; text-align: center; transition: border-color 0.3s; }
    .drop-zone:hover, .drop-zone.dragover { border-color: #007bff; background-color: #f0f8ff; }
    .file-info { background-color: #e7f3ff; border: 1px solid #b3d9ff; border-radius: 4px; padding: 15px; margin-top: 15px; }
    .preview { margin-top: 15px; max-width: 100%; max-height: 300px; }
    .btn { background-color: #007bff; color: white; border: none; padding: 10px 20px; border-radius: 4px; cursor: pointer; margin-right: 10px; }
    .btn:hover { background-color: #0056b3; }
    .progress-bar { width: 100%; height: 20px; background-color: #e9ecef; border-radius: 10px; overflow: hidden; margin-top: 10px; }
    .progress-fill { height: 100%; background-color: #28a745; width: 0; transition: width 0.3s; }
  </style>
</head>
<body>
  <h1>📁 File API Demo</h1>
  
  <!-- File Input Demo -->
  <div class="file-demo">
    <h2>Upload File dengan File Input</h2>
    
    <input type="file" id="fileInput" accept=".txt,.jpg,.png,.pdf" multiple>
    <button class="btn" onclick="processFiles()">Proses Files</button>
    <button class="btn" onclick="clearFiles()">Clear</button>
    
    <div id="fileInfo" class="file-info" style="display: none;">
      <h4>File Information:</h4>
      <div id="fileDetails"></div>
    </div>
  </div>
  
  <!-- Drag and Drop Demo -->
  <div class="file-demo">
    <h2>Drag and Drop Zone</h2>
    
    <div id="dropZone" class="drop-zone">
      <p>📂 Drop files here atau klik untuk browse</p>
      <input type="file" id="dropInput" style="display: none;" multiple>
    </div>
    
    <div id="dropFileInfo" class="file-info" style="display: none;">
      <h4>Dropped Files:</h4>
      <div id="dropFileDetails"></div>
    </div>
  </div>
  
  <!-- File Preview Demo -->
  <div class="file-demo">
    <h2>File Preview</h2>
    
    <input type="file" id="previewInput" accept="image/*" onchange="previewFile()">
    
    <div id="previewContainer" style="display: none;">
      <h4>Preview:</h4>
      <img id="filePreview" class="preview" alt="File preview">
      <div id="imageInfo" class="file-info" style="margin-top: 10px;"></div>
    </div>
  </div>
  
  <!-- File Reader Demo -->
  <div class="file-demo">
    <h2>File Reader (Text Files)</h2>
    
    <input type="file" id="textInput" accept=".txt,.csv,.json,.js,.html,.css">
    <button class="btn" onclick="readTextFile()">Read File</button>
    
    <div id="textContent" class="file-info" style="display: none; margin-top: 15px;">
      <h4>File Content:</h4>
      <pre id="fileContent" style="background-color: #f8f9fa; padding: 15px; border-radius: 4px; overflow-x: auto;"></pre>
    </div>
  </div>
  
  <script>
    // Process files from file input
    function processFiles() {
      const input = document.getElementById('fileInput');
      const files = input.files;
      
      if (files.length === 0) {
        alert('Pilih file terlebih dahulu!');
        return;
      }
      
      let fileDetails = '';
      
      for (let i = 0; i < files.length; i++) {
        const file = files[i];
        fileDetails += `
          <div style="border: 1px solid #ddd; padding: 10px; margin-bottom: 10px; border-radius: 4px;">
            <strong>Nama:</strong> ${file.name}<br>
            <strong>Tipe:</strong> ${file.type || 'Unknown'}<br>
            <strong>Ukuran:</strong> ${formatFileSize(file.size)}<br>
            <strong>Terakhir diubah:</strong> ${new Date(file.lastModified).toLocaleString()}
          </div>
        `;
      }
      
      document.getElementById('fileDetails').innerHTML = fileDetails;
      document.getElementById('fileInfo').style.display = 'block';
    }
    
    function clearFiles() {
      document.getElementById('fileInput').value = '';
      document.getElementById('fileInfo').style.display = 'none';
      document.getElementById('fileDetails').innerHTML = '';
    }
    
    // Format file size
    function formatFileSize(bytes) {
      if (bytes === 0) return '0 Bytes';
      const k = 1024;
      const sizes = ['Bytes', 'KB', 'MB', 'GB'];
      const i = Math.floor(Math.log(bytes) / Math.log(k));
      return parseFloat((bytes / Math.pow(k, i)).toFixed(2)) + ' ' + sizes[i];
    }
    
    // Drag and Drop functionality
    const dropZone = document.getElementById('dropZone');
    const dropInput = document.getElementById('dropInput');
    
    dropZone.addEventListener('click', () => dropInput.click());
    dropZone.addEventListener('dragover', (e) => {
      e.preventDefault();
      dropZone.classList.add('dragover');
    });
    dropZone.addEventListener('dragleave', () => {
      dropZone.classList.remove('dragover');
    });
    dropZone.addEventListener('drop', (e) => {
      e.preventDefault();
      dropZone.classList.remove('dragover');
      handleDroppedFiles(e.dataTransfer.files);
    });
    dropInput.addEventListener('change', (e) => {
      handleDroppedFiles(e.target.files);
    });
    
    function handleDroppedFiles(files) {
      if (files.length === 0) return;
      
      let fileDetails = '';
      
      for (let i = 0; i < files.length; i++) {
        const file = files[i];
        fileDetails += `
          <div style="border: 1px solid #ddd; padding: 10px; margin-bottom: 10px; border-radius: 4px;">
            <strong>Nama:</strong> ${file.name}<br>
            <strong>Tipe:</strong> ${file.type || 'Unknown'}<br>
            <strong>Ukuran:</strong> ${formatFileSize(file.size)}
          </div>
        `;
      }
      
      document.getElementById('dropFileDetails').innerHTML = fileDetails;
      document.getElementById('dropFileInfo').style.display = 'block';
    }
    
    // Preview image file
    function previewFile() {
      const input = document.getElementById('previewInput');
      const file = input.files[0];
      
      if (!file) return;
      
      if (!file.type.startsWith('image/')) {
        alert('File harus berupa gambar!');
        return;
      }
      
      const reader = new FileReader();
      
      reader.onload = (e) => {
        const img = document.getElementById('filePreview');
        img.src = e.target.result;
        img.style.display = 'block';
        
        // Get image dimensions
        img.onload = () => {
          document.getElementById('imageInfo').innerHTML = `
            <strong>Dimensi:</strong> ${img.naturalWidth} x ${img.naturalHeight}px<br>
            <strong>Ukuran File:</strong> ${formatFileSize(file.size)}<br>
            <strong>Tipe:</strong> ${file.type}
          `;
        };
        
        document.getElementById('previewContainer').style.display = 'block';
      };
      
      reader.readAsDataURL(file);
    }
    
    // Read text file content
    function readTextFile() {
      const input = document.getElementById('textInput');
      const file = input.files[0];
      
      if (!file) {
        alert('Pilih file teks terlebih dahulu!');
        return;
      }
      
      const reader = new FileReader();
      
      reader.onload = (e) => {
        const content = e.target.result;
        document.getElementById('fileContent').textContent = content;
        document.getElementById('textContent').style.display = 'block';
      };
      
      reader.onerror = () => {
        alert('Error reading file!');
      };
      
      reader.readAsText(file);
    }
  </script>
</body>
</html>
\end{lstlisting}

\textbf{Hasil di Browser:}
- File input dengan multiple file selection dan filtering
- Drag-and-drop zone untuk upload files
- File preview untuk image files dengan dimensions
- File reader untuk text files dengan content display
- File information showing name, type, size, dan last modified

Storage dan APIs HTML5 memungkinkan aplikasi web modern untuk menyimpan data secara lokal dan berinteraksi dengan file system \cite{w3schools-html}. Kombinasi LocalStorage, SessionStorage, IndexedDB, dan File API menyediakan fondasi kuat untuk aplikasi web yang rich dan offline-capable \cite{mdn-web-storage}.

\section{Geolocation dan Device APIs}

HTML5 menyediakan APIs untuk mengakses fitur device dan informasi lokasi pengguna. Geolocation API memungkinkan aplikasi web untuk mendapatkan lokasi fisik user, sementara Device APIs memberikan akses ke hardware seperti kamera, microphone, dan sensors \cite{mdn-geolocation}. APIs ini membuka kemungkinan untuk aplikasi web yang lebih interaktif dan personal \cite{w3schools-html}.

\subsection{Geolocation API}

Geolocation API menyediakan cara untuk mendapatkan lokasi geografis user:

\begin{itemize}
  \item \textbf{getCurrentPosition()}: Mendapatkan lokasi sekali
  \item \textbf{watchPosition()}: Monitoring lokasi secara continuous
  \item \textbf{Accuracy}: High accuracy dengan GPS atau low accuracy dengan network
  \item \textbf{Privacy}: User consent diperlukan untuk akses lokasi
  \item \textbf{Fallback}: IP-based geolocation untuk browser tanpa support
\end{itemize}

\begin{lstlisting}[caption={Geolocation API Implementation}, basicstyle=\ttfamily\small, frame=single]
<!DOCTYPE html>
<html lang="id">
<head>
  <meta charset="UTF-8">
  <meta name="viewport" content="width=device-width, initial-scale=1.0">
  <title>Geolocation API Demo</title>
  <style>
    body { font-family: Arial, sans-serif; line-height: 1.6; max-width: 800px; margin: 0 auto; padding: 20px; background-color: #f8f9fa; }
    .geo-demo { background-color: white; border-radius: 8px; padding: 20px; margin-bottom: 20px; box-shadow: 0 2px 8px rgba(0,0,0,0.1); }
    .btn { background-color: #007bff; color: white; border: none; padding: 10px 20px; border-radius: 4px; cursor: pointer; margin-right: 10px; }
    .btn:hover { background-color: #0056b3; }
    .btn:disabled { background-color: #6c757d; cursor: not-allowed; }
    .location-info { background-color: #e7f3ff; border: 1px solid #b3d9ff; border-radius: 4px; padding: 15px; margin-top: 15px; }
    .map-container { width: 100%; height: 300px; background-color: #e9ecef; border-radius: 8px; margin-top: 15px; display: flex; align-items: center; justify-content: center; }
    .coordinates { font-family: monospace; background-color: #f8f9fa; padding: 10px; border-radius: 4px; margin: 10px 0; }
    .accuracy-badge { display: inline-block; padding: 5px 10px; border-radius: 12px; font-size: 12px; font-weight: bold; }
    .accuracy-high { background-color: #d4edda; color: #155724; }
    .accuracy-medium { background-color: #fff3cd; color: #856404; }
    .accuracy-low { background-color: #f8d7da; color: #721c24; }
  </style>
</head>
<body>
  <h1>🌍 Geolocation API Demo</h1>
  
  <div class="geo-demo">
    <h2>Dapatkan Lokasi Saat Ini</h2>
    
    <div>
      <button class="btn" onclick="getCurrentLocation()">📍 Dapatkan Lokasi</button>
      <button class="btn" onclick="startWatching()">▶️ Mulai Tracking</button>
      <button class="btn" onclick="stopWatching()">⏹️ Stop Tracking</button>
      <button class="btn" onclick="clearLocation()">🗑️ Clear</button>
    </div>
    
    <div id="locationInfo" class="location-info" style="display: none;">
      <h4>Informasi Lokasi:</h4>
      <div class="coordinates">
        <strong>Latitude:</strong> <span id="latitude">-</span><br>
        <strong>Longitude:</strong> <span id="longitude">-</span><br>
        <strong>Altitude:</strong> <span id="altitude">-</span><br>
        <strong>Accuracy:</strong> <span id="accuracy">-</span> 
        <span id="accuracyBadge" class="accuracy-badge">-</span>
      </div>
      
      <div style="margin-top: 15px;">
        <strong>Kecepatan:</strong> <span id="speed">-</span><br>
        <strong>Heading:</strong> <span id="heading">-</span><br>
        <strong>Timestamp:</strong> <span id="timestamp">-</span>
      </div>
    </div>
    
    <div id="mapContainer" class="map-container" style="display: none;">
      <div>
        <h3>🗺️ Map Preview</h3>
        <p>Lokasi Anda ditampilkan pada koordinat di atas</p>
        <p><small>(Integrasi dengan Google Maps atau OpenStreetMap dapat ditambahkan)</small></p>
      </div>
    </div>
    
    <div id="trackingInfo" class="location-info" style="display: none; margin-top: 15px;">
      <h4>📊 Tracking Status:</h4>
      <p><strong>Status:</strong> <span id="trackingStatus">Stopped</span></p>
      <p><strong>Update Count:</strong> <span id="updateCount">0</span></p>
    </div>
  </div>
  
  <div class="geo-demo">
    <h2>🗺️ Reverse Geocoding (Simulasi)</h2>
    
    <button class="btn" onclick="getAddress()">🏠 Dapatkan Alamat</button>
    
    <div id="addressInfo" class="location-info" style="display: none; margin-top: 15px;">
      <h4>Alamat:</h4>
      <p id="addressText">-</p>
    </div>
  </div>
  
  <div class="geo-demo">
    <h2>⚠️ Privacy & Security</h2>
    
    <div class="location-info" style="background-color: #fff3cd; border-color: #ffeaa7;">
      <h4>🛡️ Catatan Penting:</h4>
      <ul>
        <li><strong>User Consent:</strong> Browser akan meminta izin user sebelum mengakses lokasi</li>
        <li><strong>HTTPS Required:</strong> Geolocation API hanya bekerja pada secure contexts (HTTPS)</li>
        <li><strong>Privacy:</strong> Lokasi user adalah data sensitif, gunakan dengan bertanggung jawab</li>
        <li><strong>Fallback:</strong> Selalu sediakan fallback untuk browser yang tidak support atau user menolak</li>
      </ul>
    </div>
  </div>
  
  <script>
    let watchId = null;
    let updateCount = 0;
    
    // Get current location
    function getCurrentLocation() {
      if (!navigator.geolocation) {
        alert('Browser Anda tidak mendukung Geolocation API');
        return;
      }
      
      const options = {
        enableHighAccuracy: true,
        timeout: 10000,
        maximumAge: 0
      };
      
      navigator.geolocation.getCurrentPosition(
        showPosition,
        showError,
        options
      );
    }
    
    // Show position information
    function showPosition(position) {
      const coords = position.coords;
      
      document.getElementById('latitude').textContent = coords.latitude.toFixed(6);
      document.getElementById('longitude').textContent = coords.longitude.toFixed(6);
      document.getElementById('altitude').textContent = coords.altitude ? coords.altitude.toFixed(2) + ' m' : 'Not available';
      document.getElementById('accuracy').textContent = coords.accuracy.toFixed(0) + ' m';
      document.getElementById('speed').textContent = coords.speed ? coords.speed.toFixed(2) + ' m/s' : 'Not available';
      document.getElementById('heading').textContent = coords.heading ? coords.heading.toFixed(0) + '°' : 'Not available';
      document.getElementById('timestamp').textContent = new Date(position.timestamp).toLocaleString();
      
      // Update accuracy badge
      const badge = document.getElementById('accuracyBadge');
      if (coords.accuracy < 20) {
        badge.className = 'accuracy-badge accuracy-high';
        badge.textContent = 'High Accuracy';
      } else if (coords.accuracy < 100) {
        badge.className = 'accuracy-badge accuracy-medium';
        badge.textContent = 'Medium Accuracy';
      } else {
        badge.className = 'accuracy-badge accuracy-low';
        badge.textContent = 'Low Accuracy';
      }
      
      document.getElementById('locationInfo').style.display = 'block';
      document.getElementById('mapContainer').style.display = 'flex';
      
      updateCount++;
      document.getElementById('updateCount').textContent = updateCount;
    }
    
    // Start watching position
    function startWatching() {
      if (!navigator.geolocation) {
        alert('Browser Anda tidak mendukung Geolocation API');
        return;
      }
      
      if (watchId) {
        alert('Tracking sudah berjalan!');
        return;
      }
      
      const options = {
        enableHighAccuracy: true,
        timeout: 10000,
        maximumAge: 0
      };
      
      watchId = navigator.geolocation.watchPosition(
        showPosition,
        showError,
        options
      );
      
      document.getElementById('trackingStatus').textContent = 'Running';
      document.getElementById('trackingInfo').style.display = 'block';
      
      alert('Location tracking dimulai!');
    }
    
    // Stop watching position
    function stopWatching() {
      if (watchId) {
        navigator.geolocation.clearWatch(watchId);
        watchId = null;
        document.getElementById('trackingStatus').textContent = 'Stopped';
        alert('Location tracking dihentikan!');
      }
    }
    
    // Clear location information
    function clearLocation() {
      document.getElementById('locationInfo').style.display = 'none';
      document.getElementById('mapContainer').style.display = 'none';
      document.getElementById('trackingInfo').style.display = 'none';
      updateCount = 0;
      document.getElementById('updateCount').textContent = updateCount;
      
      if (watchId) {
        navigator.geolocation.clearWatch(watchId);
        watchId = null;
      }
    }
    
    // Handle errors
    function showError(error) {
      let message = '';
      
      switch(error.code) {
        case error.PERMISSION_DENIED:
          message = 'User menolak permintaan geolocation.';
          break;
        case error.POSITION_UNAVAILABLE:
          message = 'Informasi lokasi tidak tersedia.';
          break;
        case error.TIMEOUT:
          message = 'Request timeout.';
          break;
        default:
          message = 'Error tidak diketahui.';
          break;
      }
      
      alert('Error: ' + message);
    }
    
    // Simulate reverse geocoding
    function getAddress() {
      const latitude = parseFloat(document.getElementById('latitude').textContent);
      const longitude = parseFloat(document.getElementById('longitude').textContent);
      
      if (isNaN(latitude) || isNaN(longitude)) {
        alert('Dapatkan lokasi terlebih dahulu!');
        return;
      }
      
      // Simulasi reverse geocoding (dalam aplikasi nyata, gunakan Google Geocoding API atau OpenCage)
      const mockAddresses = [
        'Jl. Sudirman No. 123, Jakarta Pusat, DKI Jakarta',
        'Jl. Malioboro No. 45, Yogyakarta, DIY',
        'Jl. Diponegoro No. 78, Surabaya, Jawa Timur',
        'Jl. Asia Afrika No. 90, Bandung, Jawa Barat'
      ];
      
      const randomAddress = mockAddresses[Math.floor(Math.random() * mockAddresses.length)];
      
      document.getElementById('addressText').textContent = 
        `${randomAddress}\nKoordinat: ${latitude.toFixed(6)}, ${longitude.toFixed(6)}`;
      document.getElementById('addressInfo').style.display = 'block';
    }
  </script>
</body>
</html>
\end{lstlisting}

\textbf{Hasil di Browser:}
- Tombol untuk mendapatkan lokasi sekali atau continuous tracking
- Display koordinat (latitude, longitude, altitude)
- Accuracy badge (High/Medium/Low) berdasarkan nilai accuracy
- Simulasi reverse geocoding untuk mendapatkan alamat
- Privacy and security notes untuk user awareness

\subsection{Device Orientation dan Motion}

Device Orientation dan Motion APIs memberikan akses ke accelerometer dan gyroscope:

\begin{itemize}
  \item \textbf{DeviceOrientationEvent}: Rotasi device dalam 3D space (alpha, beta, gamma)
  \item \textbf{DeviceMotionEvent}: Acceleration dan rotation rate
  \item \textbf{Use Cases}: Games, virtual reality, step counters, gesture control
  \item \textbf{Permission}: iOS 13+ memerlukan user permission
\end{itemize}

\begin{lstlisting}[caption={Device Orientation dan Motion}, basicstyle=\ttfamily\small, frame=single]
<!DOCTYPE html>
<html lang="id">
<head>
  <meta charset="UTF-8">
  <meta name="viewport" content="width=device-width, initial-scale=1.0">
  <title>Device Orientation & Motion</title>
  <style>
    body { font-family: Arial, sans-serif; line-height: 1.6; max-width: 800px; margin: 0 auto; padding: 20px; background-color: #f8f9fa; }
    .sensor-demo { background-color: white; border-radius: 8px; padding: 20px; margin-bottom: 20px; box-shadow: 0 2px 8px rgba(0,0,0,0.1); }
    .orientation-data { display: grid; grid-template-columns: repeat(auto-fit, minmax(200px, 1fr)); gap: 15px; margin-top: 15px; }
    .data-card { background-color: #f8f9fa; border: 1px solid #e9ecef; border-radius: 8px; padding: 15px; text-align: center; }
    .data-value { font-size: 24px; font-weight: bold; color: #007bff; margin: 10px 0; }
    .data-label { font-size: 14px; color: #6c757d; }
    .device-visual { width: 200px; height: 200px; margin: 20px auto; position: relative; }
    .device-box { width: 100px; height: 60px; background: linear-gradient(135deg, #667eea 0%, #764ba2 100%); border-radius: 8px; position: absolute; top: 50%; left: 50%; transform: translate(-50%, -50%); transition: transform 0.1s; box-shadow: 0 4px 8px rgba(0,0,0,0.2); }
    .btn { background-color: #007bff; color: white; border: none; padding: 10px 20px; border-radius: 4px; cursor: pointer; margin-right: 10px; }
    .btn:hover { background-color: #0056b3; }
    .status { padding: 10px; border-radius: 4px; margin-top: 15px; }
    .status.active { background-color: #d4edda; color: #155724; }
    .status.inactive { background-color: #f8d7da; color: #721c24; }
  </style>
</head>
<body>
  <h1>📱 Device Orientation & Motion</h1>
  
  <div class="sensor-demo">
    <h2>🧭 Device Orientation</h2>
    
    <div>
      <button class="btn" onclick="startOrientation()">▶️ Start Orientation</button>
      <button class="btn" onclick="stopOrientation()">⏹️ Stop Orientation</button>
    </div>
    
    <div class="device-visual">
      <div class="device-box" id="deviceBox"></div>
    </div>
    
    <div class="orientation-data">
      <div class="data-card">
        <div class="data-label">Alpha (Z-axis)</div>
        <div class="data-value" id="alpha">0°</div>
        <small>Rotation around z-axis</small>
      </div>
      
      <div class="data-card">
        <div class="data-label">Beta (X-axis)</div>
        <div class="data-value" id="beta">0°</div>
        <small>Front-to-back tilt</small>
      </div>
      
      <div class="data-card">
        <div class="data-label">Gamma (Y-axis)</div>
        <div class="data-value" id="gamma">0°</div>
        <small>Left-to-right tilt</small>
      </div>
    </div>
    
    <div id="orientationStatus" class="status inactive">
      Orientation tracking: Inactive
    </div>
  </div>
  
  <div class="sensor-demo">
    <h2>🏃 Device Motion (Accelerometer)</h2>
    
    <div>
      <button class="btn" onclick="startMotion()">▶️ Start Motion</button>
      <button class="btn" onclick="stopMotion()">⏹️ Stop Motion</button>
    </div>
    
    <div class="orientation-data" style="margin-top: 15px;">
      <div class="data-card">
        <div class="data-label">Acceleration X</div>
        <div class="data-value" id="accelX">0</div>
        <small>m/s²</small>
      </div>
      
      <div class="data-card">
        <div class="data-label">Acceleration Y</div>
        <div class="data-value" id="accelY">0</div>
        <small>m/s²</small>
      </div>
      
      <div class="data-card">
        <div class="data-label">Acceleration Z</div>
        <div class="data-value" id="accelZ">0</div>
        <small>m/s²</small>
      </div>
      
      <div class="data-card">
        <div class="data-label">Total Acceleration</div>
        <div class="data-value" id="totalAccel">0</div>
        <small>m/s²</small>
      </div>
    </div>
    
    <div id="motionStatus" class="status inactive" style="margin-top: 15px;">
      Motion tracking: Inactive
    </div>
  </div>
  
  <script>
    let orientationListener = null;
    let motionListener = null;
    
    // Device Orientation
    function startOrientation() {
      if (window.DeviceOrientationEvent) {
        orientationListener = handleOrientation;
        window.addEventListener('deviceorientation', handleOrientation);
        
        document.getElementById('orientationStatus').className = 'status active';
        document.getElementById('orientationStatus').textContent = 'Orientation tracking: Active';
      } else {
        alert('DeviceOrientationEvent tidak didukung di browser ini');
      }
    }
    
    function stopOrientation() {
      if (orientationListener) {
        window.removeEventListener('deviceorientation', handleOrientation);
        orientationListener = null;
        
        document.getElementById('orientationStatus').className = 'status inactive';
        document.getElementById('orientationStatus').textContent = 'Orientation tracking: Inactive';
      }
    }
    
    function handleOrientation(event) {
      const alpha = event.alpha || 0;
      const beta = event.beta || 0;
      const gamma = event.gamma || 0;
      
      document.getElementById('alpha').textContent = alpha.toFixed(1) + '°';
      document.getElementById('beta').textContent = beta.toFixed(1) + '°';
      document.getElementById('gamma').textContent = gamma.toFixed(1) + '°';
      
      // Visual feedback dengan rotate device box
      const deviceBox = document.getElementById('deviceBox');
      deviceBox.style.transform = `translate(-50%, -50%) rotateX(${-beta}deg) rotateY(${gamma}deg) rotateZ(${alpha}deg)`;
    }
    
    // Device Motion
    function startMotion() {
      if (window.DeviceMotionEvent) {
        motionListener = handleMotion;
        window.addEventListener('devicemotion', handleMotion);
        
        document.getElementById('motionStatus').className = 'status active';
        document.getElementById('motionStatus').textContent = 'Motion tracking: Active';
      } else {
        alert('DeviceMotionEvent tidak didukung di browser ini');
      }
    }
    
    function stopMotion() {
      if (motionListener) {
        window.removeEventListener('devicemotion', handleMotion);
        motionListener = null;
        
        document.getElementById('motionStatus').className = 'status inactive';
        document.getElementById('motionStatus').textContent = 'Motion tracking: Inactive';
      }
    }
    
    function handleMotion(event) {
      const acceleration = event.accelerationIncludingGravity;
      
      if (acceleration) {
        const x = acceleration.x || 0;
        const y = acceleration.y || 0;
        const z = acceleration.z || 0;
        const total = Math.sqrt(x*x + y*y + z*z);
        
        document.getElementById('accelX').textContent = x.toFixed(2);
        document.getElementById('accelY').textContent = y.toFixed(2);
        document.getElementById('accelZ').textContent = z.toFixed(2);
        document.getElementById('totalAccel').textContent = total.toFixed(2);
      }
    }
  </script>
</body>
</html>
\end{lstlisting}

\textbf{Hasil di Browser:}
- Visualisasi device orientation dengan 3D box yang berputar
- Display alpha, beta, gamma rotation values secara real-time
- Accelerometer data untuk x, y, z acceleration
- Tombol start/stop untuk mengontrol sensor monitoring
- Device motion tracking untuk gesture control

\subsection{Media Devices API (Camera & Microphone)}

Media Devices API memungkinkan akses ke camera dan microphone:

\begin{itemize}
  \item \textbf{getUserMedia()}: Akses camera dan microphone streams
  \item \textbf{Constraints}: Resolusi, facing mode, frame rate
  \item \textbf{Permissions}: User consent diperlukan untuk akses media
  \item \textbf{Recording}: MediaRecorder API untuk merekam audio/video
\end{itemize}

\begin{lstlisting}[caption={Media Devices API}, basicstyle=\ttfamily\small, frame=single]
<!DOCTYPE html>
<html lang="id">
<head>
  <meta charset="UTF-8">
  <meta name="viewport" content="width=device-width, initial-scale=1.0">
  <title>Media Devices API</title>
  <style>
    body { font-family: Arial, sans-serif; line-height: 1.6; max-width: 800px; margin: 0 auto; padding: 20px; background-color: #f8f9fa; }
    .media-demo { background-color: white; border-radius: 8px; padding: 20px; margin-bottom: 20px; box-shadow: 0 2px 8px rgba(0,0,0,0.1); }
    .video-container { position: relative; background-color: #000; border-radius: 8px; overflow: hidden; margin-bottom: 15px; }
    .video-preview { width: 100%; height: 300px; object-fit: cover; display: block; }
    .camera-controls { display: flex; gap: 10px; margin-bottom: 15px; flex-wrap: wrap; }
    .btn { background-color: #007bff; color: white; border: none; padding: 10px 20px; border-radius: 4px; cursor: pointer; }
    .btn:hover { background-color: #0056b3; }
    .btn:disabled { background-color: #6c757d; cursor: not-allowed; }
    .btn.record { background-color: #dc3545; }
    .btn.record:hover { background-color: #c82333; }
    .btn.stop { background-color: #6c757d; }
    .recording-indicator { display: none; background-color: #dc3545; color: white; padding: 5px 10px; border-radius: 4px; font-size: 12px; animation: blink 1s infinite; }
    @keyframes blink { 0%, 50% { opacity: 1; } 51%, 100% { opacity: 0; } }
    .captured-photos { display: grid; grid-template-columns: repeat(auto-fit, minmax(150px, 1fr)); gap: 10px; margin-top: 15px; }
    .photo-item { position: relative; border-radius: 4px; overflow: hidden; }
    .photo-item img { width: 100%; height: 100px; object-fit: cover; }
    .audio-visualizer { width: 100%; height: 100px; background-color: #000; border-radius: 8px; margin-top: 15px; }
  </style>
</head>
<body>
  <h1>📷 Media Devices API</h1>
  
  <div class="media-demo">
    <h2>📹 Camera Access</h2>
    
    <div class="camera-controls">
      <button class="btn" onclick="startCamera()">▶️ Start Camera</button>
      <button class="btn" onclick="stopCamera()">⏹️ Stop Camera</button>
      <button class="btn" onclick="switchCamera()">🔄 Switch Camera</button>
      <button class="btn record" onclick="takePhoto()">📸 Take Photo</button>
      <span class="recording-indicator" id="recordingIndicator">🔴 Recording</span>
    </div>
    
    <div class="video-container">
      <video id="videoPreview" class="video-preview" autoplay muted playsinline></video>
    </div>
    
    <canvas id="photoCanvas" style="display: none;"></canvas>
    
    <div id="capturedPhotos" class="captured-photos" style="display: none;">
      <h4>Captured Photos:</h4>
    </div>
  </div>
  
  <div class="media-demo">
    <h2>🎤 Microphone Access</h2>
    
    <div class="camera-controls">
      <button class="btn" onclick="startAudio()">▶️ Start Audio</button>
      <button class="btn" onclick="stopAudio()">⏹️ Stop Audio</button>
    </div>
    
    <canvas id="audioVisualizer" class="audio-visualizer"></canvas>
    
    <div id="audioStatus" style="margin-top: 15px; padding: 10px; background-color: #e9ecef; border-radius: 4px;">
      Audio Status: Stopped
    </div>
  </div>
  
  <div class="media-demo">
    <h2>⚠️ Privacy & Permissions</h2>
    
    <div style="background-color: #fff3cd; border: 1px solid #ffeaa7; border-radius: 4px; padding: 15px;">
      <h4>🔒 Important Notes:</h4>
      <ul>
        <li><strong>User Consent:</strong> Browser akan meminta izin sebelum mengakses camera/microphone</li>
        <li><strong>HTTPS Required:</strong> Media Devices API hanya bekerja pada secure contexts</li>
        <li><strong>Privacy Indicators:</strong> Browser akan menampilkan indicator saat camera/microphone aktif</li>
        <li><strong>Permission Management:</strong> User dapat revoke permission kapan saja</li>
      </ul>
    </div>
  </div>
  
  <script>
    let stream = null;
    let audioContext = null;
    let analyser = null;
    let microphone = null;
    let currentFacing = 'user'; // 'user' or 'environment'
    let capturedPhotos = [];
    
    // Start camera
    async function startCamera() {
      try {
        const constraints = {
          video: {
            facingMode: currentFacing,
            width: { ideal: 1280 },
            height: { ideal: 720 }
          },
          audio: false
        };
        
        stream = await navigator.mediaDevices.getUserMedia(constraints);
        const video = document.getElementById('videoPreview');
        video.srcObject = stream;
        
      } catch (error) {
        alert('Error accessing camera: ' + error.message);
      }
    }
    
    // Stop camera
    function stopCamera() {
      if (stream) {
        stream.getTracks().forEach(track => track.stop());
        stream = null;
        document.getElementById('videoPreview').srcObject = null;
      }
    }
    
    // Switch camera (front/rear)
    async function switchCamera() {
      stopCamera();
      currentFacing = currentFacing === 'user' ? 'environment' : 'user';
      await startCamera();
    }
    
    // Take photo
    function takePhoto() {
      const video = document.getElementById('videoPreview');
      const canvas = document.getElementById('photoCanvas');
      const photosContainer = document.getElementById('capturedPhotos');
      
      if (!video.srcObject) {
        alert('Camera tidak aktif!');
        return;
      }
      
      // Set canvas dimensions
      canvas.width = video.videoWidth;
      canvas.height = video.videoHeight;
      
      // Draw video frame to canvas
      const ctx = canvas.getContext('2d');
      ctx.drawImage(video, 0, 0, canvas.width, canvas.height);
      
      // Create photo item
      const photoData = canvas.toDataURL('image/jpeg');
      const photoItem = document.createElement('div');
      photoItem.className = 'photo-item';
      photoItem.innerHTML = `<img src="${photoData}" alt="Captured photo">`;
      
      photosContainer.appendChild(photoItem);
      photosContainer.style.display = 'grid';
      
      capturedPhotos.push(photoData);
    }
    
    // Start audio with visualization
    async function startAudio() {
      try {
        const audioStream = await navigator.mediaDevices.getUserMedia({ audio: true, video: false });
        
        audioContext = new (window.AudioContext || window.webkitAudioContext)();
        analyser = audioContext.createAnalyser();
        microphone = audioContext.createMediaStreamSource(audioStream);
        microphone.connect(analyser);
        
        analyser.fftSize = 256;
        const bufferLength = analyser.frequencyBinCount;
        const dataArray = new Uint8Array(bufferLength);
        
        const canvas = document.getElementById('audioVisualizer');
        const canvasCtx = canvas.getContext('2d');
        
        // Resize canvas
        canvas.width = canvas.offsetWidth;
        canvas.height = canvas.offsetHeight;
        
        function draw() {
          requestAnimationFrame(draw);
          
          analyser.getByteFrequencyData(dataArray);
          
          canvasCtx.fillStyle = 'rgb(0, 0, 0)';
          canvasCtx.fillRect(0, 0, canvas.width, canvas.height);
          
          const barWidth = (canvas.width / bufferLength) * 2.5;
          let barHeight;
          let x = 0;
          
          for (let i = 0; i < bufferLength; i++) {
            barHeight = dataArray[i] / 2;
            
            canvasCtx.fillStyle = `rgb(${barHeight + 100}, 50, 50)`;
            canvasCtx.fillRect(x, canvas.height - barHeight, barWidth, barHeight);
            
            x += barWidth + 1;
          }
        }
        
        draw();
        
        document.getElementById('audioStatus').textContent = 'Audio Status: Active - Microphone listening';
        document.getElementById('audioStatus').style.backgroundColor = '#d4edda';
        document.getElementById('audioStatus').style.color = '#155724';
        
      } catch (error) {
        alert('Error accessing microphone: ' + error.message);
      }
    }
    
    // Stop audio
    function stopAudio() {
      if (audioContext) {
        audioContext.close();
        audioContext = null;
        analyser = null;
        microphone = null;
      }
      
      // Clear canvas
      const canvas = document.getElementById('audioVisualizer');
      const canvasCtx = canvas.getContext('2d');
      canvasCtx.fillStyle = 'rgb(0, 0, 0)';
      canvasCtx.fillRect(0, 0, canvas.width, canvas.height);
      
      document.getElementById('audioStatus').textContent = 'Audio Status: Stopped';
      document.getElementById('audioStatus').style.backgroundColor = '#e9ecef';
      document.getElementById('audioStatus').style.color = '#333';
    }
  </script>
</body>
</html>
\end{lstlisting}

\textbf{Hasil di Browser:}
- Camera preview dengan video element
- Tombol untuk start/stop camera dan switch front/rear camera
- Take photo functionality dengan canvas capture
- Photo gallery untuk menampilkan captured photos
- Audio visualizer dengan frequency bar chart
- Privacy notes untuk user awareness

Geolocation dan Device APIs memungkinkan aplikasi web untuk mengakses fitur hardware dan informasi lokasi yang sebelumnya hanya tersedia untuk native applications \cite{w3schools-html}. Dengan permission-based access dan privacy controls, APIs ini menyediakan fondasi untuk aplikasi web yang lebih personal dan interaktif \cite{mdn-geolocation}.

\section{Web Workers dan Performance}

HTML5 memperkenalkan Web Workers untuk menjalankan JavaScript di background thread, memungkinkan aplikasi web yang responsif dan performant. Performance APIs menyediakan tools untuk mengukur dan mengoptimasi kinerja aplikasi \cite{mdn-web-workers}. Kombinasi Web Workers dan Performance APIs membuka kemungkinan untuk aplikasi web yang kompleks tanpa blocking main thread \cite{w3schools-html}.

\subsection{Web Workers Basics}

Web Workers memungkinkan eksekusi JavaScript di thread terpisah dari main thread:

\begin{itemize}
  \item \textbf{Dedicated Workers}: Worker yang terhubung ke satu script
  \item \textbf{Shared Workers}: Worker yang dapat diakses oleh multiple scripts
  \item \textbf{Service Workers}: Workers untuk caching dan offline functionality
  \item \textbf{Communication}: Message passing antara main thread dan worker
  \item \textbf{Limitations}: Tidak ada akses ke DOM, window, atau document
\end{itemize}

\begin{lstlisting}[caption={Web Workers Implementation}, basicstyle=\ttfamily\small, frame=single]
<!DOCTYPE html>
<html lang="id">
<head>
  <meta charset="UTF-8">
  <meta name="viewport" content="width=device-width, initial-scale=1.0">
  <title>Web Workers Demo</title>
  <style>
    body { font-family: Arial, sans-serif; line-height: 1.6; max-width: 900px; margin: 0 auto; padding: 20px; background-color: #f8f9fa; }
    .worker-demo { background-color: white; border-radius: 8px; padding: 20px; margin-bottom: 20px; box-shadow: 0 2px 8px rgba(0,0,0,0.1); }
    .btn { background-color: #007bff; color: white; border: none; padding: 10px 20px; border-radius: 4px; cursor: pointer; margin-right: 10px; margin-bottom: 10px; }
    .btn:hover { background-color: #0056b3; }
    .btn:disabled { background-color: #6c757d; cursor: not-allowed; }
    .progress-container { width: 100%; height: 20px; background-color: #e9ecef; border-radius: 10px; overflow: hidden; margin: 15px 0; }
    .progress-bar { height: 100%; background-color: #28a745; width: 0; transition: width 0.3s; }
    .result-container { background-color: #f8f9fa; border: 1px solid #e9ecef; border-radius: 4px; padding: 15px; margin-top: 15px; max-height: 200px; overflow-y: auto; }
    .status { padding: 10px; border-radius: 4px; margin-top: 10px; }
    .status.running { background-color: #fff3cd; color: #856404; }
    .status.completed { background-color: #d4edda; color: #155724; }
    .status.error { background-color: #f8d7da; color: #721c24; }
    .comparison-grid { display: grid; grid-template-columns: 1fr 1fr; gap: 20px; margin-top: 20px; }
  </style>
</head>
<body>
  <h1>🚀 Web Workers Demo</h1>
  
  <!-- Heavy Computation Demo -->
  <div class="worker-demo">
    <h2>🔢 Heavy Computation (Prime Numbers)</h2>
    
    <div class="comparison-grid">
      <div>
        <h3>Main Thread (Blocking)</h3>
        <div>
          <label>Calculate primes up to:</label>
          <input type="number" id="mainLimit" value="100000" min="1000" max="1000000" style="width: 100%; padding: 8px; margin: 5px 0;">
        </div>
        <button class="btn" onclick="calculatePrimesMain()" id="mainBtn">Calculate on Main Thread</button>
        
        <div class="progress-container">
          <div class="progress-bar" id="mainProgress"></div>
        </div>
        
        <div id="mainStatus" class="status" style="display: none;"></div>
        <div id="mainResult" class="result-container" style="display: none;"></div>
      </div>
      
      <div>
        <h3>Web Worker (Non-blocking)</h3>
        <div>
          <label>Calculate primes up to:</label>
          <input type="number" id="workerLimit" value="100000" min="1000" max="1000000" style="width: 100%; padding: 8px; margin: 5px 0;">
        </div>
        <button class="btn" onclick="calculatePrimesWorker()" id="workerBtn">Calculate with Web Worker</button>
        
        <div class="progress-container">
          <div class="progress-bar" id="workerProgress"></div>
        </div>
        
        <div id="workerStatus" class="status" style="display: none;"></div>
        <div id="workerResult" class="result-container" style="display: none;"></div>
      </div>
    </div>
    
    <div style="background-color: #e7f3ff; border: 1px solid #b3d9ff; border-radius: 4px; padding: 15px; margin-top: 20px;">
      <h4>📝 Notice:</h4>
      <p><strong>Main Thread:</strong> UI akan freeze saat perhitungan berlangsung</p>
      <p><strong>Web Worker:</strong> UI tetap responsif karena perhitungan di background thread</p>
      <p><strong>Try this:</strong> Klik "Calculate on Main Thread" dan coba klik button lain selama perhitungan</p>
    </div>
  </div>
  
  <!-- UI Responsiveness Test -->
  <div class="worker-demo">
    <h2>🎨 UI Responsiveness Test</h2>
    
    <div style="display: flex; gap: 20px; align-items: center; margin-bottom: 15px;">
      <div style="width: 50px; height: 50px; background-color: #007bff; border-radius: 50%; animation: pulse 2s infinite;"></div>
      <p>Animation ini akan freeze saat main thread sibuk</p>
    </div>
    
    <div style="background-color: #f0f0f0; padding: 20px; border-radius: 8px; margin: 10px 0;">
      <p>Counter: <span id="counter" style="font-size: 24px; font-weight: bold; color: #007bff;">0</span></p>
      <button class="btn" onclick="incrementCounter()">Increment Counter</button>
    </div>
    
    <style>
      @keyframes pulse {
        0%, 100% { transform: scale(1); }
        50% { transform: scale(1.2); }
      }
    </style>
  </div>
  
  <!-- Worker Code (normally in separate file) -->
  <script id="workerCode" type="javascript/worker">
    self.onmessage = function(e) {
      const limit = e.data.limit;
      const primes = [];
      
      for (let i = 2; i <= limit; i++) {
        let isPrime = true;
        for (let j = 2; j <= Math.sqrt(i); j++) {
          if (i % j === 0) {
            isPrime = false;
            break;
          }
        }
        
        if (isPrime) {
          primes.push(i);
        }
        
        // Send progress update every 1000 numbers
        if (i % 1000 === 0) {
          self.postMessage({
            type: 'progress',
            progress: Math.round((i / limit) * 100),
            current: i,
            limit: limit
          });
        }
      }
      
      self.postMessage({
        type: 'complete',
        primes: primes,
        count: primes.length
      });
    };
  </script>
  
  <script>
    let worker = null;
    let counter = 0;
    
    // Calculate primes on main thread (blocking)
    function calculatePrimesMain() {
      const limit = parseInt(document.getElementById('mainLimit').value);
      const mainBtn = document.getElementById('mainBtn');
      const mainProgress = document.getElementById('mainProgress');
      const mainStatus = document.getElementById('mainStatus');
      const mainResult = document.getElementById('mainResult');
      
      mainBtn.disabled = true;
      mainStatus.className = 'status running';
      mainStatus.textContent = 'Calculating on main thread... UI will freeze';
      mainStatus.style.display = 'block';
      mainResult.style.display = 'none';
      
      // Use setTimeout to allow UI to update before blocking
      setTimeout(() => {
        const startTime = performance.now();
        const primes = [];
        
        for (let i = 2; i <= limit; i++) {
          let isPrime = true;
          for (let j = 2; j <= Math.sqrt(i); j++) {
            if (i % j === 0) {
              isPrime = false;
              break;
            }
          }
          
          if (isPrime) {
            primes.push(i);
          }
          
          // Update progress every 1000 numbers
          if (i % 1000 === 0) {
            mainProgress.style.width = Math.round((i / limit) * 100) + '%';
          }
        }
        
        const endTime = performance.now();
        const duration = (endTime - startTime).toFixed(2);
        
        mainProgress.style.width = '100%';
        mainStatus.className = 'status completed';
        mainStatus.textContent = `Completed in ${duration}ms - Found ${primes.length} primes`;
        
        mainResult.innerHTML = `
          <h4>Results:</h4>
          <p><strong>Duration:</strong> ${duration}ms</p>
          <p><strong>Primes found:</strong> ${primes.length}</p>
          <p><strong>First 10 primes:</strong> ${primes.slice(0, 10).join(', ')}</p>
          <p><strong>Last 10 primes:</strong> ${primes.slice(-10).join(', ')}</p>
        `;
        mainResult.style.display = 'block';
        mainBtn.disabled = false;
        
      }, 100);
    }
    
    // Calculate primes with Web Worker (non-blocking)
    function calculatePrimesWorker() {
      const limit = parseInt(document.getElementById('workerLimit').value);
      const workerBtn = document.getElementById('workerBtn');
      const workerProgress = document.getElementById('workerProgress');
      const workerStatus = document.getElementById('workerStatus');
      const workerResult = document.getElementById('workerResult');
      
      workerBtn.disabled = true;
      workerStatus.className = 'status running';
      workerStatus.textContent = 'Calculating with Web Worker... UI remains responsive';
      workerStatus.style.display = 'block';
      workerResult.style.display = 'none';
      workerProgress.style.width = '0%';
      
      const startTime = performance.now();
      
      // Create worker from inline script
      const workerScript = document.getElementById('workerCode').textContent;
      const blob = new Blob([workerScript], { type: 'application/javascript' });
      worker = new Worker(URL.createObjectURL(blob));
      
      worker.onmessage = function(e) {
        const data = e.data;
        
        if (data.type === 'progress') {
          workerProgress.style.width = data.progress + '%';
          workerStatus.textContent = `Progress: ${data.progress}% (${data.current}/${data.limit})`;
        } else if (data.type === 'complete') {
          const endTime = performance.now();
          const duration = (endTime - startTime).toFixed(2);
          
          workerProgress.style.width = '100%';
          workerStatus.className = 'status completed';
          workerStatus.textContent = `Completed in ${duration}ms - Found ${data.count} primes`;
          
          workerResult.innerHTML = `
            <h4>Results:</h4>
            <p><strong>Duration:</strong> ${duration}ms</p>
            <p><strong>Primes found:</strong> ${data.count}</p>
            <p><strong>First 10 primes:</strong> ${data.primes.slice(0, 10).join(', ')}</p>
            <p><strong>Last 10 primes:</strong> ${data.primes.slice(-10).join(', ')}</p>
          `;
          workerResult.style.display = 'block';
          workerBtn.disabled = false;
          
          // Terminate worker
          worker.terminate();
          worker = null;
        }
      };
      
      worker.onerror = function(error) {
        workerStatus.className = 'status error';
        workerStatus.textContent = 'Error: ' + error.message;
        workerBtn.disabled = false;
      };
      
      // Send message to worker
      worker.postMessage({ limit: limit });
    }
    
    // Increment counter
    function incrementCounter() {
      counter++;
      document.getElementById('counter').textContent = counter;
    }
    
    // Auto-increment counter to show UI responsiveness
    setInterval(() => {
      counter++;
      document.getElementById('counter').textContent = counter;
    }, 1000);
  </script>
</body>
</html>
\end{lstlisting}

\textbf{Hasil di Browser:}
- Side-by-side comparison: Main Thread vs Web Worker untuk heavy computation
- Progress bar yang updates secara real-time untuk kedua metode
- UI responsiveness test dengan animasi dan counter
- Web Worker menjalankan perhitungan di background tanpa blocking UI
- Performance metrics showing duration dan jumlah primes found

\subsection{Performance APIs}

Performance APIs menyediakan tools untuk mengukur dan mengoptimasi kinerja aplikasi:

\begin{itemize}
  \item \textbf{Performance Timeline}: Navigation Timing, Resource Timing, User Timing
  \item \textbf{PerformanceObserver}: Monitoring performance metrics secara real-time
  \item \textbf{Memory API}: Monitoring memory usage aplikasi
  \item \textbf{Frame Timing}: Mengukur frame rate dan rendering performance
  \item \textbf{Long Tasks API}: Detecting tasks yang blocking main thread
\end{itemize}

\begin{lstlisting}[caption={Performance APIs Demo}, basicstyle=\ttfamily\small, frame=single]
<!DOCTYPE html>
<html lang="id">
<head>
  <meta charset="UTF-8">
  <meta name="viewport" content="width=device-width, initial-scale=1.0">
  <title>Performance APIs Demo</title>
  <style>
    body { font-family: Arial, sans-serif; line-height: 1.6; max-width: 900px; margin: 0 auto; padding: 20px; background-color: #f8f9fa; }
    .perf-demo { background-color: white; border-radius: 8px; padding: 20px; margin-bottom: 20px; box-shadow: 0 2px 8px rgba(0,0,0,0.1); }
    .metrics-grid { display: grid; grid-template-columns: repeat(auto-fit, minmax(200px, 1fr)); gap: 15px; margin-top: 15px; }
    .metric-card { background-color: #f8f9fa; border: 1px solid #e9ecef; border-radius: 8px; padding: 15px; text-align: center; }
    .metric-value { font-size: 24px; font-weight: bold; color: #007bff; margin: 10px 0; }
    .metric-label { font-size: 14px; color: #6c757d; }
    .btn { background-color: #007bff; color: white; border: none; padding: 10px 20px; border-radius: 4px; cursor: pointer; margin-right: 10px; margin-bottom: 10px; }
    .btn:hover { background-color: #0056b3; }
    .chart-container { background-color: #f8f9fa; border: 1px solid #e9ecef; border-radius: 8px; padding: 15px; margin-top: 15px; height: 200px; position: relative; }
    .performance-log { background-color: #f8f9fa; border: 1px solid #e9ecef; border-radius: 4px; padding: 15px; margin-top: 15px; max-height: 300px; overflow-y: auto; font-family: monospace; font-size: 12px; }
  </style>
</head>
<body>
  <h1>⚡ Performance APIs Demo</h1>
  
  <!-- Navigation Timing -->
  <div class="perf-demo">
    <h2>🌐 Navigation Timing</h2>
    
    <button class="btn" onclick="measureNavigationTiming()">📊 Measure Page Load</button>
    <button class="btn" onclick="clearLogs()">🗑️ Clear Logs</button>
    
    <div class="metrics-grid" id="navMetrics" style="display: none;">
      <div class="metric-card">
        <div class="metric-label">DNS Lookup</div>
        <div class="metric-value" id="dnsTime">-</div>
        <small>Time to resolve domain</small>
      </div>
      
      <div class="metric-card">
        <div class="metric-label">TCP Connection</div>
        <div class="metric-value" id="tcpTime">-</div>
        <small>Time to establish connection</small>
      </div>
      
      <div class="metric-card">
        <div class="metric-label">Response Time</div>
        <div class="metric-value" id="responseTime">-</div>
        <small>Time to first byte</small>
      </div>
      
      <div class="metric-card">
        <div class="metric-label">DOM Processing</div>
        <div class="metric-value" id="domTime">-</div>
        <small>Time to parse DOM</small>
      </div>
      
      <div class="metric-card">
        <div class="metric-label">Load Event</div>
        <div class="metric-value" id="loadTime">-</div>
        <small>Time to load event</small>
      </div>
      
      <div class="metric-card">
        <div class="metric-label">Total Load Time</div>
        <div class="metric-value" id="totalTime">-</div>
        <small>Complete page load</small>
      </div>
    </div>
  </div>
  
  <!-- User Timing API -->
  <div class="perf-demo">
    <h2>⏱️ User Timing API</h2>
    
    <button class="btn" onclick="startUserTiming()">▶️ Start Measurement</button>
    <button class="btn" onclick="markCheckpoint()">📍 Mark Checkpoint</button>
    <button class="btn" onclick="endUserTiming()">⏹️ End Measurement</button>
    <button class="btn" onclick="showUserTimings()">📊 Show Results</button>
    
    <div class="performance-log" id="userTimingLog" style="display: none;"></div>
  </div>
  
  <!-- Memory API -->
  <div class="perf-demo">
    <h2>💾 Memory Usage</h2>
    
    <button class="btn" onclick="checkMemory()">🔍 Check Memory</button>
    <button class="btn" onclick="createObjects()">📦 Create Test Objects</button>
    <button class="btn" onclick="gcHint()">🗑️ Request GC</button>
    
    <div class="metrics-grid" id="memoryMetrics" style="display: none; margin-top: 15px;">
      <div class="metric-card">
        <div class="metric-label">Used JS Heap</div>
        <div class="metric-value" id="usedHeap">-</div>
        <small>Memory currently in use</small>
      </div>
      
      <div class="metric-card">
        <div class="metric-label">Total JS Heap</div>
        <div class="metric-value" id="totalHeap">-</div>
        <small>Total heap size</small>
      </div>
      
      <div class="metric-card">
        <div class="metric-label">Heap Limit</div>
        <div class="metric-value" id="heapLimit">-</div>
        <small>Maximum heap size</small>
      </div>
    </div>
  </div>
  
  <!-- Frame Timing -->
  <div class="perf-demo">
    <h2>🎬 Frame Rate Monitor</h2>
    
    <button class="btn" onclick="startFrameMonitoring()">▶️ Start Monitoring</button>
    <button class="btn" onclick="stopFrameMonitoring()">⏹️ Stop Monitoring</button>
    
    <div class="metrics-grid" style="margin-top: 15px;">
      <div class="metric-card">
        <div class="metric-label">Current FPS</div>
        <div class="metric-value" id="currentFPS">-</div>
      </div>
      
      <div class="metric-card">
        <div class="metric-label">Average FPS</div>
        <div class="metric-value" id="avgFPS">-</div>
      </div>
      
      <div class="metric-card">
        <div class="metric-label">Dropped Frames</div>
        <div class="metric-value" id="droppedFrames">-</div>
      </div>
    </div>
    
    <div class="chart-container">
      <canvas id="fpsChart" width="800" height="200"></canvas>
    </div>
  </div>
  
  <script>
    // Navigation Timing API
    function measureNavigationTiming() {
      const navigation = performance.getEntriesByType('navigation')[0];
      
      if (!navigation) {
        alert('Navigation Timing API not supported');
        return;
      }
      
      // Calculate timings
      const dnsTime = navigation.domainLookupEnd - navigation.domainLookupStart;
      const tcpTime = navigation.connectEnd - navigation.connectStart;
      const responseTime = navigation.responseStart - navigation.requestStart;
      const domTime = navigation.domComplete - navigation.domLoading;
      const loadTime = navigation.loadEventEnd - navigation.loadEventStart;
      const totalTime = navigation.loadEventEnd - navigation.navigationStart;
      
      // Display metrics
      document.getElementById('dnsTime').textContent = dnsTime + 'ms';
      document.getElementById('tcpTime').textContent = tcpTime + 'ms';
      document.getElementById('responseTime').textContent = responseTime + 'ms';
      document.getElementById('domTime').textContent = domTime + 'ms';
      document.getElementById('loadTime').textContent = loadTime + 'ms';
      document.getElementById('totalTime').textContent = totalTime + 'ms';
      
      document.getElementById('navMetrics').style.display = 'grid';
    }
    
    // User Timing API
    let timingMeasurements = [];
    
    function startUserTiming() {
      performance.mark('start');
      logUserTiming('Measurement started');
    }
    
    function markCheckpoint() {
      const markName = 'checkpoint-' + Date.now();
      performance.mark(markName);
      logUserTiming(`Checkpoint marked: ${markName}`);
    }
    
    function endUserTiming() {
      performance.mark('end');
      performance.measure('total', 'start', 'end');
      logUserTiming('Measurement ended');
    }
    
    function showUserTimings() {
      const measures = performance.getEntriesByType('measure');
      const marks = performance.getEntriesByType('mark');
      
      let log = '<h4>User Timing Results:</h4>';
      
      log += '<h5>Measures:</h5>';
      measures.forEach(measure => {
        log += `<p>${measure.name}: ${measure.duration.toFixed(2)}ms</p>`;
      });
      
      log += '<h5>Marks:</h5>';
      marks.forEach(mark => {
        log += `<p>${mark.name}: ${mark.startTime.toFixed(2)}ms</p>`;
      });
      
      const logContainer = document.getElementById('userTimingLog');
      logContainer.innerHTML = log;
      logContainer.style.display = 'block';
    }
    
    function logUserTiming(message) {
      const logContainer = document.getElementById('userTimingLog');
      const timestamp = new Date().toLocaleTimeString();
      logContainer.innerHTML += `<p>[${timestamp}] ${message}</p>`;
      logContainer.style.display = 'block';
    }
    
    function clearLogs() {
      document.getElementById('userTimingLog').innerHTML = '';
      document.getElementById('userTimingLog').style.display = 'none';
      document.getElementById('navMetrics').style.display = 'none';
    }
    
    // Memory API
    function checkMemory() {
      if (performance.memory) {
        const memory = performance.memory;
        
        document.getElementById('usedHeap').textContent = formatBytes(memory.usedJSHeapSize);
        document.getElementById('totalHeap').textContent = formatBytes(memory.totalJSHeapSize);
        document.getElementById('heapLimit').textContent = formatBytes(memory.jsHeapSizeLimit);
        
        document.getElementById('memoryMetrics').style.display = 'grid';
      } else {
        alert('Memory API not supported in this browser');
      }
    }
    
    function createObjects() {
      // Create some objects to increase memory usage
      const largeArray = [];
      for (let i = 0; i < 100000; i++) {
        largeArray.push({
          id: i,
          data: 'x'.repeat(100),
          timestamp: Date.now()
        });
      }
      
      alert('Created 100,000 test objects. Check memory usage now.');
      checkMemory();
    }
    
    function gcHint() {
      // Hint to browser that GC would be beneficial
      if (window.gc) {
        window.gc();
        alert('Garbage collection requested');
      } else {
        alert('GC not available. Try closing and reopening the page.');
      }
      checkMemory();
    }
    
    function formatBytes(bytes) {
      if (bytes === 0) return '0 Bytes';
      const k = 1024;
      const sizes = ['Bytes', 'KB', 'MB', 'GB'];
      const i = Math.floor(Math.log(bytes) / Math.log(k));
      return parseFloat((bytes / Math.pow(k, i)).toFixed(2)) + ' ' + sizes[i];
    }
    
    // Frame Rate Monitoring
    let frameId = null;
    let frameCount = 0;
    let lastTime = 0;
    let fpsHistory = [];
    
    function startFrameMonitoring() {
      if (frameId) return;
      
      frameCount = 0;
      lastTime = performance.now();
      fpsHistory = [];
      
      measureFrameRate();
    }
    
    function measureFrameRate() {
      frameCount++;
      const currentTime = performance.now();
      const deltaTime = currentTime - lastTime;
      
      if (deltaTime >= 1000) {
        const fps = Math.round((frameCount * 1000) / deltaTime);
        fpsHistory.push(fps);
        
        if (fpsHistory.length > 60) {
          fpsHistory.shift();
        }
        
        document.getElementById('currentFPS').textContent = fps;
        
        const avgFPS = Math.round(fpsHistory.reduce((a, b) => a + b, 0) / fpsHistory.length);
        document.getElementById('avgFPS').textContent = avgFPS;
        
        // Estimate dropped frames (assuming 60 FPS target)
        const dropped = Math.max(0, 60 - fps);
        document.getElementById('droppedFrames').textContent = dropped;
        
        drawFPSChart();
        
        frameCount = 0;
        lastTime = currentTime;
      }
      
      frameId = requestAnimationFrame(measureFrameRate);
    }
    
    function stopFrameMonitoring() {
      if (frameId) {
        cancelAnimationFrame(frameId);
        frameId = null;
      }
    }
    
    function drawFPSChart() {
      const canvas = document.getElementById('fpsChart');
      const ctx = canvas.getContext('2d');
      
      // Clear canvas
      ctx.clearRect(0, 0, canvas.width, canvas.height);
      
      if (fpsHistory.length === 0) return;
      
      // Draw grid
      ctx.strokeStyle = '#e9ecef';
      ctx.lineWidth = 1;
      
      for (let i = 0; i <= 10; i++) {
        const y = (canvas.height / 10) * i;
        ctx.beginPath();
        ctx.moveTo(0, y);
        ctx.lineTo(canvas.width, y);
        ctx.stroke();
      }
      
      // Draw FPS line
      ctx.strokeStyle = '#007bff';
      ctx.lineWidth = 2;
      ctx.beginPath();
      
      const barWidth = canvas.width / fpsHistory.length;
      
      for (let i = 0; i < fpsHistory.length; i++) {
        const fps = fpsHistory[i];
        const x = i * barWidth;
        const y = canvas.height - (fps / 70) * canvas.height;
        
        if (i === 0) {
          ctx.moveTo(x, y);
        } else {
          ctx.lineTo(x, y);
        }
      }
      
      ctx.stroke();
      
      // Draw target line (60 FPS)
      ctx.strokeStyle = '#28a745';
      ctx.setLineDash([5, 5]);
      const targetY = canvas.height - (60 / 70) * canvas.height;
      ctx.beginPath();
      ctx.moveTo(0, targetY);
      ctx.lineTo(canvas.width, targetY);
      ctx.stroke();
      ctx.setLineDash([]);
      
      // Labels
      ctx.fillStyle = '#333';
      ctx.font = '12px Arial';
      ctx.fillText('60 FPS Target', 10, targetY - 5);
      ctx.fillText('FPS History (last 60 seconds)', 10, 20);
    }
  </script>
</body>
</html>
\end{lstlisting}

\textbf{Hasil di Browser:}
- Navigation Timing API showing DNS, TCP, response, DOM processing times
- User Timing API untuk custom performance measurements
- Memory API monitoring untuk heap usage
- Frame rate monitoring dengan real-time FPS chart
- Performance metrics dengan visual indicators

Web Workers dan Performance APIs memberikan tools yang powerful untuk mengukur dan mengoptimasi kinerja aplikasi web \cite{w3schools-html}. Dengan Web Workers, heavy computations dapat dijalankan di background tanpa blocking UI, sementara Performance APIs menyediakan visibility ke dalam setiap aspect dari application performance \cite{mdn-web-workers}.

\section{Validasi dan Aksesibilitas HTML5}

Validasi dan aksesibilitas adalah dua pilar fundamental dalam pengembangan web modern. Validasi memastikan markup HTML mengikuti standar, sementara aksesibilitas memastikan konten dapat diakses oleh semua pengguna, termasuk yang menggunakan teknologi assistif \cite{wcag-guide}. HTML5 menyediakan berbagai tools dan atribut untuk meningkatkan kedua aspek ini \cite{mdn-accessibility}.

\subsection{HTML5 Validation}

Validasi HTML memastikan dokumen mengikuti spesifikasi HTML5 dan bebas dari errors:

\begin{itemize}
  \item \textbf{W3C Validator}: Online tool untuk memeriksa markup validity
  \item \textbf{Browser DevTools}: Built-in validators di browser modern
  \item \textbf{Linting Tools}: HTMLHint, HTML-validate untuk development workflow
  \item \textbf{Common Errors}: Missing alt attributes, invalid nesting, deprecated tags
  \item \textbf{Semantic Validation}: Proper use of semantic elements
\end{itemize}

\begin{lstlisting}[caption={HTML5 Validation Best Practices}, basicstyle=\ttfamily\small, frame=single]
<!DOCTYPE html>
<html lang="id">
<head>
  <meta charset="UTF-8">
  <meta name="viewport" content="width=device-width, initial-scale=1.0">
  <title>Valid HTML5 Document Example</title>
  <style>
    body { font-family: Arial, sans-serif; line-height: 1.6; max-width: 900px; margin: 0 auto; padding: 20px; background-color: #f8f9fa; }
    .validation-demo { background-color: white; border-radius: 8px; padding: 20px; margin-bottom: 20px; box-shadow: 0 2px 8px rgba(0,0,0,0.1); }
    .valid-example { border-left: 4px solid #28a745; background-color: #f8fff8; padding: 15px; margin: 10px 0; }
    .invalid-example { border-left: 4px solid #dc3545; background-color: #fff8f8; padding: 15px; margin: 10px 0; }
    .warning-example { border-left: 4px solid #ffc107; background-color: #fffbf0; padding: 15px; margin: 10px 0; }
    .code-block { background-color: #f8f9fa; border: 1px solid #e9ecef; border-radius: 4px; padding: 15px; font-family: monospace; font-size: 14px; overflow-x: auto; }
    .btn { background-color: #007bff; color: white; border: none; padding: 10px 20px; border-radius: 4px; cursor: pointer; }
    .btn:hover { background-color: #0056b3; }
  </style>
</head>
<body>
  <h1>✅ HTML5 Validation & Best Practices</h1>
  
  <div class="validation-demo">
    <h2>📝 Valid HTML5 Structure</h2>
    
    <div class="valid-example">
      <h4>✅ Correct DOCTYPE and Basic Structure</h4>
      <div class="code-block">
&lt;!DOCTYPE html&gt;
&lt;html lang="id"&gt;
&lt;head&gt;
  &lt;meta charset="UTF-8"&gt;
  &lt;meta name="viewport" content="width=device-width, initial-scale=1.0"&gt;
  &lt;title&gt;Page Title&lt;/title&gt;
&lt;/head&gt;
&lt;body&gt;
  &lt;!-- Content goes here --&gt;
&lt;/body&gt;
&lt;/html&gt;
      </div>
    </div>
    
    <div class="valid-example">
      <h4>✅ Proper Semantic Structure</h4>
      <div class="code-block">
&lt;header&gt;
  &lt;h1&gt;Website Title&lt;/h1&gt;
  &lt;nav&gt;
    &lt;ul&gt;
      &lt;li&gt;&lt;a href="/"&gt;Home&lt;/a&gt;&lt;/li&gt;
      &lt;li&gt;&lt;a href="/about"&gt;About&lt;/a&gt;&lt;/li&gt;
    &lt;/ul&gt;
  &lt;/nav&gt;
&lt;/header&gt;

&lt;main&gt;
  &lt;article&gt;
    &lt;h2&gt;Article Title&lt;/h2&gt;
    &lt;p&gt;Article content...&lt;/p&gt;
  &lt;/article&gt;
&lt;/main&gt;

&lt;footer&gt;
  &lt;p&gt;&amp;copy; 2024 Website Name&lt;/p&gt;
&lt;/footer&gt;
      </div>
    </div>
    
    <div class="valid-example">
      <h4>✅ Accessible Images dengan Alt Text</h4>
      <div class="code-block">
&lt;!-- Good: Descriptive alt text --&gt;
&lt;img src="chart-sales.png" alt="Bar chart showing sales increased by 25% in Q4 2024"&gt;

&lt;!-- Good: Empty alt untuk decorative images --&gt;
&lt;img src="decorative-border.png" alt=""&gt;

&lt;!-- Bad: Missing alt attribute --&gt;
&lt;!-- &lt;img src="important-chart.png"&gt; --&gt;
      </div>
    </div>
    
    <div class="valid-example">
      <h4>✅ Proper Form Structure</h4>
      <div class="code-block">
&lt;form action="/submit" method="post" novalidate&gt;
  &lt;fieldset&gt;
    &lt;legend&gt;Personal Information&lt;/legend&gt;
    
    &lt;label for="name"&gt;Full Name:&lt;/label&gt;
    &lt;input type="text" id="name" name="name" required 
           aria-describedby="name-help"&gt;
    &lt;span id="name-help"&gt;Enter your full legal name&lt;/span&gt;
    
    &lt;label for="email"&gt;Email:&lt;/label&gt;
    &lt;input type="email" id="email" name="email" required&gt;
    
    &lt;button type="submit"&gt;Submit&lt;/button&gt;
  &lt;/fieldset&gt;
&lt;/form&gt;
      </div>
    </div>
  </div>
  
  <div class="validation-demo">
    <h2>⚠️ Common Validation Errors</h2>
    
    <div class="invalid-example">
      <h4>❌ Missing Required Attributes</h4>
      <div class="code-block">
&lt;!-- Error: img without alt --&gt;
&lt;img src="photo.jpg"&gt;

&lt;!-- Error: input without associated label --&gt;
&lt;input type="text" name="username"&gt;

&lt;!-- Error: Missing lang attribute on html --&gt;
&lt;html&gt;
      </div>
    </div>
    
    <div class="invalid-example">
      <h4>❌ Invalid Nesting</h4>
      <div class="code-block">
&lt;!-- Error: block element inside inline element --&gt;
&lt;span&gt;
  &lt;div&gt;This is invalid nesting&lt;/div&gt;
&lt;/span&gt;

&lt;!-- Error: p inside p --&gt;
&lt;p&gt;
  First paragraph
  &lt;p&gt;Nested paragraph (invalid)&lt;/p&gt;
&lt;/p&gt;
      </div>
    </div>
    
    <div class="warning-example">
      <h4>⚠️ Deprecated Elements (Still valid but not recommended)</h4>
      <div class="code-block">
&lt;!-- Avoid using these deprecated elements --&gt;
&lt;font color="red"&gt;Red text&lt;/font&gt;
&lt;center&gt;Centered text&lt;/center&gt;
&lt;b&gt;Bold text (use &lt;strong&gt; instead)&lt;/b&gt;
&lt;i&gt;Italic text (use &lt;em&gt; instead)&lt;/i&gt;

&lt;!-- Recommended alternatives --&gt;
&lt;span style="color: red;"&gt;Red text&lt;/span&gt;
&lt;div style="text-align: center;"&gt;Centered text&lt;/div&gt;
&lt;strong&gt;Important text&lt;/strong&gt;
&lt;em&gt;Emphasized text&lt;/em&gt;
      </div>
    </div>
  </div>
  
  <div class="validation-demo">
    <h2>🔧 Validation Tools</h2>
    
    <h3>Online Validators</h3>
    <ul>
      <li><strong>W3C Markup Validator:</strong> https://validator.w3.org/</li>
      <li><strong>HTML5 Validator:</strong> https://html5.validator.nu/</li>
      <li><strong>Validator.nu:</strong> Comprehensive HTML5 validation</li>
    </ul>
    
    <h3>Development Tools</h3>
    <ul>
      <li><strong>Browser DevTools:</strong> Built-in validation dan accessibility auditing</li>
      <li><strong>VS Code Extensions:</strong> HTMLHint, auto-close-tag, bracket-pair-colorizer</li>
      <li><strong>Linters:</strong> HTML-validate, htmllint untuk CI/CD integration</li>
    </ul>
    
    <button class="btn" onclick="window.open('https://validator.w3.org/', '_blank')">
      🔍 Open W3C Validator
    </button>
  </div>
</body>
</html>
\end{lstlisting}

\textbf{Hasil di Browser:}
- Contoh struktur HTML5 yang valid dengan semantic elements
- Perbandingan valid vs invalid markup
- Common validation errors dengan penjelasan
- Tools untuk validasi HTML online dan development
- Best practices untuk clean dan valid markup

\subsection{Web Accessibility (WCAG 2.1)}

Web Content Accessibility Guidelines (WCAG) 2.1 menyediakan standar untuk membuat konten web yang accessible:

\begin{itemize}
  \item \textbf{Perceivable}: Informasi harus dapat dipersepsikan oleh semua users
  \item \textbf{Operable}: Interface components harus dapat dioperasikan
  \item \textbf{Understandable}: Informasi dan operation harus dapat dipahami
  \item \textbf{Robust}: Konten harus cukup robust untuk berbagai technologies
  \item \textbf{Level Compliance}: A, AA, AAA levels of accessibility conformance
\end{itemize}

\begin{lstlisting}[caption={WCAG 2.1 Implementation}, basicstyle=\ttfamily\small, frame=single]
<!DOCTYPE html>
<html lang="id">
<head>
  <meta charset="UTF-8">
  <meta name="viewport" content="width=device-width, initial-scale=1.0">
  <title>Accessible Web Design - WCAG 2.1</title>
  <style>
    body { font-family: Arial, sans-serif; line-height: 1.6; max-width: 900px; margin: 0 auto; padding: 20px; background-color: #f8f9fa; }
    .accessibility-demo { background-color: white; border-radius: 8px; padding: 20px; margin-bottom: 20px; box-shadow: 0 2px 8px rgba(0,0,0,0.1); }
    .wcag-principle { background-color: #e7f3ff; border: 1px solid #b3d9ff; border-radius: 8px; padding: 20px; margin: 15px 0; }
    .example-good { border-left: 4px solid #28a745; background-color: #f8fff8; padding: 15px; margin: 10px 0; }
    .example-bad { border-left: 4px solid #dc3545; background-color: #fff8f8; padding: 15px; margin: 10px 0; }
    .skip-link { position: absolute; top: -40px; left: 0; background: #007bff; color: white; padding: 8px; text-decoration: none; z-index: 100; }
    .skip-link:focus { top: 0; }
    .btn { background-color: #007bff; color: white; border: none; padding: 10px 20px; border-radius: 4px; cursor: pointer; }
    .btn:hover { background-color: #0056b3; }
    .btn:focus { outline: 3px solid #0056b3; outline-offset: 2px; }
    .high-contrast { background-color: #000; color: #fff; padding: 15px; border-radius: 4px; }
    .aria-demo { background-color: #f8f9fa; border: 1px solid #e9ecef; padding: 15px; border-radius: 4px; margin: 10px 0; }
  </style>
</head>
<body>
  <!-- Skip to main content link untuk screen reader users -->
  <a href="#main-content" class="skip-link">Skip to main content</a>
  
  <h1>♿ Accessible Web Design (WCAG 2.1)</h1>
  
  <!-- Principle 1: Perceivable -->
  <div class="accessibility-demo">
    <h2>👁️ Principle 1: Perceivable</h2>
    
    <div class="wcag-principle">
      <h3>1.1 Text Alternatives (Level A)</h3>
      
      <div class="example-good">
        <h4>✅ Good: Images dengan Descriptive Alt Text</h4>
        <img src="chart-example.png" alt="Bar chart showing website traffic increased 50% from January to June 2024, with peak in March" style="max-width: 100%; height: auto;">
        <p><small>This image has descriptive alt text that conveys the same information as the visual content.</small></p>
      </div>
      
      <div class="example-good">
        <h4>✅ Good: Decorative Images dengan Empty Alt</h4>
        <img src="decorative-icon.png" alt="" style="width: 50px; height: 50px;">
        <p><small>Decorative images should have empty alt attribute (alt="") so screen readers ignore them.</small></p>
      </div>
      
      <div class="example-bad">
        <h4>❌ Bad: Missing atau Non-descriptive Alt Text</h4>
        <!-- Don't do this -->
        <img src="important-chart.png" alt="Chart">
        <img src="logo.png">
        <p><small>Avoid vague alt text like "chart" or missing alt attributes entirely.</small></p>
      </div>
    </div>
    
    <div class="wcag-principle">
      <h3>1.4 Distinguishable (Level AA)</h3>
      
      <div class="example-good">
        <h4>✅ Good: Sufficient Color Contrast</h4>
        <p style="color: #000000; background-color: #ffffff; padding: 10px;">
          This text has a contrast ratio of 21:1 (exceeds WCAG AAA requirement of 7:1)
        </p>
        <p style="color: #333333; background-color: #ffffff; padding: 10px;">
          This text has a contrast ratio of 12.6:1 (exceeds WCAG AA requirement of 4.5:1)
        </p>
      </div>
      
      <div class="example-bad">
        <h4>❌ Bad: Insufficient Color Contrast</h4>
        <p style="color: #cccccc; background-color: #ffffff; padding: 10px;">
          This light gray text on white background fails WCAG AA (contrast ratio: 1.6:1)
        </p>
        <p style="color: #ffff00; background-color: #ffffff; padding: 10px;">
          This yellow text on white background fails WCAG AA (contrast ratio: 1.2:1)
        </p>
      </div>
      
      <div class="example-good">
        <h4>✅ Good: High Contrast Mode Support</h4>
        <div class="high-contrast">
          <p>This content supports high contrast mode with:</p>
          <ul>
            <li>Solid borders (not just color changes)</li>
            <li>Text labels (not just icons)</li>
            <li>Underlined links (not just color)</li>
          </ul>
        </div>
      </div>
    </div>
  </div>
  
  <!-- Principle 2: Operable -->
  <div class="accessibility-demo">
    <h2>⌨️ Principle 2: Operable</h2>
    
    <div class="wcag-principle">
      <h3>2.1 Keyboard Accessible (Level A)</h3>
      
      <div class="example-good">
        <h4>✅ Good: Fully Keyboard Accessible Interface</h4>
        
        <nav aria-label="Main Navigation">
          <ul style="list-style: none; padding: 0; display: flex; gap: 10px;">
            <li><a href="#home" class="btn" style="text-decoration: none;">Home</a></li>
            <li><a href="#about" class="btn" style="text-decoration: none;">About</a></li>
            <li><a href="#contact" class="btn" style="text-decoration: none;">Contact</a></li>
          </ul>
        </nav>
        
        <form style="margin-top: 20px;">
          <label for="search">Search:</label>
          <input type="text" id="search" name="search" style="padding: 8px; margin-right: 10px;">
          <button type="submit" class="btn">Search</button>
        </form>
        
        <p><small>All interactive elements are accessible via keyboard (Tab, Enter, Space)</small></p>
      </div>
      
      <div class="example-good">
        <h4>✅ Good: Visible Focus Indicators</h4>
        <button class="btn" style="margin-right: 10px;">Button 1</button>
        <button class="btn">Button 2</button>
        <p><small>Focus indicators are clearly visible (blue outline) when navigating with keyboard.</small></p>
      </div>
    </div>
    
    <div class="wcag-principle">
      <h3>2.4 Navigable (Level AA)</h3>
      
      <div class="example-good">
        <h4>✅ Good: Descriptive Page Title dan Headings</h4>
        
        <h1>Main Page Title (H1)</h1>
        <h2>Section Heading (H2)</h2>
        <h3>Subsection Heading (H3)</h3>
        <p>Proper heading hierarchy helps screen reader users navigate content.</p>
        
        <h2>Another Section (H2)</h2>
        <p>Headings should describe the content that follows them.</p>
        
        <nav aria-label="Breadcrumb">
          <ol style="list-style: none; padding: 0;">
            <li style="display: inline;"><a href="/">Home</a> &gt; </li>
            <li style="display: inline;"><a href="/products">Products</a> &gt; </li>
            <li style="display: inline;" aria-current="page">Product Name</li>
          </ol>
        </nav>
      </div>
    </div>
  </div>
  
  <!-- Principle 3: Understandable -->
  <div class="accessibility-demo">
    <h2>🧠 Principle 3: Understandable</h2>
    
    <div class="wcag-principle">
      <h3>3.1 Readable (Level AA)</h3>
      
      <div class="example-good">
        <h4>✅ Good: Language Declaration</h4>
        <div class="aria-demo">
          <p lang="id">Ini adalah teks dalam Bahasa Indonesia.</p>
          <p lang="en">This is text in English.</p>
          <p lang="fr">Ceci est du texte en français.</p>
          <p><small>Language attributes help screen readers pronounce text correctly.</small></p>
        </div>
      </div>
      
      <div class="example-good">
        <h4>✅ Good: Unusual Words dan Abbreviations</h4>
        <p>
          The <abbr title="Web Content Accessibility Guidelines">WCAG</abbr> 
          is a technical standard developed by the 
          <abbr title="World Wide Web Consortium">W3C</abbr>.
        </p>
        <p>
          We use <dfn>progressive enhancement</dfn> (building basic functionality first, then adding enhancements) 
          to ensure accessibility.
        </p>
      </div>
    </div>
    
    <div class="wcag-principle">
      <h3>3.3 Input Assistance (Level AA)</h3>
      
      <div class="example-good">
        <h4>✅ Good: Form Labels, Instructions, dan Error Prevention</h4>
        
        <form>
          <div style="margin-bottom: 15px;">
            <label for="email-accessible">Email Address <span aria-label="required">*</span>:</label>
            <input type="email" id="email-accessible" name="email" required aria-describedby="email-error email-instructions">
            <div id="email-instructions" style="font-size: 14px; color: #666; margin-top: 5px;">
              Format: name@example.com
            </div>
            <div id="email-error" role="alert" style="color: #dc3545; margin-top: 5px; display: none;">
              Please enter a valid email address.
            </div>
          </div>
          
          <div style="margin-bottom: 15px;">
            <label for="phone-accessible">Phone Number:</label>
            <input type="tel" id="phone-accessible" name="phone" pattern="[0-9]{3}-[0-9]{4}-[0-9]{4}" 
                   placeholder="0812-3456-7890" aria-describedby="phone-instructions">
            <div id="phone-instructions" style="font-size: 14px; color: #666; margin-top: 5px;">
              Format: 0812-3456-7890
            </div>
          </div>
          
          <button type="submit" class="btn">Submit</button>
        </form>
      </div>
    </div>
  </div>
  
  <!-- ARIA Attributes -->
  <div class="accessibility-demo">
    <h2>🏷️ ARIA (Accessible Rich Internet Applications)</h2>
    
    <div class="wcag-principle">
      <h3>Common ARIA Attributes</h3>
      
      <div class="example-good">
        <h4>✅ Good: ARIA untuk Dynamic Content</h4>
        
        <!-- Live region untuk announcements -->
        <div role="status" aria-live="polite" aria-atomic="true" id="status-message" style="padding: 10px; background-color: #d4edda; border-radius: 4px; margin: 10px 0;">
          Form submitted successfully!
        </div>
        
        <!-- Modal dengan proper ARIA -->
        <div role="dialog" aria-modal="true" aria-labelledby="modal-title" aria-describedby="modal-desc" 
             style="border: 1px solid #ccc; padding: 20px; border-radius: 8px; background: white; max-width: 400px;">
          <h2 id="modal-title">Confirm Action</h2>
          <p id="modal-desc">Are you sure you want to delete this item?</p>
          <div style="display: flex; gap: 10px; margin-top: 15px;">
            <button class="btn" style="background-color: #dc3545;">Delete</button>
            <button class="btn" style="background-color: #6c757d;">Cancel</button>
          </div>
        </div>
        
        <!-- Custom components dengan ARIA -->
        <div role="tablist" aria-label="Product Information" style="margin-top: 20px;">
          <button role="tab" aria-selected="true" aria-controls="tab-description" id="tab-desc" class="btn">
            Description
          </button>
          <button role="tab" aria-selected="false" aria-controls="tab-specs" id="tab-specs-btn" class="btn">
            Specifications
          </button>
          <button role="tab" aria-selected="false" aria-controls="tab-reviews" id="tab-reviews-btn" class="btn">
            Reviews
          </button>
        </div>
        
        <div role="tabpanel" id="tab-description" aria-labelledby="tab-desc" style="padding: 15px; border: 1px solid #ccc; border-radius: 4px; margin-top: 10px;">
          <p>Product description content goes here.</p>
        </div>
      </div>
      
      <div class="example-good">
        <h4>✅ Good: Landmark Roles</h4>
        <div class="code-block" style="background-color: #f8f9fa; padding: 15px; border-radius: 4px;">
&lt;header role="banner"&gt;
  &lt;nav role="navigation" aria-label="Main"&gt;...&lt;/nav&gt;
&lt;/header&gt;

&lt;main role="main" id="main-content"&gt;
  &lt;article role="article"&gt;...&lt;/article&gt;
&lt;/main&gt;

&lt;aside role="complementary"&gt;...&lt;/aside&gt;

&lt;footer role="contentinfo"&gt;...&lt;/footer&gt;
        </div>
        <p><small>ARIA landmark roles help screen reader users navigate page structure.</small></p>
      </div>
    </div>
  </div>
  
  <!-- Testing Tools -->
  <div class="accessibility-demo">
    <h2>🧪 Accessibility Testing Tools</h2>
    
    <h3>Automated Testing</h3>
    <ul>
      <li><strong>WAVE (Web Accessibility Evaluator):</strong> Browser extension untuk visual feedback</li>
      <li><strong>axe DevTools:</strong> Automated accessibility testing untuk developers</li>
      <li><strong>Lighthouse:</strong> Built-in Chrome DevTools untuk accessibility auditing</li>
      <li><strong>Pa11y:</strong> Command-line accessibility testing tool</li>
    </ul>
    
    <h3>Screen Readers</h3>
    <ul>
      <li><strong>NVDA (NonVisual Desktop Access):</strong> Free screen reader untuk Windows</li>
      <li><strong>JAWS (Job Access With Speech):</strong> Commercial screen reader</li>
      <li><strong>VoiceOver:</strong> Built-in screen reader untuk macOS dan iOS</li>
      <li><strong>TalkBack:</strong> Built-in screen reader untuk Android</li>
    </ul>
    
    <h3>Manual Testing Checklist</h3>
    <ul>
      <li>✅ Navigate entire page menggunakan keyboard only (Tab, Shift+Tab, Enter, Space, Arrow keys)</li>
      <li>✅ Test dengan screen reader (NVDA, VoiceOver, atau TalkBack)</li>
      <li>✅ Periksa color contrast dengan tools seperti WebAIM Contrast Checker</li>
      <li>✅ Zoom page ke 200% dan pastikan content masih readable</li>
      <li>✅ Test dengan various assistive technologies</li>
      <li>✅ Periksa semua images memiliki alt text yang appropriate</li>
      <li>✅ Pastikan form labels terhubung dengan benar ke input fields</li>
    </ul>
    
    <button class="btn" onclick="window.open('https://wave.webaim.org/', '_blank')">
      🔍 Open WAVE Accessibility Tool
    </button>
  </div>
  
  <main id="main-content" class="accessibility-demo">
    <h2>📋 WCAG 2.1 Compliance Checklist</h2>
    
    <div style="display: grid; grid-template-columns: repeat(auto-fit, minmax(300px, 1fr)); gap: 20px;">
      <div style="background-color: #d4edda; padding: 15px; border-radius: 8px;">
        <h3>Level A (Minimum)</h3>
        <ul>
          <li>✓ Text alternatives untuk images</li>
          <li>✓ Keyboard accessible</li>
          <li>✓ Captions/transcripts untuk multimedia</li>
          <li>✓ Color not sole method untuk conveying information</li>
          <li>✓ Form labels present</li>
        </ul>
      </div>
      
      <div style="background-color: #fff3cd; padding: 15px; border-radius: 8px;">
        <h3>Level AA (Recommended)</h3>
        <ul>
          <li>✓ Color contrast ratio minimum 4.5:1</li>
          <li>✓ Text resizing up to 200%</li>
          <li>✓ Consistent navigation</li>
          <li>✓ Error identification and suggestions</li>
          <li>✓ Accessible forms dengan error prevention</li>
        </ul>
      </div>
      
      <div style="background-color: #f8d7da; padding: 15px; border-radius: 8px;">
        <h3>Level AAA (Enhanced)</h3>
        <ul>
          <li>✓ Color contrast ratio 7:1</li>
          <li>✓ Sign language interpretation untuk video</li>
          <li>✓ Extended audio description</li>
          <li>✓ Reading level at lower secondary education</li>
          <li>✓ Context-sensitive help</li>
        </ul>
      </div>
    </div>
    
    <p style="margin-top: 20px; padding: 15px; background-color: #e7f3ff; border-radius: 8px;">
      <strong>💡 Recommendation:</strong> Aim untuk WCAG 2.1 Level AA compliance sebagai minimum standard untuk semua web content.
    </p>
  </main>
</body>
</html>
\end{lstlisting}

\textbf{Hasil di Browser:}
- WCAG 2.1 principles dengan contoh implementasi
- Perbandingan accessible vs inaccessible design patterns
- ARIA attributes untuk dynamic content dan custom components
- Testing tools dan manual testing checklist
- WCAG compliance levels (A, AA, AAA) dengan requirements

Validasi dan aksesibilitas adalah fondasi untuk web yang inklusif dan profesional \cite{wcag-guide}. Dengan mengikuti standar WCAG 2.1 dan menggunakan semantic HTML5, developers dapat menciptakan aplikasi web yang accessible untuk semua pengguna, regardless of their abilities atau teknologi yang mereka gunakan \cite{mdn-accessibility}.


\begin{aktivitas}
  \item Buat form pendaftaran lengkap dengan validasi kustom menggunakan JavaScript API.
  \item Implementasikan penyimpanan data sederhana menggunakan \texttt{localStorage}.
  \item Gunakan audit aksesibilitas di browser untuk memeriksa elemen form yang dibuat.
\end{aktivitas}

\begin{checklist}
  \item Dapat membuat form kompleks dengan berbagai tipe input dan validasi
  \item Memahami penggunaan Web Storage dan Geolocation API
  \item Dapat menerapkan prinsip aksesibilitas pada elemen web
\end{checklist}

\begin{rangkuman}
Bab ini membahas fitur lanjutan HTML5 termasuk form kompleks, validasi keamanan, API browser (Storage, Geolocation, Web Workers), serta pentingnya aksesibilitas dan validasi markup. Fitur ini memungkinkan pembuatan aplikasi web yang interaktif, responsif, dan inklusif \cite{mdn-html}.
\end{rangkuman}

\ifSubfilesClassLoaded{
  \renewcommand{\bibname}{Daftar Pustaka}
  \bibliographystyle{plain}
  \bibliography{references}
}{}
\end{document}
