\documentclass[../main]{subfiles}
\ifSubfilesClassLoaded{\setcounter{chapter}{4}}{}
\begin{document}

\chapter{CSS3: Dasar dan Box Model}

\begin{subcpmk}
  \item Sub-CPMK 2.2: Menerapkan CSS3 untuk selector, box model, layout dasar
\end{subcpmk}

\section{CSS: Selector, Box Model, Warna, dan Typography}

CSS (Cascading Style Sheets) mengontrol tampilan dan layout halaman web \cite{mdn-css}. CSS dipisahkan dari HTML untuk memisahkan struktur konten dan presentasi, memudahkan pemeliharaan dan penggunaan ulang gaya \cite{mdn-learn-css}. Selector menentukan elemen mana yang diberi gaya: selector elemen (\texttt{p}, \texttt{h1}), class (\texttt{.nama-class}), ID (\texttt{\#nama-id}), dan selector gabungan \cite{w3c-css}.

Box model adalah fondasi layout CSS: setiap elemen direpresentasikan sebagai kotak dengan \texttt{content}, \texttt{padding}, \texttt{border}, dan \texttt{margin} \cite{csstricks}. Warna dapat dinyatakan dengan nama, hex (\texttt{\#ff0000}), RGB, atau HSL. Typography diatur dengan \texttt{font-family}, \texttt{font-size}, \texttt{line-height}, dan \texttt{text-align} \cite{mdn-css}. Memahami box model dan tipografi penting untuk desain yang konsisten \cite{webdev}.

\section{Layout Dasar dengan CSS}

Layout dasar CSS meliputi \texttt{display} (block, inline, inline-block), \texttt{position} (static, relative, absolute, fixed), dan \texttt{float} \cite{mdn-css}. Elemen block menempati lebar penuh dan ditumpuk vertikal; elemen inline mengalir dalam baris. \texttt{position: relative} memposisikan relatif terhadap posisi normal; \texttt{position: absolute} relatif terhadap ancestor yang di-position \cite{csstricks}. Layout modern lebih banyak menggunakan Flexbox dan Grid (Bab VI), tetapi pemahaman display dan position tetap berguna untuk kasus spesifik \cite{w3c-css}.


\begin{aktivitas}
  \item Terapkan CSS eksternal ke halaman HTML dari Bab III.
  \item Gunakan box model (padding, border, margin) untuk mengatur jarak elemen.
\end{aktivitas}

\begin{checklist}
  \item Dapat menggunakan selector CSS
  \item Memahami box model
  \item Dapat mengatur warna dan typography
\end{checklist}

\begin{rangkuman}
CSS mengontrol tampilan halaman web melalui selector, box model, warna, typography, dan properti layout. Pemisahan struktur (HTML) dan presentasi (CSS) adalah prinsip dasar desain web \cite{mdn-learn-css}.
\end{rangkuman}

\ifSubfilesClassLoaded{
  \renewcommand{\bibname}{Daftar Pustaka}
  \bibliographystyle{plain}
  \bibliography{references}
}{}
\end{document}
