\section{Membuat Endpoint dan Konsumsi dengan Fetch/AJAX}

Endpoint REST dapat dibuat di PHP atau Node.js: route menerima request, memproses, mengembalikan JSON dengan header \texttt{Content-Type: application/json} \cite{rest-tutorial}, \cite{express}. Dari client, Fetch API digunakan untuk mengirim request HTTP dan menerima respons \cite{mdn-fetch}. Contoh: \texttt{fetch(url).then(r => r.json()).then(data => ...)} \cite{mdn-learn-js}. Fetch mendukung method, header, dan body; cocok untuk GET, POST, PUT, DELETE \cite{mdn-fetch}. Konsumsi API dari client memungkinkan pembaruan konten tanpa reload halaman ( pendekatan SPA-like) \cite{js-info}.
