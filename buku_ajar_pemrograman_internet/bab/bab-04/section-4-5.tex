\section{Storage dan APIs HTML5}

HTML5 memperkenalkan berbagai APIs untuk menyimpan data di client-side dan mengakses fitur device. Web Storage API menyediakan cara yang lebih efisien dan aman untuk menyimpan data dibandingkan cookies \cite{mdn-web-storage}. IndexedDB dan File API memungkinkan penyimpanan data yang lebih kompleks dan struktur \cite{w3schools-html}.

\subsection{LocalStorage dan SessionStorage}

Web Storage API menyediakan dua mekanisme penyimpanan key-value di browser:

\begin{itemize}
  \item \textbf{localStorage}: Data persisten, tidak hilang saat browser ditutup
  \item \textbf{sessionStorage}: Data hanya untuk satu session, hilang saat tab ditutup
  \item \textbf{Storage Limit}: Biasanya 5-10MB per domain
  \item \textbf{Synchronous API}: Operasi storage bersifat synchronous
\end{itemize}

\begin{lstlisting}[caption={Web Storage API}, basicstyle=\ttfamily\small, frame=single]
<!DOCTYPE html>
<html lang="id">
<head>
  <meta charset="UTF-8">
  <meta name="viewport" content="width=device-width, initial-scale=1.0">
  <title>Web Storage API</title>
  <style>
    body { font-family: Arial, sans-serif; line-height: 1.6; max-width: 800px; margin: 0 auto; padding: 20px; background-color: #f8f9fa; }
    .storage-demo { background-color: white; border-radius: 8px; padding: 20px; margin-bottom: 20px; box-shadow: 0 2px 8px rgba(0,0,0,0.1); }
    .form-group { margin-bottom: 15px; }
    label { display: block; margin-bottom: 5px; font-weight: bold; }
    input { width: 100%; padding: 8px; border: 1px solid #ddd; border-radius: 4px; box-sizing: border-box; }
    .btn { background-color: #007bff; color: white; border: none; padding: 10px 20px; border-radius: 4px; cursor: pointer; margin-right: 10px; }
    .btn:hover { background-color: #0056b3; }
    .storage-info { background-color: #e7f3ff; border: 1px solid #b3d9ff; border-radius: 4px; padding: 15px; margin-top: 20px; }
    .data-display { background-color: #f8f9fa; border: 1px solid #e9ecef; border-radius: 4px; padding: 15px; margin-top: 15px; max-height: 200px; overflow-y: auto; }
  </style>
</head>
<body>
  <h1>Web Storage API Demo</h1>
  
  <!-- LocalStorage Demo -->
  <div class="storage-demo">
    <h2>💾 LocalStorage (Persistent)</h2>
    
    <div class="form-group">
      <label for="localKey">Key:</label>
      <input type="text" id="localKey" placeholder="Masukkan key (contoh: username)">
    </div>
    
    <div class="form-group">
      <label for="localValue">Value:</label>
      <input type="text" id="localValue" placeholder="Masukkan value (contoh: JohnDoe)">
    </div>
    
    <div>
      <button class="btn" onclick="saveToLocal()">Simpan ke LocalStorage</button>
      <button class="btn" onclick="loadFromLocal()">Load dari LocalStorage</button>
      <button class="btn" onclick="clearLocal()">Clear LocalStorage</button>
    </div>
    
    <div class="storage-info">
      <h4>Data LocalStorage:</h4>
      <div id="localData" class="data-display">Belum ada data</div>
    </div>
  </div>
  
  <!-- SessionStorage Demo -->
  <div class="storage-demo">
    <h2>⏱️ SessionStorage (Session Only)</h2>
    
    <div class="form-group">
      <label for="sessionKey">Key:</label>
      <input type="text" id="sessionKey" placeholder="Masukkan key">
    </div>
    
    <div class="form-group">
      <label for="sessionValue">Value:</label>
      <input type="text" id="sessionValue" placeholder="Masukkan value">
    </div>
    
    <div>
      <button class="btn" onclick="saveToSession()">Simpan ke SessionStorage</button>
      <button class="btn" onclick="loadFromSession()">Load dari SessionStorage</button>
      <button class="btn" onclick="clearSession()">Clear SessionStorage</button>
    </div>
    
    <div class="storage-info">
      <h4>Data SessionStorage:</h4>
      <div id="sessionData" class="data-display">Belum ada data</div>
    </div>
  </div>
  
  <!-- Storage Info -->
  <div class="storage-demo">
    <h2>📊 Storage Information</h2>
    <div class="storage-info">
      <p><strong>LocalStorage Size:</strong> <span id="localSize">0</span> items</p>
      <p><strong>SessionStorage Size:</strong> <span id="sessionSize">0</span> items</p>
      <p><strong>Remaining Space:</strong> ~5-10MB (browser dependent)</p>
    </div>
    <button class="btn" onclick="updateStorageInfo()">Update Info</button>
  </div>
  
  <script>
    // LocalStorage Functions
    function saveToLocal() {
      const key = document.getElementById('localKey').value;
      const value = document.getElementById('localValue').value;
      
      if (key && value) {
        localStorage.setItem(key, value);
        displayLocalData();
        updateStorageInfo();
        alert('Data disimpan ke LocalStorage!');
      } else {
        alert('Key dan value harus diisi!');
      }
    }
    
    function loadFromLocal() {
      const key = document.getElementById('localKey').value;
      const value = localStorage.getItem(key);
      
      if (value) {
        document.getElementById('localValue').value = value;
        alert('Data ditemukan: ' + value);
      } else {
        alert('Data tidak ditemukan!');
      }
    }
    
    function clearLocal() {
      localStorage.clear();
      displayLocalData();
      updateStorageInfo();
      alert('LocalStorage dibersihkan!');
    }
    
    function displayLocalData() {
      const dataDiv = document.getElementById('localData');
      let html = '';
      
      for (let i = 0; i < localStorage.length; i++) {
        const key = localStorage.key(i);
        const value = localStorage.getItem(key);
        html += `<p><strong>${key}:</strong> ${value}</p>`;
      }
      
      dataDiv.innerHTML = html || 'Belum ada data';
    }
    
    // SessionStorage Functions
    function saveToSession() {
      const key = document.getElementById('sessionKey').value;
      const value = document.getElementById('sessionValue').value;
      
      if (key && value) {
        sessionStorage.setItem(key, value);
        displaySessionData();
        updateStorageInfo();
        alert('Data disimpan ke SessionStorage!');
      } else {
        alert('Key dan value harus diisi!');
      }
    }
    
    function loadFromSession() {
      const key = document.getElementById('sessionKey').value;
      const value = sessionStorage.getItem(key);
      
      if (value) {
        document.getElementById('sessionValue').value = value;
        alert('Data ditemukan: ' + value);
      } else {
        alert('Data tidak ditemukan!');
      }
    }
    
    function clearSession() {
      sessionStorage.clear();
      displaySessionData();
      updateStorageInfo();
      alert('SessionStorage dibersihkan!');
    }
    
    function displaySessionData() {
      const dataDiv = document.getElementById('sessionData');
      let html = '';
      
      for (let i = 0; i < sessionStorage.length; i++) {
        const key = sessionStorage.key(i);
        const value = sessionStorage.getItem(key);
        html += `<p><strong>${key}:</strong> ${value}</p>`;
      }
      
      dataDiv.innerHTML = html || 'Belum ada data';
    }
    
    function updateStorageInfo() {
      document.getElementById('localSize').textContent = localStorage.length;
      document.getElementById('sessionSize').textContent = sessionStorage.length;
    }
    
    // Initialize
    displayLocalData();
    displaySessionData();
    updateStorageInfo();
  </script>
</body>
</html>
\end{lstlisting}

\textbf{Hasil di Browser:}
- Form untuk menyimpan key-value pairs ke LocalStorage dan SessionStorage
- Display data yang tersimpan secara real-time
- Tombol untuk save, load, dan clear storage
- Storage information showing jumlah items
- Demo interaktif dengan JavaScript API

\subsection{IndexedDB}

IndexedDB adalah database NoSQL di browser untuk penyimpanan data struktur yang kompleks:

\begin{itemize}
  \item \textbf{Object Store}: Penyimpanan data seperti tabel dalam database
  \item \textbf{Index}: Untuk pencarian data yang efisien
  \item \textbf{Transactions}: Operasi database dalam transaksi
  \item \textbf{Async API}: Operasi bersifat asynchronous dengan promises
  \item \textbf{Large Storage}: Dapat menyimpan data dalam jumlah besar (hundreds of MB)
\end{itemize}

\begin{lstlisting}[caption={IndexedDB Implementation}, basicstyle=\ttfamily\small, frame=single]
<!DOCTYPE html>
<html lang="id">
<head>
  <meta charset="UTF-8">
  <meta name="viewport" content="width=device-width, initial-scale=1.0">
  <title>IndexedDB Demo</title>
  <style>
    body { font-family: Arial, sans-serif; line-height: 1.6; max-width: 900px; margin: 0 auto; padding: 20px; background-color: #f8f9fa; }
    .db-demo { background-color: white; border-radius: 8px; padding: 20px; margin-bottom: 20px; box-shadow: 0 2px 8px rgba(0,0,0,0.1); }
    .form-group { margin-bottom: 15px; }
    label { display: block; margin-bottom: 5px; font-weight: bold; }
    input { width: 100%; padding: 8px; border: 1px solid #ddd; border-radius: 4px; box-sizing: border-box; }
    .btn { background-color: #007bff; color: white; border: none; padding: 10px 20px; border-radius: 4px; cursor: pointer; margin-right: 10px; }
    .btn:hover { background-color: #0056b3; }
    .data-table { width: 100%; border-collapse: collapse; margin-top: 15px; }
    .data-table th, .data-table td { border: 1px solid #ddd; padding: 10px; text-align: left; }
    .data-table th { background-color: #f2f2f2; }
    .status { padding: 10px; border-radius: 4px; margin-top: 10px; }
    .status.success { background-color: #d4edda; color: #155724; }
    .status.error { background-color: #f8d7da; color: #721c24; }
  </style>
</head>
<body>
  <h1>🗄️ IndexedDB Demo</h1>
  
  <div class="db-demo">
    <h2>Tambah Data Mahasiswa</h2>
    
    <div class="form-group">
      <label for="studentId">ID Mahasiswa:</label>
      <input type="number" id="studentId" placeholder="Contoh: 12345">
    </div>
    
    <div class="form-group">
      <label for="studentName">Nama:</label>
      <input type="text" id="studentName" placeholder="Contoh: John Doe">
    </div>
    
    <div class="form-group">
      <label for="studentEmail">Email:</label>
      <input type="email" id="studentEmail" placeholder="Contoh: john@email.com">
    </div>
    
    <div class="form-group">
      <label for="studentMajor">Jurusan:</label>
      <input type="text" id="studentMajor" placeholder="Contoh: Teknik Informatika">
    </div>
    
    <div>
      <button class="btn" onclick="addStudent()">Tambah Mahasiswa</button>
      <button class="btn" onclick="loadAllStudents()">Load Semua Data</button>
      <button class="btn" onclick="clearAllData()">Hapus Semua Data</button>
    </div>
    
    <div id="status" class="status" style="display: none;"></div>
  </div>
  
  <div class="db-demo">
    <h2>Data Mahasiswa</h2>
    <table class="data-table">
      <thead>
        <tr>
          <th>ID</th>
          <th>Nama</th>
          <th>Email</th>
          <th>Jurusan</th>
          <th>Aksi</th>
        </tr>
      </thead>
      <tbody id="studentsTable">
        <tr>
          <td colspan="5" style="text-align: center; color: #666;">Belum ada data</td>
        </tr>
      </tbody>
    </table>
  </div>
  
  <script>
    let db;
    const DB_NAME = 'StudentDB';
    const STORE_NAME = 'students';
    const DB_VERSION = 1;
    
    // Initialize IndexedDB
    function initDB() {
      return new Promise((resolve, reject) => {
        const request = indexedDB.open(DB_NAME, DB_VERSION);
        
        request.onerror = () => reject(request.error);
        request.onsuccess = () => {
          db = request.result;
          resolve(db);
        };
        
        request.onupgradeneeded = (event) => {
          const database = event.target.result;
          
          // Create object store dengan keyPath 'id'
          if (!database.objectStoreNames.contains(STORE_NAME)) {
            const store = database.createObjectStore(STORE_NAME, { keyPath: 'id' });
            
            // Create indexes untuk pencarian
            store.createIndex('name', 'name', { unique: false });
            store.createIndex('email', 'email', { unique: true });
            store.createIndex('major', 'major', { unique: false });
          }
        };
      });
    }
    
    // Add student to IndexedDB
    async function addStudent() {
      try {
        const id = parseInt(document.getElementById('studentId').value);
        const name = document.getElementById('studentName').value;
        const email = document.getElementById('studentEmail').value;
        const major = document.getElementById('studentMajor').value;
        
        if (!id || !name || !email || !major) {
          showStatus('Semua field harus diisi!', 'error');
          return;
        }
        
        const student = { id, name, email, major, createdAt: new Date().toISOString() };
        
        const transaction = db.transaction([STORE_NAME], 'readwrite');
        const store = transaction.objectStore(STORE_NAME);
        
        await new Promise((resolve, reject) => {
          const request = store.add(student);
          request.onsuccess = () => resolve();
          request.onerror = () => reject(request.error);
        });
        
        showStatus('Mahasiswa berhasil ditambahkan!', 'success');
        clearForm();
        loadAllStudents();
        
      } catch (error) {
        showStatus('Error: ' + error.message, 'error');
      }
    }
    
    // Load all students from IndexedDB
    async function loadAllStudents() {
      try {
        const transaction = db.transaction([STORE_NAME], 'readonly');
        const store = transaction.objectStore(STORE_NAME);
        
        const students = await new Promise((resolve, reject) => {
          const request = store.getAll();
          request.onsuccess = () => resolve(request.result);
          request.onerror = () => reject(request.error);
        });
        
        displayStudents(students);
        
      } catch (error) {
        showStatus('Error loading data: ' + error.message, 'error');
      }
    }
    
    // Display students in table
    function displayStudents(students) {
      const tbody = document.getElementById('studentsTable');
      
      if (students.length === 0) {
        tbody.innerHTML = '<tr><td colspan="5" style="text-align: center; color: #666;">Belum ada data</td></tr>';
        return;
      }
      
      tbody.innerHTML = students.map(student => `
        <tr>
          <td>${student.id}</td>
          <td>${student.name}</td>
          <td>${student.email}</td>
          <td>${student.major}</td>
          <td>
            <button class="btn" style="padding: 5px 10px; font-size: 12px;" 
                    onclick="deleteStudent(${student.id})">Hapus</button>
          </td>
        </tr>
      `).join('');
    }
    
    // Delete student from IndexedDB
    async function deleteStudent(id) {
      try {
        const transaction = db.transaction([STORE_NAME], 'readwrite');
        const store = transaction.objectStore(STORE_NAME);
        
        await new Promise((resolve, reject) => {
          const request = store.delete(id);
          request.onsuccess = () => resolve();
          request.onerror = () => reject(request.error);
        });
        
        showStatus('Mahasiswa berhasil dihapus!', 'success');
        loadAllStudents();
        
      } catch (error) {
        showStatus('Error deleting student: ' + error.message, 'error');
      }
    }
    
    // Clear all data from IndexedDB
    async function clearAllData() {
      try {
        const transaction = db.transaction([STORE_NAME], 'readwrite');
        const store = transaction.objectStore(STORE_NAME);
        
        await new Promise((resolve, reject) => {
          const request = store.clear();
          request.onsuccess = () => resolve();
          request.onerror = () => reject(request.error);
        });
        
        showStatus('Semua data berhasil dihapus!', 'success');
        loadAllStudents();
        
      } catch (error) {
        showStatus('Error clearing data: ' + error.message, 'error');
      }
    }
    
    // Clear form
    function clearForm() {
      document.getElementById('studentId').value = '';
      document.getElementById('studentName').value = '';
      document.getElementById('studentEmail').value = '';
      document.getElementById('studentMajor').value = '';
    }
    
    // Show status message
    function showStatus(message, type) {
      const statusDiv = document.getElementById('status');
      statusDiv.textContent = message;
      statusDiv.className = 'status ' + type;
      statusDiv.style.display = 'block';
      
      setTimeout(() => {
        statusDiv.style.display = 'none';
      }, 3000);
    }
    
    // Initialize database when page loads
    initDB().then(() => {
      console.log('IndexedDB initialized successfully');
      loadAllStudents();
    }).catch(error => {
      console.error('Error initializing IndexedDB:', error);
      showStatus('Error initializing database: ' + error.message, 'error');
    });
  </script>
</body>
</html>
\end{lstlisting}

\textbf{Hasil di Browser:}
- Form untuk menambah data mahasiswa dengan ID, nama, email, dan jurusan
- Data disimpan dalam IndexedDB dengan object store dan indexes
- Tabel menampilkan semua data mahasiswa yang tersimpan
- Fitur CRUD (Create, Read, Update, Delete) operations
- Status notifications untuk feedback user

\subsection{File API}

File API memungkinkan web applications untuk berinteraksi dengan file di device user:

\begin{itemize}
  \item \textbf{File Input}: Mengakses file melalui file input atau drag-and-drop
  \item \textbf{FileReader}: Membaca konten file (text, data URL, binary)
  \item \textbf{FileWriter}: Menulis file (dengan File System Access API)
  \item \textbf{Blob}: Mengolah binary large objects
  \item \textbf{FileList}: Koleksi file dari input atau drag-and-drop
\end{itemize}

\begin{lstlisting}[caption={File API Implementation}, basicstyle=\ttfamily\small, frame=single]
<!DOCTYPE html>
<html lang="id">
<head>
  <meta charset="UTF-8">
  <meta name="viewport" content="width=device-width, initial-scale=1.0">
  <title>File API Demo</title>
  <style>
    body { font-family: Arial, sans-serif; line-height: 1.6; max-width: 900px; margin: 0 auto; padding: 20px; background-color: #f8f9fa; }
    .file-demo { background-color: white; border-radius: 8px; padding: 20px; margin-bottom: 20px; box-shadow: 0 2px 8px rgba(0,0,0,0.1); }
    .drop-zone { border: 2px dashed #ccc; border-radius: 8px; padding: 40px; text-align: center; transition: border-color 0.3s; }
    .drop-zone:hover, .drop-zone.dragover { border-color: #007bff; background-color: #f0f8ff; }
    .file-info { background-color: #e7f3ff; border: 1px solid #b3d9ff; border-radius: 4px; padding: 15px; margin-top: 15px; }
    .preview { margin-top: 15px; max-width: 100%; max-height: 300px; }
    .btn { background-color: #007bff; color: white; border: none; padding: 10px 20px; border-radius: 4px; cursor: pointer; margin-right: 10px; }
    .btn:hover { background-color: #0056b3; }
    .progress-bar { width: 100%; height: 20px; background-color: #e9ecef; border-radius: 10px; overflow: hidden; margin-top: 10px; }
    .progress-fill { height: 100%; background-color: #28a745; width: 0; transition: width 0.3s; }
  </style>
</head>
<body>
  <h1>📁 File API Demo</h1>
  
  <!-- File Input Demo -->
  <div class="file-demo">
    <h2>Upload File dengan File Input</h2>
    
    <input type="file" id="fileInput" accept=".txt,.jpg,.png,.pdf" multiple>
    <button class="btn" onclick="processFiles()">Proses Files</button>
    <button class="btn" onclick="clearFiles()">Clear</button>
    
    <div id="fileInfo" class="file-info" style="display: none;">
      <h4>File Information:</h4>
      <div id="fileDetails"></div>
    </div>
  </div>
  
  <!-- Drag and Drop Demo -->
  <div class="file-demo">
    <h2>Drag and Drop Zone</h2>
    
    <div id="dropZone" class="drop-zone">
      <p>📂 Drop files here atau klik untuk browse</p>
      <input type="file" id="dropInput" style="display: none;" multiple>
    </div>
    
    <div id="dropFileInfo" class="file-info" style="display: none;">
      <h4>Dropped Files:</h4>
      <div id="dropFileDetails"></div>
    </div>
  </div>
  
  <!-- File Preview Demo -->
  <div class="file-demo">
    <h2>File Preview</h2>
    
    <input type="file" id="previewInput" accept="image/*" onchange="previewFile()">
    
    <div id="previewContainer" style="display: none;">
      <h4>Preview:</h4>
      <img id="filePreview" class="preview" alt="File preview">
      <div id="imageInfo" class="file-info" style="margin-top: 10px;"></div>
    </div>
  </div>
  
  <!-- File Reader Demo -->
  <div class="file-demo">
    <h2>File Reader (Text Files)</h2>
    
    <input type="file" id="textInput" accept=".txt,.csv,.json,.js,.html,.css">
    <button class="btn" onclick="readTextFile()">Read File</button>
    
    <div id="textContent" class="file-info" style="display: none; margin-top: 15px;">
      <h4>File Content:</h4>
      <pre id="fileContent" style="background-color: #f8f9fa; padding: 15px; border-radius: 4px; overflow-x: auto;"></pre>
    </div>
  </div>
  
  <script>
    // Process files from file input
    function processFiles() {
      const input = document.getElementById('fileInput');
      const files = input.files;
      
      if (files.length === 0) {
        alert('Pilih file terlebih dahulu!');
        return;
      }
      
      let fileDetails = '';
      
      for (let i = 0; i < files.length; i++) {
        const file = files[i];
        fileDetails += `
          <div style="border: 1px solid #ddd; padding: 10px; margin-bottom: 10px; border-radius: 4px;">
            <strong>Nama:</strong> ${file.name}<br>
            <strong>Tipe:</strong> ${file.type || 'Unknown'}<br>
            <strong>Ukuran:</strong> ${formatFileSize(file.size)}<br>
            <strong>Terakhir diubah:</strong> ${new Date(file.lastModified).toLocaleString()}
          </div>
        `;
      }
      
      document.getElementById('fileDetails').innerHTML = fileDetails;
      document.getElementById('fileInfo').style.display = 'block';
    }
    
    function clearFiles() {
      document.getElementById('fileInput').value = '';
      document.getElementById('fileInfo').style.display = 'none';
      document.getElementById('fileDetails').innerHTML = '';
    }
    
    // Format file size
    function formatFileSize(bytes) {
      if (bytes === 0) return '0 Bytes';
      const k = 1024;
      const sizes = ['Bytes', 'KB', 'MB', 'GB'];
      const i = Math.floor(Math.log(bytes) / Math.log(k));
      return parseFloat((bytes / Math.pow(k, i)).toFixed(2)) + ' ' + sizes[i];
    }
    
    // Drag and Drop functionality
    const dropZone = document.getElementById('dropZone');
    const dropInput = document.getElementById('dropInput');
    
    dropZone.addEventListener('click', () => dropInput.click());
    dropZone.addEventListener('dragover', (e) => {
      e.preventDefault();
      dropZone.classList.add('dragover');
    });
    dropZone.addEventListener('dragleave', () => {
      dropZone.classList.remove('dragover');
    });
    dropZone.addEventListener('drop', (e) => {
      e.preventDefault();
      dropZone.classList.remove('dragover');
      handleDroppedFiles(e.dataTransfer.files);
    });
    dropInput.addEventListener('change', (e) => {
      handleDroppedFiles(e.target.files);
    });
    
    function handleDroppedFiles(files) {
      if (files.length === 0) return;
      
      let fileDetails = '';
      
      for (let i = 0; i < files.length; i++) {
        const file = files[i];
        fileDetails += `
          <div style="border: 1px solid #ddd; padding: 10px; margin-bottom: 10px; border-radius: 4px;">
            <strong>Nama:</strong> ${file.name}<br>
            <strong>Tipe:</strong> ${file.type || 'Unknown'}<br>
            <strong>Ukuran:</strong> ${formatFileSize(file.size)}
          </div>
        `;
      }
      
      document.getElementById('dropFileDetails').innerHTML = fileDetails;
      document.getElementById('dropFileInfo').style.display = 'block';
    }
    
    // Preview image file
    function previewFile() {
      const input = document.getElementById('previewInput');
      const file = input.files[0];
      
      if (!file) return;
      
      if (!file.type.startsWith('image/')) {
        alert('File harus berupa gambar!');
        return;
      }
      
      const reader = new FileReader();
      
      reader.onload = (e) => {
        const img = document.getElementById('filePreview');
        img.src = e.target.result;
        img.style.display = 'block';
        
        // Get image dimensions
        img.onload = () => {
          document.getElementById('imageInfo').innerHTML = `
            <strong>Dimensi:</strong> ${img.naturalWidth} x ${img.naturalHeight}px<br>
            <strong>Ukuran File:</strong> ${formatFileSize(file.size)}<br>
            <strong>Tipe:</strong> ${file.type}
          `;
        };
        
        document.getElementById('previewContainer').style.display = 'block';
      };
      
      reader.readAsDataURL(file);
    }
    
    // Read text file content
    function readTextFile() {
      const input = document.getElementById('textInput');
      const file = input.files[0];
      
      if (!file) {
        alert('Pilih file teks terlebih dahulu!');
        return;
      }
      
      const reader = new FileReader();
      
      reader.onload = (e) => {
        const content = e.target.result;
        document.getElementById('fileContent').textContent = content;
        document.getElementById('textContent').style.display = 'block';
      };
      
      reader.onerror = () => {
        alert('Error reading file!');
      };
      
      reader.readAsText(file);
    }
  </script>
</body>
</html>
\end{lstlisting}

\textbf{Hasil di Browser:}
- File input dengan multiple file selection dan filtering
- Drag-and-drop zone untuk upload files
- File preview untuk image files dengan dimensions
- File reader untuk text files dengan content display
- File information showing name, type, size, dan last modified

Storage dan APIs HTML5 memungkinkan aplikasi web modern untuk menyimpan data secara lokal dan berinteraksi dengan file system \cite{w3schools-html}. Kombinasi LocalStorage, SessionStorage, IndexedDB, dan File API menyediakan fondasi kuat untuk aplikasi web yang rich dan offline-capable \cite{mdn-web-storage}.
