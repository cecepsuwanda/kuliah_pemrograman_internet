\section{Forms Dasar HTML}

Forms adalah elemen fundamental HTML untuk mengumpulkan data dari pengguna. Forms memungkinkan interaksi antara user dan website, mulai dari login form hingga survey kompleks \cite{w3schools-html}. Pemahaman forms adalah kunci untuk web development yang interaktif \cite{mdn-html}.

\subsection{Basic Form Structure}

Form HTML dibangun menggunakan tag-tag berikut:

\begin{itemize}
  \item \texttt{<form>} - Container utama untuk form
  \item \texttt{<input>} - Elemen input untuk berbagai tipe data
  \item \texttt{<label>} - Label untuk accessibility dan user experience
  \item \texttt{<button>} - Tombol untuk submit atau aksi lain
  \item \texttt{<textarea>} - Input teks multi-line
  \item \texttt{<select>} dan \texttt{<option>} - Dropdown/select box
  \item \texttt{<fieldset>} dan \texttt{<legend>} - Grouping related fields
\end{itemize}

\begin{lstlisting}[caption={Contoh Form Dasar}, basicstyle=\ttfamily\small, frame=single]
<!DOCTYPE html>
<html>
<head>
  <title>Contoh Form Dasar</title>
</head>
<body>
  <h1>Demonstrasi HTML Forms</h1>
  
  <!-- Form sederhana -->
  <h3>Form Kontak Sederhana</h3>
  <form action="/submit-contact" method="post">
    <!-- Text Input dengan Label -->
    <p>
      <label for="nama">Nama Lengkap:</label><br>
      <input type="text" id="nama" name="nama" size="30">
    </p>
    
    <p>
      <label for="email">Email:</label><br>
      <input type="text" id="email" name="email" size="30">
    </p>
    
    <!-- Password Input -->
    <p>
      <label for="password">Password:</label><br>
      <input type="password" id="password" name="password" size="30">
    </p>
    
    <!-- Submit Button -->
    <p>
      <input type="submit" value="Kirim">
      <input type="reset" value="Reset">
    </p>
  </form>
  
  <!-- Form dengan berbagai input types -->
  <h3>Form dengan Berbagai Input Types</h3>
  <form action="/register" method="post">
    <fieldset>
      <legend>Informasi Pribadi</legend>
      
      <p>
        <label for="fullname">Nama Lengkap:</label><br>
        <input type="text" id="fullname" name="fullname" maxlength="50">
      </p>
      
      <p>
        <label for="age">Umur:</label><br>
        <input type="text" id="age" name="age" size="5" maxlength="3">
      </p>
      
      <p>
        <label for="gender">Jenis Kelamin:</label><br>
        <input type="radio" id="male" name="gender" value="L">
        <label for="male">Laki-laki</label>
        <input type="radio" id="female" name="gender" value="P">
        <label for="female">Perempuan</label>
      </p>
    </fieldset>
    
    <fieldset>
      <legend>Preferensi</legend>
      
      <p>
        <label for="newsletter">Subscribe Newsletter:</label><br>
        <input type="checkbox" id="newsletter" name="newsletter" value="yes">
        <label for="newsletter">Ya, saya ingin menerima newsletter</label>
      </p>
      
      <p>
        <label for="country">Negara:</label><br>
        <select id="country" name="country">
          <option value="">-- Pilih Negara --</option>
          <option value="ID">Indonesia</option>
          <option value="MY">Malaysia</option>
          <option value="SG">Singapore</option>
          <option value="US">United States</option>
        </select>
      </p>
    </fieldset>
    
    <p>
      <label for="comments">Komentar:</label><br>
      <textarea id="comments" name="comments" rows="4" cols="40">
        Tulis komentar Anda di sini...
      </textarea>
    </p>
    
    <p>
      <input type="submit" value="Daftar">
      <input type="button" value="Batal" onclick="alert('Form dibatalkan!')">
    </p>
  </form>
</body>
</html>
\end{lstlisting}

\textbf{Hasil di Browser:}
- Form dengan fieldset untuk grouping related fields
- Label yang terhubung dengan input untuk accessibility
- Berbagai input types: text, password, radio, checkbox, select, textarea
- Submit dan reset button dengan fungsi berbeda
- Legend untuk fieldset description

\subsection{Input Types Detail}

HTML menyediakan berbagai tipe input untuk keperluan spesifik:

\begin{itemize}
  \item \textbf{Text Inputs}: text, password, hidden
  \item \textbf{Choice Inputs}: radio, checkbox
  \item \textbf{Date/Time}: date, time, datetime-local
  \item \textbf{Numeric}: number, range, tel
  \item \textbf{File Inputs}: file, image
  \item \textbf{Action Inputs}: submit, reset, button
\end{itemize}

\begin{lstlisting}[caption={Berbagai Input Types}, basicstyle=\ttfamily\small, frame=single]
<!DOCTYPE html>
<html>
<head>
  <title>Berbagai Input Types</title>
</head>
<body>
  <h1>Demonstrasi Input Types</h1>
  
  <form action="/process" method="post">
    <!-- Text Inputs -->
    <fieldset>
      <legend>Text Inputs</legend>
      
      <p><label for="username">Username:</label><br>
         <input type="text" id="username" name="username" 
                placeholder="Masukkan username" maxlength="20" required>
      </p>
      
      <p><label for="search">Pencarian:</label><br>
         <input type="search" id="search" name="search" 
                placeholder="Cari..." results="5">
      </p>
      
      <p><label for="url">Website:</label><br>
         <input type="url" id="url" name="url" 
                placeholder="https://www.example.com">
      </p>
      
      <p><label for="email">Email:</label><br>
         <input type="email" id="email" name="email" 
                placeholder="nama@email.com" required>
      </p>
      
      <p><label for="password">Password:</label><br>
         <input type="password" id="password" name="password" 
                placeholder="Minimal 8 karakter" minlength="8" required>
      </p>
      
      <p><label for="hidden">Hidden Field:</label><br>
         <input type="hidden" id="hidden" name="csrf_token" value="abc123">
      </p>
    </fieldset>
    
    <!-- Numeric Inputs -->
    <fieldset>
      <legend>Numeric Inputs</legend>
      
      <p><label for="age">Umur:</label><br>
         <input type="number" id="age" name="age" 
                min="1" max="120" step="1" value="25">
      </p>
      
      <p><label for="range">Rating:</label><br>
         <input type="range" id="range" name="rating" 
                min="1" max="10" value="5" oninput="showValue(this.value)">
         <span id="rangeValue">5</span>
      </p>
      
      <p><label for="phone">Telepon:</label><br>
         <input type="tel" id="phone" name="phone" 
                placeholder="0812-3456-789" pattern="[0-9]{3}-[0-9]{4}-[0-9]{4}">
      </p>
    </fieldset>
    
    <!-- Date/Time Inputs -->
    <fieldset>
      <legend>Date and Time Inputs</legend>
      
      <p><label for="birthdate">Tanggal Lahir:</label><br>
         <input type="date" id="birthdate" name="birthdate" 
                min="1900-01-01" max="2024-12-31">
      </p>
      
      <p><label for="meeting-time">Waktu Meeting:</label><br>
         <input type="time" id="meeting-time" name="meeting_time" 
                min="09:00" max="17:00">
      </p>
      
      <p><label for="event-datetime">Acara:</label><br>
         <input type="datetime-local" id="event-datetime" name="event_datetime">
      </p>
      
      <p><label for="month">Bulan:</label><br>
         <input type="month" id="month" name="month" value="2024-12">
      </p>
      
      <p><label for="week">Minggu:</label><br>
         <input type="week" id="week" name="week" value="2024-W52">
      </p>
    </fieldset>
    
    <!-- Choice Inputs -->
    <fieldset>
      <legend>Choice Inputs</legend>
      
      <p><label for="hobbies">Hobi:</label><br>
         <input type="checkbox" id="reading" name="hobbies[]" value="membaca">
         <label for="reading">Membaca</label><br>
         
         <input type="checkbox" id="sports" name="hobbies[]" value="olahraga">
         <label for="sports">Olahraga</label><br>
         
         <input type="checkbox" id="music" name="hobbies[]" value="musik">
         <label for="music">Musik</label>
      </p>
      
      <p><label for="education">Pendidikan Terakhir:</label><br>
         <input type="radio" id="sma" name="education" value="SMA">
         <label for="sma">SMA</label><br>
         
         <input type="radio" id="s1" name="education" value="S1">
         <label for="s1">S1/Diploma</label><br>
         
         <input type="radio" id="s2" name="education" value="S2">
         <label for="s2">S2/Sarjana</label>
      </p>
    </fieldset>
    
    <!-- File Inputs -->
    <fieldset>
      <legend>File Inputs</legend>
      
      <p><label for="avatar">Avatar:</label><br>
         <input type="file" id="avatar" name="avatar" 
                accept="image/*" capture="user">
      </p>
      
      <p><label for="documents">Upload Dokumen:</label><br>
         <input type="file" id="documents" name="documents[]" 
                accept=".pdf,.doc,.docx" multiple>
      </p>
      
      <p><label for="photo">Ambil Foto:</label><br>
         <input type="image" id="photo" name="photo" 
                accept="image/*" alt="Camera input">
      </p>
    </fieldset>
    
    <!-- Action Buttons -->
    <fieldset>
      <legend>Form Actions</legend>
      
      <p>
         <input type="submit" value="Submit Form" style="background-color: #4CAF50; color: white;">
         <input type="reset" value="Reset Form" style="background-color: #f44336; color: white;">
         <input type="button" value="Batal" onclick="alert('Dibatalkan!')" 
                style="background-color: #ff9800; color: white;">
      </p>
    </fieldset>
  </form>
  
  <script>
    function showValue(value) {
      document.getElementById('rangeValue').textContent = value;
    }
  </script>
</body>
</html>
\end{lstlisting}

\textbf{Hasil di Browser:}
- Text inputs dengan placeholder dan validation
- Numeric inputs dengan min/max/step controls
- Date/time inputs dengan native calendar picker
- Radio buttons untuk single choice
- Checkboxes untuk multiple choices
- File inputs dengan drag-and-drop support
- Styled action buttons dengan different behaviors

\subsection{Form Attributes dan Validation}

HTML5 menyediakan atribut untuk validasi dan user experience:

\begin{itemize}
  \item \textbf{Validation}: required, pattern, min, max, minlength, maxlength
  \item \textbf{User Experience}: placeholder, autofocus, autocomplete, readonly, disabled
  \item \textbf{Form Attributes}: action, method, enctype, target, novalidate
\end{itemize}

\begin{lstlisting}[caption={Form Attributes dan Validation}, basicstyle=\ttfamily\small, frame=single]
<!DOCTYPE html>
<html>
<head>
  <title>Form Validation dan Attributes</title>
  <style>
    .form-group {
      margin-bottom: 15px;
    }
    label {
      display: block;
      margin-bottom: 5px;
      font-weight: bold;
    }
    input, select, textarea {
      width: 100%;
      padding: 8px;
      border: 1px solid #ddd;
      border-radius: 4px;
      box-sizing: border-box;
    }
    input:focus, select:focus, textarea:focus {
      border-color: #4CAF50;
      outline: none;
    }
    .error {
      border-color: #f44336;
    }
    .success {
      border-color: #4CAF50;
    }
  </style>
</head>
<body>
  <h1>Form dengan Validation dan UX Attributes</h1>
  
  <form action="/register" method="post" novalidate>
    <!-- Required Fields -->
    <fieldset>
      <legend>Informasi Wajib</legend>
      
      <div class="form-group">
        <label for="fullname">Nama Lengkap:</label>
        <input type="text" id="fullname" name="fullname" 
               required minlength="3" maxlength="50"
               placeholder="Wajib diisi (3-50 karakter)">
      </div>
      
      <div class="form-group">
        <label for="email">Email:</label>
        <input type="email" id="email" name="email" 
               required placeholder="email@domain.com">
      </div>
      
      <div class="form-group">
        <label for="password">Password:</label>
        <input type="password" id="password" name="password" 
               required minlength="8" maxlength="20"
               placeholder="Minimal 8 karakter">
      </div>
      
      <div class="form-group">
        <label for="confirm-password">Konfirmasi Password:</label>
        <input type="password" id="confirm-password" name="confirm_password" 
               required minlength="8" maxlength="20"
               placeholder="Ketik ulang password">
      </div>
    </fieldset>
    
    <!-- Optional Fields -->
    <fieldset>
      <legend>Informasi Opsional</legend>
      
      <div class="form-group">
        <label for="phone">Telepon:</label>
        <input type="tel" id="phone" name="phone" 
               pattern="[0-9]{3}-[0-9]{4}-[0-9]{4}"
               placeholder="0812-3456-789">
      </div>
      
      <div class="form-group">
        <label for="website">Website:</label>
        <input type="url" id="website" name="website" 
               placeholder="https://www.example.com">
      </div>
      
      <div class="form-group">
        <label for="age">Umur:</label>
        <input type="number" id="age" name="age" 
               min="17" max="100" step="1" value="25">
      </div>
    </fieldset>
    
    <!-- Special Attributes -->
    <fieldset>
      <legend>Form Attributes</legend>
      
      <div class="form-group">
        <label for="search">Pencarian:</label>
        <input type="search" id="search" name="search" 
               autofocus autocomplete="on"
               placeholder="Cari produk..." results="10">
      </div>
      
      <div class="form-group">
        <label for="comments">Komentar Tambahan:</label>
        <textarea id="comments" name="comments" rows="4" 
                  placeholder="Tulis komentar opsional di sini..."></textarea>
      </div>
      
      <div class="form-group">
        <label for="newsletter">
          <input type="checkbox" id="newsletter" name="newsletter" value="yes" checked>
          Saya ingin menerima newsletter dan promo
        </label>
      </div>
    </fieldset>
    
    <!-- Form Actions -->
    <div class="form-group">
      <input type="submit" value="Daftar Sekarang" 
             style="background-color: #4CAF50; color: white; padding: 10px 20px; border: none; border-radius: 4px; cursor: pointer;">
      
      <input type="reset" value="Reset Form" 
             style="background-color: #f44336; color: white; padding: 10px 20px; border: none; border-radius: 4px; cursor: pointer; margin-left: 10px;">
    </div>
  </form>
  
  <script>
    // Client-side validation example
    document.querySelector('form').addEventListener('submit', function(e) {
      const password = document.getElementById('password').value;
      const confirmPassword = document.getElementById('confirm-password').value;
      
      if (password !== confirmPassword) {
        e.preventDefault();
        alert('Password tidak cocok!');
        document.getElementById('confirm-password').classList.add('error');
      } else {
        document.getElementById('confirm-password').classList.add('success');
      }
    });
  </script>
</body>
</html>
\end{lstlisting}

\textbf{Hasil di Browser:}
- Required fields dengan browser validation
- Pattern validation untuk phone format
- Min/max validation untuk numeric input
- Autofocus pada search field
- Checked state untuk newsletter checkbox
- Client-side validation dengan JavaScript
- Styled submit dan reset buttons

\subsection{Contoh Lengkap: Multi-Step Registration Form}

Berikut contoh registration form yang kompleks dengan multiple sections:

\begin{lstlisting}[caption={Multi-Step Registration Form}, basicstyle=\ttfamily\small, frame=single]
<!DOCTYPE html>
<html>
<head>
  <title>Multi-Step Registration</title>
  <style>
    .form-container {
      max-width: 600px;
      margin: 0 auto;
      padding: 20px;
      border: 1px solid #ddd;
      border-radius: 8px;
    }
    .step {
      display: none;
    }
    .step.active {
      display: block;
    }
    .step-indicator {
      text-align: center;
      margin-bottom: 20px;
    }
    .step-dot {
      display: inline-block;
      width: 12px;
      height: 12px;
      border-radius: 50%;
      background-color: #ccc;
      margin: 0 5px;
    }
    .step-dot.active {
      background-color: #4CAF50;
    }
    .form-group {
      margin-bottom: 15px;
    }
    label {
      display: block;
      margin-bottom: 5px;
      font-weight: bold;
    }
    input, select {
      width: 100%;
      padding: 8px;
      border: 1px solid #ddd;
      border-radius: 4px;
    }
    .btn {
      background-color: #4CAF50;
      color: white;
      padding: 10px 20px;
      border: none;
      border-radius: 4px;
      cursor: pointer;
    }
    .btn-secondary {
      background-color: #6c757d;
    }
  </style>
</head>
<body>
  <h1>Form Registrasi Multi-Step</h1>
  
  <div class="form-container">
    <!-- Step Indicators -->
    <div class="step-indicator">
      <span class="step-dot active" id="step1-dot"></span>
      <span class="step-dot" id="step2-dot"></span>
      <span class="step-dot" id="step3-dot"></span>
    </div>
    
    <form action="/register" method="post" id="registrationForm">
      <!-- Step 1: Personal Information -->
      <div class="step active" id="step1">
        <h2>Langkah 1: Informasi Pribadi</h2>
        
        <div class="form-group">
          <label for="firstName">Nama Depan:</label>
          <input type="text" id="firstName" name="firstName" required>
        </div>
        
        <div class="form-group">
          <label for="lastName">Nama Belakang:</label>
          <input type="text" id="lastName" name="lastName" required>
        </div>
        
        <div class="form-group">
          <label for="email">Email:</label>
          <input type="email" id="email" name="email" required>
        </div>
        
        <div class="form-group">
          <label for="phone">Telepon:</label>
          <input type="tel" id="phone" name="phone">
        </div>
        
        <button type="button" class="btn" onclick="nextStep(2)">Selanjutnya</button>
      </div>
      
      <!-- Step 2: Account Information -->
      <div class="step" id="step2">
        <h2>Langkah 2: Informasi Akun</h2>
        
        <div class="form-group">
          <label for="username">Username:</label>
          <input type="text" id="username" name="username" required minlength="4">
        </div>
        
        <div class="form-group">
          <label for="password">Password:</label>
          <input type="password" id="password" name="password" required minlength="8">
        </div>
        
        <div class="form-group">
          <label for="confirmPassword">Konfirmasi Password:</label>
          <input type="password" id="confirmPassword" name="confirmPassword" required>
        </div>
        
        <div class="form-group">
          <label for="securityQuestion">Pertanyaan Keamanan:</label>
          <select id="securityQuestion" name="securityQuestion" required>
            <option value="">-- Pilih Pertanyaan --</option>
            <option value="pet">Nama hewan peliharaan pertama?</option>
            <option value="school">Nama sekolah dasar?</option>
            <option value="city">Kota kelahiran?</option>
          </select>
        </div>
        
        <div class="form-group">
          <label for="securityAnswer">Jawaban:</label>
          <input type="text" id="securityAnswer" name="securityAnswer" required>
        </div>
        
        <button type="button" class="btn btn-secondary" onclick="previousStep(1)">Sebelumnya</button>
        <button type="button" class="btn" onclick="nextStep(3)">Selanjutnya</button>
      </div>
      
      <!-- Step 3: Preferences -->
      <div class="step" id="step3">
        <h2>Langkah 3: Preferensi</h2>
        
        <div class="form-group">
          <label for="newsletter">
            <input type="checkbox" id="newsletter" name="newsletter" value="yes" checked>
            Saya ingin menerima newsletter
          </label>
        </div>
        
        <div class="form-group">
          <label for="notifications">
            <input type="checkbox" id="notifications" name="notifications" value="yes">
            Aktifkan notifikasi email
          </label>
        </div>
        
        <div class="form-group">
          <label for="language">Bahasa Preferensi:</label>
          <select id="language" name="language">
            <option value="id">Bahasa Indonesia</option>
            <option value="en">English</option>
            <option value="zh">中文</option>
          </select>
        </div>
        
        <div class="form-group">
          <label for="timezone">Timezone:</label>
          <select id="timezone" name="timezone">
            <option value="WIB">WIB (GMT+7)</option>
            <option value="WITA">WITA (GMT+8)</option>
            <option value="WIT">WIT (GMT+9)</option>
          </select>
        </div>
        
        <button type="button" class="btn btn-secondary" onclick="previousStep(2)">Sebelumnya</button>
        <button type="submit" class="btn">Daftar Sekarang</button>
      </div>
    </form>
  </div>
  
  <script>
    let currentStep = 1;
    
    function showStep(stepNumber) {
      // Hide all steps
      document.querySelectorAll('.step').forEach(step => {
        step.classList.remove('active');
      });
      
      // Show current step
      document.getElementById('step' + stepNumber).classList.add('active');
      
      // Update indicators
      document.querySelectorAll('.step-dot').forEach(dot => {
        dot.classList.remove('active');
      });
      document.getElementById('step' + stepNumber + '-dot').classList.add('active');
      
      currentStep = stepNumber;
    }
    
    function nextStep(step) {
      if (validateStep(currentStep)) {
        showStep(step);
      }
    }
    
    function previousStep(step) {
      showStep(step);
    }
    
    function validateStep(step) {
      let isValid = true;
      
      if (step === 1) {
        const firstName = document.getElementById('firstName').value;
        const email = document.getElementById('email').value;
        
        if (!firstName || !email) {
          alert('Nama dan email wajib diisi!');
          isValid = false;
        }
      } else if (step === 2) {
        const username = document.getElementById('username').value;
        const password = document.getElementById('password').value;
        const confirmPassword = document.getElementById('confirmPassword').value;
        
        if (!username || !password || password !== confirmPassword) {
          alert('Username dan password harus diisi dan cocok!');
          isValid = false;
        }
      }
      
      return isValid;
    }
  </script>
</body>
</html>
\end{lstlisting}

\textbf{Hasil di Browser:}
- Multi-step form dengan step indicators
- Validasi antar step sebelum lanjut
- Smooth transitions antar sections
- Progress indicators dengan dots
- Form validation yang komprehensif
- User-friendly navigation dengan previous/next buttons

Forms yang baik memungkinkan pengumpulan data yang efektif dan user-friendly \cite{w3schools-html}. Kombinasi berbagai input types, validation attributes, dan user experience features akan menciptakan forms yang optimal untuk berbagai keperluan \cite{mdn-html}.
