\section{Form dan Elemen Input HTML5}

Form HTML memungkinkan pengguna mengirim data ke server \cite{mdn-html}. Elemen \texttt{<form>} membungkus berbagai kontrol input dan mendefinisikan atribut \texttt{action} (URL tujuan) dan \texttt{method} (GET atau POST). Setiap input harus memiliki atribut \texttt{name} agar nilainya dapat dikirim bersama request \cite{mdn-learn-html}. Metode GET menampilkan data di URL dan cocok untuk pencarian; metode POST mengirim data di body request dan cocok untuk data sensitif atau besar \cite{mdn-http}.

HTML5 menyediakan berbagai tipe input melalui atribut \texttt{type}: \texttt{text}, \texttt{email}, \texttt{password}, \texttt{number}, \texttt{date}, \texttt{checkbox}, \texttt{radio}, \texttt{submit}, dan lainnya \cite{whatwg}. Tipe baru seperti \texttt{email}, \texttt{url}, \texttt{number} memberikan validasi dasar di browser. Elemen \texttt{<label>} menghubungkan teks dengan input untuk aksesibilitas; atribut \texttt{for} pada label harus sesuai dengan \texttt{id} pada input \cite{w3c-wai}.

\begin{lstlisting}[caption={Contoh Form Registrasi}, basicstyle=\ttfamily\small, frame=single]
<form action="/register" method="post">
  <label for="nama">Nama:</label>
  <input type="text" id="nama" name="nama" required>
  <label for="email">Email:</label>
  <input type="email" id="email" name="email" required>
  <button type="submit">Daftar</button>
</form>
\end{lstlisting}
