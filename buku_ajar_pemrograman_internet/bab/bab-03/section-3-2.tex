\section{Elemen Semantik, Link, Gambar, dan Tabel}

HTML5 memperkenalkan elemen semantik yang memberikan makna pada struktur konten \cite{mdn-html}. Elemen seperti \texttt{<header>}, \texttt{<nav>}, \texttt{<main>}, \texttt{<article>}, \texttt{<section>}, dan \texttt{<footer>} membantu mesin pencari dan teknologi asistif memahami struktur halaman. Penggunaan elemen semantik juga memperbaiki aksesibilitas sesuai panduan W3C WAI \cite{w3c-wai}.

Link dibuat dengan elemen \texttt{<a>} dan atribut \texttt{href} yang menyatakan URL tujuan \cite{w3c-html}. Gambar dimasukkan dengan \texttt{<img src="..." alt="...">}; atribut \texttt{alt} wajib untuk aksesibilitas. Tabel dibangun dengan \texttt{<table>}, \texttt{<tr>}, \texttt{<th>}, dan \texttt{<td>} untuk menampilkan data tabuler. Setiap elemen memiliki atribut dan perilaku yang dijelaskan dalam dokumentasi WHATWG \cite{whatwg}.

\begin{table}[h]
\centering
\small
\begin{tabular}{|l|p{8cm}|}
\hline
\textbf{Elemen} & \textbf{Fungsi} \\
\hline
\texttt{<header>} & Kepala halaman atau bagian \\
\hline
\texttt{<nav>} & Navigasi utama \\
\hline
\texttt{<main>} & Konten utama halaman \\
\hline
\texttt{<article>} & Konten mandiri (artikel, posting) \\
\hline
\texttt{<section>} & Pengelompokan konten tematik \\
\hline
\texttt{<footer>} & Bagian kaki halaman \\
\hline
\end{tabular}
\caption{Elemen Semantik HTML5}
\end{table}
