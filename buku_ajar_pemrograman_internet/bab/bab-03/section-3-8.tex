\section{Tables dan Lists}

Tables dan lists adalah elemen fundamental HTML untuk mengorganisir data dan informasi secara terstruktur. Tables digunakan untuk data tabular, sedangkan lists untuk item-item yang berurutan \cite{w3schools-html}. Pemahaman yang baik tentang tables dan lists adalah kunci untuk presentasi data yang efektif \cite{mdn-html}.

\subsection{HTML Tables}

Tables dibangun menggunakan tag-tag yang mendefinisikan struktur baris dan kolom:

\begin{itemize}
  \item \texttt{<table>} - Container utama untuk table
  \item \texttt{<tr>} - Table row (baris)
  \item \texttt{<th>} - Table header cell (sel header)
  \item \texttt{<td>} - Table data cell (sel data)
  \item \texttt{<thead>} - Group header rows
  \item \texttt{<tbody>} - Group body rows
  \item \texttt{<tfoot>} - Group footer rows
  \item \texttt{<caption>} - Caption/judul table
\end{itemize}

\begin{lstlisting}[caption={Contoh Basic Table}, basicstyle=\ttfamily\small, frame=single]
<!DOCTYPE html>
<html>
<head>
  <title>Contoh Table Dasar</title>
</head>
<body>
  <h1>Demonstrasi HTML Tables</h1>
  
  <!-- Table sederhana -->
  <h3>Table Sederhana</h3>
  <table border="1">
    <tr>
      <th>Nama</th>
      <th>Umur</th>
      <th>Kota</th>
    </tr>
    <tr>
      <td>Andi</td>
      <td>25</td>
      <td>Jakarta</td>
    </tr>
    <tr>
      <td>Budi</td>
      <td>30</td>
      <td>Bandung</td>
    </tr>
    <tr>
      <td>Cici</td>
      <td>28</td>
      <td>Surabaya</td>
    </tr>
  </table>
  
  <!-- Table dengan caption dan styling -->
  <h3>Table dengan Caption dan Styling</h3>
  <table border="1" cellpadding="5" cellspacing="0" width="100%">
    <caption>Data Karyawan Perusahaan</caption>
    <tr style="background-color: #f0f0f0;">
      <th style="padding: 10px;">ID</th>
      <th style="padding: 10px;">Nama</th>
      <th style="padding: 10px;">Departemen</th>
      <th style="padding: 10px;">Gaji</th>
    </tr>
    <tr>
      <td style="padding: 8px; text-align: center;">001</td>
      <td style="padding: 8px;">John Doe</td>
      <td style="padding: 8px;">IT</td>
      <td style="padding: 8px; text-align: right;">Rp 10.000.000</td>
    </tr>
    <tr style="background-color: #f9f9f9;">
      <td style="padding: 8px; text-align: center;">002</td>
      <td style="padding: 8px;">Jane Smith</td>
      <td style="padding: 8px;">Marketing</td>
      <td style="padding: 8px; text-align: right;">Rp 8.500.000</td>
    </tr>
  </table>
</body>
</html>
\end{lstlisting}

\textbf{Hasil di Browser:}
- Table dengan border dan struktur baris-kolom yang jelas
- Header row dengan background warna berbeda
- Caption di atas table
- Cell padding untuk spacing yang lebih baik
- Data alignment yang sesuai (center untuk ID, right untuk gaji)

\subsection{Advanced Table Features}

Tables memiliki fitur lanjutan untuk struktur yang lebih kompleks:

\begin{itemize}
  \item \textbf{Colspan}: Menggabungkan multiple kolom
  \item \textbf{Rowspan}: Menggabungkan multiple baris
  \item \textbf{Table Sections}: Thead, tbody, tfoot
  \item \textbf{Table Attributes}: Border, cellpadding, cellspacing, width
\end{itemize}

\begin{lstlisting}[caption={Advanced Table Features}, basicstyle=\ttfamily\small, frame=single]
<!DOCTYPE html>
<html>
<head>
  <title>Advanced Table Features</title>
</head>
<body>
  <h1>Advanced Table Demonstrations</h1>
  
  <!-- Table dengan colspan dan rowspan -->
  <h3>Colspan dan Rowspan</h3>
  <table border="1" cellpadding="5" width="100%">
    <tr>
      <th rowspan="2">No</th>
      <th colspan="2">Informasi Mahasiswa</th>
    </tr>
    <tr>
      <th>Nama</th>
      <th>Nilai</th>
    </tr>
    <tr>
      <td>1</td>
      <td>Andi</td>
      <td>A</td>
    </tr>
    <tr>
      <td>2</td>
      <td>Budi</td>
      <td>B</td>
    </tr>
  </table>
  
  <!-- Table dengan thead, tbody, tfoot -->
  <h3>Table Sections</h3>
  <table border="1" cellpadding="5" width="100%">
    <thead>
      <tr style="background-color: #e0e0e0;">
        <th>Produk</th>
        <th>Q1</th>
        <th>Q2</th>
        <th>Q3</th>
        <th>Q4</th>
        <th>Total</th>
      </tr>
    </thead>
    <tbody>
      <tr>
        <td><strong>Laptop</strong></td>
        <td>100</td>
        <td>120</td>
        <td>90</td>
        <td>110</td>
        <td><strong>420</strong></td>
      </tr>
      <tr>
        <td><strong>Smartphone</strong></td>
        <td>80</td>
        <td>95</td>
        <td>110</td>
        <td>125</td>
        <td><strong>410</strong></td>
      </tr>
    </tbody>
    <tfoot>
      <tr style="background-color: #f0f0f0; font-weight: bold;">
        <td>Total</td>
        <td>180</td>
        <td>215</td>
        <td>200</td>
        <td>235</td>
        <td>830</td>
      </tr>
    </tfoot>
  </table>
  
  <!-- Complex table dengan multiple features -->
  <h3>Complex Table Example</h3>
  <table border="1" cellpadding="8" cellspacing="2" width="100%">
    <caption>Jadwal Kuliah Semester Ganjil</caption>
    <thead>
      <tr style="background-color: #4CAF50; color: white;">
        <th rowspan="2">Hari</th>
        <th colspan="2">Pagi</th>
        <th colspan="2">Sore</th>
      </tr>
      <tr style="background-color: #4CAF50; color: white;">
        <th>08:00</th>
        <th>10:00</th>
        <th>13:00</th>
        <th>15:00</th>
      </tr>
    </thead>
    <tbody>
      <tr>
        <td rowspan="2"><strong>Senin</strong></td>
        <td>Matematika</td>
        <td>Fisika</td>
        <td>Kimia</td>
        <td>Biologi</td>
      </tr>
      <tr>
        <td colspan="2" style="text-align: center; font-style: italic;">
          Lab Fisika
        </td>
        <td colspan="2" style="text-align: center; font-style: italic;">
          Lab Kimia
        </td>
      </tr>
      <tr>
        <td rowspan="2"><strong>Selasa</strong></td>
        <td>Bahasa Indonesia</td>
        <td>Bahasa Inggris</td>
        <td>Sejarah</td>
        <td>Geografi</td>
      </tr>
      <tr>
        <td colspan="2" style="text-align: center; font-style: italic;">
          Debat Bahasa
        </td>
        <td colspan="2" style="text-align: center; font-style: italic;">
          Presentasi Sejarah
        </td>
      </tr>
    </tbody>
  </table>
</body>
</html>
\end{lstlisting}

\textbf{Hasil di Browser:}
- Colspan menggabungkan 2 kolom menjadi 1
- Rowspan menggabungkan 2 baris menjadi 1
- Thead dengan background hijau untuk header
- Tbody dengan data utama
- Tfoot dengan total dan background abu-abu
- Complex table structure dengan multiple rowspan/colspan

\subsection{HTML Lists}

Lists digunakan untuk menampilkan item-item yang berurutan atau tidak berurutan:

\begin{itemize}
  \item \textbf{Unordered List}: `<ul>` dengan `<li>` (bullet points)
  \item \textbf{Ordered List}: `<ol>` dengan `<li>` (numbered)
  \item \textbf{Definition List}: `<dl>`, `<dt>`, `<dd>` (term dan definition)
  \item \textbf{Nested Lists}: Lists di dalam lists
\end{itemize}

\begin{lstlisting}[caption={Berbagai Jenis Lists}, basicstyle=\ttfamily\small, frame=single]
<!DOCTYPE html>
<html>
<head>
  <title>Contoh HTML Lists</title>
</head>
<body>
  <h1>Demonstrasi HTML Lists</h1>
  
  <!-- Unordered List -->
  <h3>Unordered List (UL)</h3>
  <p>Fitur HTML5:</p>
  <ul>
    <li>Semantic elements</li>
    <li>Multimedia support</li>
    <li>Form validation</li>
    <li>Local storage</li>
    <li>Geolocation API</li>
  </ul>
  
  <!-- Ordered List -->
  <h3>Ordered List (OL)</h3>
  <p>Langkah belajar web development:</p>
  <ol>
    <li>Learn HTML fundamentals</li>
    <li>Master CSS styling</li>
    <li>Understand JavaScript basics</li>
    <li>Practice with projects</li>
    <li>Build portfolio</li>
  </ol>
  
  <!-- Ordered List dengan type berbeda -->
  <h3>Ordered List Types</h3>
  <p>Popular programming languages:</p>
  <ol type="A">
    <li>JavaScript</li>
    <li>Python</li>
    <li>Java</li>
  </ol>
  
  <ol type="I">
    <li>First place</li>
    <li>Second place</li>
    <li>Third place</li>
  </ol>
  
  <ol type="a">
    <li>Installation guide</li>
    <li>Configuration steps</li>
    <li>Troubleshooting tips</li>
  </ol>
  
  <!-- Definition List -->
  <h3>Definition List (DL)</h3>
  <p>Web development terms:</p>
  <dl>
    <dt>HTML</dt>
    <dd>HyperText Markup Language - bahasa markup untuk membuat halaman web</dd>
    
    <dt>CSS</dt>
    <dd>Cascading Style Sheets - bahasa styling untuk tampilan web</dd>
    
    <dt>JavaScript</dt>
    <dd>Bahasa scripting untuk interaktivitas web</dd>
    
    <dt>API</dt>
    <dd>Application Programming Interface - antarmuka untuk komunikasi antar software</dd>
  </dl>
</body>
</html>
\end{lstlisting}

\textbf{Hasil di Browser:}
- Unordered list dengan bullet points (•)
- Ordered list dengan numbering (1, 2, 3)
- Ordered list dengan letters (A, B, C)
- Ordered list dengan roman numbers (I, II, III)
- Definition list dengan term bold dan definition indented

\subsection{Nested Lists}

Lists dapat disusun secara hierarkis dengan nested lists:

\begin{lstlisting}[caption={Nested Lists}, basicstyle=\ttfamily\small, frame=single]
<!DOCTYPE html>
<html>
<head>
  <title>Nested Lists</title>
</head>
<body>
  <h1>Nested Lists Demonstrations</h1>
  
  <!-- Nested unordered lists -->
  <h3>Teknologi Web Development</h3>
  <ul>
    <li>
      <strong>Frontend Technologies</strong>
      <ul>
        <li>HTML5</li>
        <li>CSS3</li>
        <li>JavaScript ES6+</li>
        <li>
          Frameworks
          <ul>
            <li>React</li>
            <li>Vue.js</li>
            <li>Angular</li>
          </ul>
        </li>
      </ul>
    </li>
    <li>
      <strong>Backend Technologies</strong>
      <ul>
        <li>Node.js</li>
        <li>PHP</li>
        <li>Python</li>
        <li>
          Databases
          <ul>
            <li>MySQL</li>
            <li>MongoDB</li>
            <li>PostgreSQL</li>
          </ul>
        </li>
      </ul>
    </li>
    <li>
      <strong>DevOps Tools</strong>
      <ul>
        <li>Git</li>
        <li>Docker</li>
        <li>CI/CD Pipelines</li>
      </ul>
    </li>
  </ul>
  
  <!-- Mixed nested lists -->
  <h3>Curriculum Web Development</h3>
  <ol>
    <li>
      Semester 1: Fundamentals
      <ul>
        <li>HTML & CSS Basics</li>
        <li>JavaScript Fundamentals</li>
        <li>Responsive Design Principles</li>
      </ul>
    </li>
    <li>
      Semester 2: Intermediate
      <ul>
        <li>Advanced CSS & Frameworks</li>
        <li>DOM Manipulation</li>
        <li>Async JavaScript</li>
      </ul>
    </li>
    <li>
      Semester 3: Advanced
      <ul>
        <li>Full-Stack Development</li>
        <li>Performance Optimization</li>
        <li>Security Best Practices</li>
      </ul>
    </li>
  </ol>
</body>
</html>
\end{lstlisting}

\textbf{Hasil di Browser:}
- Hierarki lists dengan multiple levels
- Berbagai bullet styles untuk setiap level
- Mixed ordered dan unordered lists
- Struktur yang jelas untuk informasi kompleks

\subsection{Contoh Lengkap: Data Dashboard}

Berikut contoh dashboard yang menggabungkan tables dan lists:

\begin{lstlisting}[caption={Dashboard dengan Tables dan Lists}, basicstyle=\ttfamily\small, frame=single]
<!DOCTYPE html>
<html>
<head>
  <title>Data Dashboard</title>
  <style>
    .dashboard {
      font-family: Arial, sans-serif;
      margin: 20px;
    }
    .section {
      margin-bottom: 30px;
      padding: 20px;
      border: 1px solid #ddd;
      border-radius: 8px;
    }
    table {
      width: 100%;
      border-collapse: collapse;
      margin-bottom: 20px;
    }
    th, td {
      padding: 12px;
      text-align: left;
      border-bottom: 1px solid #ddd;
    }
    th {
      background-color: #f2f2f2;
      font-weight: bold;
    }
    .status-active { color: green; }
    .status-pending { color: orange; }
    .status-inactive { color: red; }
  </style>
</head>
<body>
  <div class="dashboard">
    <h1>Project Management Dashboard</h1>
    
    <!-- Summary Table -->
    <div class="section">
      <h2>Project Summary</h2>
      <table>
        <thead>
          <tr>
            <th>Project Name</th>
            <th>Status</th>
            <th>Progress</th>
            <th>Deadline</th>
          </tr>
        </thead>
        <tbody>
          <tr>
            <td>Website Redesign</td>
            <td><span class="status-active">Active</span></td>
            <td>75%</td>
            <td>2024-12-31</td>
          </tr>
          <tr>
            <td>Mobile App</td>
            <td><span class="status-pending">Pending</span></td>
            <td>30%</td>
            <td>2025-01-15</td>
          </tr>
          <tr>
            <td>API Integration</td>
            <td><span class="status-inactive">Inactive</span></td>
            <td>100%</td>
            <td>2024-11-30</td>
          </tr>
        </tbody>
      </table>
    </div>
    
    <!-- Task Lists -->
    <div class="section">
      <h2>Active Tasks</h2>
      <ul>
        <li>
          <strong>Website Redesign Tasks:</strong>
          <ul>
            <li>Homepage mockup approval</li>
            <li>Database optimization</li>
            <li>Performance testing</li>
            <li>User acceptance testing</li>
          </ul>
        </li>
        <li>
          <strong>Mobile App Development:</strong>
          <ul>
            <li>UI/UX design finalization</li>
            <li>Backend API integration</li>
            <li>Testing on multiple devices</li>
          </ul>
        </li>
      </ul>
    </div>
    
    <!-- Team Information -->
    <div class="section">
      <h2>Team Members</h2>
      <table>
        <thead>
          <tr>
            <th>Name</th>
            <th>Role</th>
            <th>Current Tasks</th>
          </tr>
        </thead>
        <tbody>
          <tr>
            <td>John Doe</td>
            <td>Frontend Developer</td>
            <td>Website Redesign</td>
          </tr>
          <tr>
            <td>Jane Smith</td>
            <td>Backend Developer</td>
            <td>API Integration</td>
          </tr>
          <tr>
            <td>Bob Johnson</td>
            <td>UI/UX Designer</td>
            <td>Mobile App Design</td>
          </tr>
        </tbody>
      </table>
    </div>
    
    <!-- Priority Lists -->
    <div class="section">
      <h2>Priority Action Items</h2>
      <ol>
        <li>
          <strong>Critical Priority:</strong>
          <ul>
            <li>Fix production bugs</li>
            <li>Security audit completion</li>
          </ul>
        </li>
        <li>
          <strong>High Priority:</strong>
          <ul>
            <li>Complete website redesign</li>
            <li>Deploy mobile app beta</li>
          </ul>
        </li>
        <li>
          <strong>Medium Priority:</strong>
          <ul>
            <li>Documentation update</li>
            <li>Team training sessions</li>
          </ul>
        </li>
      </ol>
    </div>
  </div>
</body>
</html>
\end{lstlisting}

\textbf{Hasil di Browser:}
- Dashboard dengan 3 sections terorganisir
- Summary table dengan status berwarna
- Task lists dengan hierarki yang jelas
- Team member table dengan informasi lengkap
- Priority action items dengan ordered list
- Kombinasi tables dan lists untuk data management

Tables dan lists yang baik memungkinkan presentasi data yang terstruktur dan mudah dibaca \cite{w3schools-html}. Penggunaan yang tepat akan meningkatkan user experience dan kemudahan dalam mengakses informasi \cite{mdn-html}.
