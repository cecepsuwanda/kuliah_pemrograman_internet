\section{Images dan Media}

Images dan media adalah elemen penting dalam web development yang membuat konten lebih menarik dan informatif. HTML menyediakan tag `<img>` untuk menampilkan gambar dengan berbagai format dan pengaturan \cite{w3schools-html}. Pemahaman yang baik tentang image handling adalah fundamental untuk web development modern \cite{mdn-html}.

\subsection{Basic Image Tag}

Tag `<img>` memiliki atribut penting untuk menampilkan gambar:

\begin{itemize}
  \item \texttt{src} - Source/path file gambar (wajib)
  \item \texttt{alt} - Alternative text (wajib untuk accessibility)
  \item \texttt{width} - Lebar gambar dalam pixels atau persentase
  \item \texttt{height} - Tinggi gambar dalam pixels atau persentase
  \item \texttt{title} - Tooltip saat hover
  \item \texttt{border} - Border gambar
  \item \texttt{align} - Alignment (left, right, center, top, bottom)
\end{itemize}

\begin{lstlisting}[caption={Contoh Basic Image}, basicstyle=\ttfamily\small, frame=single]
<!DOCTYPE html>
<html>
<head>
  <title>Contoh Image Dasar</title>
</head>
<body>
  <h1>Demonstrasi Image HTML</h1>
  
  <!-- Image dengan semua atribut penting -->
  <img src="sunset.jpg" alt="Sunset di pantai" width="400" height="300" 
       title="Pemandangan sunset yang indah">
  
  <!-- Image dengan ukuran berbeda -->
  <h3>Image dengan Ukuran Berbeda:</h3>
  <p>Image kecil:</p>
  <img src="logo.png" alt="Company Logo" width="100" height="50">
  
  <p>Image sedang:</p>
  <img src="product.jpg" alt="Product Image" width="200" height="150">
  
  <p>Image besar:</p>
  <img src="banner.jpg" alt="Website Banner" width="600" height="200">
  
  <!-- Image dengan alignment -->
  <h3>Image Alignment:</h3>
  <p>
    <img src="icon-left.png" alt="Icon" width="50" height="50" align="left">
    Teks ini mengalir di sebelah kanan image. Image di sebelah kiri dengan alignment left.
    Lorem ipsum dolor sit amet, consectetur adipiscing elit. Sed do eiusmod tempor incididunt ut labore.
  </p>
  
  <p>
    <img src="icon-right.png" alt="Icon" width="50" height="50" align="right">
    Teks ini mengalir di sebelah kiri image. Image di sebelah kanan dengan alignment right.
    Lorem ipsum dolor sit amet, consectetur adipiscing elit. Sed do eiusmod tempor incididunt ut labore.
  </p>
</body>
</html>
\end{lstlisting}

\textbf{Hasil di Browser:}
- Gambar ditampilkan dengan ukuran sesuai atribut width/height
- Text mengalir di sekitar gambar sesuai alignment
- Tooltip muncul saat hover pada gambar
- Alt text tidak terlihat tapi penting untuk accessibility dan SEO

\subsection{Image Formats dan Optimization}

HTML mendukung berbagai format gambar dengan karakteristik berbeda:

\begin{itemize}
  \item \textbf{JPEG/JPG}: Cocok untuk foto, ukuran kecil, lossy compression
  \item \textbf{PNG}: Cocok untuk grafis, transparansi, lossless compression
  \item \textbf{GIF}: Animasi, 256 colors, ukuran kecil
  \item \textbf{SVG}: Vector graphics, scalable, tanpa pixelation
  \item \textbf{WebP}: Modern format, ukuran kecil, kualitas baik
\end{itemize}

\begin{lstlisting}[caption={Berbagai Format Image}, basicstyle=\ttfamily\small, frame=single]
<!DOCTYPE html>
<html>
<head>
  <title>Format Image</title>
</head>
<body>
  <h1>Demonstrasi Format Image</h1>
  
  <!-- JPEG untuk foto -->
  <h3>JPEG - Untuk Foto</h3>
  <img src="photo.jpg" alt="Foto landscape" width="300" height="200">
  <p><small>Format: JPEG | Ukuran: 45KB | Cocok untuk foto kompleks</small></p>
  
  <!-- PNG untuk grafis -->
  <h3>PNG - Untuk Grafis</h3>
  <img src="logo.png" alt="Logo transparan" width="200" height="100">
  <p><small>Format: PNG | Ukuran: 12KB | Mendukung transparansi</small></p>
  
  <!-- GIF untuk animasi -->
  <h3>GIF - Untuk Animasi</h3>
  <img src="loading.gif" alt="Loading animation" width="100" height="100">
  <p><small>Format: GIF | Ukuran: 8KB | Mendukung animasi</small></p>
  
  <!-- SVG untuk vector -->
  <h3>SVG - Vector Graphics</h3>
  <img src="icon.svg" alt="SVG Icon" width="100" height="100">
  <p><small>Format: SVG | Ukuran: 2KB | Vector, tidak pecah saat diperbesar</small></p>
  
  <!-- WebP modern format -->
  <h3>WebP - Modern Format</h3>
  <img src="banner.webp" alt="WebP Banner" width="400" height="200">
  <p><small>Format: WebP | Ukuran: 25KB | Format modern, kompresi baik</small></p>
</body>
</html>
\end{lstlisting}

\textbf{Hasil di Browser:}
- JPEG menampilkan foto dengan kualitas baik namun ukuran lebih besar
- PNG menampilkan grafis dengan transparansi (jika ada)
- GIF menampilkan animasi bergerak
- SVG tetap tajam saat diperbesar (vector)
- WebP menampilkan kualitas baik dengan ukuran lebih kecil

\subsection{Responsive Images}

Responsive images memastikan gambar tampil optimal di berbagai ukuran layar:

\begin{itemize}
  \item \textbf{Fluid Images}: Width dalam persentase
  \item \textbf{Srcset}: Multiple sources untuk berbagai resolusi
  \item \textbf{Picture Element}: Art direction untuk berbagai kondisi
\end{itemize}

\begin{lstlisting}[caption={Responsive Images}, basicstyle=\ttfamily\small, frame=single]
<!DOCTYPE html>
<html>
<head>
  <title>Responsive Images</title>
  <style>
    .responsive-img {
      max-width: 100%;
      height: auto;
    }
  </style>
</head>
<body>
  <h1>Responsive Image Techniques</h1>
  
  <!-- Fluid Image dengan CSS -->
  <h3>Fluid Image (CSS)</h3>
  <img src="landscape.jpg" alt="Landscape" class="responsive-img">
  
  <!-- Srcset untuk berbagai resolusi -->
  <h3>Srcset untuk Multiple Resolutions</h3>
  <img src="photo-small.jpg" 
       srcset="photo-small.jpg 480w,
               photo-medium.jpg 800w,
               photo-large.jpg 1200w"
       sizes="(max-width: 600px) 480px,
              (max-width: 900px) 800px,
              1200px"
       alt="Foto responsif">
  
  <!-- Picture Element untuk art direction -->
  <h3>Picture Element untuk Art Direction</h3>
  <picture>
    <source media="(min-width: 900px)" srcset="banner-large.jpg">
    <source media="(min-width: 600px)" srcset="banner-medium.jpg">
    <img src="banner-small.jpg" alt="Banner responsif">
  </picture>
  
  <!-- Picture untuk format berbeda -->
  <h3>Picture untuk Format Support</h3>
  <picture>
    <source srcset="photo.webp" type="image/webp">
    <source srcset="photo.jpg" type="image/jpeg">
    <img src="photo.jpg" alt="Foto dengan fallback">
  </picture>
</body>
</html>
\end{lstlisting}

\textbf{Hasil di Browser:}
- Fluid images menyesuaikan lebar container
- Srcset memilih gambar yang sesuai berdasarkan ukuran layar
- Picture element memilih source berdasarkan media query atau format support
- Gambar optimal untuk setiap device dan kondisi

\subsection{Background Images}

Background images dapat ditambahkan melalui HTML attributes atau CSS:

\begin{itemize}
  \item \textbf{HTML Background}: Atribut background pada body atau table
  \item \textbf{CSS Background}: Lebih fleksibel dan powerful
\end{itemize}

\begin{lstlisting}[caption={Background Images}, basicstyle=\ttfamily\small, frame=single]
<!DOCTYPE html>
<html>
<head>
  <title>Background Images</title>
  <style>
    /* CSS Background untuk body */
    body {
      background-image: url('pattern.jpg');
      background-repeat: repeat;
      background-attachment: fixed;
      background-size: cover;
    }
    
    /* Background untuk div */
    .hero-section {
      background-image: url('hero-bg.jpg');
      background-size: cover;
      background-position: center;
      height: 400px;
      color: white;
      text-align: center;
      padding: 50px;
    }
    
    .pattern-box {
      background-image: url('small-pattern.png');
      background-repeat: repeat;
      border: 2px solid #333;
      padding: 20px;
      margin: 10px;
    }
  </style>
</head>
<body>
  <!-- Background dengan CSS (recommended) -->
  <div class="hero-section">
    <h1>Hero Section dengan Background</h1>
    <p>Background image yang menutupi seluruh area</p>
  </div>
  
  <!-- Pattern background -->
  <h3>Pattern Background:</h3>
  <div class="pattern-box">
    <h4>Box dengan Pattern</h4>
    <p>Pattern diulang untuk mengisi background</p>
  </div>
  
  <!-- HTML Background (deprecated) -->
  <h3>HTML Background Attribute:</h3>
  <table background="table-bg.jpg" border="1">
    <tr>
      <td style="color: white; padding: 10px;">Cell 1</td>
      <td style="color: white; padding: 10px;">Cell 2</td>
    </tr>
  </table>
  
  <p><small>Catatan: HTML background attribute deprecated, gunakan CSS</small></p>
</body>
</html>
\end{lstlisting}

\textbf{Hasil di Browser:}
- Body dengan background pattern yang fixed
- Hero section dengan background image full width
- Box dengan pattern background yang diulang
- Table dengan background (deprecated method)
- CSS background lebih fleksibel dan recommended

\subsection{Image Maps Revisited}

Image maps untuk navigasi interaktif dengan gambar:

\begin{lstlisting}[caption={Interactive Image Map}, basicstyle=\ttfamily\small, frame=single]
<!DOCTYPE html>
<html>
<head>
  <title>Interactive Image Map</title>
</head>
<body>
  <h1>Interactive Campus Map</h1>
  
  <!-- Campus map dengan multiple clickable areas -->
  <img src="campus-map.png" alt="Campus Map" usemap="#campusmap" 
       width="600" height="400" style="border: 1px solid #ccc;">
  
  <map name="campusmap">
    <!-- Gedung Rektorat -->
    <area shape="rect" coords="100,100,200,180" 
          href="rektorat.html" alt="Gedung Rektorat"
          title="Gedung Rektorat">
    
    <!-- Perpustakaan -->
    <area shape="rect" coords="250,150,350,230" 
          href="perpustakaan.html" alt="Perpustakaan"
          title="Perpustakaan">
    
    <!-- Laboratorium -->
    <area shape="circle" coords="450,200,40" 
          href="lab.html" alt="Laboratorium"
          title="Laboratorium Komputer">
    
    <!-- Lapangan Olahraga -->
    <area shape="poly" coords="50,300,150,320,200,280,250,100,280" 
          href="olahraga.html" alt="Lapangan Olahraga"
          title="Lapangan Olahraga">
    
    <!-- Area default -->
    <area shape="default" href="campus-info.html" 
          alt="Informasi Kampus" title="Informasi Umum Kampus">
  </map>
  
  <h3>Legenda Peta:</h3>
  <ul>
    <li><b>Gedung Rektorat:</b> Administrasi utama</li>
    <li><b>Perpustakaan:</b> Sumber belajar dan referensi</li>
    <li><b>Laboratorium:</b> Praktikum dan riset</li>
    <li><b>Lapangan Olahraga:</b> Fasilitas olahraga</li>
  </ul>
</body>
</html>
\end{lstlisting}

\textbf{Hasil di Browser:}
- Peta kampus dengan multiple area yang bisa diklik
- Setiap area memiliki tooltip saat hover
- Cursor berubah pointer saat di area clickable
- Link ke halaman informasi yang relevan

\subsection{Best Practices untuk Images}

Praktik terbaik dalam menggunakan images di website:

\begin{itemize}
  \item \textbf{Alt Text}: Selalu gunakan alt text yang deskriptif
  \item \textbf{File Size}: Optimasi ukuran file untuk loading cepat
  \item \textbf{Responsive}: Gunakan images yang responsif
  \item \textbf{Format}: Pilih format yang sesuai dengan kebutuhan
  \item \textbf{Lazy Loading}: Load images saat dibutuhkan
  \item \textbf{Compression}: Compress images tanpa mengurangi kualitas
\end{itemize}

\begin{lstlisting}[caption={Best Practices Implementation}, basicstyle=\ttfamily\small, frame=single]
<!DOCTYPE html>
<html>
<head>
  <title>Image Best Practices</title>
  <style>
    .image-container {
      margin: 20px 0;
    }
    
    .optimized-image {
      max-width: 100%;
      height: auto;
      border-radius: 8px;
      box-shadow: 0 4px 8px rgba(0,0,0,0.1);
      transition: transform 0.3s ease;
    }
    
    .optimized-image:hover {
      transform: scale(1.05);
    }
    
    .image-caption {
      text-align: center;
      font-style: italic;
      color: #666;
      margin-top: 8px;
    }
  </style>
</head>
<body>
  <h1>Image Best Practices Demo</h1>
  
  <!-- Image dengan alt text deskriptif -->
  <div class="image-container">
    <img src="mountain-landscape.jpg" 
         alt="Pemandangan gunung dengan danau biru di pagi hari, 
               terlihat kabut tipis di puncak gunung dan cahaya matahari 
               yang bersinar melalui awan" 
         class="optimized-image"
         width="600" height="400"
         loading="lazy">
    <p class="image-caption">Gambar 1: Pemandangan alam yang indah</p>
  </div>
  
  <!-- Image dengan figure dan figcaption -->
  <figure class="image-container">
    <img src="city-skyline.jpg" 
         alt="Skyline kota modern di malam hari dengan lampu-lampu gedung 
               yang menerangi jalan-jalan utama" 
         class="optimized-image"
         width="600" height="400"
         loading="lazy">
    <figcaption>Gambar 2: Pemandangan kota di malam hari</figcaption>
  </figure>
  
  <!-- Multiple images gallery -->
  <h3>Image Gallery:</h3>
  <div style="display: grid; grid-template-columns: repeat(auto-fit, minmax(200px, 1fr)); gap: 20px;">
    <div class="image-container">
      <img src="gallery1.jpg" 
           alt="Close-up bunga merah dengan tetesan air" 
           class="optimized-image"
           loading="lazy">
      <p class="image-caption">Bunga Merah</p>
    </div>
    
    <div class="image-container">
      <img src="gallery2.jpg" 
           alt="Jembatan kayu di tengah hutan hijau" 
           class="optimized-image"
           loading="lazy">
      <p class="image-caption">Jembatan Hutan</p>
    </div>
    
    <div class="image-container">
      <img src="gallery3.jpg" 
           alt="Pantai dengan pasir putih dan ombak biru" 
           class="optimized-image"
           loading="lazy">
      <p class="image-caption">Pantai Indah</p>
    </div>
  </div>
  
  <h3>Optimization Notes:</h3>
  <ul>
    <li>✅ Semua images memiliki alt text yang deskriptif</li>
    <li>✅ Menggunakan loading="lazy" untuk performance</li>
    <li>✅ Responsive dengan max-width: 100%</li>
    <li>✅ Optimized untuk web (compressed)</li>
    <li>✅ Semantic dengan figure dan figcaption</li>
  </ul>
</body>
</html>
\end{lstlisting}

\textbf{Hasil di Browser:}
- Images dengan hover effects dan transitions
- Alt text yang deskriptif untuk accessibility
- Lazy loading untuk performance
- Responsive gallery dengan CSS Grid
- Semantic markup dengan figure dan figcaption
- Caption yang informatif untuk setiap gambar

Pemahaman images dan media yang baik memungkinkan penciptaan website yang visual menarik, cepat, dan accessible \cite{w3schools-html}. Kombinasi teknik-teknik ini akan menghasilkan pengalaman visual yang optimal untuk pengguna \cite{mdn-html}.
