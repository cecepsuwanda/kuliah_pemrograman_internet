\section{Multimedia HTML5}

HTML5 merevolusi multimedia web dengan menyediakan elemen-elemen native untuk audio, video, dan grafis tanpa memerlukan plugin pihak ketiga seperti Flash atau Silverlight \cite{mdn-multimedia}. Elemen-elemen multimedia HTML5 memungkinkan pengembangan aplikasi web yang kaya media dengan performa yang lebih baik dan compatibility yang lebih luas \cite{w3schools-html}.

\subsection{Video Element}

Elemen `<video>` adalah cara standar untuk menampilkan video di web tanpa plugin:

\begin{itemize}
  \item \textbf{Atribut Utama}: \texttt{src}, \texttt{controls}, \texttt{width}, \texttt{height}
  \item \textbf{Playback Control}: \texttt{autoplay}, \texttt{loop}, \texttt{muted}, \texttt{preload}
  \item \textbf{Multiple Sources}: \texttt{<source>} untuk berbagai format video
  \item \textbf{Fallback Content}: Konten untuk browser yang tidak support video
  \item \textbf{Styling}: CSS untuk custom controls dan layout
\end{itemize}

\begin{lstlisting}[caption={HTML5 Video Element}, basicstyle=\ttfamily\small, frame=single]
<!DOCTYPE html>
<html lang="id">
<head>
  <meta charset="UTF-8">
  <meta name="viewport" content="width=device-width, initial-scale=1.0">
  <title>HTML5 Video Player</title>
  <style>
    body { font-family: Arial, sans-serif; line-height: 1.6; margin: 0; padding: 20px; background-color: #f4f4f4; }
    .video-container { max-width: 800px; margin: 0 auto; background-color: #000; border-radius: 8px; overflow: hidden; box-shadow: 0 4px 20px rgba(0,0,0,0.3); }
    .video-player { width: 100%; height: 450px; display: block; }
    .video-controls { background-color: rgba(0,0,0,0.7); padding: 15px; text-align: center; }
    .video-info { color: white; margin-bottom: 10px; }
    .video-gallery { display: grid; grid-template-columns: repeat(auto-fit, minmax(300px, 1fr)); gap: 20px; margin-top: 30px; }
    .video-item { background-color: white; border-radius: 8px; overflow: hidden; box-shadow: 0 2px 8px rgba(0,0,0,0.1); }
    .video-item h3 { margin: 0; padding: 15px; background-color: #2c3e50; color: white; }
    .video-item video { width: 100%; height: 200px; object-fit: cover; }
    .custom-controls { display: flex; justify-content: center; gap: 10px; margin-top: 15px; }
    .btn { background-color: #007bff; color: white; border: none; padding: 8px 16px; border-radius: 4px; cursor: pointer; }
    .btn:hover { background-color: #0056b3; }
  </style>
</head>
<body>
  <h1>HTML5 Video Player</h1>
  
  <div class="video-container">
    <video class="video-player" controls poster="video-poster.jpg">
      <source src="videos/tutorial-html5.mp4" type="video/mp4">
      <source src="videos/tutorial-html5.webm" type="video/webm">
      <source src="videos/tutorial-html5.ogv" type="video/ogg">
      Browser Anda tidak mendukung elemen video.
    </video>
    
    <div class="video-controls">
      <div class="video-info">
        <h3>Tutorial HTML5</h3>
        <p>Durasi: 5:30</p>
        <p>Format: MP4/WebM/OGG</p>
      </div>
      
      <div class="custom-controls">
        <button class="btn" onclick="playPause()">Play/Pause</button>
        <button class="btn" onclick="muteUnmute()">Mute/Unmute</button>
        <button class="btn" onclick="changeSpeed()">Kecepatan</button>
        <button class="btn" onclick="toggleFullscreen()">Fullscreen</button>
      </div>
    </div>
  </div>
  
  <h2>Video Gallery</h2>
  <div class="video-gallery">
    <div class="video-item">
      <h3>Video 1: Introduction to HTML5</h3>
      <video controls poster="intro-poster.jpg">
        <source src="videos/intro.mp4" type="video/mp4">
        <source src="videos/intro.webm" type="video/webm">
      </video>
    </div>
    
    <div class="video-item">
      <h3>Video 2: Semantic Elements</h3>
      <video controls poster="semantic-poster.jpg">
        <source src="videos/semantic.mp4" type="video/mp4">
        <source src="videos/semantic.webm" type="video/webm">
      </video>
    </div>
    
    <div class="video-item">
      <h3>Video 3: Form Validation</h3>
      <video controls poster="validation-poster.jpg">
        <source src="videos/validation.mp4" type="video/mp4">
        <source src="videos/validation.webm" type="video/webm">
      </video>
    </div>
  </div>
  
  <script>
    const video = document.querySelector('.video-player');
    
    function playPause() {
      if (video.paused) {
        video.play();
      } else {
        video.pause();
      }
    }
    
    function muteUnmute() {
      video.muted = !video.muted;
    }
    
    function changeSpeed() {
      video.playbackRate = video.playbackRate === 1 ? 1.5 : 1;
    }
    
    function toggleFullscreen() {
      if (!document.fullscreenElement) {
        video.requestFullscreen();
      } else {
        document.exitFullscreen();
      }
    }
  </script>
</body>
</html>
\end{lstlisting}

\textbf{Hasil di Browser:}
- Video player dengan controls standar HTML5
- Multiple video sources untuk compatibility
- Custom JavaScript controls untuk play/pause, mute, speed, fullscreen
- Video gallery dengan responsive grid layout
- Poster image yang ditampilkan sebelum video dimuat

\subsection{Audio Element}

Elemen `<audio>` menyediakan cara standar untuk menampilkan audio di web:

\begin{itemize}
  \item \textbf{Basic Usage}: \texttt{<audio>} dengan \texttt{controls}
  \item \textbf{Multiple Sources}: \texttt{<source>} untuk berbagai format audio
  \item \textbf{Audio Formats}: MP3, WAV, OGG, AAC
  \item \textbf{Attributes}: \texttt{autoplay}, \texttt{loop}, \texttt{preload}, \texttt{volume}
  \item \textbf{JavaScript API}: Kontrol audio melalui JavaScript
\end{itemize}

\begin{lstlisting}[caption={HTML5 Audio Element}, basicstyle=\ttfamily\small, frame=single]
<!DOCTYPE html>
<html lang="id">
<head>
  <meta charset="UTF-8">
  <meta name="viewport" content="width=device-width, initial-scale=1.0">
  <title>HTML5 Audio Player</title>
  <style>
    body { font-family: Arial, sans-serif; line-height: 1.6; margin: 0; padding: 20px; background-color: #f8f9fa; }
    .audio-player { max-width: 600px; margin: 0 auto; background: linear-gradient(135deg, #667eea 0%, #764ba2 100%); border-radius: 12px; padding: 20px; box-shadow: 0 8px 32px rgba(0,0,0,0.1); }
    .audio-controls { display: flex; align-items: center; gap: 15px; margin-bottom: 20px; }
    .audio-info { color: white; text-align: center; }
    .progress-bar { width: 100%; height: 6px; background-color: rgba(255,255,255,0.3); border-radius: 3px; overflow: hidden; margin-bottom: 10px; }
    .progress-fill { height: 100%; background-color: #4CAF50; width: 0; transition: width 0.1s ease; }
    .playlist { background-color: rgba(255,255,255,0.1); border-radius: 8px; padding: 15px; }
    .playlist-item { padding: 10px; border-bottom: 1px solid rgba(255,255,255,0.2); cursor: pointer; transition: background-color 0.3s; }
    .playlist-item:hover { background-color: rgba(255,255,255,0.2); }
    .playlist-item.active { background-color: rgba(255,255,255,0.3); font-weight: bold; }
    .custom-controls { display: flex; justify-content: center; gap: 10px; margin-top: 15px; }
    .btn { background-color: rgba(255,255,255,0.2); color: #333; border: 1px solid rgba(255,255,255,0.3); padding: 8px 16px; border-radius: 4px; cursor: pointer; }
    .btn:hover { background-color: rgba(255,255,255,0.3); }
  </style>
</head>
<body>
  <h1>HTML5 Audio Player</h1>
  
  <div class="audio-player">
    <audio id="audioPlayer" controls>
      <source src="audio/music.mp3" type="audio/mpeg">
      <source src="audio/music.ogg" type="audio/ogg">
      <source src="audio/music.wav" type="audio/wav">
      Browser Anda tidak mendukung elemen audio.
    </audio>
    
    <div class="audio-info">
      <h3>Music Collection</h3>
      <p>Sedang dimainkan: <span id="currentTrack">Unknown</span></p>
      <p>Format: MP3/OGG/WAV</p>
    </div>
    
    <div class="progress-bar">
      <div class="progress-fill" id="progressFill"></div>
    </div>
    
    <div class="playlist">
      <h4>Playlist</h4>
      <div class="playlist-item active" onclick="playTrack(0)">
        <strong>1. Introduction to Web Development</strong><br>
        <small>Duration: 3:45</small>
      </div>
      <div class="playlist-item" onclick="playTrack(1)">
        <strong>2. HTML5 Semantic Elements</strong><br>
        <small>Duration: 5:20</small>
      </div>
      <div class="playlist-item" onclick="playTrack(2)">
        <strong>3. CSS Grid and Flexbox</strong><br>
        <small>Duration: 4:15</small>
      </div>
      <div class="playlist-item" onclick="playTrack(3)">
        <strong>4. JavaScript ES6+</strong><br>
        <small>Duration: 6:30</small>
      </div>
    </div>
    
    <div class="custom-controls">
      <button class="btn" onclick="previousTrack()">Previous</button>
      <button class="btn" onclick="togglePlayPause()">Play/Pause</button>
      <button class="btn" onclick="nextTrack()">Next</button>
      <button class="btn" onclick="toggleMute()">Mute</button>
      <button class="btn" onclick="changeVolume(0.5)">Volume 50%</button>
      <button class="btn" onclick="changeVolume(1)">Volume 100%</button>
    </div>
  </div>
  
  <script>
    const audio = document.getElementById('audioPlayer');
    const tracks = [
      { name: 'Introduction to Web Development', file: 'music.mp3', duration: '3:45' },
      { name: 'HTML5 Semantic Elements', file: 'semantic.mp3', duration: '5:20' },
      { name: 'CSS Grid and Flexbox', file: 'css.mp3', duration: '4:15' },
      { name: 'JavaScript ES6+', file: 'javascript.mp3', duration: '6:30' }
    ];
    let currentTrackIndex = 0;
    
    function playTrack(index) {
      currentTrackIndex = index;
      const track = tracks[index];
      audio.src = track.file;
      document.getElementById('currentTrack').textContent = track.name;
      
      document.querySelectorAll('.playlist-item').forEach((item, i) => {
        item.classList.toggle('active', i === index);
      });
      
      audio.play();
      updateProgress();
    }
    
    function updateProgress() {
      const progressFill = document.getElementById('progressFill');
      const duration = audio.duration;
      const currentTime = audio.currentTime;
      
      if (duration) {
        const progress = (currentTime / duration) * 100;
        progressFill.style.width = progress + '%';
      }
    }
    
    function togglePlayPause() {
      if (audio.paused) {
        audio.play();
      } else {
        audio.pause();
      }
    }
    
    function previousTrack() {
      if (currentTrackIndex > 0) {
        playTrack(currentTrackIndex - 1);
      }
    }
    
    function nextTrack() {
      if (currentTrackIndex < tracks.length - 1) {
        playTrack(currentTrackIndex + 1);
      }
    }
    
    function toggleMute() {
      audio.muted = !audio.muted;
    }
    
    function changeVolume(level) {
      audio.volume = level;
    }
    
    audio.addEventListener('timeupdate', updateProgress);
  </script>
</body>
</html>
\end{lstlisting}

\textbf{Hasil di Browser:}
- Audio player dengan multiple format sources
- Custom playlist dengan track selection
- Progress bar animasi untuk playback progress
- Volume controls dan mute functionality
- Gradient background dan modern styling

\subsection{Canvas API untuk Grafis}

Canvas API menyediakan area drawing 2D yang dapat dimanipulasi dengan JavaScript:

\begin{itemize}
  \item \textbf{Drawing Methods}: \texttt{fillRect()}, \texttt{strokeRect()}, \texttt{drawImage()}
  \item \textbf{Text Rendering}: \texttt{fillText()}, \texttt{strokeText()}, font customization
  \item \textbf{Shapes}: \texttt{arc()}, \texttt{lineTo()}, \texttt{quadraticCurveTo()}
  \item \textbf{Transforms}: \texttt{translate()}, \texttt{rotate()}, \texttt{scale()}
  \item \textbf{Gradients}: Linear dan radial gradients
  \item \textbf{Animation}: \texttt{requestAnimationFrame()} untuk smooth animations
  \item \textbf{Image Manipulation}: \texttt{getImageData()}, \texttt{putImageData()}
\end{itemize}

\begin{lstlisting}[caption={Canvas API untuk Grafis}, basicstyle=\ttfamily\small, frame=single]
<!DOCTYPE html>
<html lang="id">
<head>
  <meta charset="UTF-8">
  <meta name="viewport" content="width=device-width, initial-scale=1.0">
  <title>Canvas API - Drawing Application</title>
  <style>
    body { font-family: Arial, sans-serif; margin: 0; padding: 20px; background-color: #f0f0f0; }
    .canvas-container { max-width: 800px; margin: 0 auto; background-color: white; border-radius: 8px; box-shadow: 0 4px 20px rgba(0,0,0,0.1); padding: 20px; }
    .canvas-controls { display: flex; flex-wrap: wrap; gap: 10px; margin-bottom: 20px; justify-content: center; }
    .control-group { display: flex; flex-direction: column; align-items: center; gap: 5px; }
    .btn { background-color: #007bff; color: white; border: none; padding: 8px 16px; border-radius: 4px; cursor: pointer; font-size: 14px; }
    .btn:hover { background-color: #0056b3; }
    .btn.active { background-color: #28a745; }
    .color-picker { display: flex; align-items: center; gap: 5px; }
    input[type="color"] { width: 50px; height: 40px; border: none; border-radius: 4px; cursor: pointer; }
    .size-slider { display: flex; align-items: center; gap: 5px; }
    input[type="range"] { width: 100px; }
    #drawingCanvas { border: 2px solid #ddd; border-radius: 4px; cursor: crosshair; display: block; margin: 0 auto; }
  </style>
</head>
<body>
  <h1>Canvas Drawing Application</h1>
  
  <div class="canvas-container">
    <h2>Drawing Tools</h2>
    
    <div class="canvas-controls">
      <div class="control-group">
        <label>Drawing Tool:</label>
        <div>
          <button class="btn active" onclick="setTool('pen')">✏️ Pen</button>
          <button class="btn" onclick="setTool('eraser')">🧹 Eraser</button>
          <button class="btn" onclick="setTool('line')">📏 Line</button>
          <button class="btn" onclick="setTool('rectangle')">▢ Rectangle</button>
          <button class="btn" onclick="setTool('circle')">⭕ Circle</button>
          <button class="btn" onclick="setTool('text')">📝 Text</button>
        </div>
      </div>
      
      <div class="control-group">
        <label>Color:</label>
        <div class="color-picker">
          <input type="color" id="strokeColor" value="#000000" title="Stroke Color">
          <input type="color" id="fillColor" value="#ff0000" title="Fill Color">
        </div>
      </div>
      
      <div class="control-group">
        <label>Stroke Width:</label>
        <div class="size-slider">
          <input type="range" id="strokeWidth" min="1" max="20" value="2">
          <span id="strokeWidthValue">2px</span>
        </div>
      </div>
    </div>
    
    <h2>Drawing Canvas</h2>
    <canvas id="drawingCanvas" width="760" height="400"></canvas>
  </div>
  
  <script>
    const canvas = document.getElementById('drawingCanvas');
    const ctx = canvas.getContext('2d');
    
    let isDrawing = false;
    let currentTool = 'pen';
    let strokeColor = '#000000';
    let fillColor = '#ff0000';
    let strokeWidth = 2;
    
    canvas.addEventListener('mousedown', startDrawing);
    canvas.addEventListener('mousemove', draw);
    canvas.addEventListener('mouseup', stopDrawing);
    
    function startDrawing(e) {
      isDrawing = true;
      const rect = canvas.getBoundingClientRect();
      const x = e.clientX - rect.left;
      const y = e.clientY - rect.top;
      
      ctx.beginPath();
      ctx.moveTo(x, y);
    }
    
    function draw(e) {
      if (!isDrawing) return;
      
      const rect = canvas.getBoundingClientRect();
      const x = e.clientX - rect.left;
      const y = e.clientY - rect.top;
      
      ctx.strokeStyle = strokeColor;
      ctx.fillStyle = fillColor;
      ctx.lineWidth = strokeWidth;
      ctx.lineCap = 'round';
      
      if (currentTool === 'pen') {
        ctx.lineTo(x, y);
        ctx.stroke();
      } else if (currentTool === 'line') {
        ctx.lineTo(x, y);
        ctx.stroke();
      } else if (currentTool === 'rectangle') {
        ctx.strokeRect(x - 25, y - 25, 50, 50);
      } else if (currentTool === 'circle') {
        ctx.beginPath();
        ctx.arc(x, y, 25, 0, 2 * Math.PI);
        ctx.stroke();
      }
    }
    
    function stopDrawing() {
      isDrawing = false;
    }
    
    function setTool(tool) {
      currentTool = tool;
      document.querySelectorAll('.btn').forEach(btn => {
        btn.classList.remove('active');
      });
      event.target.classList.add('active');
    }
    
    document.getElementById('strokeColor').addEventListener('change', (e) => {
      strokeColor = e.target.value;
    });
    
    document.getElementById('fillColor').addEventListener('change', (e) => {
      fillColor = e.target.value;
    });
    
    document.getElementById('strokeWidth').addEventListener('input', (e) => {
      strokeWidth = e.target.value;
      document.getElementById('strokeWidthValue').textContent = e.target.value + 'px';
    });
  </script>
</body>
</html>
\end{lstlisting}

\textbf{Hasil di Browser:}
- Interactive drawing canvas dengan multiple tools
- Color picker untuk stroke dan fill colors
- Size slider untuk stroke width
- Drawing tools: pen, eraser, line, rectangle, circle, text

\subsection{Best Practices untuk Multimedia HTML5}

Praktik terbaik dalam implementasi multimedia HTML5:

\begin{itemize}
  \item \textbf{Progressive Enhancement}: Fallback untuk browser lama
  \item \textbf{Format Support}: Multiple sources untuk compatibility
  \item \textbf{Performance Optimization}: Lazy loading dan preloading
  \item \textbf{Responsive Design}: Media yang responsif untuk semua devices
  \item \textbf{Accessibility}: Alternative text dan captions
  \item \textbf{File Size Optimization}: Compressed media untuk loading cepat
  \item \textbf{User Controls}: Intuitive controls dan feedback
\end{itemize}

\begin{lstlisting}[caption={Multimedia Best Practices}, basicstyle=\ttfamily\small, frame=single]
<!DOCTYPE html>
<html lang="id">
<head>
  <meta charset="UTF-8">
  <meta name="viewport" content="width=device-width, initial-scale=1.0">
  <title>Multimedia Best Practices</title>
  <style>
    body { font-family: Arial, sans-serif; line-height: 1.6; margin: 0; padding: 20px; background-color: #f8f9fa; }
    .media-container { max-width: 1200px; margin: 0 auto; }
    .media-section { background-color: white; border-radius: 8px; padding: 20px; margin-bottom: 30px; box-shadow: 0 4px 8px rgba(0,0,0,0.1); }
    .media-section h2 { color: #2c3e50; border-bottom: 2px solid #4CAF50; padding-bottom: 10px; }
    .video-grid { display: grid; grid-template-columns: repeat(auto-fit, minmax(300px, 1fr)); gap: 20px; }
    .video-item { position: relative; border-radius: 8px; overflow: hidden; box-shadow: 0 2px 8px rgba(0,0,0,0.1); }
    .video-item video { width: 100%; height: 200px; object-fit: cover; }
    .video-overlay { position: absolute; top: 0; left: 0; right: 0; bottom: 0; background: rgba(0,0,0,0.7); display: flex; align-items: center; justify-content: center; opacity: 0; transition: opacity 0.3s; }
    .play-button { position: absolute; top: 50%; left: 50%; transform: translate(-50%, -50%); background-color: rgba(255,255,255,0.9); color: #333; border: none; padding: 10px 20px; border-radius: 50%; cursor: pointer; font-size: 24px; opacity: 0; transition: opacity 0.3s; }
  </style>
</head>
<body>
  <h1>Multimedia HTML5 Best Practices</h1>
  
  <div class="media-container">
    <div class="media-section">
      <h2>📹 Responsive Video Gallery</h2>
      <p>Video dengan lazy loading dan responsive design:</p>
      
      <div class="video-grid">
        <div class="video-item">
          <video controls preload="metadata" poster="video1-poster.jpg">
            <source src="https://commondatastorage.googleapis.com/gtv-videos-bucket/sample/BigBuckBunny.mp4" 
                   type="video/mp4" media="(min-width: 800px)">
            <source src="https://commondatastorage.googleapis.com/gtv-videos-bucket/sample/BigBuckBunny.mp4" 
                   type="video/mp4" media="(max-width: 799px)">
            <track label="Indonesian" src="subtitles-id.vtt" kind="subtitles" srclang="id" default>
            Browser Anda tidak mendukung elemen video.
          </video>
          <div class="video-overlay">
            <div class="play-button">▶️</div>
          </div>
        </div>
      </div>
    </div>
  </div>
  
  <script>
    // Lazy loading untuk video
    const videos = document.querySelectorAll('video');
    const videoObserver = new IntersectionObserver((entries) => {
      entries.forEach(entry => {
        const video = entry.target;
        if (entry.isIntersecting) {
          video.src = video.dataset.src;
        }
      });
    });
    
    videos.forEach(video => {
      videoObserver.observe(video);
    });
  </script>
</body>
</html>
\end{lstlisting}

\textbf{Hasil di Browser:}
- Responsive video gallery dengan lazy loading
- Video dengan multiple sources dan media queries
- Intersection Observer untuk lazy loading
- Modern design dengan overlays dan transitions

Multimedia HTML5 memberikan fondasi kuat untuk aplikasi web modern yang kaya media \cite{w3schools-html}. Dengan pemahaman video, audio, dan canvas API yang baik, developer dapat menciptakan pengalaman multimedia yang interaktif, performant, dan accessible \cite{mdn-html}.
