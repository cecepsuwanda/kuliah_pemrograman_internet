\documentclass[../main]{subfiles}
\ifSubfilesClassLoaded{\setcounter{chapter}{2}}{}
\begin{document}

\chapter{HTML5 Dasar: Struktur, Teks, dan Media}

\begin{subcpmk}
  \item Sub-CPMK 2.1: Membuat halaman web dengan HTML5 (struktur, teks, media, semantic elements)
\end{subcpmk}

\section{Struktur Dasar Dokumen HTML5}

HTML (HyperText Markup Language) adalah bahasa markup standar untuk membuat halaman web \cite{mdn-html}. HTML5 adalah versi terbaru dari spesifikasi HTML yang dikembangkan oleh WHATWG dan W3C, menyediakan elemen semantik baru, dukungan media, dan API yang lebih kaya \cite{whatwg}. Setiap halaman web dibangun dari dokumen HTML yang mendefinisikan struktur dan konten yang akan ditampilkan browser.

Dokumen HTML5 minimal terdiri dari deklarasi tipe dokumen \texttt{<!DOCTYPE html>}, elemen \texttt{<html>} sebagai akar, \texttt{<head>} untuk metadata, dan \texttt{<body>} untuk konten yang terlihat \cite{mdn-learn-html}. Bagian \texttt{<head>} biasanya berisi judul halaman, meta charset untuk encoding UTF-8, dan referensi ke file CSS atau script eksternal. Bagian \texttt{<body>} berisi semua konten yang akan ditampilkan kepada pengguna.

\begin{contoh}
Contoh struktur dasar dokumen HTML5 \cite{w3schools}:
\end{contoh}

\begin{lstlisting}[caption={Struktur Dasar HTML5}, basicstyle=\ttfamily\small, frame=single]
<!DOCTYPE html>
<html lang="id">
<head>
  <meta charset="UTF-8">
  <title>Halaman Pertama</title>
</head>
<body>
  <h1>Selamat Datang</h1>
  <p>Ini adalah paragraf pertama.</p>
</body>
</html>
\end{lstlisting}

\section{Tag-Tag Fundamental HTML}

HTML (HyperText Markup Language) menggunakan tag-tag untuk mendefinisikan struktur dan format konten web. Tag HTML ditulis dalam kurung sudut `<tag>` dan biasanya berpasangan dengan tag penutup `</tag>` \cite{w3schools-html}. Pemahaman tag-tag fundamental ini adalah dasar untuk semua pengembangan web \cite{mdn-html}.

\subsection{Heading Tags}

Heading tags digunakan untuk membuat hierarki konten dari yang paling penting hingga yang kurang penting:

\begin{itemize}
  \item \texttt{<h1>} - Heading paling penting (biasanya judul utama)
  \item \texttt{<h2>} - Subheading penting
  \item \texttt{<h3>} - Subheading level 3
  \item \texttt{<h4>} - Subheading level 4
  \item \texttt{<h5>} - Subheading level 5
  \item \texttt{<h6>} - Subheading paling kecil
\end{itemize}

\begin{lstlisting}[caption={Contoh Heading Tags}, basicstyle=\ttfamily\small, frame=single]
<!DOCTYPE html>
<html>
<head>
  <title>Contoh Heading</title>
</head>
<body>
  <h1>Judul Utama Halaman</h1>
  <h2>Bab 1: Pengenalan HTML</h2>
  <h3>1.1 Sejarah HTML</h3>
  <h4>1.1.1 HTML 1.0</h4>
  <h5>1.1.1.1 Fitur Awal</h5>
  <h6>1.1.1.1.1 Detail Teknis</h6>
</body>
</html>
\end{lstlisting}

\textbf{Hasil di Browser:}
Browser akan menampilkan teks dengan ukuran font yang berbeda-beda:
- `<h1>`: Font terbesar dan tebal
- `<h6>`: Font terkecil dan kurang tebal
- Setiap heading otomatis memiliki line break sebelum dan sesudahnya

\subsection{Paragraph dan Line Break}

Tag paragraph dan line break mengatur alur teks dalam dokumen:

\begin{itemize}
  \item \texttt{<p>} - Membuat paragraf baru (dengan jarak spasi)
  \item \texttt{<br>} - Line break (pindah baris tanpa spasi)
  \item \texttt{<hr>} - Horizontal rule (garis horizontal pembatas)
\end{itemize}

\begin{lstlisting}[caption={Contoh Paragraph dan Line Break}, basicstyle=\ttfamily\small, frame=single]
<!DOCTYPE html>
<html>
<head>
  <title>Contoh Paragraph</title>
</head>
<body>
  <p>Ini adalah paragraf pertama. 
  Paragraf akan otomatis wrap text dan memiliki jarak dengan paragraf lain.</p>
  
  <p>Ini paragraf kedua.<br>
  Teks ini pindah ke baris baru dengan tag br.<br>
  Dan ini baris ketiga.</p>
  
  <hr>
  
  <p>Garis horizontal di atas memisahkan konten.</p>
</body>
</html>
\end{lstlisting}

\textbf{Hasil di Browser:}
- Setiap `<p>` memiliki jarak vertikal antar paragraf
- `<br>` membuat teks pindah baris tanpa jarak tambahan
- `<hr>` menampilkan garis horizontal pembatas konten

\subsection{Divisions dan Spans}

Tag `<div>` dan `<span>` adalah container untuk mengelompokkan elemen:

\begin{itemize}
  \item \texttt{<div>} - Block-level container (mengambil full width)
  \item \texttt{<span>} - Inline container (mengambil width sesuai konten)
\end{itemize}

\begin{lstlisting}[caption={Contoh Div dan Span}, basicstyle=\ttfamily\small, frame=single]
<!DOCTYPE html>
<html>
<head>
  <title>Contoh Div dan Span</title>
</head>
<body>
  <div style="background-color: lightblue; padding: 10px;">
    <h3>Container Div</h3>
    <p>Div adalah block-level element yang mengambil full width.</p>
  </div>
  
  <p>Teks normal dengan <span style="color: red; font-weight: bold;">span merah tebal</span> di dalamnya.</p>
  
  <div style="border: 1px solid black; margin: 10px;">
    <span style="background-color: yellow;">Span 1</span>
    <span style="background-color: lightgreen;">Span 2</span>
    <span style="background-color: orange;">Span 3</span>
  </div>
</body>
</html>
\end{lstlisting}

\textbf{Hasil di Browser:}
- `<div>` menampilkan kotak dengan background biru muda
- Teks dalam `<span>` memiliki warna dan style berbeda
- `<div>` kedua menampilkan 3 span dengan background berbeda dalam satu baris

\subsection{HTML Comments}

Comments digunakan untuk dokumentasi dan tidak akan ditampilkan di browser:

\begin{lstlisting}[caption={Contoh HTML Comments}, basicstyle=\ttfamily\small, frame=single]
<!DOCTYPE html>
<html>
<head>
  <title>Contoh Comments</title>
</head>
<body>
  <!-- Ini adalah comment yang tidak akan ditampilkan -->
  <h1>Judul Halaman</h1>
  
  <!-- 
    Comment bisa 
    multiple lines
    untuk dokumentasi
  -->
  
  <p>Ini paragraf yang akan ditampilkan.</p>
  <!-- <p>Paragraf ini di-comment, tidak akan muncul</p> -->
</body>
</html>
\end{lstlisting}

\textbf{Hasil di Browser:}
- Hanya `<h1>` dan `<p>` pertama yang ditampilkan
- Semua comments tidak terlihat di browser
- Berguna untuk debugging dan dokumentasi kode

\subsection{Contoh Lengkap: Halaman Web Sederhana}

Berikut contoh lengkap halaman web menggunakan tag-tag fundamental HTML:

\begin{lstlisting}[caption={Halaman Web Lengkap dengan Tag Fundamental}, basicstyle=\ttfamily\small, frame=single]
<!DOCTYPE html>
<html>
<head>
  <title>Toko Buku Online</title>
</head>
<body>
  <!-- Header Section -->
  <div style="text-align: center; background-color: #f0f0f0; padding: 20px;">
    <h1>Toko Buku "ilmu"</h1>
    <hr>
    <p>Koleksi buku berkualitas untuk semua usia</p>
  </div>
  
  <!-- Main Content -->
  <div style="padding: 20px;">
    <h2>Buku Terlaris Minggu Ini</h2>
    
    <div style="border: 1px solid #ccc; margin: 10px; padding: 10px;">
      <h3>JavaScript untuk Pemula</h3>
      <p>Penulis: <span style="font-style: italic;">John Doe</span></p>
      <p>Harga: <span style="color: red; font-weight: bold;">Rp 150.000</span></p>
      <p>Deskripsi: Buku ini cocok untuk pemula yang ingin belajar JavaScript dari dasar.</p>
    </div>
    
    <div style="border: 1px solid #ccc; margin: 10px; padding: 10px;">
      <h3>HTML5 dan CSS3 Modern</h3>
      <p>Penulis: <span style="font-style: italic;">Jane Smith</span></p>
      <p>Harga: <span style="color: red; font-weight: bold;">Rp 200.000</span></p>
      <p>Deskripsi: Panduan lengkap membuat website modern dengan HTML5 dan CSS3.</p>
    </div>
  </div>
  
  <!-- Footer -->
  <div style="background-color: #333; color: white; text-align: center; padding: 10px;">
    <p>&copy; 2024 Toko Buku "ilmu". All rights reserved.</p>
    <br>
    <p>Contact: info@tokoilmu.com | Phone: 021-12345678</p>
  </div>
</body>
</html>
\end{lstlisting}

\textbf{Hasil di Browser:}
- Header dengan background abu-abu dan judul toko
- Konten utama dengan 2 card buku terlaris
- Footer dengan background gelap dan informasi kontak
- Layout terstruktur dengan penggunaan `<div>` dan styling inline

Pemahaman tag-tag fundamental ini adalah fondasi untuk membangun halaman web yang kompleks dan terstruktur \cite{w3schools-html}.

\section{HTML5 Semantic Elements Lengkap}

HTML5 memperkenalkan elemen-elemen semantik yang memberikan makna struktur pada dokumen web. Elemen semantik tidak hanya mengatur tampilan, tetapi juga memberikan konteks pada konten untuk browser, search engine, dan teknologi assistif \cite{mdn-html}. Penggunaan elemen semantik yang tepat adalah fondasi web modern yang accessible dan SEO-friendly \cite{w3schools-html}.

\subsection{Semantic Structural Elements}

Elemen struktural semantik mendefinisikan bagian-bagian utama halaman:

\begin{itemize}
  \item \texttt{<header>} - Kepala halaman atau section
  \item \texttt{<nav>} - Navigasi utama website
  \item \texttt{<main>} - Konten utama dan unik halaman
  \item \texttt{<article>} - Konten mandiri yang berdiri sendiri
  \item \texttt{<section>} - Pengelompokan konten tematik
  \item \texttt{<aside>} - Konten samping/tambahan
  \item \texttt{<footer>} - Kaki halaman atau section
\end{itemize}

\begin{lstlisting}[caption={Contoh Semantic Structure}, basicstyle=\ttfamily\small, frame=single]
<!DOCTYPE html>
<html lang="id">
<head>
  <meta charset="UTF-8">
  <meta name="viewport" content="width=device-width, initial-scale=1.0">
  <title>Website Modern dengan Semantic HTML5</title>
  <style>
    body { font-family: Arial, sans-serif; line-height: 1.6; margin: 0; padding: 0; }
    header { background-color: #2c3e50; color: white; padding: 1rem; text-align: center; }
    nav { background-color: #333; padding: 1rem; }
    nav ul { list-style-type: none; margin: 0; padding: 0; display: flex; justify-content: center; }
    nav li { margin: 0 1rem; }
    nav a { color: white; text-decoration: none; padding: 0.5rem 1rem; border-radius: 4px; }
    main { padding: 2rem; max-width: 1200px; margin: 0 auto; }
    article { background-color: white; margin-bottom: 2rem; padding: 2rem; border-radius: 8px; }
    section { margin-bottom: 2rem; }
    aside { background-color: #e1f5fe; padding: 2rem; border-radius: 8px; margin-bottom: 2rem; }
    footer { background-color: #333; color: white; text-align: center; padding: 2rem; margin-top: 2rem; }
  </style>
</head>
<body>
  <header>
    <h1>TechBlog Indonesia</h1>
  </header>
  <nav>
    <ul>
      <li><a href="#beranda">Beranda</a></li>
      <li><a href="#tutorial">Tutorial</a></li>
      <li><a href="#review">Review</a></li>
      <li><a href="#tentang">Tentang</a></li>
    </ul>
  </nav>
  <main>
    <article id="beranda">
      <header><h2>Artikel Terbaru: HTML5 Semantic Elements</h2></header>
      <section>
        <h3>Pengenalan Semantic HTML5</h3>
        <p>HTML5 membawa revolusi dalam cara kita menulis dan memahami struktur dokumen web.</p>
      </section>
    </article>
    <aside>
      <h3>Artikel Terkait</h3>
      <ul><li><a href="#article2">CSS Grid vs Flexbox</a></li></ul>
    </aside>
  </main>
  <footer>
    <p>&copy; 2024 TechBlog Indonesia</p>
  </footer>
</body>
</html>
\end{lstlisting}

\textbf{Hasil di Browser:}
- Header dengan branding dan navigasi
- Main content dengan featured article
- Aside dengan related articles
- Footer dengan copyright
- Semantic structure yang jelas dan accessible

\input{bab-03/section-3-4}
\section{Link dan Navigasi}

Link (hyperlink) adalah elemen fundamental HTML yang memungkinkan pengguna berpindah antar halaman atau website. Link dibuat menggunakan tag `<a>` (anchor) dengan atribut `href` yang menentukan URL tujuan \cite{w3schools-html}. Pemahaman link yang baik adalah kunci untuk navigasi website yang intuitif \cite{mdn-html}.

\subsection{Basic Link Syntax}

Sintaks dasar link menggunakan tag `<a>` dengan atribut penting:

\begin{itemize}
  \item \texttt{href} - Hypertext reference (URL tujuan)
  \item \texttt{target} - Cara membuka link (\_self, \_blank, \_parent, \_top)
  \item \texttt{title} - Tooltip yang muncul saat hover
  \item \texttt{rel} - Relationship antar dokumen (nofollow, noopener, noreferrer)
\end{itemize}

\begin{lstlisting}[caption={Contoh Basic Link}, basicstyle=\ttfamily\small, frame=single]
<!DOCTYPE html>
<html>
<head>
  <title>Contoh Link Dasar</title>
</head>
<body>
  <h1>Demonstrasi Link HTML</h1>
  
  <!-- Link ke halaman lain di website yang sama -->
  <p>Kunjungi <a href="about.html">halaman about</a> untuk informasi lebih lanjut.</p>
  
  <!-- Link ke website eksternal -->
  <p>Kunjungi <a href="https://www.google.com">Google</a> untuk pencarian.</p>
  
  <!-- Link yang membuka di tab baru -->
  <p>Buka <a href="https://www.github.com" target="_blank">GitHub</a> di tab baru.</p>
  
  <!-- Link dengan tooltip -->
  <p><a href="https://www.wikipedia.org" title="Ensiklopedia online">Wikipedia</a> adalah sumber informasi.</p>
  
  <!-- Link dengan rel="nofollow" -->
  <p><a href="https://example.com" rel="nofollow">External Link</a> (nofollow untuk SEO).</p>
</body>
</html>
\end{lstlisting}

\textbf{Hasil di Browser:}
- Teks yang di-link akan berwarna biru dan bergaris bawah
- Saat hover, cursor berubah menjadi pointer
- `target="_blank"` membuka link di tab/jendela baru
- `title` menampilkan tooltip saat mouse hover
- Link yang sudah dikunjungi akan berwarna ungu

\subsection{Jenis-Jenis Link}

HTML mendukung berbagai jenis link untuk keperluan berbeda:

\begin{itemize}
  \item \textbf{Internal Links}: Link ke halaman dalam website yang sama
  \item \textbf{External Links}: Link ke website lain
  \item \textbf{Email Links}: Membuka email client
  \item \textbf{Phone Links}: Membuka phone dialer
  \item \textbf{Anchor Links}: Link ke bagian dalam halaman yang sama
  \item \textbf{Download Links}: Link untuk download file
\end{itemize}

\begin{lstlisting}[caption={Berbagai Jenis Link}, basicstyle=\ttfamily\small, frame=single]
<!DOCTYPE html>
<html>
<head>
  <title>Jenis-Jenis Link</title>
</head>
<body>
  <h1>Demonstrasi Berbagai Link</h1>
  
  <!-- Internal Link -->
  <h2>Internal Navigation</h2>
  <ul>
    <li><a href="index.html">Beranda</a></li>
    <li><a href="about.html">Tentang Kami</a></li>
    <li><a href="services.html">Layanan</a></li>
    <li><a href="contact.html">Kontak</a></li>
  </ul>
  
  <!-- External Links -->
  <h2>External Resources</h2>
  <ul>
    <li><a href="https://www.facebook.com" target="_blank">Facebook</a></li>
    <li><a href="https://www.twitter.com" target="_blank">Twitter</a></li>
    <li><a href="https://www.linkedin.com" target="_blank">LinkedIn</a></li>
  </ul>
  
  <!-- Email dan Phone Links -->
  <h2>Contact Information</h2>
  <p>Email: <a href="mailto:info@company.com">info@company.com</a></p>
  <p>Phone: <a href="tel:+628123456789">+62 812-3456-789</a></p>
  
  <!-- Anchor Links -->
  <h2>Quick Navigation</h2>
  <p>Langsung ke: 
    <a href="#section1">Section 1</a> | 
    <a href="#section2">Section 2</a> | 
    <a href="#section3">Section 3</a>
  </p>
  
  <!-- Download Links -->
  <h2>Downloads</h2>
  <ul>
    <li><a href="documents/brochure.pdf" download>Company Brochure (PDF)</a></li>
    <li><a href="software/setup.exe" download>Software Setup</a></li>
    <li><a href="images/logo.png" download>Company Logo</a></li>
  </ul>
  
  <!-- Target Sections untuk Anchor Links -->
  <hr>
  <h2 id="section1">Section 1: Pengenalan</h2>
  <p>Ini adalah konten section 1...</p>
  
  <h2 id="section2">Section 2: Produk</h2>
  <p>Ini adalah konten section 2...</p>
  
  <h2 id="section3">Section 3: Kontak</h2>
  <p>Ini adalah konten section 3...</p>
</body>
</html>
\end{lstlisting}

\textbf{Hasil di Browser:}
- Internal links navigasi antar halaman website
- External links membuka website lain (bisa di tab baru)
- Email links membuka default email client
- Phone links membuka phone dialer di mobile
- Anchor links scroll ke bagian tertentu dalam halaman
- Download links memulai download file

\subsection{Image Links dan Link Buttons}

Link tidak hanya berupa teks, tetapi juga bisa berupa gambar atau tombol:

\begin{itemize}
  \item \textbf{Image Links}: Gambar yang bisa diklik
  \item \textbf{Button Links}: Tombol yang berfungsi sebagai link
  \item \textbf{Icon Links}: Ikon kecil yang bisa diklik
\end{itemize}

\begin{lstlisting}[caption={Image Links dan Button Links}, basicstyle=\ttfamily\small, frame=single]
<!DOCTYPE html>
<html>
<head>
  <title>Image dan Button Links</title>
</head>
<body>
  <h1>Link dengan Gambar dan Tombol</h1>
  
  <!-- Image Link -->
  <h2>Image Links</h2>
  <p>Klik logo untuk menuju homepage:</p>
  <a href="https://www.example.com" target="_blank">
    <img src="logo.png" alt="Company Logo" width="200" height="100">
  </a>
  
  <!-- Multiple Image Links -->
  <p>Social Media:</p>
  <a href="https://www.facebook.com" target="_blank">
    <img src="facebook-icon.png" alt="Facebook" width="32" height="32">
  </a>
  <a href="https://www.twitter.com" target="_blank">
    <img src="twitter-icon.png" alt="Twitter" width="32" height="32">
  </a>
  <a href="https://www.instagram.com" target="_blank">
    <img src="instagram-icon.png" alt="Instagram" width="32" height="32">
  </a>
  
  <!-- Button Links -->
  <h2>Button Links</h2>
  <a href="register.html" style="background-color: #4CAF50; color: white; 
     padding: 10px 20px; text-decoration: none; border-radius: 5px;">
    Daftar Sekarang
  </a>
  
  <a href="login.html" style="background-color: #008CBA; color: white; 
     padding: 10px 20px; text-decoration: none; border-radius: 5px;">
    Login
  </a>
  
  <!-- Icon dengan Text Links -->
  <h2>Icon dengan Text</h2>
  <a href="download.html">
    <img src="download-icon.png" alt="Download" width="16" height="16">
    Download Software
  </a>
  
  <a href="help.html">
    <img src="help-icon.png" alt="Help" width="16" height="16">
    Bantuan
  </a>
</body>
</html>
\end{lstlisting}

\textbf{Hasil di Browser:}
- Gambar yang bisa diklik (ada border biru saat belum dikunjungi)
- Social media icons yang bisa diklik
- Tombol dengan background warna yang bisa diklik
- Kombinasi icon dan teks untuk navigasi yang lebih intuitif

\subsection{Image Maps}

Image maps memungkinkan satu gambar memiliki multiple area yang bisa diklik dengan link berbeda:

\begin{itemize}
  \item \texttt{<map>} - Mendefinisikan image map
  \item \texttt{<area>} - Mendefinisikan area yang bisa diklik
  \item Atribut area: \texttt{shape} (rect, circle, poly), \texttt{coords}, \texttt{href}
\end{itemize}

\begin{lstlisting}[caption={Contoh Image Map}, basicstyle=\ttfamily\small, frame=single]
<!DOCTYPE html>
<html>
<head>
  <title>Contoh Image Map</title>
</head>
<body>
  <h1>Image Map Demo</h1>
  
  <!-- Image dengan Map -->
  <p>Klik pada bagian gambar:</p>
  <img src="office-map.png" alt="Office Layout" usemap="#officemap" 
       width="400" height="300">
  
  <!-- Definisi Map -->
  <map name="officemap">
    <!-- Area persegi panjang untuk ruangan 1 -->
    <area shape="rect" coords="50,50,150,150" 
          href="room1.html" alt="Ruangan 1">
    
    <!-- Area lingkaran untuk ruangan 2 -->
    <area shape="circle" coords="250,100,30" 
          href="room2.html" alt="Ruangan 2">
    
    <!-- Area polygon untuk ruangan 3 -->
    <area shape="poly" coords="100,200,200,200,150,250,50,250" 
          href="room3.html" alt="Ruangan 3">
    
    <!-- Area default untuk bagian lain -->
    <area shape="default" href="office-info.html" alt="Informasi Kantor">
  </map>
  
  <h3>Keterangan Area:</h3>
  <ul>
    <li><b>Ruangan 1:</b> Area persegi (50,50) hingga (150,150)</li>
    <li><b>Ruangan 2:</b> Area lingkaran pusat (250,100) radius 30</li>
    <li><b>Ruangan 3:</b> Area polygon dengan 5 titik</li>
    <li><b>Default:</b> Area lain yang tidak didefinisikan</li>
  </ul>
</body>
</html>
\end{lstlisting}

\textbf{Hasil di Browser:}
- Satu gambar dengan multiple area yang bisa diklik
- Cursor berubah pointer saat hover di area yang bisa diklik
- Setiap area memiliki link tujuan yang berbeda
- Berguna untuk peta denah, diagram, atau gambar interaktif

\subsection{Best Practices untuk Link dan Navigasi}

Praktik terbaik dalam membuat link dan navigasi yang user-friendly:

\begin{itemize}
  \item \textbf{Descriptive Text}: Gunakan teks yang jelas menggambarkan tujuan link
  \item \textbf{Accessibility}: Tambahkan `title` attribute untuk screen readers
  \item \textbf{Target Blank}: Gunakan `target="_blank"` untuk external links
  \item \textbf{Security}: Tambahkan `rel="noopener noreferrer"` untuk external links
  \item \textbf{Visual Feedback}: Berikan indikator visual untuk hover dan active states
  \item \textbf{Consistent Navigation}: Menu navigasi yang konsisten di semua halaman
\end{itemize}

\begin{lstlisting}[caption={Best Practices Example}, basicstyle=\ttfamily\small, frame=single]
<!DOCTYPE html>
<html>
<head>
  <title>Best Practices Navigation</title>
  <style>
    /* CSS untuk navigasi yang baik */
    .nav-link {
      display: inline-block;
      padding: 10px 15px;
      text-decoration: none;
      color: #333;
      border-radius: 5px;
      transition: all 0.3s ease;
    }
    
    .nav-link:hover {
      background-color: #f0f0f0;
      color: #0066cc;
    }
    
    .nav-link:active {
      background-color: #e0e0e0;
    }
    
    .external-link::after {
      content: " ↗";
      font-size: 0.8em;
    }
  </style>
</head>
<body>
  <h1>Navigation Best Practices</h1>
  
  <!-- Navigasi Utama -->
  <nav>
    <a href="home.html" class="nav-link">Beranda</a>
    <a href="products.html" class="nav-link">Produk</a>
    <a href="services.html" class="nav-link">Layanan</a>
    <a href="about.html" class="nav-link">Tentang</a>
    <a href="contact.html" class="nav-link">Kontak</a>
  </nav>
  
  <!-- External Links dengan indikator -->
  <h2>External Resources</h2>
  <p>
    Kunjungi <a href="https://www.w3schools.com" target="_blank" 
                rel="noopener noreferrer" class="external-link" 
                title="Tutorial Web Development">W3Schools</a> 
    untuk belajar web development.
  </p>
  
  <p>
    Download <a href="software.zip" download 
                title="Download software terbaru">software kami</a> 
    gratis.
  </p>
  
  <!-- Breadcrumb Navigation -->
  <h2>Breadcrumb Navigation</h2>
  <p>
    <a href="home.html">Beranda</a> &gt; 
    <a href="products.html">Produk</a> &gt; 
    <a href="laptops.html">Laptop</a> &gt; 
    Gaming Laptop
  </p>
</body>
</html>
\end{lstlisting}

\textbf{Hasil di Browser:}
- Navigasi dengan hover effects dan transitions
- External links dengan indikator panah
- Breadcrumb untuk navigasi hierarkis
- Tooltip untuk accessibility
- Visual feedback untuk interaksi user

Link dan navigasi yang baik memungkinkan pengguna berpindah antar halaman dengan intuitif dan efisien \cite{w3schools-html}. Kombinasi berbagai jenis link dan best practices akan menciptakan pengalaman navigasi yang optimal untuk pengguna \cite{mdn-html}.

\section{Images dan Media}

Images dan media adalah elemen penting dalam web development yang membuat konten lebih menarik dan informatif. HTML menyediakan tag `<img>` untuk menampilkan gambar dengan berbagai format dan pengaturan \cite{w3schools-html}. Pemahaman yang baik tentang image handling adalah fundamental untuk web development modern \cite{mdn-html}.

\subsection{Basic Image Tag}

Tag `<img>` memiliki atribut penting untuk menampilkan gambar:

\begin{itemize}
  \item \texttt{src} - Source/path file gambar (wajib)
  \item \texttt{alt} - Alternative text (wajib untuk accessibility)
  \item \texttt{width} - Lebar gambar dalam pixels atau persentase
  \item \texttt{height} - Tinggi gambar dalam pixels atau persentase
  \item \texttt{title} - Tooltip saat hover
  \item \texttt{border} - Border gambar
  \item \texttt{align} - Alignment (left, right, center, top, bottom)
\end{itemize}

\begin{lstlisting}[caption={Contoh Basic Image}, basicstyle=\ttfamily\small, frame=single]
<!DOCTYPE html>
<html>
<head>
  <title>Contoh Image Dasar</title>
</head>
<body>
  <h1>Demonstrasi Image HTML</h1>
  
  <!-- Image dengan semua atribut penting -->
  <img src="sunset.jpg" alt="Sunset di pantai" width="400" height="300" 
       title="Pemandangan sunset yang indah">
  
  <!-- Image dengan ukuran berbeda -->
  <h3>Image dengan Ukuran Berbeda:</h3>
  <p>Image kecil:</p>
  <img src="logo.png" alt="Company Logo" width="100" height="50">
  
  <p>Image sedang:</p>
  <img src="product.jpg" alt="Product Image" width="200" height="150">
  
  <p>Image besar:</p>
  <img src="banner.jpg" alt="Website Banner" width="600" height="200">
  
  <!-- Image dengan alignment -->
  <h3>Image Alignment:</h3>
  <p>
    <img src="icon-left.png" alt="Icon" width="50" height="50" align="left">
    Teks ini mengalir di sebelah kanan image. Image di sebelah kiri dengan alignment left.
    Lorem ipsum dolor sit amet, consectetur adipiscing elit. Sed do eiusmod tempor incididunt ut labore.
  </p>
  
  <p>
    <img src="icon-right.png" alt="Icon" width="50" height="50" align="right">
    Teks ini mengalir di sebelah kiri image. Image di sebelah kanan dengan alignment right.
    Lorem ipsum dolor sit amet, consectetur adipiscing elit. Sed do eiusmod tempor incididunt ut labore.
  </p>
</body>
</html>
\end{lstlisting}

\textbf{Hasil di Browser:}
- Gambar ditampilkan dengan ukuran sesuai atribut width/height
- Text mengalir di sekitar gambar sesuai alignment
- Tooltip muncul saat hover pada gambar
- Alt text tidak terlihat tapi penting untuk accessibility dan SEO

\subsection{Image Formats dan Optimization}

HTML mendukung berbagai format gambar dengan karakteristik berbeda:

\begin{itemize}
  \item \textbf{JPEG/JPG}: Cocok untuk foto, ukuran kecil, lossy compression
  \item \textbf{PNG}: Cocok untuk grafis, transparansi, lossless compression
  \item \textbf{GIF}: Animasi, 256 colors, ukuran kecil
  \item \textbf{SVG}: Vector graphics, scalable, tanpa pixelation
  \item \textbf{WebP}: Modern format, ukuran kecil, kualitas baik
\end{itemize}

\begin{lstlisting}[caption={Berbagai Format Image}, basicstyle=\ttfamily\small, frame=single]
<!DOCTYPE html>
<html>
<head>
  <title>Format Image</title>
</head>
<body>
  <h1>Demonstrasi Format Image</h1>
  
  <!-- JPEG untuk foto -->
  <h3>JPEG - Untuk Foto</h3>
  <img src="photo.jpg" alt="Foto landscape" width="300" height="200">
  <p><small>Format: JPEG | Ukuran: 45KB | Cocok untuk foto kompleks</small></p>
  
  <!-- PNG untuk grafis -->
  <h3>PNG - Untuk Grafis</h3>
  <img src="logo.png" alt="Logo transparan" width="200" height="100">
  <p><small>Format: PNG | Ukuran: 12KB | Mendukung transparansi</small></p>
  
  <!-- GIF untuk animasi -->
  <h3>GIF - Untuk Animasi</h3>
  <img src="loading.gif" alt="Loading animation" width="100" height="100">
  <p><small>Format: GIF | Ukuran: 8KB | Mendukung animasi</small></p>
  
  <!-- SVG untuk vector -->
  <h3>SVG - Vector Graphics</h3>
  <img src="icon.svg" alt="SVG Icon" width="100" height="100">
  <p><small>Format: SVG | Ukuran: 2KB | Vector, tidak pecah saat diperbesar</small></p>
  
  <!-- WebP modern format -->
  <h3>WebP - Modern Format</h3>
  <img src="banner.webp" alt="WebP Banner" width="400" height="200">
  <p><small>Format: WebP | Ukuran: 25KB | Format modern, kompresi baik</small></p>
</body>
</html>
\end{lstlisting}

\textbf{Hasil di Browser:}
- JPEG menampilkan foto dengan kualitas baik namun ukuran lebih besar
- PNG menampilkan grafis dengan transparansi (jika ada)
- GIF menampilkan animasi bergerak
- SVG tetap tajam saat diperbesar (vector)
- WebP menampilkan kualitas baik dengan ukuran lebih kecil

\subsection{Responsive Images}

Responsive images memastikan gambar tampil optimal di berbagai ukuran layar:

\begin{itemize}
  \item \textbf{Fluid Images}: Width dalam persentase
  \item \textbf{Srcset}: Multiple sources untuk berbagai resolusi
  \item \textbf{Picture Element}: Art direction untuk berbagai kondisi
\end{itemize}

\begin{lstlisting}[caption={Responsive Images}, basicstyle=\ttfamily\small, frame=single]
<!DOCTYPE html>
<html>
<head>
  <title>Responsive Images</title>
  <style>
    .responsive-img {
      max-width: 100%;
      height: auto;
    }
  </style>
</head>
<body>
  <h1>Responsive Image Techniques</h1>
  
  <!-- Fluid Image dengan CSS -->
  <h3>Fluid Image (CSS)</h3>
  <img src="landscape.jpg" alt="Landscape" class="responsive-img">
  
  <!-- Srcset untuk berbagai resolusi -->
  <h3>Srcset untuk Multiple Resolutions</h3>
  <img src="photo-small.jpg" 
       srcset="photo-small.jpg 480w,
               photo-medium.jpg 800w,
               photo-large.jpg 1200w"
       sizes="(max-width: 600px) 480px,
              (max-width: 900px) 800px,
              1200px"
       alt="Foto responsif">
  
  <!-- Picture Element untuk art direction -->
  <h3>Picture Element untuk Art Direction</h3>
  <picture>
    <source media="(min-width: 900px)" srcset="banner-large.jpg">
    <source media="(min-width: 600px)" srcset="banner-medium.jpg">
    <img src="banner-small.jpg" alt="Banner responsif">
  </picture>
  
  <!-- Picture untuk format berbeda -->
  <h3>Picture untuk Format Support</h3>
  <picture>
    <source srcset="photo.webp" type="image/webp">
    <source srcset="photo.jpg" type="image/jpeg">
    <img src="photo.jpg" alt="Foto dengan fallback">
  </picture>
</body>
</html>
\end{lstlisting}

\textbf{Hasil di Browser:}
- Fluid images menyesuaikan lebar container
- Srcset memilih gambar yang sesuai berdasarkan ukuran layar
- Picture element memilih source berdasarkan media query atau format support
- Gambar optimal untuk setiap device dan kondisi

\subsection{Background Images}

Background images dapat ditambahkan melalui HTML attributes atau CSS:

\begin{itemize}
  \item \textbf{HTML Background}: Atribut background pada body atau table
  \item \textbf{CSS Background}: Lebih fleksibel dan powerful
\end{itemize}

\begin{lstlisting}[caption={Background Images}, basicstyle=\ttfamily\small, frame=single]
<!DOCTYPE html>
<html>
<head>
  <title>Background Images</title>
  <style>
    /* CSS Background untuk body */
    body {
      background-image: url('pattern.jpg');
      background-repeat: repeat;
      background-attachment: fixed;
      background-size: cover;
    }
    
    /* Background untuk div */
    .hero-section {
      background-image: url('hero-bg.jpg');
      background-size: cover;
      background-position: center;
      height: 400px;
      color: white;
      text-align: center;
      padding: 50px;
    }
    
    .pattern-box {
      background-image: url('small-pattern.png');
      background-repeat: repeat;
      border: 2px solid #333;
      padding: 20px;
      margin: 10px;
    }
  </style>
</head>
<body>
  <!-- Background dengan CSS (recommended) -->
  <div class="hero-section">
    <h1>Hero Section dengan Background</h1>
    <p>Background image yang menutupi seluruh area</p>
  </div>
  
  <!-- Pattern background -->
  <h3>Pattern Background:</h3>
  <div class="pattern-box">
    <h4>Box dengan Pattern</h4>
    <p>Pattern diulang untuk mengisi background</p>
  </div>
  
  <!-- HTML Background (deprecated) -->
  <h3>HTML Background Attribute:</h3>
  <table background="table-bg.jpg" border="1">
    <tr>
      <td style="color: white; padding: 10px;">Cell 1</td>
      <td style="color: white; padding: 10px;">Cell 2</td>
    </tr>
  </table>
  
  <p><small>Catatan: HTML background attribute deprecated, gunakan CSS</small></p>
</body>
</html>
\end{lstlisting}

\textbf{Hasil di Browser:}
- Body dengan background pattern yang fixed
- Hero section dengan background image full width
- Box dengan pattern background yang diulang
- Table dengan background (deprecated method)
- CSS background lebih fleksibel dan recommended

\subsection{Image Maps Revisited}

Image maps untuk navigasi interaktif dengan gambar:

\begin{lstlisting}[caption={Interactive Image Map}, basicstyle=\ttfamily\small, frame=single]
<!DOCTYPE html>
<html>
<head>
  <title>Interactive Image Map</title>
</head>
<body>
  <h1>Interactive Campus Map</h1>
  
  <!-- Campus map dengan multiple clickable areas -->
  <img src="campus-map.png" alt="Campus Map" usemap="#campusmap" 
       width="600" height="400" style="border: 1px solid #ccc;">
  
  <map name="campusmap">
    <!-- Gedung Rektorat -->
    <area shape="rect" coords="100,100,200,180" 
          href="rektorat.html" alt="Gedung Rektorat"
          title="Gedung Rektorat">
    
    <!-- Perpustakaan -->
    <area shape="rect" coords="250,150,350,230" 
          href="perpustakaan.html" alt="Perpustakaan"
          title="Perpustakaan">
    
    <!-- Laboratorium -->
    <area shape="circle" coords="450,200,40" 
          href="lab.html" alt="Laboratorium"
          title="Laboratorium Komputer">
    
    <!-- Lapangan Olahraga -->
    <area shape="poly" coords="50,300,150,320,200,280,250,100,280" 
          href="olahraga.html" alt="Lapangan Olahraga"
          title="Lapangan Olahraga">
    
    <!-- Area default -->
    <area shape="default" href="campus-info.html" 
          alt="Informasi Kampus" title="Informasi Umum Kampus">
  </map>
  
  <h3>Legenda Peta:</h3>
  <ul>
    <li><b>Gedung Rektorat:</b> Administrasi utama</li>
    <li><b>Perpustakaan:</b> Sumber belajar dan referensi</li>
    <li><b>Laboratorium:</b> Praktikum dan riset</li>
    <li><b>Lapangan Olahraga:</b> Fasilitas olahraga</li>
  </ul>
</body>
</html>
\end{lstlisting}

\textbf{Hasil di Browser:}
- Peta kampus dengan multiple area yang bisa diklik
- Setiap area memiliki tooltip saat hover
- Cursor berubah pointer saat di area clickable
- Link ke halaman informasi yang relevan

\subsection{Best Practices untuk Images}

Praktik terbaik dalam menggunakan images di website:

\begin{itemize}
  \item \textbf{Alt Text}: Selalu gunakan alt text yang deskriptif
  \item \textbf{File Size}: Optimasi ukuran file untuk loading cepat
  \item \textbf{Responsive}: Gunakan images yang responsif
  \item \textbf{Format}: Pilih format yang sesuai dengan kebutuhan
  \item \textbf{Lazy Loading}: Load images saat dibutuhkan
  \item \textbf{Compression}: Compress images tanpa mengurangi kualitas
\end{itemize}

\begin{lstlisting}[caption={Best Practices Implementation}, basicstyle=\ttfamily\small, frame=single]
<!DOCTYPE html>
<html>
<head>
  <title>Image Best Practices</title>
  <style>
    .image-container {
      margin: 20px 0;
    }
    
    .optimized-image {
      max-width: 100%;
      height: auto;
      border-radius: 8px;
      box-shadow: 0 4px 8px rgba(0,0,0,0.1);
      transition: transform 0.3s ease;
    }
    
    .optimized-image:hover {
      transform: scale(1.05);
    }
    
    .image-caption {
      text-align: center;
      font-style: italic;
      color: #666;
      margin-top: 8px;
    }
  </style>
</head>
<body>
  <h1>Image Best Practices Demo</h1>
  
  <!-- Image dengan alt text deskriptif -->
  <div class="image-container">
    <img src="mountain-landscape.jpg" 
         alt="Pemandangan gunung dengan danau biru di pagi hari, 
               terlihat kabut tipis di puncak gunung dan cahaya matahari 
               yang bersinar melalui awan" 
         class="optimized-image"
         width="600" height="400"
         loading="lazy">
    <p class="image-caption">Gambar 1: Pemandangan alam yang indah</p>
  </div>
  
  <!-- Image dengan figure dan figcaption -->
  <figure class="image-container">
    <img src="city-skyline.jpg" 
         alt="Skyline kota modern di malam hari dengan lampu-lampu gedung 
               yang menerangi jalan-jalan utama" 
         class="optimized-image"
         width="600" height="400"
         loading="lazy">
    <figcaption>Gambar 2: Pemandangan kota di malam hari</figcaption>
  </figure>
  
  <!-- Multiple images gallery -->
  <h3>Image Gallery:</h3>
  <div style="display: grid; grid-template-columns: repeat(auto-fit, minmax(200px, 1fr)); gap: 20px;">
    <div class="image-container">
      <img src="gallery1.jpg" 
           alt="Close-up bunga merah dengan tetesan air" 
           class="optimized-image"
           loading="lazy">
      <p class="image-caption">Bunga Merah</p>
    </div>
    
    <div class="image-container">
      <img src="gallery2.jpg" 
           alt="Jembatan kayu di tengah hutan hijau" 
           class="optimized-image"
           loading="lazy">
      <p class="image-caption">Jembatan Hutan</p>
    </div>
    
    <div class="image-container">
      <img src="gallery3.jpg" 
           alt="Pantai dengan pasir putih dan ombak biru" 
           class="optimized-image"
           loading="lazy">
      <p class="image-caption">Pantai Indah</p>
    </div>
  </div>
  
  <h3>Optimization Notes:</h3>
  <ul>
    <li>✅ Semua images memiliki alt text yang deskriptif</li>
    <li>✅ Menggunakan loading="lazy" untuk performance</li>
    <li>✅ Responsive dengan max-width: 100%</li>
    <li>✅ Optimized untuk web (compressed)</li>
    <li>✅ Semantic dengan figure dan figcaption</li>
  </ul>
</body>
</html>
\end{lstlisting}

\textbf{Hasil di Browser:}
- Images dengan hover effects dan transitions
- Alt text yang deskriptif untuk accessibility
- Lazy loading untuk performance
- Responsive gallery dengan CSS Grid
- Semantic markup dengan figure dan figcaption
- Caption yang informatif untuk setiap gambar

Pemahaman images dan media yang baik memungkinkan penciptaan website yang visual menarik, cepat, dan accessible \cite{w3schools-html}. Kombinasi teknik-teknik ini akan menghasilkan pengalaman visual yang optimal untuk pengguna \cite{mdn-html}.

\section{Multimedia HTML5}

HTML5 merevolusi multimedia web dengan menyediakan elemen-elemen native untuk audio, video, dan grafis tanpa memerlukan plugin pihak ketiga seperti Flash atau Silverlight \cite{mdn-multimedia}. Elemen-elemen multimedia HTML5 memungkinkan pengembangan aplikasi web yang kaya media dengan performa yang lebih baik dan compatibility yang lebih luas \cite{w3schools-html}.

\subsection{Video Element}

Elemen `<video>` adalah cara standar untuk menampilkan video di web tanpa plugin:

\begin{itemize}
  \item \textbf{Atribut Utama}: \texttt{src}, \texttt{controls}, \texttt{width}, \texttt{height}
  \item \textbf{Playback Control}: \texttt{autoplay}, \texttt{loop}, \texttt{muted}, \texttt{preload}
  \item \textbf{Multiple Sources}: \texttt{<source>} untuk berbagai format video
  \item \textbf{Fallback Content}: Konten untuk browser yang tidak support video
  \item \textbf{Styling}: CSS untuk custom controls dan layout
\end{itemize}

\begin{lstlisting}[caption={HTML5 Video Element}, basicstyle=\ttfamily\small, frame=single]
<!DOCTYPE html>
<html lang="id">
<head>
  <meta charset="UTF-8">
  <meta name="viewport" content="width=device-width, initial-scale=1.0">
  <title>HTML5 Video Player</title>
  <style>
    body { font-family: Arial, sans-serif; line-height: 1.6; margin: 0; padding: 20px; background-color: #f4f4f4; }
    .video-container { max-width: 800px; margin: 0 auto; background-color: #000; border-radius: 8px; overflow: hidden; box-shadow: 0 4px 20px rgba(0,0,0,0.3); }
    .video-player { width: 100%; height: 450px; display: block; }
    .video-controls { background-color: rgba(0,0,0,0.7); padding: 15px; text-align: center; }
    .video-info { color: white; margin-bottom: 10px; }
    .video-gallery { display: grid; grid-template-columns: repeat(auto-fit, minmax(300px, 1fr)); gap: 20px; margin-top: 30px; }
    .video-item { background-color: white; border-radius: 8px; overflow: hidden; box-shadow: 0 2px 8px rgba(0,0,0,0.1); }
    .video-item h3 { margin: 0; padding: 15px; background-color: #2c3e50; color: white; }
    .video-item video { width: 100%; height: 200px; object-fit: cover; }
    .custom-controls { display: flex; justify-content: center; gap: 10px; margin-top: 15px; }
    .btn { background-color: #007bff; color: white; border: none; padding: 8px 16px; border-radius: 4px; cursor: pointer; }
    .btn:hover { background-color: #0056b3; }
  </style>
</head>
<body>
  <h1>HTML5 Video Player</h1>
  
  <div class="video-container">
    <video class="video-player" controls poster="video-poster.jpg">
      <source src="videos/tutorial-html5.mp4" type="video/mp4">
      <source src="videos/tutorial-html5.webm" type="video/webm">
      <source src="videos/tutorial-html5.ogv" type="video/ogg">
      Browser Anda tidak mendukung elemen video.
    </video>
    
    <div class="video-controls">
      <div class="video-info">
        <h3>Tutorial HTML5</h3>
        <p>Durasi: 5:30</p>
        <p>Format: MP4/WebM/OGG</p>
      </div>
      
      <div class="custom-controls">
        <button class="btn" onclick="playPause()">Play/Pause</button>
        <button class="btn" onclick="muteUnmute()">Mute/Unmute</button>
        <button class="btn" onclick="changeSpeed()">Kecepatan</button>
        <button class="btn" onclick="toggleFullscreen()">Fullscreen</button>
      </div>
    </div>
  </div>
  
  <h2>Video Gallery</h2>
  <div class="video-gallery">
    <div class="video-item">
      <h3>Video 1: Introduction to HTML5</h3>
      <video controls poster="intro-poster.jpg">
        <source src="videos/intro.mp4" type="video/mp4">
        <source src="videos/intro.webm" type="video/webm">
      </video>
    </div>
    
    <div class="video-item">
      <h3>Video 2: Semantic Elements</h3>
      <video controls poster="semantic-poster.jpg">
        <source src="videos/semantic.mp4" type="video/mp4">
        <source src="videos/semantic.webm" type="video/webm">
      </video>
    </div>
    
    <div class="video-item">
      <h3>Video 3: Form Validation</h3>
      <video controls poster="validation-poster.jpg">
        <source src="videos/validation.mp4" type="video/mp4">
        <source src="videos/validation.webm" type="video/webm">
      </video>
    </div>
  </div>
  
  <script>
    const video = document.querySelector('.video-player');
    
    function playPause() {
      if (video.paused) {
        video.play();
      } else {
        video.pause();
      }
    }
    
    function muteUnmute() {
      video.muted = !video.muted;
    }
    
    function changeSpeed() {
      video.playbackRate = video.playbackRate === 1 ? 1.5 : 1;
    }
    
    function toggleFullscreen() {
      if (!document.fullscreenElement) {
        video.requestFullscreen();
      } else {
        document.exitFullscreen();
      }
    }
  </script>
</body>
</html>
\end{lstlisting}

\textbf{Hasil di Browser:}
- Video player dengan controls standar HTML5
- Multiple video sources untuk compatibility
- Custom JavaScript controls untuk play/pause, mute, speed, fullscreen
- Video gallery dengan responsive grid layout
- Poster image yang ditampilkan sebelum video dimuat

\subsection{Audio Element}

Elemen `<audio>` menyediakan cara standar untuk menampilkan audio di web:

\begin{itemize}
  \item \textbf{Basic Usage}: \texttt{<audio>} dengan \texttt{controls}
  \item \textbf{Multiple Sources}: \texttt{<source>} untuk berbagai format audio
  \item \textbf{Audio Formats}: MP3, WAV, OGG, AAC
  \item \textbf{Attributes}: \texttt{autoplay}, \texttt{loop}, \texttt{preload}, \texttt{volume}
  \item \textbf{JavaScript API}: Kontrol audio melalui JavaScript
\end{itemize}

\begin{lstlisting}[caption={HTML5 Audio Element}, basicstyle=\ttfamily\small, frame=single]
<!DOCTYPE html>
<html lang="id">
<head>
  <meta charset="UTF-8">
  <meta name="viewport" content="width=device-width, initial-scale=1.0">
  <title>HTML5 Audio Player</title>
  <style>
    body { font-family: Arial, sans-serif; line-height: 1.6; margin: 0; padding: 20px; background-color: #f8f9fa; }
    .audio-player { max-width: 600px; margin: 0 auto; background: linear-gradient(135deg, #667eea 0%, #764ba2 100%); border-radius: 12px; padding: 20px; box-shadow: 0 8px 32px rgba(0,0,0,0.1); }
    .audio-controls { display: flex; align-items: center; gap: 15px; margin-bottom: 20px; }
    .audio-info { color: white; text-align: center; }
    .progress-bar { width: 100%; height: 6px; background-color: rgba(255,255,255,0.3); border-radius: 3px; overflow: hidden; margin-bottom: 10px; }
    .progress-fill { height: 100%; background-color: #4CAF50; width: 0; transition: width 0.1s ease; }
    .playlist { background-color: rgba(255,255,255,0.1); border-radius: 8px; padding: 15px; }
    .playlist-item { padding: 10px; border-bottom: 1px solid rgba(255,255,255,0.2); cursor: pointer; transition: background-color 0.3s; }
    .playlist-item:hover { background-color: rgba(255,255,255,0.2); }
    .playlist-item.active { background-color: rgba(255,255,255,0.3); font-weight: bold; }
    .custom-controls { display: flex; justify-content: center; gap: 10px; margin-top: 15px; }
    .btn { background-color: rgba(255,255,255,0.2); color: #333; border: 1px solid rgba(255,255,255,0.3); padding: 8px 16px; border-radius: 4px; cursor: pointer; }
    .btn:hover { background-color: rgba(255,255,255,0.3); }
  </style>
</head>
<body>
  <h1>HTML5 Audio Player</h1>
  
  <div class="audio-player">
    <audio id="audioPlayer" controls>
      <source src="audio/music.mp3" type="audio/mpeg">
      <source src="audio/music.ogg" type="audio/ogg">
      <source src="audio/music.wav" type="audio/wav">
      Browser Anda tidak mendukung elemen audio.
    </audio>
    
    <div class="audio-info">
      <h3>Music Collection</h3>
      <p>Sedang dimainkan: <span id="currentTrack">Unknown</span></p>
      <p>Format: MP3/OGG/WAV</p>
    </div>
    
    <div class="progress-bar">
      <div class="progress-fill" id="progressFill"></div>
    </div>
    
    <div class="playlist">
      <h4>Playlist</h4>
      <div class="playlist-item active" onclick="playTrack(0)">
        <strong>1. Introduction to Web Development</strong><br>
        <small>Duration: 3:45</small>
      </div>
      <div class="playlist-item" onclick="playTrack(1)">
        <strong>2. HTML5 Semantic Elements</strong><br>
        <small>Duration: 5:20</small>
      </div>
      <div class="playlist-item" onclick="playTrack(2)">
        <strong>3. CSS Grid and Flexbox</strong><br>
        <small>Duration: 4:15</small>
      </div>
      <div class="playlist-item" onclick="playTrack(3)">
        <strong>4. JavaScript ES6+</strong><br>
        <small>Duration: 6:30</small>
      </div>
    </div>
    
    <div class="custom-controls">
      <button class="btn" onclick="previousTrack()">Previous</button>
      <button class="btn" onclick="togglePlayPause()">Play/Pause</button>
      <button class="btn" onclick="nextTrack()">Next</button>
      <button class="btn" onclick="toggleMute()">Mute</button>
      <button class="btn" onclick="changeVolume(0.5)">Volume 50%</button>
      <button class="btn" onclick="changeVolume(1)">Volume 100%</button>
    </div>
  </div>
  
  <script>
    const audio = document.getElementById('audioPlayer');
    const tracks = [
      { name: 'Introduction to Web Development', file: 'music.mp3', duration: '3:45' },
      { name: 'HTML5 Semantic Elements', file: 'semantic.mp3', duration: '5:20' },
      { name: 'CSS Grid and Flexbox', file: 'css.mp3', duration: '4:15' },
      { name: 'JavaScript ES6+', file: 'javascript.mp3', duration: '6:30' }
    ];
    let currentTrackIndex = 0;
    
    function playTrack(index) {
      currentTrackIndex = index;
      const track = tracks[index];
      audio.src = track.file;
      document.getElementById('currentTrack').textContent = track.name;
      
      document.querySelectorAll('.playlist-item').forEach((item, i) => {
        item.classList.toggle('active', i === index);
      });
      
      audio.play();
      updateProgress();
    }
    
    function updateProgress() {
      const progressFill = document.getElementById('progressFill');
      const duration = audio.duration;
      const currentTime = audio.currentTime;
      
      if (duration) {
        const progress = (currentTime / duration) * 100;
        progressFill.style.width = progress + '%';
      }
    }
    
    function togglePlayPause() {
      if (audio.paused) {
        audio.play();
      } else {
        audio.pause();
      }
    }
    
    function previousTrack() {
      if (currentTrackIndex > 0) {
        playTrack(currentTrackIndex - 1);
      }
    }
    
    function nextTrack() {
      if (currentTrackIndex < tracks.length - 1) {
        playTrack(currentTrackIndex + 1);
      }
    }
    
    function toggleMute() {
      audio.muted = !audio.muted;
    }
    
    function changeVolume(level) {
      audio.volume = level;
    }
    
    audio.addEventListener('timeupdate', updateProgress);
  </script>
</body>
</html>
\end{lstlisting}

\textbf{Hasil di Browser:}
- Audio player dengan multiple format sources
- Custom playlist dengan track selection
- Progress bar animasi untuk playback progress
- Volume controls dan mute functionality
- Gradient background dan modern styling

\subsection{Canvas API untuk Grafis}

Canvas API menyediakan area drawing 2D yang dapat dimanipulasi dengan JavaScript:

\begin{itemize}
  \item \textbf{Drawing Methods}: \texttt{fillRect()}, \texttt{strokeRect()}, \texttt{drawImage()}
  \item \textbf{Text Rendering}: \texttt{fillText()}, \texttt{strokeText()}, font customization
  \item \textbf{Shapes}: \texttt{arc()}, \texttt{lineTo()}, \texttt{quadraticCurveTo()}
  \item \textbf{Transforms}: \texttt{translate()}, \texttt{rotate()}, \texttt{scale()}
  \item \textbf{Gradients}: Linear dan radial gradients
  \item \textbf{Animation}: \texttt{requestAnimationFrame()} untuk smooth animations
  \item \textbf{Image Manipulation}: \texttt{getImageData()}, \texttt{putImageData()}
\end{itemize}

\begin{lstlisting}[caption={Canvas API untuk Grafis}, basicstyle=\ttfamily\small, frame=single]
<!DOCTYPE html>
<html lang="id">
<head>
  <meta charset="UTF-8">
  <meta name="viewport" content="width=device-width, initial-scale=1.0">
  <title>Canvas API - Drawing Application</title>
  <style>
    body { font-family: Arial, sans-serif; margin: 0; padding: 20px; background-color: #f0f0f0; }
    .canvas-container { max-width: 800px; margin: 0 auto; background-color: white; border-radius: 8px; box-shadow: 0 4px 20px rgba(0,0,0,0.1); padding: 20px; }
    .canvas-controls { display: flex; flex-wrap: wrap; gap: 10px; margin-bottom: 20px; justify-content: center; }
    .control-group { display: flex; flex-direction: column; align-items: center; gap: 5px; }
    .btn { background-color: #007bff; color: white; border: none; padding: 8px 16px; border-radius: 4px; cursor: pointer; font-size: 14px; }
    .btn:hover { background-color: #0056b3; }
    .btn.active { background-color: #28a745; }
    .color-picker { display: flex; align-items: center; gap: 5px; }
    input[type="color"] { width: 50px; height: 40px; border: none; border-radius: 4px; cursor: pointer; }
    .size-slider { display: flex; align-items: center; gap: 5px; }
    input[type="range"] { width: 100px; }
    #drawingCanvas { border: 2px solid #ddd; border-radius: 4px; cursor: crosshair; display: block; margin: 0 auto; }
  </style>
</head>
<body>
  <h1>Canvas Drawing Application</h1>
  
  <div class="canvas-container">
    <h2>Drawing Tools</h2>
    
    <div class="canvas-controls">
      <div class="control-group">
        <label>Drawing Tool:</label>
        <div>
          <button class="btn active" onclick="setTool('pen')">✏️ Pen</button>
          <button class="btn" onclick="setTool('eraser')">🧹 Eraser</button>
          <button class="btn" onclick="setTool('line')">📏 Line</button>
          <button class="btn" onclick="setTool('rectangle')">▢ Rectangle</button>
          <button class="btn" onclick="setTool('circle')">⭕ Circle</button>
          <button class="btn" onclick="setTool('text')">📝 Text</button>
        </div>
      </div>
      
      <div class="control-group">
        <label>Color:</label>
        <div class="color-picker">
          <input type="color" id="strokeColor" value="#000000" title="Stroke Color">
          <input type="color" id="fillColor" value="#ff0000" title="Fill Color">
        </div>
      </div>
      
      <div class="control-group">
        <label>Stroke Width:</label>
        <div class="size-slider">
          <input type="range" id="strokeWidth" min="1" max="20" value="2">
          <span id="strokeWidthValue">2px</span>
        </div>
      </div>
    </div>
    
    <h2>Drawing Canvas</h2>
    <canvas id="drawingCanvas" width="760" height="400"></canvas>
  </div>
  
  <script>
    const canvas = document.getElementById('drawingCanvas');
    const ctx = canvas.getContext('2d');
    
    let isDrawing = false;
    let currentTool = 'pen';
    let strokeColor = '#000000';
    let fillColor = '#ff0000';
    let strokeWidth = 2;
    
    canvas.addEventListener('mousedown', startDrawing);
    canvas.addEventListener('mousemove', draw);
    canvas.addEventListener('mouseup', stopDrawing);
    
    function startDrawing(e) {
      isDrawing = true;
      const rect = canvas.getBoundingClientRect();
      const x = e.clientX - rect.left;
      const y = e.clientY - rect.top;
      
      ctx.beginPath();
      ctx.moveTo(x, y);
    }
    
    function draw(e) {
      if (!isDrawing) return;
      
      const rect = canvas.getBoundingClientRect();
      const x = e.clientX - rect.left;
      const y = e.clientY - rect.top;
      
      ctx.strokeStyle = strokeColor;
      ctx.fillStyle = fillColor;
      ctx.lineWidth = strokeWidth;
      ctx.lineCap = 'round';
      
      if (currentTool === 'pen') {
        ctx.lineTo(x, y);
        ctx.stroke();
      } else if (currentTool === 'line') {
        ctx.lineTo(x, y);
        ctx.stroke();
      } else if (currentTool === 'rectangle') {
        ctx.strokeRect(x - 25, y - 25, 50, 50);
      } else if (currentTool === 'circle') {
        ctx.beginPath();
        ctx.arc(x, y, 25, 0, 2 * Math.PI);
        ctx.stroke();
      }
    }
    
    function stopDrawing() {
      isDrawing = false;
    }
    
    function setTool(tool) {
      currentTool = tool;
      document.querySelectorAll('.btn').forEach(btn => {
        btn.classList.remove('active');
      });
      event.target.classList.add('active');
    }
    
    document.getElementById('strokeColor').addEventListener('change', (e) => {
      strokeColor = e.target.value;
    });
    
    document.getElementById('fillColor').addEventListener('change', (e) => {
      fillColor = e.target.value;
    });
    
    document.getElementById('strokeWidth').addEventListener('input', (e) => {
      strokeWidth = e.target.value;
      document.getElementById('strokeWidthValue').textContent = e.target.value + 'px';
    });
  </script>
</body>
</html>
\end{lstlisting}

\textbf{Hasil di Browser:}
- Interactive drawing canvas dengan multiple tools
- Color picker untuk stroke dan fill colors
- Size slider untuk stroke width
- Drawing tools: pen, eraser, line, rectangle, circle, text

\subsection{Best Practices untuk Multimedia HTML5}

Praktik terbaik dalam implementasi multimedia HTML5:

\begin{itemize}
  \item \textbf{Progressive Enhancement}: Fallback untuk browser lama
  \item \textbf{Format Support}: Multiple sources untuk compatibility
  \item \textbf{Performance Optimization}: Lazy loading dan preloading
  \item \textbf{Responsive Design}: Media yang responsif untuk semua devices
  \item \textbf{Accessibility}: Alternative text dan captions
  \item \textbf{File Size Optimization}: Compressed media untuk loading cepat
  \item \textbf{User Controls}: Intuitive controls dan feedback
\end{itemize}

\begin{lstlisting}[caption={Multimedia Best Practices}, basicstyle=\ttfamily\small, frame=single]
<!DOCTYPE html>
<html lang="id">
<head>
  <meta charset="UTF-8">
  <meta name="viewport" content="width=device-width, initial-scale=1.0">
  <title>Multimedia Best Practices</title>
  <style>
    body { font-family: Arial, sans-serif; line-height: 1.6; margin: 0; padding: 20px; background-color: #f8f9fa; }
    .media-container { max-width: 1200px; margin: 0 auto; }
    .media-section { background-color: white; border-radius: 8px; padding: 20px; margin-bottom: 30px; box-shadow: 0 4px 8px rgba(0,0,0,0.1); }
    .media-section h2 { color: #2c3e50; border-bottom: 2px solid #4CAF50; padding-bottom: 10px; }
    .video-grid { display: grid; grid-template-columns: repeat(auto-fit, minmax(300px, 1fr)); gap: 20px; }
    .video-item { position: relative; border-radius: 8px; overflow: hidden; box-shadow: 0 2px 8px rgba(0,0,0,0.1); }
    .video-item video { width: 100%; height: 200px; object-fit: cover; }
    .video-overlay { position: absolute; top: 0; left: 0; right: 0; bottom: 0; background: rgba(0,0,0,0.7); display: flex; align-items: center; justify-content: center; opacity: 0; transition: opacity 0.3s; }
    .play-button { position: absolute; top: 50%; left: 50%; transform: translate(-50%, -50%); background-color: rgba(255,255,255,0.9); color: #333; border: none; padding: 10px 20px; border-radius: 50%; cursor: pointer; font-size: 24px; opacity: 0; transition: opacity 0.3s; }
  </style>
</head>
<body>
  <h1>Multimedia HTML5 Best Practices</h1>
  
  <div class="media-container">
    <div class="media-section">
      <h2>📹 Responsive Video Gallery</h2>
      <p>Video dengan lazy loading dan responsive design:</p>
      
      <div class="video-grid">
        <div class="video-item">
          <video controls preload="metadata" poster="video1-poster.jpg">
            <source src="https://commondatastorage.googleapis.com/gtv-videos-bucket/sample/BigBuckBunny.mp4" 
                   type="video/mp4" media="(min-width: 800px)">
            <source src="https://commondatastorage.googleapis.com/gtv-videos-bucket/sample/BigBuckBunny.mp4" 
                   type="video/mp4" media="(max-width: 799px)">
            <track label="Indonesian" src="subtitles-id.vtt" kind="subtitles" srclang="id" default>
            Browser Anda tidak mendukung elemen video.
          </video>
          <div class="video-overlay">
            <div class="play-button">▶️</div>
          </div>
        </div>
      </div>
    </div>
  </div>
  
  <script>
    // Lazy loading untuk video
    const videos = document.querySelectorAll('video');
    const videoObserver = new IntersectionObserver((entries) => {
      entries.forEach(entry => {
        const video = entry.target;
        if (entry.isIntersecting) {
          video.src = video.dataset.src;
        }
      });
    });
    
    videos.forEach(video => {
      videoObserver.observe(video);
    });
  </script>
</body>
</html>
\end{lstlisting}

\textbf{Hasil di Browser:}
- Responsive video gallery dengan lazy loading
- Video dengan multiple sources dan media queries
- Intersection Observer untuk lazy loading
- Modern design dengan overlays dan transitions

Multimedia HTML5 memberikan fondasi kuat untuk aplikasi web modern yang kaya media \cite{w3schools-html}. Dengan pemahaman video, audio, dan canvas API yang baik, developer dapat menciptakan pengalaman multimedia yang interaktif, performant, dan accessible \cite{mdn-html}.

\section{Tables dan Lists}

Tables dan lists adalah elemen fundamental HTML untuk mengorganisir data dan informasi secara terstruktur. Tables digunakan untuk data tabular, sedangkan lists untuk item-item yang berurutan \cite{w3schools-html}. Pemahaman yang baik tentang tables dan lists adalah kunci untuk presentasi data yang efektif \cite{mdn-html}.

\subsection{HTML Tables}

Tables dibangun menggunakan tag-tag yang mendefinisikan struktur baris dan kolom:

\begin{itemize}
  \item \texttt{<table>} - Container utama untuk table
  \item \texttt{<tr>} - Table row (baris)
  \item \texttt{<th>} - Table header cell (sel header)
  \item \texttt{<td>} - Table data cell (sel data)
  \item \texttt{<thead>} - Group header rows
  \item \texttt{<tbody>} - Group body rows
  \item \texttt{<tfoot>} - Group footer rows
  \item \texttt{<caption>} - Caption/judul table
\end{itemize}

\begin{lstlisting}[caption={Contoh Basic Table}, basicstyle=\ttfamily\small, frame=single]
<!DOCTYPE html>
<html>
<head>
  <title>Contoh Table Dasar</title>
</head>
<body>
  <h1>Demonstrasi HTML Tables</h1>
  
  <!-- Table sederhana -->
  <h3>Table Sederhana</h3>
  <table border="1">
    <tr>
      <th>Nama</th>
      <th>Umur</th>
      <th>Kota</th>
    </tr>
    <tr>
      <td>Andi</td>
      <td>25</td>
      <td>Jakarta</td>
    </tr>
    <tr>
      <td>Budi</td>
      <td>30</td>
      <td>Bandung</td>
    </tr>
    <tr>
      <td>Cici</td>
      <td>28</td>
      <td>Surabaya</td>
    </tr>
  </table>
  
  <!-- Table dengan caption dan styling -->
  <h3>Table dengan Caption dan Styling</h3>
  <table border="1" cellpadding="5" cellspacing="0" width="100%">
    <caption>Data Karyawan Perusahaan</caption>
    <tr style="background-color: #f0f0f0;">
      <th style="padding: 10px;">ID</th>
      <th style="padding: 10px;">Nama</th>
      <th style="padding: 10px;">Departemen</th>
      <th style="padding: 10px;">Gaji</th>
    </tr>
    <tr>
      <td style="padding: 8px; text-align: center;">001</td>
      <td style="padding: 8px;">John Doe</td>
      <td style="padding: 8px;">IT</td>
      <td style="padding: 8px; text-align: right;">Rp 10.000.000</td>
    </tr>
    <tr style="background-color: #f9f9f9;">
      <td style="padding: 8px; text-align: center;">002</td>
      <td style="padding: 8px;">Jane Smith</td>
      <td style="padding: 8px;">Marketing</td>
      <td style="padding: 8px; text-align: right;">Rp 8.500.000</td>
    </tr>
  </table>
</body>
</html>
\end{lstlisting}

\textbf{Hasil di Browser:}
- Table dengan border dan struktur baris-kolom yang jelas
- Header row dengan background warna berbeda
- Caption di atas table
- Cell padding untuk spacing yang lebih baik
- Data alignment yang sesuai (center untuk ID, right untuk gaji)

\subsection{Advanced Table Features}

Tables memiliki fitur lanjutan untuk struktur yang lebih kompleks:

\begin{itemize}
  \item \textbf{Colspan}: Menggabungkan multiple kolom
  \item \textbf{Rowspan}: Menggabungkan multiple baris
  \item \textbf{Table Sections}: Thead, tbody, tfoot
  \item \textbf{Table Attributes}: Border, cellpadding, cellspacing, width
\end{itemize}

\begin{lstlisting}[caption={Advanced Table Features}, basicstyle=\ttfamily\small, frame=single]
<!DOCTYPE html>
<html>
<head>
  <title>Advanced Table Features</title>
</head>
<body>
  <h1>Advanced Table Demonstrations</h1>
  
  <!-- Table dengan colspan dan rowspan -->
  <h3>Colspan dan Rowspan</h3>
  <table border="1" cellpadding="5" width="100%">
    <tr>
      <th rowspan="2">No</th>
      <th colspan="2">Informasi Mahasiswa</th>
    </tr>
    <tr>
      <th>Nama</th>
      <th>Nilai</th>
    </tr>
    <tr>
      <td>1</td>
      <td>Andi</td>
      <td>A</td>
    </tr>
    <tr>
      <td>2</td>
      <td>Budi</td>
      <td>B</td>
    </tr>
  </table>
  
  <!-- Table dengan thead, tbody, tfoot -->
  <h3>Table Sections</h3>
  <table border="1" cellpadding="5" width="100%">
    <thead>
      <tr style="background-color: #e0e0e0;">
        <th>Produk</th>
        <th>Q1</th>
        <th>Q2</th>
        <th>Q3</th>
        <th>Q4</th>
        <th>Total</th>
      </tr>
    </thead>
    <tbody>
      <tr>
        <td><strong>Laptop</strong></td>
        <td>100</td>
        <td>120</td>
        <td>90</td>
        <td>110</td>
        <td><strong>420</strong></td>
      </tr>
      <tr>
        <td><strong>Smartphone</strong></td>
        <td>80</td>
        <td>95</td>
        <td>110</td>
        <td>125</td>
        <td><strong>410</strong></td>
      </tr>
    </tbody>
    <tfoot>
      <tr style="background-color: #f0f0f0; font-weight: bold;">
        <td>Total</td>
        <td>180</td>
        <td>215</td>
        <td>200</td>
        <td>235</td>
        <td>830</td>
      </tr>
    </tfoot>
  </table>
  
  <!-- Complex table dengan multiple features -->
  <h3>Complex Table Example</h3>
  <table border="1" cellpadding="8" cellspacing="2" width="100%">
    <caption>Jadwal Kuliah Semester Ganjil</caption>
    <thead>
      <tr style="background-color: #4CAF50; color: white;">
        <th rowspan="2">Hari</th>
        <th colspan="2">Pagi</th>
        <th colspan="2">Sore</th>
      </tr>
      <tr style="background-color: #4CAF50; color: white;">
        <th>08:00</th>
        <th>10:00</th>
        <th>13:00</th>
        <th>15:00</th>
      </tr>
    </thead>
    <tbody>
      <tr>
        <td rowspan="2"><strong>Senin</strong></td>
        <td>Matematika</td>
        <td>Fisika</td>
        <td>Kimia</td>
        <td>Biologi</td>
      </tr>
      <tr>
        <td colspan="2" style="text-align: center; font-style: italic;">
          Lab Fisika
        </td>
        <td colspan="2" style="text-align: center; font-style: italic;">
          Lab Kimia
        </td>
      </tr>
      <tr>
        <td rowspan="2"><strong>Selasa</strong></td>
        <td>Bahasa Indonesia</td>
        <td>Bahasa Inggris</td>
        <td>Sejarah</td>
        <td>Geografi</td>
      </tr>
      <tr>
        <td colspan="2" style="text-align: center; font-style: italic;">
          Debat Bahasa
        </td>
        <td colspan="2" style="text-align: center; font-style: italic;">
          Presentasi Sejarah
        </td>
      </tr>
    </tbody>
  </table>
</body>
</html>
\end{lstlisting}

\textbf{Hasil di Browser:}
- Colspan menggabungkan 2 kolom menjadi 1
- Rowspan menggabungkan 2 baris menjadi 1
- Thead dengan background hijau untuk header
- Tbody dengan data utama
- Tfoot dengan total dan background abu-abu
- Complex table structure dengan multiple rowspan/colspan

\subsection{HTML Lists}

Lists digunakan untuk menampilkan item-item yang berurutan atau tidak berurutan:

\begin{itemize}
  \item \textbf{Unordered List}: `<ul>` dengan `<li>` (bullet points)
  \item \textbf{Ordered List}: `<ol>` dengan `<li>` (numbered)
  \item \textbf{Definition List}: `<dl>`, `<dt>`, `<dd>` (term dan definition)
  \item \textbf{Nested Lists}: Lists di dalam lists
\end{itemize}

\begin{lstlisting}[caption={Berbagai Jenis Lists}, basicstyle=\ttfamily\small, frame=single]
<!DOCTYPE html>
<html>
<head>
  <title>Contoh HTML Lists</title>
</head>
<body>
  <h1>Demonstrasi HTML Lists</h1>
  
  <!-- Unordered List -->
  <h3>Unordered List (UL)</h3>
  <p>Fitur HTML5:</p>
  <ul>
    <li>Semantic elements</li>
    <li>Multimedia support</li>
    <li>Form validation</li>
    <li>Local storage</li>
    <li>Geolocation API</li>
  </ul>
  
  <!-- Ordered List -->
  <h3>Ordered List (OL)</h3>
  <p>Langkah belajar web development:</p>
  <ol>
    <li>Learn HTML fundamentals</li>
    <li>Master CSS styling</li>
    <li>Understand JavaScript basics</li>
    <li>Practice with projects</li>
    <li>Build portfolio</li>
  </ol>
  
  <!-- Ordered List dengan type berbeda -->
  <h3>Ordered List Types</h3>
  <p>Popular programming languages:</p>
  <ol type="A">
    <li>JavaScript</li>
    <li>Python</li>
    <li>Java</li>
  </ol>
  
  <ol type="I">
    <li>First place</li>
    <li>Second place</li>
    <li>Third place</li>
  </ol>
  
  <ol type="a">
    <li>Installation guide</li>
    <li>Configuration steps</li>
    <li>Troubleshooting tips</li>
  </ol>
  
  <!-- Definition List -->
  <h3>Definition List (DL)</h3>
  <p>Web development terms:</p>
  <dl>
    <dt>HTML</dt>
    <dd>HyperText Markup Language - bahasa markup untuk membuat halaman web</dd>
    
    <dt>CSS</dt>
    <dd>Cascading Style Sheets - bahasa styling untuk tampilan web</dd>
    
    <dt>JavaScript</dt>
    <dd>Bahasa scripting untuk interaktivitas web</dd>
    
    <dt>API</dt>
    <dd>Application Programming Interface - antarmuka untuk komunikasi antar software</dd>
  </dl>
</body>
</html>
\end{lstlisting}

\textbf{Hasil di Browser:}
- Unordered list dengan bullet points (•)
- Ordered list dengan numbering (1, 2, 3)
- Ordered list dengan letters (A, B, C)
- Ordered list dengan roman numbers (I, II, III)
- Definition list dengan term bold dan definition indented

\subsection{Nested Lists}

Lists dapat disusun secara hierarkis dengan nested lists:

\begin{lstlisting}[caption={Nested Lists}, basicstyle=\ttfamily\small, frame=single]
<!DOCTYPE html>
<html>
<head>
  <title>Nested Lists</title>
</head>
<body>
  <h1>Nested Lists Demonstrations</h1>
  
  <!-- Nested unordered lists -->
  <h3>Teknologi Web Development</h3>
  <ul>
    <li>
      <strong>Frontend Technologies</strong>
      <ul>
        <li>HTML5</li>
        <li>CSS3</li>
        <li>JavaScript ES6+</li>
        <li>
          Frameworks
          <ul>
            <li>React</li>
            <li>Vue.js</li>
            <li>Angular</li>
          </ul>
        </li>
      </ul>
    </li>
    <li>
      <strong>Backend Technologies</strong>
      <ul>
        <li>Node.js</li>
        <li>PHP</li>
        <li>Python</li>
        <li>
          Databases
          <ul>
            <li>MySQL</li>
            <li>MongoDB</li>
            <li>PostgreSQL</li>
          </ul>
        </li>
      </ul>
    </li>
    <li>
      <strong>DevOps Tools</strong>
      <ul>
        <li>Git</li>
        <li>Docker</li>
        <li>CI/CD Pipelines</li>
      </ul>
    </li>
  </ul>
  
  <!-- Mixed nested lists -->
  <h3>Curriculum Web Development</h3>
  <ol>
    <li>
      Semester 1: Fundamentals
      <ul>
        <li>HTML & CSS Basics</li>
        <li>JavaScript Fundamentals</li>
        <li>Responsive Design Principles</li>
      </ul>
    </li>
    <li>
      Semester 2: Intermediate
      <ul>
        <li>Advanced CSS & Frameworks</li>
        <li>DOM Manipulation</li>
        <li>Async JavaScript</li>
      </ul>
    </li>
    <li>
      Semester 3: Advanced
      <ul>
        <li>Full-Stack Development</li>
        <li>Performance Optimization</li>
        <li>Security Best Practices</li>
      </ul>
    </li>
  </ol>
</body>
</html>
\end{lstlisting}

\textbf{Hasil di Browser:}
- Hierarki lists dengan multiple levels
- Berbagai bullet styles untuk setiap level
- Mixed ordered dan unordered lists
- Struktur yang jelas untuk informasi kompleks

\subsection{Contoh Lengkap: Data Dashboard}

Berikut contoh dashboard yang menggabungkan tables dan lists:

\begin{lstlisting}[caption={Dashboard dengan Tables dan Lists}, basicstyle=\ttfamily\small, frame=single]
<!DOCTYPE html>
<html>
<head>
  <title>Data Dashboard</title>
  <style>
    .dashboard {
      font-family: Arial, sans-serif;
      margin: 20px;
    }
    .section {
      margin-bottom: 30px;
      padding: 20px;
      border: 1px solid #ddd;
      border-radius: 8px;
    }
    table {
      width: 100%;
      border-collapse: collapse;
      margin-bottom: 20px;
    }
    th, td {
      padding: 12px;
      text-align: left;
      border-bottom: 1px solid #ddd;
    }
    th {
      background-color: #f2f2f2;
      font-weight: bold;
    }
    .status-active { color: green; }
    .status-pending { color: orange; }
    .status-inactive { color: red; }
  </style>
</head>
<body>
  <div class="dashboard">
    <h1>Project Management Dashboard</h1>
    
    <!-- Summary Table -->
    <div class="section">
      <h2>Project Summary</h2>
      <table>
        <thead>
          <tr>
            <th>Project Name</th>
            <th>Status</th>
            <th>Progress</th>
            <th>Deadline</th>
          </tr>
        </thead>
        <tbody>
          <tr>
            <td>Website Redesign</td>
            <td><span class="status-active">Active</span></td>
            <td>75%</td>
            <td>2024-12-31</td>
          </tr>
          <tr>
            <td>Mobile App</td>
            <td><span class="status-pending">Pending</span></td>
            <td>30%</td>
            <td>2025-01-15</td>
          </tr>
          <tr>
            <td>API Integration</td>
            <td><span class="status-inactive">Inactive</span></td>
            <td>100%</td>
            <td>2024-11-30</td>
          </tr>
        </tbody>
      </table>
    </div>
    
    <!-- Task Lists -->
    <div class="section">
      <h2>Active Tasks</h2>
      <ul>
        <li>
          <strong>Website Redesign Tasks:</strong>
          <ul>
            <li>Homepage mockup approval</li>
            <li>Database optimization</li>
            <li>Performance testing</li>
            <li>User acceptance testing</li>
          </ul>
        </li>
        <li>
          <strong>Mobile App Development:</strong>
          <ul>
            <li>UI/UX design finalization</li>
            <li>Backend API integration</li>
            <li>Testing on multiple devices</li>
          </ul>
        </li>
      </ul>
    </div>
    
    <!-- Team Information -->
    <div class="section">
      <h2>Team Members</h2>
      <table>
        <thead>
          <tr>
            <th>Name</th>
            <th>Role</th>
            <th>Current Tasks</th>
          </tr>
        </thead>
        <tbody>
          <tr>
            <td>John Doe</td>
            <td>Frontend Developer</td>
            <td>Website Redesign</td>
          </tr>
          <tr>
            <td>Jane Smith</td>
            <td>Backend Developer</td>
            <td>API Integration</td>
          </tr>
          <tr>
            <td>Bob Johnson</td>
            <td>UI/UX Designer</td>
            <td>Mobile App Design</td>
          </tr>
        </tbody>
      </table>
    </div>
    
    <!-- Priority Lists -->
    <div class="section">
      <h2>Priority Action Items</h2>
      <ol>
        <li>
          <strong>Critical Priority:</strong>
          <ul>
            <li>Fix production bugs</li>
            <li>Security audit completion</li>
          </ul>
        </li>
        <li>
          <strong>High Priority:</strong>
          <ul>
            <li>Complete website redesign</li>
            <li>Deploy mobile app beta</li>
          </ul>
        </li>
        <li>
          <strong>Medium Priority:</strong>
          <ul>
            <li>Documentation update</li>
            <li>Team training sessions</li>
          </ul>
        </li>
      </ol>
    </div>
  </div>
</body>
</html>
\end{lstlisting}

\textbf{Hasil di Browser:}
- Dashboard dengan 3 sections terorganisir
- Summary table dengan status berwarna
- Task lists dengan hierarki yang jelas
- Team member table dengan informasi lengkap
- Priority action items dengan ordered list
- Kombinasi tables dan lists untuk data management

Tables dan lists yang baik memungkinkan presentasi data yang terstruktur dan mudah dibaca \cite{w3schools-html}. Penggunaan yang tepat akan meningkatkan user experience dan kemudahan dalam mengakses informasi \cite{mdn-html}.

\section{Frames dan Iframes}

Frames dan iframes adalah elemen HTML untuk menampilkan multiple dokumen dalam satu halaman browser. Frames memungkinkan pembagian halaman menjadi beberapa bagian independen, sedangkan iframes menyematkan halaman lain dalam konteks halaman saat ini \cite{w3schools-html}. Meskipun frames sudah deprecated, pemahaman konsep ini penting untuk legacy web development \cite{mdn-html}.

\subsection{HTML Frames}

Frames membagi browser window menjadi multiple panel dengan dokumen terpisah:

\begin{itemize}
  \item \texttt{<frameset>} - Mendefinisikan layout frames
  \item \texttt{<frame>} - Individual frame panel
  \item \texttt{<noframes>} - Fallback untuk browser yang tidak support frames
  \item \texttt{<base>} - Base URL untuk semua links dalam frameset
\end{itemize}

\begin{lstlisting}[caption={Contoh Frames Layout}, basicstyle=\ttfamily\small, frame=single]
<!DOCTYPE html>
<html>
<head>
  <title>Contoh Frames Layout</title>
</head>
<frameset rows="80,*" border="1">
  <!-- Header Frame -->
  <frame src="header.html" name="header" scrolling="no" noresize>
  
  <!-- Main Content Area dengan Columns -->
  <frameset cols="200,*" border="1">
    <!-- Navigation Frame -->
    <frame src="navigation.html" name="navigation" scrolling="auto">
    
    <!-- Content Frame -->
    <frame src="content.html" name="content" scrolling="auto">
  </frameset>
  
  <!-- Fallback untuk browser tanpa support frames -->
  <noframes>
    <body>
      <h1>Browser Anda tidak mendukung frames</h1>
      <p>Silakan gunakan browser modern atau <a href="no-frames.html">versi tanpa frames</a></p>
    </body>
  </noframes>
</frameset>
</html>
\end{lstlisting}

\textbf{Hasil di Browser:}
- Layout dengan header di atas (80px height)
- Navigation di kiri (200px width)
- Content area di kanan (remaining space)
- Setiap frame dapat di-scroll independently
- Noframes message untuk browser yang tidak support frames

\subsection{Frame Attributes dan Targeting}

Frames memiliki atribut untuk mengontrol behavior dan navigation:

\begin{itemize}
  \item \textbf{Layout}: rows, cols, border, frameborder
  \item \textbf{Behavior}: scrolling, noresize, marginwidth, marginheight
  \item \textbf{Targeting}: name, target untuk link navigation
\end{itemize}

\begin{lstlisting}[caption={Frame Attributes dan Targeting}, basicstyle=\ttfamily\small, frame=single]
<!-- header.html -->
<!DOCTYPE html>
<html>
<head>
  <title>Header Frame</title>
  <style>
    body {
      margin: 0;
      padding: 10px;
      background-color: #333;
      color: white;
      font-family: Arial, sans-serif;
    }
    h1 {
      margin: 0;
      font-size: 18px;
    }
    .nav-link {
      color: white;
      text-decoration: none;
      margin-right: 15px;
    }
    .nav-link:hover {
      text-decoration: underline;
    }
  </style>
</head>
<body>
  <h1>Website Navigation</h1>
  <nav>
    <a href="content.html" target="content" class="nav-link">Beranda</a>
    <a href="about.html" target="content" class="nav-link">Tentang</a>
    <a href="contact.html" target="content" class="nav-link">Kontak</a>
  </nav>
</body>
</html>

<!-- navigation.html -->
<!DOCTYPE html>
<html>
<head>
  <title>Navigation Frame</title>
  <style>
    body {
      margin: 0;
      padding: 10px;
      background-color: #f4f4f4;
      font-family: Arial, sans-serif;
    }
    ul {
      list-style-type: none;
      padding: 0;
    }
    li {
      margin-bottom: 8px;
    }
    a {
      display: block;
      padding: 8px;
      text-decoration: none;
      color: #333;
      border-radius: 4px;
    }
    a:hover {
      background-color: #e0e0e0;
    }
    .active {
      background-color: #4CAF50;
      color: white;
    }
  </style>
</head>
<body>
  <h3>Main Menu</h3>
  <ul>
    <li><a href="page1.html" target="content">Halaman 1</a></li>
    <li><a href="page2.html" target="content" class="active">Halaman 2</a></li>
    <li><a href="page3.html" target="content">Halaman 3</a></li>
    <li><a href="page4.html" target="content">Halaman 4</a></li>
  </ul>
  
  <h3>Sub Menu</h3>
  <ul>
    <li><a href="sub1.html" target="content">Sub Halaman 1</a></li>
    <li><a href="sub2.html" target="content">Sub Halaman 2</a></li>
    <li><a href="sub3.html" target="content">Sub Halaman 3</a></li>
  </ul>
</body>
</html>

<!-- content.html -->
<!DOCTYPE html>
<html>
<head>
  <title>Main Content Frame</title>
  <style>
    body {
      margin: 20px;
      font-family: Arial, sans-serif;
      line-height: 1.6;
    }
    .content-section {
      margin-bottom: 30px;
      padding: 20px;
      border: 1px solid #ddd;
      border-radius: 8px;
    }
    h2 {
      color: #333;
      border-bottom: 2px solid #4CAF50;
      padding-bottom: 10px;
    }
  </style>
</head>
<body>
  <div class="content-section">
    <h2>Selamat Datang di Halaman Utama</h2>
    <p>Ini adalah konten utama yang dimuat dalam frame. 
       Navigation dari frame kiri dapat mengubah konten di area ini.</p>
    <p>Frame technology memungkinkan pembagian halaman menjadi 
       beberapa bagian yang independen namun tetap dalam satu window.</p>
  </div>
  
  <div class="content-section">
    <h2>Keuntungan Frame Layout</h2>
    <ul>
      <li><strong>Independent Scrolling:</strong> Setiap frame dapat di-scroll sendiri</li>
      <li><strong>Persistent Navigation:</strong> Navigation tetap visible saat konten berubah</li>
      <li><strong>Modular Design:</strong> Setiap bagian dapat dikelola terpisah</li>
      <li><strong>Targeted Loading:</strong> Hanya bagian yang berubah yang perlu di-load ulang</li>
    </ul>
  </div>
</body>
</html>
\end{lstlisting}

\textbf{Hasil di Browser:}
- Header frame dengan navigation links
- Navigation frame dengan menu yang clickable
- Content frame dengan informasi utama
- Links dengan target="content" mengubah content frame
- Setiap frame dapat di-scroll independently
- Layout yang terstruktur dengan 3 panel

\subsection{HTML Iframes}

Iframes (inline frames) menyematkan dokumen HTML lain dalam halaman saat ini:

\begin{itemize}
  \item \texttt{<iframe>} - Inline frame element
  \item \textbf{Atribut Utama}: src, width, height, name, frameborder, scrolling
  \item \textbf{Security}: sandbox, allow, allowfullscreen, allowpaymentrequest
  \item \textbf{Responsive}: seamless, loading, referrerpolicy
\end{itemize}

\begin{lstlisting}[caption={Contoh Iframe Implementations}, basicstyle=\ttfamily\small, frame=single]
<!DOCTYPE html>
<html>
<head>
  <title>Contoh Iframe Implementations</title>
  <style>
    body {
      font-family: Arial, sans-serif;
      margin: 20px;
      line-height: 1.6;
    }
    .iframe-container {
      margin: 20px 0;
    }
    .iframe-wrapper {
      border: 1px solid #ddd;
      border-radius: 8px;
      overflow: hidden;
      box-shadow: 0 4px 8px rgba(0,0,0,0.1);
    }
    .responsive-iframe {
      width: 100%;
      height: 400px;
      border: none;
    }
    .iframe-grid {
      display: grid;
      grid-template-columns: repeat(auto-fit, minmax(300px, 1fr));
      gap: 20px;
    }
    .iframe-card {
      border: 1px solid #ddd;
      border-radius: 8px;
      overflow: hidden;
    }
    .iframe-card h3 {
      margin: 0;
      padding: 15px;
      background-color: #f8f9fa;
      border-bottom: 1px solid #ddd;
    }
  </style>
</head>
<body>
  <h1>Demonstrasi HTML Iframes</h1>
  
  <!-- Basic Iframe -->
  <div class="iframe-container">
    <h3>Basic Iframe</h3>
    <div class="iframe-wrapper">
      <iframe src="https://www.example.com" 
              width="600" height="400" 
              frameborder="0" 
              scrolling="auto"
              title="External Website">
      </iframe>
    </div>
    <p><small>Iframe menampilkan website eksternal dalam halaman ini.</small></p>
  </div>
  
  <!-- Iframe dengan Security Sandbox -->
  <div class="iframe-container">
    <h3>Iframe dengan Security Sandbox</h3>
    <div class="iframe-wrapper">
      <iframe src="trusted-content.html" 
              width="100%" height="300" 
              sandbox="allow-same-origin allow-scripts allow-forms"
              title="Trusted Content">
      </iframe>
    </div>
    <p><small>Sandbox membatasi kemampuan iframe untuk keamanan.</small></p>
  </div>
  
  <!-- Responsive Iframe -->
  <div class="iframe-container">
    <h3>Responsive Iframe</h3>
    <div class="iframe-wrapper">
      <iframe src="https://www.youtube.com/embed/dQw4w9WgXcQ" 
              class="responsive-iframe"
              allowfullscreen
              title="Video Player">
      </iframe>
    </div>
    <p><small>Iframe video yang responsif dengan aspect ratio.</small></p>
  </div>
  
  <!-- Multiple Iframes Grid -->
  <div class="iframe-container">
    <h3>Multiple Iframes Grid Layout</h3>
    <div class="iframe-grid">
      <div class="iframe-card">
        <h3>Google Maps</h3>
        <iframe src="https://www.google.com/maps/embed?pb=!1m18!1m12!1m3!2d-2.9203175,106.8468725!3d15.8420928!2m3!1f0x1f0x1f0x1f0x1f0" 
                width="100%" height="200" 
                frameborder="0" 
                allowfullscreen
                title="Location Map">
        </iframe>
      </div>
      
      <div class="iframe-card">
        <h3>Weather Widget</h3>
        <iframe src="https://www.weatherwidget.com/wv.php?cid=12345" 
                width="100%" height="200" 
                frameborder="0" 
                scrolling="no"
                title="Weather Information">
        </iframe>
      </div>
      
      <div class="iframe-card">
        <h3>Social Media Feed</h3>
        <iframe src="https://www.facebook.com/plugins/page.php?href=https://www.facebook.com/yourpage&amp;tabs=timeline" 
                width="100%" height="200" 
                frameborder="0" 
                scrolling="no"
                title="Facebook Timeline">
        </iframe>
      </div>
    </div>
    <p><small>Grid layout dengan multiple iframes untuk dashboard.</small></p>
  </div>
  
  <!-- Iframe dengan JavaScript Control -->
  <div class="iframe-container">
    <h3>Iframe dengan JavaScript Control</h3>
    <div class="iframe-wrapper">
      <iframe id="controlled-iframe" 
              src="about:blank" 
              width="100%" height="300" 
              frameborder="0"
              title="Controlled Iframe">
      </iframe>
    </div>
    
    <p>
      <button onclick="loadContent('https://www.example.com')">Load Example.com</button>
      <button onclick="loadContent('https://www.wikipedia.org')">Load Wikipedia</button>
      <button onclick="reloadIframe()">Reload Iframe</button>
      <button onclick="clearIframe()">Clear Iframe</button>
    </p>
  </div>
  
  <script>
    function loadContent(url) {
      document.getElementById('controlled-iframe').src = url;
    }
    
    function reloadIframe() {
      const iframe = document.getElementById('controlled-iframe');
      iframe.src = iframe.src;
    }
    
    function clearIframe() {
      document.getElementById('controlled-iframe').src = 'about:blank';
    }
  </script>
</body>
</html>
\end{lstlisting}

\textbf{Hasil di Browser:}
- Basic iframe dengan external website
- Sandboxed iframe dengan security restrictions
- Responsive iframe yang menyesuaikan lebar container
- Grid layout dengan multiple iframes
- JavaScript-controlled iframe dengan dynamic content loading
- Video iframe dengan fullscreen capability

\subsection{Iframe Security dan Best Practices}

Iframes memiliki pertimbangan keamanan dan best practices penting:

\begin{itemize}
  \item \textbf{Security}: Gunakan sandbox untuk membatasi kemampuan iframe
  \item \textbf{Same-Origin Policy}: Iframe dari domain sama memiliki akses penuh
  \item \textbf{Responsive}: Gunakan responsive design untuk mobile compatibility
  \item \textbf{Performance}: Lazy loading dan optimization
  \item \textbf{Accessibility}: Title dan fallback content
  \item \textbf{SEO}: Iframe content tidak di-index oleh search engines
\end{itemize}

\begin{lstlisting}[caption={Secure Iframe Implementation}, basicstyle=\ttfamily\small, frame=single]
<!DOCTYPE html>
<html>
<head>
  <title>Secure Iframe Best Practices</title>
  <style>
    body {
      font-family: Arial, sans-serif;
      margin: 20px;
    }
    .iframe-container {
      max-width: 800px;
      margin: 0 auto;
    }
    .secure-iframe {
      width: 100%;
      height: 500px;
      border: 1px solid #ddd;
      border-radius: 8px;
    }
    .iframe-info {
      background-color: #f8f9fa;
      padding: 15px;
      border-radius: 8px;
      margin-top: 10px;
    }
    .warning {
      background-color: #fff3cd;
      border: 1px solid #ffeaa7;
      padding: 10px;
      border-radius: 4px;
      margin: 10px 0;
    }
  </style>
</head>
<body>
  <h1>Secure Iframe Implementation</h1>
  
  <div class="iframe-container">
    <h3>Trusted Content Iframe</h3>
    <div class="iframe-info">
      <p><strong>Source:</strong> trusted-content.html</p>
      <p><strong>Sandbox:</strong> allow-same-origin allow-scripts allow-forms</p>
      <p><strong>Security:</strong> Dibatasi untuk domain yang sama</p>
    </div>
    
    <iframe src="trusted-content.html" 
            class="secure-iframe"
            sandbox="allow-same-origin allow-scripts allow-forms allow-popups"
            loading="lazy"
            referrerpolicy="no-referrer-when-downgrade"
            title="Trusted Content">
    </iframe>
    
    <div class="warning">
      <h4>⚠️ Security Considerations:</h4>
      <ul>
        <li><strong>XSS Protection:</strong> Gunakan sandbox untuk mencegah XSS</li>
        <li><strong>Clickjacking:</strong> Gunakan X-Frame-Options header</li>
        <li><strong>Data Privacy:</strong> Hindari sensitive data dalam iframe</li>
        <li><strong>Performance:</strong> Gunakan loading="lazy" untuk optimasi</li>
      </ul>
    </div>
  </div>
  
  <!-- Fallback untuk browser tanpa iframe support -->
  <noscript>
    <div class="warning">
      <h4>JavaScript Disabled</h4>
      <p>Iframe memerlukan JavaScript untuk fungsi optimal. 
         Silakan <a href="enable-javascript.html">aktifkan JavaScript</a> 
         untuk pengalaman penuh.</p>
    </div>
  </noscript>
  
  <!-- Iframe dengan Communication Example -->
  <div class="iframe-container">
    <h3>Iframe Communication</h3>
    <div class="iframe-info">
      <p><strong>PostMessage API:</strong> Komunikasi antar window</p>
    </div>
    
    <iframe id="communication-iframe" 
            class="secure-iframe"
            src="iframe-content.html"
            sandbox="allow-same-origin allow-scripts">
    </iframe>
    
    <p>
      <button onclick="sendMessageToIframe()">Kirim Pesan ke Iframe</button>
      <button onclick="resizeIframe()">Resize Iframe</button>
    </p>
  </div>
  
  <script>
    // Kirim pesan ke iframe
    function sendMessageToIframe() {
      const iframe = document.getElementById('communication-iframe');
      const message = {
        type: 'resize',
        width: 600,
        height: 400
      };
      
      iframe.contentWindow.postMessage(message, '*');
    }
    
    // Resize iframe
    function resizeIframe() {
      const iframe = document.getElementById('communication-iframe');
      iframe.style.width = '100%';
      iframe.style.height = '600px';
    }
    
    // Terima pesan dari iframe
    window.addEventListener('message', function(event) {
      if (event.origin !== window.location.origin) {
        return; // Security check
      }
      
      console.log('Pesan diterima dari iframe:', event.data);
      
      if (event.data.type === 'ready') {
        console.log('Iframe siap menerima pesan');
      }
    });
  </script>
</body>
</html>
\end{lstlisting}

\textbf{Hasil di Browser:}
- Secure iframe dengan sandbox restrictions
- Security considerations dan warnings
- PostMessage communication antar window
- Lazy loading untuk performance
- Fallback content untuk JavaScript disabled
- Responsive design dengan proper sizing

\subsection{Frames vs Iframes - Perbandingan}

Perbandingan antara frames dan iframes:

\begin{table}[h]
\centering
\small
\begin{tabular}{|l|p{6cm}|p{6cm}|}
\hline
\textbf{Aspek} & \textbf{Frames} & \textbf{Iframes} \\
\hline
\textbf{Status} & Deprecated (HTML5) & Supported (HTML5) \\
\hline
\textbf{Syntax} & Terpisah (frameset) & Inline (dalam body) \\
\hline
\textbf{Layout} & Window division & Embedded content \\
\hline
\textbf{Navigation} & Target attribute & PostMessage API \\
\hline
\textbf{SEO} & Buruk (multiple documents) & Buruk (embedded content) \\
\hline
\textbf{Mobile} & Tidak support & Support dengan responsive \\
\hline
\textbf{Use Case} & Legacy applications & Modern embedding \\
\hline
\end{tabular}
\caption{Perbandingan Frames dan Iframes}
\end{table}

\subsection{Modern Alternatives}

Frames sudah deprecated, gunakan alternatif modern:

\begin{itemize}
  \item \textbf{CSS Grid/Flexbox}: Untuk layout multi-column
  \item \textbf{AJAX/SPA}: Single Page Applications dengan dynamic content
  \item \textbf{Web Components}: Custom reusable elements
  \item \textbf{Server-Side Includes}: SSI atau template engines
  \item \textbf{JavaScript Frameworks}: React, Vue, Angular untuk complex UI
\end{itemize}

Frames dan iframes adalah teknologi legacy yang penting untuk dipahami dalam konteks web development historis dan maintenance aplikasi lama \cite{w3schools-html}. Namun, untuk pengembangan modern, disarankan menggunakan alternatif yang lebih aman dan SEO-friendly \cite{mdn-html}.


\begin{aktivitas}
  \item Buat halaman HTML5 dengan struktur semantik (\texttt{header}, \texttt{nav}, \texttt{main}, \texttt{footer}) untuk profil diri.
  \item Tambahkan link, gambar, video, dan tabel ke halaman tersebut.
\end{aktivitas}

\begin{latihan}
  \item Jelaskan peran \texttt{<!DOCTYPE html>} dan bagian \texttt{head} serta \texttt{body}!
  \item Sebutkan minimal 4 elemen semantik HTML5 dan kegunaannya!
  \item Bagaimana elemen \texttt{<video>} dan \texttt{<audio>} memudahkan integrasi multimedia dibandingkan metode lama?
\end{latihan}

\begin{checklist}
  \item Dapat membuat dokumen HTML5 minimal
  \item Dapat menggunakan elemen semantik, link, gambar, multimedia, dan tabel
\end{checklist}

\begin{rangkuman}
Bab ini membahas struktur dasar HTML5, elemen semantik, link, gambar, multimedia (video/audio), dan tabel. HTML5 menyediakan elemen yang kaya untuk mendefinisikan struktur dan konten media secara native \cite{mdn-learn-html}.
\end{rangkuman}

\ifSubfilesClassLoaded{
  \renewcommand{\bibname}{Daftar Pustaka}
  \bibliographystyle{plain}
  \bibliography{references}
}{}
\end{document}
