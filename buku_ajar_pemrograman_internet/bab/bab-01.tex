\documentclass[../main]{subfiles}
\ifSubfilesClassLoaded{\setcounter{chapter}{0}}{}
\begin{document}

\chapter{Pendahuluan dan Orientasi Buku}

\section{Tujuan Buku Ajar}

Buku ajar ini dirancang sebagai panduan komprehensif untuk menguasai \textit{Pemrograman Internet} dengan memanfaatkan teknologi web modern (\cite{mdn-learn}). Fokus utama buku ini adalah pada penguasaan arsitektur web, pemrograman client-side (HTML, CSS, JavaScript), pemrograman server-side, integrasi basis data, dan prinsip keamanan dasar. Tujuan spesifik buku ini adalah memberikan landasan yang kuat bagi mahasiswa untuk mengembangkan aplikasi web yang fungsional dan aman.

Pembelajaran pemrograman internet mencakup berbagai lapisan teknologi yang saling melengkapi. Di sisi client, mahasiswa akan mempelajari struktur konten (HTML), presentasi visual (CSS), dan interaktivitas (JavaScript) sesuai standar web yang ditetapkan oleh W3C dan WHATWG \cite{whatwg}. Di sisi server, mahasiswa akan diperkenalkan pada pemrosesan request, koneksi basis data, dan integrasi aplikasi web dinamis.

Setelah mempelajari buku ini secara menyeluruh, mahasiswa diharapkan mampu:
\begin{enumerate}
  \item Memahami arsitektur client-server dan protokol HTTP \cite{mdn-http}
  \item Membangun antarmuka web responsif dengan HTML5, CSS3, dan JavaScript
  \item Mengimplementasikan logika server-side dan integrasi basis data
  \item Mengintegrasikan aplikasi web dengan PHP dan MySQL
  \item Menerapkan prinsip keamanan web dasar
\end{enumerate}

\section{Keterkaitan Buku Ajar dengan RPS Berbasis OBE}

Buku ajar ini dirancang selaras dengan Rencana Pembelajaran Semester (RPS) mata kuliah Pemrograman Internet yang berbasis Outcome-Based Education (OBE). Keterkaitan ini diwujudkan melalui pemetaan eksplisit setiap bab ke Sub-CPMK dan CPMK yang telah ditetapkan dalam silabus. Pendekatan ini memastikan bahwa materi pembelajaran, aktivitas, dan asesmen sejalan dengan capaian yang diharapkan oleh program studi.

\textbf{Alignment dengan CPL dan CPMK:} Setiap bab dalam buku ini dipetakan ke Sub-CPMK yang berkontribusi pada pencapaian CPMK dan CPL. Bab II--IV mencakup Sub-CPMK 1.1, 1.2 (arsitektur web) dan Sub-CPMK 2.1 (HTML). Bab V--VIII mencakup Sub-CPMK 2.2, 2.3 (CSS dan JavaScript). Bab IX--XIV mencakup Sub-CPMK 3.1, 3.2 (server-side, database) dan Sub-CPMK 4.1, 4.2 (API, keamanan).

\textbf{Integrasi Metode Pembelajaran:} Buku ini mengintegrasikan metode pembelajaran yang tercantum dalam RPS, meliputi ceramah interaktif, demonstrasi dan live coding, praktikum terbimbing, Problem-Based Learning (PBL), Project-Based Learning, peer review, dan studi kasus. Setiap bab menyediakan aktivitas yang mendukung metode tersebut sesuai konteks materi \cite{freecodecamp}.

\textbf{Sistem Penilaian:} Komponen asesmen dalam buku ini sejalan dengan bobot penilaian RPS: Tugas Individu (15\%), Kuis (10\%), Praktikum (15\%), UTS (20\%), Proyek Kelompok (20\%), dan UAS (20\%).

\section{Petunjuk Penggunaan Buku Ajar}

\subsection{Untuk Mahasiswa}

Buku ini dirancang untuk dipelajari secara bertahap selama 16 pertemuan (2 SKS). Sebelum perkuliahan, bacalah Sub-CPMK di awal setiap bab untuk memahami target pembelajaran, pelajari materi pokok dengan seksama, dan jalankan semua contoh kode menggunakan browser dan editor teks. Mencatat pertanyaan atau konsep yang belum dipahami akan memperkaya diskusi di kelas.

Selama perkuliahan, manfaatkan waktu untuk diskusi, praktikum, dan live coding. Kerjakan aktivitas pembelajaran secara aktif, tanyakan hal-hal yang belum jelas, dan berpartisipasi dalam peer review terhadap tampilan atau kode web yang dibuat rekan. Setelah perkuliahan, kerjakan latihan dan refleksi, lakukan asesmen mandiri, centang checklist kompetensi yang telah dikuasai, dan kerjakan proyek mini untuk memperdalam pemahaman \cite{webdev}.

\subsection{Untuk Dosen}

Buku ini dapat digunakan sebagai bahan ajar utama, sumber latihan dan tugas, referensi untuk menyusun soal ujian, dan panduan untuk merancang aktivitas pembelajaran. Dokumentasi daring seperti MDN, W3Schools, dan OWASP dapat dilengkapi sebagai referensi tambahan sesuai kebutuhan \cite{mdn-learn}.

\section{Konteks Kurikulum OBE}

\textbf{Outcome-Based Education (OBE)} adalah pendekatan pembelajaran yang berfokus pada pencapaian hasil (\textit{outcomes}) yang terukur. Dalam OBE, proses pembelajaran dirancang secara sistematis untuk memastikan mahasiswa mencapai kompetensi yang telah ditetapkan. Prinsip utama OBE meliputi: kejelasan fokus pada outcomes, perancangan kurikulum secara top-down dari outcomes, ekspektasi tinggi, dan kesempatan beragam untuk mencapai outcomes.

Buku ini mengimplementasikan OBE melalui komponen-komponen yang terstruktur. Setiap bab diawali dengan Sub-CPMK yang eksplisit; materi disusun dari dasar ke lanjut secara sistematis; aktivitas beragam (latihan, studi kasus, proyek, peer review) mendukung pencapaian kompetensi; asesmen terukur dengan rubrik yang jelas; dan checklist memberikan umpan balik berkelanjutan bagi mahasiswa.

\begin{table}[h]
\centering
\small
\begin{tabular}{|l|p{8cm}|}
\hline
\textbf{Komponen OBE} & \textbf{Implementasi dalam Buku} \\
\hline
Outcomes yang Jelas & Sub-CPMK eksplisit di setiap bab \\
\hline
Pembelajaran Terstruktur & Materi dari arsitektur web $\rightarrow$ HTML $\rightarrow$ CSS $\rightarrow$ JS $\rightarrow$ server $\rightarrow$ basis data $\rightarrow$ keamanan \\
\hline
Aktivitas Beragam & Latihan coding, studi kasus web, proyek aplikasi, peer review \\
\hline
Asesmen Terukur & Rubrik penilaian untuk tugas, praktikum, proyek \\
\hline
Feedback Berkelanjutan & Checklist untuk self-assessment per bab \\
\hline
\end{tabular}
\caption{Implementasi OBE dalam Buku Ajar Pemrograman Internet}
\end{table}

\section{Peta Konsep Pemrograman Internet}

Mata kuliah Pemrograman Internet mencakup topik-topik utama yang saling terkait dan disusun mengikuti alur pembelajaran dari dasar ke lanjut \cite{mdn-learn}. Peta konsep ini membantu mahasiswa memvisualisasikan struktur materi secara keseluruhan.

\begin{enumerate}
  \item \textbf{Bab II}: Landasan Teori --- Arsitektur web, HTTP, client-server
  \item \textbf{Bab III}: HTML5 --- Struktur dokumen, elemen semantik, link, gambar, tabel
  \item \textbf{Bab IV}: HTML5 Form --- Form dan elemen input
  \item \textbf{Bab V}: CSS3 --- Selector, box model, warna, typography
  \item \textbf{Bab VI}: CSS3 Layout --- Flexbox, Grid, desain responsif
  \item \textbf{Bab VII}: JavaScript --- Sintaks, DOM, event handling
  \item \textbf{Bab VIII}: JavaScript --- Validasi form, integrasi dengan HTML/CSS
  \item \textbf{Bab IX}: Server-Side --- Pengenalan PHP/Node.js, request-response
  \item \textbf{Bab X}: Basis Data --- Integrasi, CRUD
  \item \textbf{Bab XI}: Session, Cookie, Autentikasi
  \item \textbf{Bab XII}: REST API --- Endpoint, Fetch/AJAX
  \item \textbf{Bab XIII}: Keamanan Web --- XSS, CSRF, validasi, sanitasi
  \item \textbf{Bab XIV}: Evaluasi dan Integrasi Kompetensi
\end{enumerate}

\textbf{Alur Pembelajaran:} Bab II--IV membangun fondasi arsitektur dan struktur konten; Bab V--VIII membangun antarmuka dan interaktivitas; Bab IX--XII membangun logika server, basis data, dan API; Bab XIII menambahkan lapisan keamanan; Bab XIV mengintegrasikan seluruh kompetensi melalui evaluasi komprehensif.


\begin{rangkuman}
Bab ini memperkenalkan tujuan buku ajar Pemrograman Internet, keterkaitan dengan RPS berbasis OBE, petunjuk penggunaan, dan konteks kurikulum OBE. Pemahaman yang baik tentang struktur dan pendekatan buku ini akan membantu Anda memaksimalkan pembelajaran.

\textbf{Poin Kunci:}
\begin{itemize}
  \item Buku ini dirancang dengan pendekatan OBE yang fokus pada pencapaian kompetensi terukur
  \item Setiap bab dipetakan ke Sub-CPMK yang berkontribusi pada CPL
  \item Materi mencakup HTML, CSS, JavaScript, server-side, basis data, API, dan keamanan web
  \item Gunakan komponen OBE (Sub-CPMK, aktivitas, latihan, asesmen, checklist) secara optimal
\end{itemize}
\end{rangkuman}

\ifSubfilesClassLoaded{
  \renewcommand{\bibname}{Daftar Pustaka}
  \bibliographystyle{plain}
  \bibliography{references}
}{}
\end{document}
