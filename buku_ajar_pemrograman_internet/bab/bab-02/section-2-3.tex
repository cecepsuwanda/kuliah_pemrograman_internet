\section{Client-Side vs Server-Side}

Pemrograman web dibagi menjadi dua ranah utama: client-side dan server-side \cite{mdn-learn}. Client-side merujuk pada kode yang dijalankan di browser pengguna, sedangkan server-side merujuk pada kode yang dijalankan di server. Keduanya memiliki peran yang saling melengkapi dan sering bekerja bersama dalam satu aplikasi web yang lengkap.

Teknologi client-side meliputi HTML (struktur konten), CSS (tampilan dan layout), dan JavaScript (perilaku dan interaktivitas) \cite{mdn-html}, \cite{mdn-css}, \cite{mdn-js}. Kode client-side bersifat terbuka—pengguna dapat melihat dan memeriksa kode melalui alat pengembang browser. Oleh karena itu, logika sensitif dan data rahasia tidak boleh disimpan atau diproses di client-side.

Teknologi server-side mencakup PHP, Node.js, Python (Django, Flask), Java (Spring), dan bahasa lain yang berjalan di server \cite{php-manual}, \cite{nodejs}. Kode server-side bersifat tertutup dari pengguna; server memproses permintaan, mengakses basis data, menerapkan logika bisnis, dan mengembalikan hasil. Data sensitif dan validasi penting harus dilakukan di server untuk keamanan \cite{owasp-input}.
