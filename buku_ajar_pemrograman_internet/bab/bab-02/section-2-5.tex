\section{Ringkasan Landasan Teori}

Landasan teori pemrograman internet mencakup arsitektur client-server, protokol HTTP, dan pemisahan peran client-side serta server-side. Arsitektur client-server adalah fondasi di mana browser (client) berkomunikasi dengan web server melalui pertukaran permintaan dan respons HTTP \cite{mdn-http}.

Protokol HTTP mendefinisikan metode (GET, POST, PUT, DELETE), status code, dan format pesan yang memungkinkan interoperabilitas antara berbagai klien dan server. Di sisi lain, pemahaman yang jelas tentang client-side versus server-side membantu pengembang memilih teknologi dan lapisan yang tepat untuk setiap kebutuhan \cite{devdocs}. Referensi daring seperti MDN, W3C, dan WHATWG menyediakan dokumentasi yang komprehensif untuk mempelajari standar web secara mendalam \cite{whatwg}.
