\section{PHP dan Node.js: Request-Response}

PHP adalah bahasa server-side yang tertanam dalam HTML dan dijalankan oleh web server \cite{php-manual}. Variabel \texttt{\$\_GET} dan \texttt{\$\_POST} menyimpan data dari request; \texttt{\$\_SERVER} berisi informasi server dan request \cite{w3schools-php}. Node.js menjalankan JavaScript di server; framework Express menyederhanakan penanganan route dan middleware \cite{express}, \cite{mdn-express}. Baik PHP maupun Node.js dapat membaca body request, mengembalikan HTML atau JSON, dan mengatur header respons \cite{nodejs}. Pemilihan teknologi bergantung pada kebutuhan proyek, tim, dan ekosistem yang tersedia \cite{php-right-way}.

\begin{lstlisting}[caption={Contoh PHP Memproses Form}, basicstyle=\ttfamily\small, frame=single]
<?php
if ($_SERVER['REQUEST_METHOD'] === 'POST') {
  $nama = htmlspecialchars($_POST['nama']);
  echo "Halo, " . $nama;
}
?>
\end{lstlisting}
